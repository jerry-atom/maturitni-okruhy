\chapter{Firmy a jejich formy}

\begin{description}
    \item[Právní osobnost] Vzniká narozením a zaniká smrtí. Schopnost být nositelem práv a povinností (právní subjektivita).
    \item[Svéprávnost] Způsobilost k právním úkonům - právně jednat (uzavírat smlouvy). Plná svéprávnost platí dovršením 18 let.
\end{description}

\paragraph{Fyzické osoby}

\begin{itemize}
    \item je jeden konkrétní člověk, který platí daň z příjmu
    \item podnikatelé podnikající na základě živnostenského či jiného oprávnění, zaměstnanci
    \item způsobilost podnikat nabývá dovršením 18 let
\end{itemize}

\textbf{Členění fyzických osob}:
\begin{enumerate}
    \item Podnikatelé
    \begin{itemize}
        \item Podnikající na základě živnostenského oprávnění dle živnostenského zákona
        \item Podnikající na základě jiného oprávnění dle jiných zákonů (lékaři, veterináři)
    \end{itemize}
    \item Zaměstnanci
    \begin{itemize}
        \item Osoby schopné vstupovat do pracovně právních vztahů se svými zaměstnavateli, zde se řídíme především zákoníkem práce
    \end{itemize}
    \item Spotřebitel
    \begin{itemize}
        \item Spotřebitel je každý člověk, který sám uzavírá smlouvu s podnikatelem nebo s ním jedná (NOZ)
    \end{itemize}
\end{enumerate}

\paragraph*{Právnické osoby}
\begin{itemize}
    \item je společnost několika osob, která je oprávněna vstupovat do právních vztahů a jednat svým jménem
    \item jsou zapsány v obchodním rejstříku
    \item obchodní společnosti, družstva, státní podniky
\end{itemize}

\textbf{Členění právnických osob}:
\begin{enumerate}
    \item Korporace (obchodní korporace a spolky, politické strany)
    \item Fundace (nadace a nadační fondy)
    \item Ústavy (propojení majetkové a osobní složky)
\end{enumerate}

\textbf{Členění obchodních společností}:
\begin{enumerate}
    \item Osobní společnosti
    \begin{itemize}
        \item veřejná obchodní společnost (v.o.s.)
        \item komanditní společnost (k.s.)
    \end{itemize}
    \item Kapitálové společnosti
    \begin{itemize}
        \item společnost s ručením omezeným (s.r.o.)
        \item akciová společnost (a.s.)
    \end{itemize}
\end{enumerate}


\textbf{Živnost} je soustavná Činnost provozovaná samostatně, vlastním jménem, na vlastní
odpovědnost, za účelem dosažení zisku a za podmínek stanovených tímto zákonem

\textbf{Živnost není}
\begin{itemize}
    \item činnosti vyhrazené zákonem státu nebo určené právnické osoby
    \item využívání výsledků duševní tvůrčí činnosti (autorský zákon, vynálezy)
    \item povolání lékařů, veterinářů, advokátů, notářů, znalců, daňových poradců, makléřů
    \item činnost bank, pojišťoven, penzijních fondů, burz, pořádání loterií, hornická činnost, výroba a rozvod elektřiny, plynu a tepla, zemědělství, provoz drah, telekomunikační sítě, výroba léčiv, zprostředkovávání zaměstnání, výchova a vzdělávání ve školách \ldots
    \item pronájem nemovitostí, bytu a nebytových prostor
\end{itemize}

\textbf{Všeobecné podmínky pro podnikání}:
\begin{itemize}
    \item plná svéprávnost (18 let a způsobilost k právním úkonům)
    \item bezúhonnost (dle výpisu z TR, který není starší více než 3 měsíce)
\end{itemize}

\subparagraph{Zvláštní podmínky pro podnikání:}
\begin{itemize}
    \item odborná a jiná způsobilost (u řemeslných živností se požaduje vyučení v oboru)
\end{itemize}

\subparagraph{Případy, kdy nelze provozovat činnost:}
\begin{itemize}
    \item když soud uložil podnikateli zákaz činnosti v oboru
    \item když na majetek podnikatele byl prohlášen konkurz
    \item podnikat nelze v případě, pokud návrh na konkurz byl zrušen pro nedostatek majetku
\end{itemize}

\subparagraph{Emancipace}
\begin{itemize}
    \item osvobození z podřízeného postavení, zrovnoprávnění
    \item boj za rovnoprávnost žen
\end{itemize}

\textbf{Odpovědný zástupce} je fyzická osoba ustanovená podnikatelem, která odpovídá za řádný provoz živnosti a dodržování živnostenskoprávních předpisů a která je obvykle v pracovně právním nebo jiném vztahu k podnikateli.

Odpovědného zástupce je povinen ustanovit:
\begin{enumerate}
    \item podnikatel - FO, který nesplňuje zvláštní podmínky provozování živnosti
    \item podnikatel - zahraniční FO, který nemá na území ČR povolen pobyt
    \item podnikatel - PO se sídlem v ČR.
    \item podnikatel - zahraniční PO
\end{enumerate}

\textbf{Rozdělení živností}:
\begin{enumerate}
    \item ohlašovací
        \begin{itemize}
            \item Řemeslné - podmínkou je odborná způsobilost získaná vyučením v oboru
            \item Vázané - podmínkou je odborná způsobilost (autoškola)
            \item Volné - poskytování služeb pro zemědělství, zahradnictví, lesnictví, myslivost
        \end{itemize}				
    \item koncesované - lze je provozovat pouze na základě udělení koncese-
\end{enumerate}	

\textbf{Zánik Živnostenského oprávnění}:
\begin{itemize}
    \item smrtí podnikatele
    \item zánikem právnické osoby, výmazem zahraniční osoby z obchodního rejstříku
    \item uplynutím doby, pokud bylo vydáno na dobu určitou
    \item rozhodnutím živnostenského úřadu o zrušení živnostenského oprávnění
\end{itemize}

\textbf{Živnostenský zákon dále stanovuje}:
\begin{itemize}
    \item povinnosti podnikatelů	
        \begin{itemize}
            \item povinnost dokladovat kontrolním orgánům způsob nabytí prodávaného zboží
            \item povinnost zajistit, aby v provozovně, byla přítomna osoba znalá českého nebo sloven. jazyka
        \end{itemize}
    \item na požádání vydávat kupujícímu doklady o prodeji zboží
    \item každý podnikatel dostane své IČ (identifikační číslo)
    \item Živnostenské úřady vedou živnostenský rejstřík v elektronické podobě
    \item velká část Živnostenského rejstříku je veřejná - obsahuje údaje o podnikatelích
    \item DIČ = daňové identifikační číslo
    \item Živnostenský rejstřík - informační systém veřejné správy.
\end{itemize}

