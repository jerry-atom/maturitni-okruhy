\chapter{Marketing}

Marketing je nauka o trhu, podnikatelská koncepce. Marketing je proces řízení, jehož výsledkem je poznání, předvídání, ovlivňování a v konečné fázi uspokojení střeb a přání zákazníka efektivním a výhodným způsobem, zajišťujícím splnění cílů organizace.

\paragraph*{Historie marketingu-fáze vývoje}
\begin{description}
    \item[Výrobní koncepce] (co nejlevnější výrobek)
        \begin{itemize}
            \item představiteli této koncepce byli Henry Ford v USA a Tomáš Baťa u nás
        \end{itemize}
    \item[Výrobková koncepce] (kvalitní výrobek)
        \begin{itemize}
            \item výroba je menší, světová krize
            \item důležitá vysoká kvalita, velký důraz se klade na technický rozvoj a inovaci výrobků
        \end{itemize}
    \item[Prodejní koncepce] (přeceňování úlohy reklamy)
        \begin{itemize}
            \item po 2.světové válce, výrobek se lépe prodává s reklamou-tisk, tv, rozhlas
        \end{itemize}
    \item[Marketingová koncepce] (poznej potřeby svého zákazníka a teprve pak vyráběj)
        \begin{itemize}
            \item státy zničené válkou po obnově hospodářství
            \item hospodářství roste, trh je nasycený
            \item poznejme naše potencionální zákazníky, jejich potřeby, a teprve vyrábějme
        \end{itemize}
    \item[Sociální marketing] (zohlední nejen potřeby zákazníka, ale i celé společnosti)
        \begin{itemize}
            \item nejnovější vývojový stupeň marketingu
            \item bere se ohled na společenské zájmy
        \end{itemize}
\end{description}

\paragraph*{Tomáš Baťa}
    \begin{itemize}
        \item {Československý podnikatel}
        \item Vytvořil ve Zlíně obuvnickou firmu Baťa, postupně rozsáhlý komplex výroby, obchodu, dopravy, služeb a financí
        \item Jeden z největších podnikatelů své doby
    \end{itemize}

\paragraph*{Prodej $\times$ Marketing}
Prodej musíme chápat jako jednu z činností podniku. Prodej je podmnožinou ve velké množině marketingových činností.

Marketing není samostatná podniková funkce, musí prolínat všemi činnostmi podniku.

\paragraph*{Informační systém marketingu}
    Základní zdroje informací:
    \begin{itemize}
        \item Vnitřní zdroje-informace, které má firma sama k dispozici. Sem patří informace z podnikového účetnictví a statistické evidence, z ekonomických rozborů.
        \item Vnější zdroje-informace, které nejsou firmě běžně dostupné a zjistitelné. Je třeba informace získat z vnějších zdrojů (statistické přehledy, články v odbor.časopisech, odbor. konference nebo prostřednictvím \textbf{marketingového výzkumu}.
    \end{itemize}

K metodám marketingového výzkumu patří:
\begin{itemize}
    \item Pozorování (např.známé počítání aut u cest pro zjištění průjezdnosti)
    \item Experiment (zjišťování reakce zákazníků na změnu ceny, reklamy, obalu\ldots)
    \item Průzkum trhu (např.formou dotazníků v prodejnách,pohovorů se zákazníky)
\end{itemize}

Informace můžeme členit také:
\begin{itemize}
    \item Primární - info byly získány za účelem marketingového využití
    \item Sekundární - info původně sloužily jinému účelu (uce,statistika), ale jsou využitelné 1 pro marketing
\end{itemize}

Informace jiných věd:
Psychologie (chování kupujících), sociologie (chování skupin lidí), demografie (věkové rozložení).

Tři základní subjekty, o kterých jsou zjišťovány informace:
O zákaznících, o konkurenci, o vlastní firmě.

\textbf{Segmentace trhu}:
Souvisí s cíleným marketingem a znamená rozčlenění trhu na specifické segmenty zákazníků, na které se firma zaměří svým marketingovým mixem.

\textbf{Základní marketingové strategie}
\begin{enumerate}
    \item hromadný marketing - univerzální výrobek nabízíme všem zákazníkům (sůl)
    \item diferencovaný marketing - výrobek v několika obměnách nabízíme všem zákazníkům (cukrovinky, drogerie, auta)
    \item cílený marketing - výrobce nabízí konkrétnímu segmentu zákazníků specifický výrobek (automobil)
\end{enumerate}

Postup při aplikaci cíleného marketingu:
\begin{itemize}
    \item Rozčleníme trh na tzv. \textbf{tržní segmenty}
    \item Zaměříme se na jeden nebo více z nich (tzv. tržní zacílení - segment rodinných automobilů)
    \item Hledáme a určujeme konkrétní nástroje a prostředky pro získání potencionálních zákazníků (tzv. tržní umístění)
\end{itemize}

Hlediska segmentace trhů spotřebních:
\begin{itemize}
    \item Geografická (územní)
    \item Demografická (věk, vzdálenost, pohlaví, povolání, národnost)
    \item Psychografická (podle sociální třídy, osobnosti)
    \item Podle chování (věrnost značce)
\end{itemize}

Hlediska segmentace trhů průmyslových:
\begin{itemize}
    \item Geografická (velikost firmy)
    \item Kritéria provozu (technologie, potřeby služeb a servisu)
    \item Nákupní kritéria (firmy, které chtějí nízkou cenu, kvalitu, servis)
\end{itemize}

\textbf{Výhody segmentace}
\begin{itemize}
    \item Lepší uspokojení konkrétních potřeb zákazníka
    \item Efektivnější stimulace a distribuce výrobku nebo služby
    \item Získání konkurenční převahy ve vybraném segmentu trhu (mýdlo na hlavu)
\end{itemize}

\textbf{Charakteristiky určitého segmentu}
\begin{itemize}
    \item Velikost a síla segmentu (kolik segment obsahuje zákazníků a jaká je jejich kupní síla)
    \item Vývojový trend segmentu
    \item Síla konkurence v daném segmentu, a to současná i potencionální, substituující výrobky (jiné výrobky uspokojí stejnou potřebu zákazníka, maso vepřové mohou nahradit hovězím)
\end{itemize}

\paragraph*{Potrfoliová matice růst - podíl $\times$ analýza BCG}

Bostonská matice pochází od poradenské firmy Boston Consulting Group (BCG), odtud také její název \emph{BCG matice} nebo \emph{Bostonská matice}. Používá se pro hodnocení porfolia produktů podniku.

[Obr. Bostonská matice]

Kvadranty portfoliové matice
\begin{description}
    \item[Otazníky] (Question marks) jsou takové SBU\footnote{strategická obchodní jednotka (Strategic Bussines Unit)} firmy, které se uskutečňují na trzích s vysokým tempem růstu, ale mají na trzích malý relativní tržní podíl. (elektro auta-TESLA)
    \item[Hvězdy] (Stars) pokud je SBU-otazník úspěšný stává se hvězdou a má vysoký podíl na trhu (Red Bull)
    \item[Peněžní krávy] (Cash cows) tyto SBU už nemají tak velké tempo růstu na trhu, proto už nemají, tak vysoké vklady a naopak samy jsou nejdůležitější zdroj příjmů pro firmu (Škoda-Octavia)
    \item[Bídní psi] (Dogs) jsou takové SBU, které mají slabý tržní podíl a nízké tempo růstu. Tyto SBU znamenají nízké zisky nebo dokonce ztrátu (Microsoft Lumia)
\end{description}

\paragraph*{SWOT analýzy}
\begin{itemize}
    \item analýza vnitřního prostředí obchodní jednotky (silné a slabé stránky) a analýza okolí (příležitosti a ohrožení)
    \item při této analýze firma kriticky hodnotí sama sebe v porovnání s konkurencí
    \item musí zohledňovat 1 další vlivy o okolí í např. politický vývoj, (egislativní vývoj..
\end{itemize}

[Obr. SWOT analýzy]

\paragraph*{Marketingový Mix (4P)}
MM je jedním ze základních nástrojů marketingu. Pomáhá nám rozčlenit, na co se zaměřit při tvorbě marketingového plánu.

MM se skládá z následujících prvků:
\begin{itemize}
    \item Výrobek (product)
    \item Cena (price)
    \item Propagace (promotion)
    \item Distribuce (placement)
\end{itemize}

\begin{description}
    \item[P1 - VÝROBEK] Výrobek je statek či služba, který se stává předmětem směny na trhu a je určen k uspokojení potřeb zákazníka. Marketing hovoří o \uv{komplexním výrobku}.
    \item[P2 - CENA] Je jedním z nejdůležitějších nástrojů MM a její správná volba je velmi náročná.
    \item[P3 - PROPAGACE] Propagace je forma komunikace mezi prodávajícím a kupujícím, jejímž cílem je větší prodej výrobku nebo služby. (podpora prodeje, reklama, personál\ldots)
    \item[P4 - DISTRIBUCE] Distribuční cesty - prodejní cena = souhrn všeho co zajistí tok zboží od výrobce k zákazníkovi (MO, VO\ldots)
\end{description}

[Obr. Komplexní výrobek se skládá z]

\paragraph*{Životní cyklus výrobku-fáze}
\begin{enumerate}
    \item Uvedení na trh-v této fázi firma do výrobku a jeho prosazení na trhu hodně investuje a je ztrátová
    \item Růst-objem prodeje se zvětšuje a firma začíná dosahovat zisky
    \item Zralost-prodej a zisk již tolik nerostou, dosahují vrcholu a začínají stagnovat pro firmu je výhodné, aby tato fáze trvala, co nejdéle
    \item Pokles-zájem zákazníků o výrobek slábne a firma, chce-li přežít, musí přijít s výrobkem inovovaným
\end{enumerate}

\paragraph*{Průběh S-křivky}
Je samozřejmě u různých výrobků odlišný a na její tvar má vliv celá řada faktorů, ovšem zákonité snížení zájmu zákazníků jednou určitě přijde a fáze poklesu je neodvratná.

\paragraph*{Značka výrobku}
Značka je nedílnou součástí komplexního výrobku. Značka odlišuje výrobek od ostatních obdobných výrobků na trhu. Značka by měla být registrována formou ochranné známky, aby byla chráněna před zneužitím.

Typy značek:
\begin{enumerate}
    \item značka výrobce (adidas, Coca Cola)
    \item značka obchodu (Ahold má značku Albert)
\end{enumerate}

