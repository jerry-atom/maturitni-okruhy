\chapter{Bankovnictví}

Bankovní soustava - historie do roku 1990)

Do roku 1990 byl jednoúrovňový bankovní systém s výrazným monopolem státní banky
československé.

Bankovní systém od roku 1990- charakteristika, licence, bankovní dohled

Od roku 1990 je bankovní systém dvouúrovňový

1. Centrální banka ČNB - státní instituce, nepodnikatelský subjekt

2. obchodní banky - podnikatelské subjekty

V roce 2004 se bankovní sektor stabilizoval a vstupem do EU v květnu 2004 lze využít
princip jednotné licence, která vychází ze svobody poskytování služeb a svobody usazování,
jakožto jedněch ze základních zásad, na nichž stojí EU.

Jednotná licence představuje další možnosti pro podnikání zahraničních bank z EU v ČR, ale i
našich bank na území členských států EU a ESVO a nemusí procházet licenčním řízením v
hostitelském státě. Princip také mění postavení bankovního dohledu CNB.

ČNB - základní charakteristika

ČNB je centrální bankou českého státu. Má postavení ústředního orgánu státní správy v
oblasti měny, bankovnictví a vydávání obecně závazných předpisů. Je právnickou osobou,
která usměrňuje peněžní trh z měnových hledisek, reguluje činnost bank a spořitelen
bankovními ekonomickými nástroji, emituje peníze a hospodaří podle zásad stanovených
vládou. Její postavení a funkce jsou především měnově řídící a nikoliv podnikatelské, CNB
nepracuje na komerčních principech.

Nejvyšší řídící orgán, cuvernér

Nejvyšším řídícím orgánem je bankovní rada; v čele stojí guvernér CNB - jmenuje jej a
odvolává prezident. V současné době je guvernérem Jiří Rusnok

Základní úkoly centrální banky a její nezávislost



Hlavním cílem ČNB je zabezpečovat stabilitu české měny. Za tímto účelem plní tyto
funkce:

- určuje a prosazuje vnitřní a vnější měnovou politiku

- sleduje množství peněz v oběhu, emituje (vydává) nové peníze a opotřebované nebo
neplatné peníze stahuje z oběhu

- dohlíží nad činností obchodních bank, poskytuje bankám úvěry a ukládá jejich depozita
(banka bank)

67
\newpage
- vede účty státního rozpočtu

- spravuje měnové rezervy ve zlatě a devizách
- obchoduje s cennými papíry

- je vrcholnou institucí bankovního dozoru

Učinnost měnové politiky centrální banky je přímo úměrná její nezávislosti (především na
vládě)

Přímé nástroje - základní charakteristika

CNB disponuje řadou nástrojů, pomocí kterých prosazuje své cíle a měnovou politiku. Tyto
nástroje můžeme rozdělit na přímé (administrativní, omezující volné tržní hospodářství) a
nepřímé, které využívají tržních zákonů a plošně působí na ostatní subjekty finančního trhu.

Pravidla likvidity, úvěrové kontingenty, povinné vklady, doporučení, výzvy, dohody

Přímé nástroje mají velký vliv na finanční hospodářství, proto jich centrální banka využívá
jen výjimečně a na přechodnou dobu. K těmto nástrojům patří:

e | Pravidla likvidity - centrální banka určuje obchodním bankám, jaký mají mít vztah
mezi aktivy a pasivy. Patří sem například ukazatel kapitálové přiměřenosti. Aktivní
operace banky mohou činit maximálně 125% vlastního kapitálu, 8% z hlediska
pokladní hotovosti k aktivům.

« | Povinné vklady - povinné vedení běžných účtů státních institucí u centrální banky.

« | Úvěrové kontingenty - určení limitních úvěrů a úvěrových stropů. Patří mezi velmi
razantní přímé nástroje.

Nepřímé nástroje - základní charakteristika

využívají tržních zákonů a plošně působí na ostatní subjekty finančního trhu. Využívají
tržních zákonů a plošně působí na ostatní subjekty finančního trhu. využívají tržních zákonů a
plošně působí na ostatní subjekty finančního trhu

Povinné minimální rezervy, operace na volném trhu, diskontní sazba, lombardní úvěr,
konverze měny, swapové obchody

Mezi nepřímé nástroje centrální banky patří:

e © Diskontní sazba - úroková sazba, za kterou si mohou komerční banky půjčit peníze
od centrální banky. Centrální banka výší této sazby ovlivňuje peněžní zásobu
komerčních bank, podle které určují banky výši poskytovaných úvěrů. Diskontní
sazba představuje dolní mez krátkodobých úrokových sazeb na peněžním trhu.
Zvýšení diskontní sazby pomáhá snižovat inflaci, její snížení naopak vede k

expanzivnímu navyšování zásoby peněz.

68
\newpage
« © Repo sazba - Při repo operacích centrální banka přijímá od bank přebytečnou
likviditu a na oplátku jim předává dohodnuté cenné papíry. Obě strany se zároveň
zavazují, že po uplynutí doby splatnosti centrální banka jako dlužník vrátí věřitelské
bance zapůjčenou jistinu zvýšenou o dohodnutý úrok a věřitelská banka vrátí
poskytnuté cenné papíry. Základní doba trvání těchto operací je 14 dní, úrok při této
operaci je nazýván repo sazbou (refinanční sazbou). ČNB podle americké aukční
procedury přijímá přednostně nabídky bank požadující nejnižší úrokovou sazbu. Při
vyšší repo sazbě dochází ke zdražení peněz, banky si půjčují méně, naopak jsou
ochotné poskytnout samy své prostředky centrální bance, dochází ke stahování peněz
z oběhu a zmírňování inflace.

« © Lombardní sazba - představuje úrokovou sazbu při operacích, kdy si banky vypůjčují
likviditu oproti zástavě cenných papírů. V současné době je vzhledem k trvalému
přebytku likvidity bank tato možnost využívána minimálně. Lombardní sazba
představuje horní mez krátkodobých úrokových sazeb na peněžním trhu - je vyšší než
diskontní sazba nebo repo sazba. Zvýšení lombardní sazby má za následek menší
půjčky obchodních bank, peníze v oběhu jsou omezeny, což vede ke snižování inflace.

« | Operace na otevřeném trhu - Centrální banka nakupuje a prodává na volném
peněžním trhu státní cenné papíry (státní pokladniční poukázky, popř. státní
dluhopisy). Obchodní banky si mohou půjčit u centrální banky peníze tím, že jí prodají
své cenné papíry a dohodnou se na budoucím zpětném odkupu (repo obchodě), nebo
může jít o obchod bez budoucích ujednání (tzv. promptní obchod). Pokud centrální
banka prodává státní cenné papíry, dochází k odčerpávání peněz z obchodních bank a
zpomalení oběhu peněz. Tato restriktivní monetární politika vede ke snížení inflace.
Naopak při expanzivní politice, tj. nákupu cenných papírů centrální bankou, dochází k
nárůstu peněžní zásoby v oběhu. Emise státních cenných papírů slouží 1 ke krytí
přechodného nedostatku peněz ve státní pokladně nebo ke krytí schodku státního
rozpočtu.

e | Povinné minimální rezervy - Centrální banka předepisuje obchodním bankám určité
procento z vkladů, které si u ní musí uložit ve formě neúročené povinné minimální
rezervy. Tyto peníze jsou mimo oběh a působí protiinflačně. Výše povinných
minimálních rezerv ovlivňuje úvěrovou kapacitu obchodních bank, důsledkem toho i
výši úrokových sazeb.

e« | Konverze a swapy - Centrální banka nakupuje a prodává cizí měny za koruny
obchodním bankám. Tyto operace mají vliv na měnové kurzy. Dochází ke:

« © konverzi - promptnímu obchodu v aktuálním kurzu bez následných zpětných operací

e © swapu - kombinaci promptního obchodu s následnou zpětnou operací -
prodává/nakupuje se za aktuální kurz a budoucí zpětný odkup/prodej se odehrává za
předem dohodnutého kurzu

Při prodeji deviz centrální bankou dochází ke stahování českých korun z oběhu a zpomaluje
se oběh peněz, při nákupu deviz se české koruny přilévají do oběhu.

69
\newpage
Banky obchodní - základní charakteristika

Obchodní banky jsou podnikatelskými subjekty, které podnikají za účelem dosažení zisku.
Poskytují tyto služby: depozitní (vkladové) a úvěrové operace, převody peněz a další služby
zprostředkovatelské povahy.

Zisk banky je tvořen čistými bankovními úroky (získané úroky mínus úroky vydané) a
poplatky za služby. Současnou tendencí je poskytování stále většího rozpětí bankovních
služeb.

Operace obchodních bank

Bilance banky - aktiva a pasiva

VKLADY

1. pasivní operace - banka přijímá peníze, je v dlužnické pozici

- netermínované vklady (úročeny velmi nízkým procentem)

- termínované vklady (termínované účty, vkladní knížky)

nebo z hlediska měny:

- korunové (v KČ)

- devizové (V cizí měně)

ÚVĚRY

Banka zde sleduje dva cíle: výnosnost a návratnost úvěru

2. aktivní operace - banka poskytuje úvěry, vystupuje v roli věřitele

- úvěry od ČNB - refinanční operace za repo sazbu, lombardní úvěr, Nouzový úvěr

- úvěry od ostatních bank - korunové úvěry např za sazbu Pribor

- emise bankovních obligací - emise bývá v mld. Objemech a je dlouhodobým zdrojem.

- emise hypotéčních zástavních listů

70
\newpage
Pojišťovnictví

Pojišťovnictví můžeme charakterizovat jako specifický ekonomický obr řešící minimalizaci
rizik ekonomických i neekonomických činností člověka.

Smyslem pojišťovnictví je zabezpečit pojištěného pro případ nahodilých nepříznivých
událostí. Samotné pojištění nemůže zabránit ztrátám, ale může zmírnit jejich následky

Pojišťovny

Dnes působí na českém trhu kolem 50 pojišťovacích ústavů, VIG RE zajišťovna, a.s. jako
první česká zajišťovna.

Právní formy - akciová společnost (nejčastější forma)

- družstevní organizace

Zajišťovny

Je právnická osoba, která přebírá na základě smlouvy jistou část rizik pojištění pojišťoven a
zajišťoven. Tento typ pojištění sjednávají pojišťovny pro určitá pojištění nebo celé portfolio
pojištění, většinou formou podílové spoluúčasti na pojistném 1 škodách.

Státní dozor nad pojišťovnictvím vykonává Ministerstvo financí

Nemůže dělat každý. Je třeba vysoký kapitál v minimální výši 1 000 000 000 Kč

Pojištění, riziková událost - koho se týká

pro jednotlivce: pří úrazu, při dožití určitého věku, při léčení v cizině, \ldots

pro kolektiv: při požáru, při odcizení majetku organizace, \ldots

pro hospodářství: pomáhá zajišťovat plynulý chod ekonomiky, omezuje počet bankrotů, \ldots
Členění pojištění z hlediska povinnosti uzavření

a) povinné pojištění - jsou zákonem uložena a to firmám i osobám, sledují zajištění
sociálních jistot lidí a zabezpeční proti škodám způsobenými jiným osobami na provozu
motorových vozidel

1. zákonné sociální pojištění osob (dle zákona o sociálním pojištění)

- správcem tohoto pojištění je správa sociálního zabezpečení

- z tohoto pojištění jsou vypláceny nemocenské dávky, důchody, podpory v nezaměstnanosti
2. Zákonné zdravotní pojištění osob (dle zákona o zdravotním pojištění)

- toto pojištění zpracovávají zdravotní pojišťovny

Ji
\newpage
3. Zákonné pojištění odpovědnosti za škodu z provozu motorového vozidla
- od roku 2000 toto pojištění poskytují vybrané největší pojišťovny
4. Zákonné pojištění pracovních úrazů a nemocí z povolání zaměstnanců

- tato pojištění jsou povinní uzavírat zaměstnavatelé pro své zaměstnance (Česká pojišťovna,
Kooperativa)

b) Dobrovolná pojištění
- uzavírá se na komerční bázi
- nejsou povinná

- klient, který má zájem pojistit se proti určitým rizikům si vybírá z pestré nabídky
komerčních pojišťoven.

Členění pojistných služeb

1. Životní pojištění - toto pojištění fyzických osob pomáhá chránit tyto osoby a jejich rodiny
proti rizikům těžkých úrazů, jejich trvalých následků, vážných nemocí a následné ztráty
příjmu, případně při úmrtí pojištěného pomáhá nahradit zdroj příjmu jeho pozůstalým

- rizikové (za nižší pojistné poskytuje vysokou pojistnou ochranu, nedojde-li k pojistné
události zamká bez náhrady)

- rezervotvorné (pojistné je vyšší, protože obsahuje spořící složku. Pojistná částka plus podíly
na zisku jsou vypláceny při pojistné události nebo na konci sjednané pojistné doby

2. Neživotní pojištění - zahrnuje především pojištění movitostí a nemovitostí.

Pojmy - pojištěný, pojistné plnění, pojistná událost, pojistitel, pojistné, pojistník,
pojistná částka, pojistka, pojistná smlouva, pojistný zájem, obmyšlená osoba

Pojistitel - pojišťovna, má právo pojistné a povinnost vyplatit pojistné plnění v případě
pojistné události

Pojistník - subjekt, který uzavřel s pojistitelem pojistnou smlouvu. Má povinnost platit
pojistné

Pojištěný - subjekt, na jehož majetek, život, zdraví nebo odpovědnost za škodu se pojištění
vztahuje. Má právo na pojistné plnění

Obmyšlená osoba - subjekt, kterému v případě pojištění ve prospěch jiné osoby vznikne v
případě pojistné události právo na pojistné plnění Pojistná událost - je nahodilá událost, při
které vzniká nárok na pojistné plnění. Její nahlášení (a případné doložení, že se jedná
skutečně o pojistnou událost) je povinností pojištěného

72
\newpage
Pojistné - představuje cenu za poskytnutí pojistné ochrany.

Pojistné plnění - je částka, kterou pojišťovna při pojištění vyplácí v případě pojistné události.
Pojistná částka - je nejvyšší finanční částka, jež může být vyplacena, dojde-li k pojistné
události. Hodnota této částky je sjednána při uzavírání pojistné smlouvy a je v této smlouvě
uvedena.

Pojistka je písemným potvrzení pojistitele o uzavření pojistné smlouvy. U některých
pojišťoven je vystavováno tzv. potvrzení o akceptaci pojištění, tedy o přijetí rizika do
pojištění.

Pojistná smlouva - je smlouvou o finančních službách, ve které se pojistitel zavazuje v
případě vzniku pojistné události poskytnout pojistníkovi nebo třetí osobě ve sjednaném
rozsahu pojistné plnění a pojistník se zavazuje platit pojistiteli pojistné.

Pojistný zájem - je oprávněná potřeba ochrany před následky pojistné události. Pojistník má

pojistný zájem na vlastním životě a zdraví. Pojistník má pojistný zájem na vlastním
majetku.

Náležitosti pojistné smlouvy

Základní náležitosti pojistné smlouvy

- smluvní strany (pojistník, pojistitel, pojištěný)

- předmět pojištění (na co smlouva je - na úraz, na majetek\ldots)

- začátek platnosti pojistné smlouvy, někdy 1 konec

- výše pojistného (kolik a jak často platí klient)

- výše pojistného plnění (kolik pojišťovna zaplatí a podmínky, za kterých zaplatí)

- všeobecné podmínky

73
\newpage