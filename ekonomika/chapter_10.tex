\chapter{Personalistika}

Výrobní faktory nezbytné k zajištění výroby -- práce, přírodní zdroje a kapitál.

Personalistika je soubor činností podniku, jejichž cílem je zajistit, rozmístit a udržet pro podnik kvalitní zaměstnance - práce s lidmi.

Zákoník práce -- nejdůležitější ustanovení: \textbf{Mezi zaměstnanci a zaměstnavateli vznikají pracovněprávní vztahy, které se řídí zákoníkem práce}

Závislá práce -- není samostatná činnost, není podnikatel.

\paragraph*{Parcovní poměr}
\begin{itemize}
    \item způsobilost FO vstupovat do pracovněprávních vztahů jako zaměstnanec vzniká dnem dosažení 15 let
    \item každý zaměstnanec má v průběhu pracovního poměru obecnou zodpovědnost vůči zaměstnavateli za jím způsobené škody
\end{itemize}

\paragraph*{Pracovněprávní vztah}
Může mít podobu:
\begin{enumerate}
    \item dohody o práci konané mimo pracovní poměr:
        \begin{enumerate}
            \item Dohoda o provedení práce
                \begin{itemize} 
                    \item Může být uzavřena na max. 300 hodin ročně u jednoho zaměstnavatele
                    \item Musí být uzavřena písemně a musí obsahovat dobu,na jakou je uzavřena
                    \item Vhodná především pro krátkodobé brigády
                    \item U této formy nevzniká povinnost platit SaZ pojištění, pokud mzda nepřekročí 10000 Kč měsíčně
                    \item Je povinnost uhradit daň z příjmu
                \end{itemize}
            \item Dohoda o pracovní činnosti
                \begin{itemize}
                    \item Sjednává se na práci, která nesmí svým rozsahem převýšit v průměru, polovinu stanovené týdenní pracovní doby
                    \item Musí být uzavřena písemně
                    \item Vzniká povinnost uhradit SaZ pojištění, daň z příjmu
                \end{itemize}
        \end{enumerate}
    \item pracovní poměr
        \begin{enumerate}
            \item Vznik pracovního poměru:
                \begin{itemize}
                    \item volbou -- zvolení poslanci, zvolení řídící pracovníci družstev
                    \item jmenováním -- vedoucí pracovníci
                    \item uzavřením pracovní smlouvy -- tato forma je nejběžnější
                \end{itemize}
            \item Změny pracovního poměru:
                \begin{itemize}
                    \item převedení na jinou práci -- zdravotní potíže, těhotenství
                    \item přeložení na jiné místo výkonu práce -- se souhlasem zaměstnance
                \end{itemize}
            \item Skončení pracovního poměru:
                \begin{itemize}
                    \item dohodou -- je nutný souhlas obou stran, není stanovena výpověd.lhůta
                    \item zrušením ve zkušební době -- nemusí uvádět důvody, je třeba písemně oznámit 3 dny předem
                    \item okamžité zrušení:
                        \begin{itemize}
                            \item Zaměstnavatel:
                                \begin{itemize}
                                    \item pokud zaměstnanec porušil pracovní kázeň
                                    \item odsouzen pro trestný čin k nepodmíněnému trestu odnětí svobody na dobu delší než 1 rok
                                \end{itemize}
                            \item Zaměstnanec:
                                \begin{itemize}
                                    \item podle lékařského posudku nemůže déle konat práci bez vážného ohrožení svého zdraví
                                    \item zaměstnavatel nevyplatil mzdu do 15 dnů po uplynutí splatnosti
                                \end{itemize}
                        \end{itemize}
                    \item výpověď -- je jednostranný právní akt, musí být dána písemně
                        \begin{itemize}
                            \item Výpověď ze strany zaměstnance -- výpovědní doba je 2 měsíce
                            \item Výpověď ze strany zaměstnavatele
                                \begin{itemize}
                                    \item stane-li se zaměstnanec nadbytečný má nárok na 2 měsíce výpovědní doby a odstupné ve výši jednoho měsíčního platu (0-1 rok), dvou platů (1-2 roky), tří platů (2-? roky)
                                    \item smrtí
                                \end{itemize}
                        \end{itemize}
                \end{itemize}
        \end{enumerate}
\end{enumerate}

Zaměstnavatel nesmí dát výpověď zaměstnanci pracovně neschopnému na nemocenské, povolanému k výkonu vojenské služby, uvolněnému pro výkon veřejné funkce, těhotné 	zaměstnankyni nebo trvale pečující alespoň o jedno dítě mladší než tři roky.

\section*{Pracovní smlouva}
\begin{itemize}
    \item Pracovní náležitosti pracovní smlouvy
        \begin{itemize}
            \item druh práce
            \item místo výkonu práce
            \item den nástupu do práce
            \item účastníci smlouvy
        \end{itemize}
    \item V pracovní smlouvě mohou být i další ujednání, nejsou však povinná
        \begin{itemize}
            \item zda se jedná o prac. poměr na dobu určitou či neurčitou, není-li udáno, jedná se o smlouvu na dobu neurčitou
            \item zkušební doba -- není-li zkušení doba ve smlouvě písemně sjednána, neplatí
                \begin{itemize}
                    \item maximální zkušební doba je 3 měsíce
                    \item nelze ji sjednat v případě pracovního poměru na dobu neurčitou
                \end{itemize}			
        \end{itemize}
\end{itemize}

Zaměstnavatel je povinen uzavřít pracovní smlouvu písemně a jedno vyhotovení smlouvy vydat zaměstnanci.

\paragraph*{Povinnosti zaměstnavatele}
\begin{itemize}
    \item přidělovat zaměstnanci práci podle sjednané smlouvy
    \item platit mu mzdu
    \item vytvářet podmínky pro plnění jeho pracovních úkolů
\end{itemize}

\paragraph*{Práva zaměstnavatele}
\begin{itemize}
    \item pracovní řád, odebrat vzorek při alkoholu a drogách	
\end{itemize}

\paragraph*{Povinnosti zaměstnance}
\begin{itemize}
    \item konat práci podle pokynů zaměstnavatele
    \item konat práci osobně a ve stanovené pracovní době
    \item dodržovat pracovní kázeň
\end{itemize}

\paragraph*{Práva zaměstnance}
\begin{itemize}
    \item mzda, přestávky, ochranné pomůcky	
\end{itemize}

\paragraph*{Pracovní doba a doba odpočinku}
Pracovní doba je nejvýše 40 hodin týdně, u mladistvých do 18 let maximálně 30 hodin týdně.

\section*{Mzda}

Zaměstnavatelé samostatně rozhodují o uplatnění formy základní mzdy:
\begin{itemize}
    \item mzda úkolová -- mzda závisí na množství kvalitní práce
    \item mzda časová -- zaměstannec je odměňován podle času práce
    \item mzda podílová -- určitý podíl na dosažených výsledcích (obchodník)
\end{itemize}

\paragraph*{Zaručená mzda}
Stanovuje nařízení vlády, a to tak že při týdenní pracovní době (40 hodin) stanoví 8 tarifních skupin podle namáhavosti, složitosti a odpovědnosti stanoví minimální mzdu. (12200-24400 Kč)

Mzda nesmí být nižší než minimální mzda stanovená vládním nařízením.

Hrubá minimální mzda platná pro rok 2018 je 12 200 Kč. Srovnáme-li s průměrnou nominální mzdou 29 504 Kč, zjistíme, že tato minimální mzda je velmi nízká.

Stanovení minimální mzdy má mnoho souvislostí a dopadů:
\begin{itemize}
    \item minimální mzda je nad hranicí životního minima -- 3410 Kč
    \item příspěvek na bydlení a existenční minimum (osoby bez trvalého bydliště)
\end{itemize}

\emph{Valorizace mezd} znamená zvyšování mezd při znehodnocení peněz (inflaci).

\section*{Bezpečnost a ochrana zdraví při práci}

Zaměstnavatelé jsou v rozsahu své působivosti povinni vytvářet podmínky pro bezpečnou a zdraví neohrožující práci v souladu s předpisy o bezpečnosti práce a bezpečnosti technických zařízení. Zaměstnavatelé mají povinnost odškodňovat pracovní úrazy a nemoci z povolání 	zaměstnanců.

Každý rok musí být zaměstnanci proškolení.

Zaměstnavatelé jsou ze zákona povinni platit zákonné úrazové pojištění zaměstnanců.

\paragraph*{Zákoník práce}
\begin{description}
    \item[Dovolená] nárok na 4 týdny (2 týdny vkuse), placené volno, někdo i 5 týdnů
    \item[Odbory] sdružení zaměstnanců, založené s cílem prosazovat jejich pracovní, hospodářské, politické a sociální zájmy
    \item[Stávka] je forma kolektivního protestu zaměstnanců
    \item[Výluka] je zamezení práce zaměstnavatelem (opak stávky)
    \item[Kolektivní smlouva] smlouva mezi odbory a zaměstnavatelem o mzdách a dalších ujednáních, týkajících se zaměstnanců
    \item[Odměňování zaměstnanců]
        \begin{itemize}
            \item Přímé -- plat,mzda
            \item Nepřímé -- zaměstnanecké výhody (podnikové půjčky, stravenky, příspěvek na dovolenou)
            \item Nefinanční -- pochvala, stáž, studijní dovolená				
        \end{itemize}
    \item[Náhrady mzdy]
        \begin{itemize}
            \item zákonem -- za dovolenou nebo svátek
            \item překážky v práci-svědectví u soudu, výpadek dodávky energie
        \end{itemize}
\end{description}

\section*{Složky mzdy}
\begin{itemize}			
    \item Osobní ohodnocení -- vyjadřuje kvalitu práce zaměstnance
    \item Příplatky -- v noci(10\%), víkendy (10\%), ztížené prac. prostředí (10\% z min. mzdy), přesčas (25\% hod. mzdy), ve svátek
    \item Prémie a odměny -- vyplácejí se za určité výsledky práce
\end{itemize}

\paragraph*{Význam úřadu práce}
Státní úřad, poskytuje informace z oblasti pracovního trhu ČR a EU, eviduje uchazeče o zaměstnání a volná pracovní místa. \\
Pobočky v různých místech ČR. Nabízí rekvalifikační kurzy.

\section*{Zdravotní a Sociální pojištění}
\paragraph*{Zdravotní pojištění}
Toto pojištění spravují zdravotní pojišťovny (největší VZP) a hradí z něj lékařům jejich práci a léky. Za děti, studenty, ženy na mateřské, vojáky, registrované nezaměstnané a důchodce platí toto pojištění stát ze státního rozpočtu.

\paragraph*{Sociální pojištění}
Tyto příjmy spravuje státní instituce správa sociálního zabezpečení
\begin{itemize}
    \item Nemocenské pojištění -- hrazeny nemocenské dávky
    \item Důchodové pojištění -- jsou z něj vypláceny důchody (starobní, ,vdovské,sirotčí)
    \item Příspěvek na státní politiku zaměstnanosti -- z něj se vyplácí podpory v nezaměstnanosti (také rekvalifikace)
\end{itemize}

Zaměstnanci odvádějí ze své hrubé mzdy 11\% pro účely SaZP a zaměstnavatelé povinně odvádí navíc 34\% z této hrubé mzdy zaměstnance (superhrubá mzda).

Podnikatelé vypočítávají své SaZP tak, že ze svého zisku (výnosy-náklady) vypočítají 50\% základ, ze kterého odvedou 45\% na účely SaZP.

