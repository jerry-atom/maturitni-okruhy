\chapter{Finanční trhy}

STRUKTURA FINANČNÍHO TRHU A NA ČEM JE ZALOŽEN

Finanční trh je založený na nabídce relativně volných peněz (které jejich majitelé nechtějí nechat ležet ladem a ztrácet
na hodnotě díky inflaci, ale chtějí je zhodnotit) a poptávce po penězích (kdy podnikatelé vědí jak vydělat, ale nemají
dostatek kapitálu). Střetem této nabídky a poptávky se vytváří cena peněz (výše úroků, tržní cena cenných papírů apod.)
Členění finančního trhu: peněžní trh, kapitálový trh, trh drahých kovů, devizový trh

ÚROK A ÚROKOVÁ MÍRA

Úrok je peněžitá odměna za půjčení peněz. Velikost úroku se obvykle vyjadřuje pomocí úrokové míry (sazby), která je
procentním vyjádřením zvýšení půjčené částky za určité časové období.

Úrokové sazby zjednodušeně říkají, jak velkou část zaplatí dlužník navíc, zjednodušeně jde o měřítko ceny peněz. U
spoření říkají, kolik peněz dostanete navíc, u půjček zase, kolik musíte zaplatit navíc k půjčené částce. Vyhlašují se na
určité období (den, měsíc, rok). Úrok je peněžitá odměna za půjčení peněz.

MOTIVY POPTÁVKY PO PENĚZÍCH

Poptávka po penězích je představována množstvím peněz v držbě jednotlivých ekonomických subjektů (domácností,
podniků, vlády) Hlavním motivem, proč člověk drží peníze v hotovosti a ztrácí tak úrok, je možnost disponovat těmito
penězi.

Druhy poptávky po penězích

1) Transakční poptávka - vyplývá z funkce peněz jako prostředku směny - jednotlivé subjekty potřebují peníze na
běžné nákupy výrobků a služeb, podniky na úhradu svých nákladů.

2) Majetková poptávka - vyplývá z funkce uchovatele hodnoty. Znamená, že ekonomické subjekty drží určitou část
svého majetku v podobě peněz jako vysoce likvidní formy majetku. Formy struktury jsou různé a působí na ní celá řada
faktorů (výnos, splatnost, riziko, atd.). Úspory => investice.

Úrok je nákladem na držbu peněz - budou-li úrokové sazby vysoké, je držba peněz nákladná, jelikož znamená ztrátu
úroku, které by přinesly.

FINANČNÍ INSTITUCE
Instituce finančního trhu jsou subjekty, které podnikají s penězi.

Instituce finančního trhu se liší charakterem poskytovaných služeb a také, zda mohou být zakládány bez povolení státu
(např. nebankovní instituce poskytující úvěry) či zda pro svojí činnost potřebují licenci (např. banky, pojišťovny,
makléřské společnosti, burzy apod.).

Česká národní banka - Komerční banky - Družstevní záložny (spořitelní a úvěrová družstva) - Stavební spořitelny -
Investiční společnosti, investiční fondy a podílové fondy - Pojišťovny - aj.

Na mezinárodní úrovni působí finanční instituce, které plní regulační, dohledovou a stabilizační funkci. Jsou to např.
Evropská centrální banka (pravidla finančního trhu), Mezinárodní měnový fond (stabilizace) nebo Evropská bank pro
obnovu a rozvoj (pomoc zemím s přechodem na tržní ekonomiku).

CENNÝ PAPÍR

Cenné papír je písemnou formou zachycený právní vztah mezi dvěma (i více) subjekty, tato písemnost má určité
náležitosti stanovené zákonem, a pak může být samostatně obchodovatelná.

Cenné papíry peněžního trhu zachycují vztah dlužnický (věřitel x dlužník)
\newpage
PENĚŽNÍ TRH - CHARAKTERISTIKA

Finanční trh vzniká na základě střetu poptávky a nabídky, kde je poptávkou investování a nabídkou spoření. Jinými slovy
poptávajícími jsou ti, kteří poptávají peníze a nabízejícími jsou ti, kteří peníze půjčují, tedy nabízí. Na peněžním trhu se
obchoduje s krátkodobými vklady a úvěry ve formě krátkodobých CP se splatností do 1 roku

ŠEK

je CP, kterým výstavce šeku dává příkaz bance, aby osobě uvedené na šeku nebo doručiteli zaplatila z jeho účtu částku,
na kterou je šek vystavený. Šek je splatný na viděnou (po předložení) do 8 dnů v téže zemi.

Podstatné náležitosti šeku:

- označení CP slovem šek

- částka (číslicí i slovy)

- jméno toho, kdo má platit (banka)

- místo, kde se má platit

- datum a místo vystavení šeku

- podpis výstavce

Použití šeku:

- proplacení v hotovosti - zaplacení šekem - zúčtování šeku - proplacení či jeho vklad na účet

SMĚNKY
Směnka je obchodovatelný cenný papír, který slouží k platbě nebo jako zajišťovací nástroj. Jedná se o nejjednodušší a
často nejrychlejší formu úvěru. Jasně z ní vyplývají závazkové vztahy a tím je ideálním prostředkem pro obchodování.

Druhy směnek

Základním členění směnek je na vlastní a cizí. Jednoduše řečeno, pokud osoba, která směnku vystavila, se zaváže
zaplatit v daném termínu a místě, jde o směnku vlastní. Směnka vlastní obsahuje slovo „zaplatím“. Za směnku cizí musí



























. bh .. , .. . .-
zaplatit někdo jiný, než ten kdo jí vystavil a vyskytuje se termín
1 “
Véřitel 2. osoba „zaplaťte .
„prožá zboží 3 osobša
W , „prodává s 2. vsoké VÝSAD VĚ VĚMŘU 5 PP 20
Věňitel Dhé nk Z Ao tací obě ase -------- P | zoplozit důsníkoví vé
prospěch věširelh
- prodávě zboži * upsdavý zoněníku ve i
- majitel směnku (| pospěch věřitele
Dhůni:

- BÍV OBE ŘLONÍCH SDĚKÝ
věřiteli







Náležitosti směnky:

* označení, že se jedná o směnku (doslovné označení v textu listiny)
* příkaz zaplatit danou finanční sumu (bez dalších podmínek)

* jméno a adresa toho, kdo má směnku zaplatit

* informace o splatnosti

* informace o místě zaplacení

* jméno věřitele, tedy toho, jemuž má být zaplaceno

+ datum a místo vystavení směnky

* podpis výstavce směnky
\newpage
POKLADNIČNÍ POUKÁZKY -
emituje stát prostřednictvím ČNB, je to CP, který slouží ke krytí přechodného schodku (deficitu) státního rozpočtu

- určen pro obchodování na mezibankovním trhu pro velké investory

- nominální hodnota jedné poukázky bývá 10mil. Kč, objem jedné emise se pohybuje v řádech miliard korun

- o tyto CP je velký zájem, protože stát v pozici dlužníka je velmi solidní zárukou navrácení peněz včetně úrokového výnosu

DEPOZITNÍ CERTIFIKÁT
- vydává banka (dlužník) a potvrzuje jím přijetí jednorázového termínovaného vkladu od klienta (věřitele). Certifikáty nakupují
podniky, občané, banky a jiné subjekty. Jsou vystavovány na částku 10 tisíc Kč. Čím vyšší je jejich nominální hodnota tím vyšší je úrok.

Kapitálový trh

- obchoduje se s dlouhodobými termínovanými vklady, úvěry a zdroji financí ve formě dlouhodobých CP se splatností nad 1 rok

Podoby CP : 1. Listinné - je vydán v podobě listiny
2. Zaknihované - neexistují fyzicky, jsou evidovány v elektronické podobě na účtu majitelů

ČLENĚNÍ CP
1 Podle právního nároku

dlužnické- představují nárok na splacení dluhu ( např. dluhopisy, směnky),
7 majetkové- představují podíl na majetku (např. akcie, podílové listy),

dispoziční - představují právo nakládat s určitým majetkem (např. konosament, skladištní list).
2 Podle doby splatnosti

peněžní - krátkodobé, splatné do 1 roku,

kapitálové - dlouhodobé, splatné za dobu delší než 1 rok.
3 Podle podoby

zaknihované - mají jen podobu zápisu v určitém registru (nyní Centrální depozitář cenných papírů, a. s.),
listinné - mají podobu speciální listiny.
4 Podle formy

Cenné papíry na doručitele - jméno majitele není na cenném papíru uvedeno, převádějí se pouhým předáním novému majiteli.

Cenné papíry na řad - jméno majitele cenného papíru je na něm uvedeno, někdy s doložkou „na řad“ (zákonné ji obsahovat nemusí),
převádějí se rubopisem a předáním. Uvedením doložky "nikoli na řad" se cenný papír na řad změní v cenný papír na jméno.

Cenné papíry na jméno - jméno majitele cenného papíru je na něm uvedeno, někdy s doložkou „nikoli na řad“, převádějí se smlouvou o
postoupení pohledávky a předáním.

5 Podle obchodovatelnosti



Obchodovatelní - nakupují se a prodávají na sekundárním trhu. Jsou převoditelné.

Neobchodovatelné - emitent zakazuje jejich nákup a prodej na sekundárním trhu. Prodávají se jen na trhu primárním, např. Vkladní
knížky.

6 Podle emise:
CP hromadně vydávané - např. Akcie, dluhopisy

CP individuálně vydávané - např. Směnka, šek
\newpage
Akcie



Akcie je majetkovým cenným papírem a její vlastník se stává spolumajitelem firmy. Souček akcií tvoří základní kapitál.
Práva akcionáře, jako společníka:
podílet se na zisku (výplata dividend)
podílet se na řízení společnosti (hlasovat na valné hromadě akcionářů)
podílet se na likvidačním zůstatku společnosti, pokud jde firma do likvidace
Akcie může být emitována:
1. Přímo za nominální hodnotu,
2. S emisním ážiem (za vyšší cenu než nominální),
3. S emisním disážiem (za nižší cenu než nominální) - současný zákon o obchodních korporacích neumožňuje.

Rozlišujeme:
1.Kmenové akcie

© Na jméno - převoditelné rubopisem neboli indosamentem,

© Na majitele - volně převoditelné, na doručitele, od 2017 pouze jako zaknihovaný CP nebo imobilizovaný CP.
2. Speciální

© Zaměstnanecké (pro zaměstnance firmy, akcie vystaveny na jméno, max. 5% objemu emise akcií),

© Prioritní akcie (přednostní výplata dividendy, ovšem majitel nemá právo hlasovat na valné hromadě).

Členění dle fyzické podoby:

© Akcie materializované (listinné - fyzicky vytištěné na papíře - mají plášť, kuponový arch pro výplaty dividend a talón pro
získání nového kuponového archu, když je předchozí vyčerpán)

© Akcie dematerializované (zaknihované - většinou jako údaje v paměti počítače).

© Dividenda je příjmem z kapitálového majetku a jste povinni dle zákona o dani z příjmu fyzických osob ji zdanit srážkovou
daní 15%.

Skládá se ze dvou částí :

a) plášť

- obchodní jméno a sídlo společnosti, číselné označení akcie, jmenovitá hodnota akcie, označení, zda je akcie na jméno nebo na
majitele, výše ZK, počet akcií v době vydání akcie, datum vydání akcie, podpisy dvou členů představenstva

b) kupónový arch s talónem

- jednotlivé kupóny se odstřihávají (detašují) a vyplácí se podíl na zisku, který se nazývá dividenda (vyjadřuje se V %). Dividenda by
měla být vyšší než úrok v bankách, v opačném případě by občané ukládali svůj kapitál do bank - poslední ústřižek
kupónového archu je talón - poukázka na nový kupónový arch.

Podílové listy

jsou vydávány podílovými fondy, tyto fondy zakládají investiční společnosti nebo banky, které tak shromažďují peníze od investorů
(občanů, firem) formou prodeje svých podílových listů. Takto získané prostředky se investují do nákupu různých druhů CP, tzv.
portfolio a tím rozloží možné riziko. V současnosti existují v ČR pouze tzv. otevřené podílové fondy. To znamená že podílník má právo
kdykoliv svůj podílový list prodat zpět podílovému fondu.

Jaké potencionální zhodnocení fond nabízí, závisí především na aktivech, do kterých investuje.

Rozeznáváme fondy: a) akciové d) smíšené
b) dluhopisové e) zajištění - minimální zhodnocení investic
c) peněžního trhu
\newpage
Obligace

- dlouhodobý úvěrový CP, v němž se vydavatel zavazuje jeho majiteli splatit dlužnou nominální částku a vyplácet
výnosy k určitému datu. Splatnost je zpravidla pevně stanovena. Vydavatel si emisí obligací opatřuje finanční zdroje
převážně na uskutečnění svých investičních záměrů. Obligace podléhají při emisi schválení ministerstvem financí-
emise obligací se pohybují v řádek stovek milionů a jejich emitenti bývají největšími ekonomickými subjekty v
národním hospodářství.

Členění obligací z hlediska emitentů:

1) státní - emituje je vláda

2) komunální - emitují územně samosprávné celky, např. obce
3) bankovní

4) podnikové - vyšší úročení, vyšší riziko

Hypoteční zástavní listy

- emituje banka, která má od ČNB povolení k této činnosti. Jejich prodejem banka získá prostředky, které dále používá.
Tyto listy jsou kryty (ručeny) splátkami hypotečních úvěrů a zastavenými nemovitostmi. Vzhledem k dokonalému jištění
je nízké úročení. Bývají veřejné obchodovatelné, doba splatnosti 5 let.

Hypotéční zástavní listy mají podobu:
© listinnou (cenný papír materializovaný)
© zaknihovanou (cenný papír dematerializovaný)

© Mohou býti neveřejně obchodovatelné a veřejně obchodovatelné

Deriváty

- využívají se nejen u CP kapitálového trhu, ale i u deviz. Předmětem koupě nebo prodeje je pouze určité právo, nikoliv
věc hmatatelná. Jejich podstatou je forma termínovaného obchodu, tzn. že dochází k určitému zpoždění mezi
sjednáním obchodu a jeho plněním. V době sjednání jsou jasně definovány podmínky obchodu, tedy cena a datum
plnění. V den splatnosti dochází k dodání finančních instrumentů a jejich proplácení.

Mají funkci: 1) zajišťovací - proti rizikům výkyvu kurzu
2) spekulativní - spekuluji, abych vydělal. V praxi běžnější

Druhy derivátů:
Forwardy, Futures - pevné termínované obchody k určitému budoucímu datu za předem dohodnutou cenu

Swapy - prodej CP či deviz v aktuálním kurzu se současným podpisem smlouvy o budoucím odkoupení zpět za předem
dohodnutou cenu

Opce - je právo k budoucímu datu uskutečnit obchod s CP nebo devizami za předem sjednanou cenu
\newpage
Investování volných peněžních prostředků
Kam uložit peníze? 13 základních investičních možností

1. Bankovní účty

2. Podílové fondy

3. ETF (ETF jsou fondy obchodované na burzách. Jejich výhodou oproti klasickým podílovým fondům je, že můžete fond koupit a
prodat kdykoli a vždy vidíte jeho aktuální cenu (u klasických podílových fondů je hodnota podílového listu přeceňována nejčastěji
jednou denně, občas i jednou týdně). Zároveň jsou zpravidla levnější na pořízení i na správu (mají nižší poplatky).

4. Investiční certifikáty

5. Penzijní připojištění

6. Stavební spoření

7. Životní pojištění

8. Akcie

9. Dluhopisy

10. Nemovitosti

11. Zlato

12. Ropa a další komodity

13. FOREX (Na Forexu se obchoduje s měnovými páry - tedy dvojicemi měn. Posiluje-li jedna měna, cena měnového páru roste.
Posiluje-li druhá z páru, cena klesá. I na tomto trhu se obchoduje s pákovým efektem a investice jsou tak značně rizikové. \uv{Vyčistit}
účet je mnohem snadnější než ho zhodnotit do závratných výšek)

Trh drahých kovů
je pro nás jen doplňkovým trhem, protože prostřednictvím drahých kovů probíhá jen min. finančních transakcí.

Za nejdůležitější trhy drahých kovů jsou všeobecně považovány trhy zlata a stříbra, přičemž sem bývají zahrnovány i
trhy platiny a palladia.

Nejdůležitější institucí je burza drahých kovů v Londýně.

Devizový trh

- Na devizovém trhu je stejně jako na ostatních trzích cena určována poptávkou a nabídkou po dané komoditě.
Obchoduje se zde s měnami různých zemí. Střetávají se zde zájmy kupujících a prodávajících různé měnové jednotky.
Na základě poptávky a nabídky je určen poměr v jakém se dané množství měny smění. Devizové trhy tvoří banky,
finanční firmy atd.

Poptávka a nabídka po měně dané země bývá zpravidla určena vývojem mezinárodního obchodu, vývojem úrokových
měr v zemi, inflačním očekáváním, platební bilancí země, celkovým stavem ekonomiky. Naše měna je volně směnitelná
od roku 1995.
\newpage