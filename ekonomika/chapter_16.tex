\chapter{Národní hospodářství}

\textbf{Makroekonomie} je obor ekonomické teorie, který se zabývá zkoumáním ekonomického systému jako celku, sleduje vztahy mezi agregátními veličinami např. HDP, agregátní nabídka a poptávka, inflace, nezaměstnanost, úroková míra, měnový kurz.

\textbf{Národní hospodářství} je systém ekonomických subjektů na území daného státu a vztahy mezi nimi. Souhrn hospodářských činností na území daného státu, kterých se účastní tři ekonomické subjekty: STÁT, PODNIKY, DOMÁCNOSTI

\textbf{Subjekty národního hospodářství}
\begin{enumerate}
    \item Ziskový a neziskový sektor
    \item Domácnosti, firma, stát
    \item FO, PO
    \item Právní formy subjektů
\end{enumerate}

\textbf{Sektory národního hospodářství}
\begin{enumerate}
    \item Primární sektor -- prvovýroba (získávání surovin z přírody, např. zemědělství, rybolov, těžební průmysl)
    \item Sekundární sektor -- zpracování toho co vyprodukuje prim. sektor, např. strojírenský, textilní, stavební, potravinářský, výroba a rozvod plynu, elektřiny a vody
    \item Terciální sektor -- služby (obchod, doprava, pošta, školství, zdravotnictví, bydlení, obrana)
    \item Kvartární sektor -- věda a výzkum
\end{enumerate}

\textbf{Magický čtyřúhelník} -- co všechno je třeba sledovat při hodnocení národního hospodářství jako celku

[Obr. Magický čtyřůhelník]

\paragraph{Hrubý domácí produkt (HDP)}
Je souhrn statků a služeb vyjádřený v penězích vytvořený za určité období výrobními faktory
(práce, přírodní zdroje, kapitál) na území státu, bez ohledu na to, zda jsou vlastněny občany
státu nebo cizinci.

\paragraph{Hrubý národní produkt (HNP)}
Je souhrn statků a služeb vyjádřený v penězích vytvořený za určité období výrobními faktory
ve vlastnictví občanů příslušné země, bez ohledu na to, zda výroba probíhala na území státu
nebo v zahraničí.

Český statistický úřad počítá HDP třemi rovnocennými metodami:
\begin{enumerate}
    \item Produkční metoda
    \item Výdajová metoda
    \item Důchodová metoda
\end{enumerate}

\textbf{Produkční metoda} sčítá přidané hodnoty při výrobě (produkci statků a služeb. Je rozdílem mezi produkcí a mezi spotřebou

\textbf{Výdajová metoda} sčítá výdaje všech subjektů státu za finální statky a služby, tj. spotřeba domácností, výdaje vlády, tvorba kapitálu firem a Čisté vývozy \\
\textit{HDP = spotřeba domácnosti + investice soukromých firem + vládní nákupy + čistý vývoz}

\textbf{Důchodová metoda} vychází z fáze rozdělování, kdy každý z účastníků výroby získává svůj podíl na vyrobených statcích a službách jako odměnu za vynaložení svých výrobních faktorů. Zaměstnanci dostávají mzdy a platy, majitelé půdy rentu, majitelé kapitálu čistý úrok a zisky. U této metody do výpočtů nemohou být zahrnovány platby firem jiným firmám. Pro úplnost nesmíme zapomenout přičíst odpisy a nepřímé daně (především DPH) \\
\textit{HDP = mzdy + renty + zisky + úroky + opotřebení investic + nepřímé daně} \\
\textit{HDP = národní důchod + opotřebení investic + nepřímé daně}

\textbf{Běžné ceny} -- HDP v cenách sledovaného roku neočištěné od vlivu znehodnocení peněz (od inflace)

\textbf{Stálé ceny} -- HDP přepočítané ve stálých cenách zvoleného roku, přepočítáním na stálé ceny očistíme výsledky od inflačního znehodnocení peněz

\textbf{Skutečný HDP} -- pomocí produkční nebo důchodové metody vypočítaný skutečně dosažený HDP dané ekonomiky

\textbf{Potenciální HDP} -- HDP, který by ekonomika dosáhla při využití všech svých výrobních zdrojů

\textbf{Čisté ekonomické bohatství} -- makroekonomický ukazatel čisté ekonomické bohatství (NEW -- Net Economic Welfare) vychází ze známého ukazatele hrubého národního produktu, který je zvýšen o nelegálně produkované výrobky a služby (šedá a černá ekonomika), je zvýšen o výrobky produkované ve volném čase pro svou potřebu a na druhé straně je snížen o negativní dopady hospodářské činnosti na kvalitu našeho života (především dopady na životní prostředí tzv. externality) \\
\textit{NEW = HNP + šedá a černá ekonomika, produkce pro vlastní potřebu apod. - negativní dopady hospodářské činnosti}

\textbf{Negativní externality} -- špatný vliv výrobců na Životní prostředí (třetí poškozenou osobou stranou, které výrobci nic nezaplatí za zničené prostředí, jsou lidé žijící v blízkosti továren, průmyslových zón a dopravních komunikací.

\textbf{Pozitivní externality} -- podnikatelskou činností můžeme někomu přinést i užitek, aniž by se o to třetí osoba jakkoliv zasloužila nebo zaplatila -- např. přivedením infrastruktury (dálnice, elektrifikace, plynofikace)

\textbf{Šedá ekonomika} -- je souhrnem ekonomických vztahů, které porušují běžné etické a morální normy společnosti, ale většinou jsou na hranici zákona a jsou těžko právně postižitelné. Nejrozšířenějším ekonomickým vztahem šedé ekonomiky je podplácení/

\textbf{Černá ekonomika} -- je souhrn ekonomických vztahů, které porušují zákony dané země, popř. zákony mezinárodní. Jedná se o protiprávní ekonomické vztahy, které, pokud jsou odhaleny a prokázány, jsou trestně postižitelné. Zahrneme sem hospodářskou kriminalitu jednotlivců (krádeže, zpronevěry, padělání, daňové úniky apod.), ale i hospodářskou trestnou činnost organizovaných zločinců -- mafií.

\section*{Ekonomika a neziskový sektor}

Neziskový sektor může ve společnosti fungovat pouze za předpokladu, že tato společnost je schopna část svých prostředků (výrobních faktorů i hotových produktů -- statků a služeb) na tuto činnost vyčlenit, aniž by to vedlo k sebedestrukci.

Zdroje pro neziskový sektor musí společnost hledat ve svém fungujícím ekonomickém systému (v tržním hospodářství je to podnikatelský = ziskový sektor).

V neziskovém sektoru jsou typické statky, u kterých nemůžeme jednotlivce vyloučit ze spotřeby (policie chrání občany, vzdělání získají všechny děti, lékař zachrání život každému, pouliční lampy svítí všem). Hovoříme o veřejných statcích, které získají příslušníci určité společnosti bez přímé protihodnoty (neplatí za ně, nebo platí cenu netržní)

Většinu statků a služeb ale musíme zaplatit, chceme-li je spotřebovávat (když nebudeme mít peníze na dovolenou v zahraničí, tak na ni nepojedeme a život půjde dál) -- zde plně působí tržní prostředí a hovoříme o ziskovém sektoru.

Principy rozdělování:
\begin{itemize}
    \item v ziskovém sektoru rozdělujeme výsledný produkt podle množství, kvality a tržní úspěšnosti práce
    \item v neziskovém sektoru rozdělujeme podle potřeb
\end{itemize}

\paragraph{Státní neziskové organizace}
Patří sem především státní školství, zdravotnictví, instituce na ochranu životního prostředí,kulturních památek, celá oblast státní správy atd.

\paragraph{Nestátní neziskové organizace}
Církevní organizace, spolky (dříve občanská sdružení), ústavy, fundace (nadace, nadační fondy), politické strany

\paragraph{EU}
Evropa začala ve druhé polovině 20. Století zaostávat v dynamice ekonomického vývoje za ostatními světovými hospodářskými centry (USA, Japonsko, jihovýchodní Asie). To si uvědomovali čelní představitelé nejvýznamnějších evropských států a již od konce druhé světové války začali vyvíjet politické aktivity k hospodářskému sjednocení Evropy

\textbf{Integrace má dvě podoby}
\begin{enumerate}   
    \item federalistická -- státy se sjednotí politicky a ekonomická integrace následuje po politické integraci, tento proces byl typický pro USA
    \item funkcionalistická -- státy se propojují nejdříve ekonomickými vazbami a úplná politická integrace je završením tohoto ekonomického integračního procesu. (Evropa)
\end{enumerate}

\textbf{Integrace} (spojování do větších celků) má velký ekonomický význam -- umožňuje zhromadnění výroby a efektivnější využívání zdrojů, působí jak na mikroekonomické úrovní (spojování firem) tak na makroekonomické úrovni.

Stupně mezinárodní ekonomické integrace:
\begin{enumerate}   
    \item Pásmo volného obchodu
    \item Celní unie
    \item Společný trh
    \item Hospodářská unie
    \item Úplná ekonomická unie
\end{enumerate}

EU je v současné době ve fázi přechodu od hospodářské unie k úplné ekonomické unii. EU má 28 členských států včetně CR. \textbf{Evropská měnová unie (eurozóna) zavedla jednotnou měnu -- Euro}.

\paragraph{Evropská unie má 3 pilíře}
\begin{enumerate}
    \item hospodářská a měnová unie -- zavedení společné měny eura
    \item spolupráce v oblasti justice a vnitřních věcí
    \item společná zahraniční a bezpečnostní politika    
\end{enumerate}