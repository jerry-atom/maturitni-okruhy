\chapter{Národní hospodářství}

Makroekonomie - obor ekonomické teorie, který se zabývá zkoumáním ekonomického
systému jako celku, sleduje vztahy mezi agregátními veličinami např. HDP, agregátní nabídka
a poptávka, inflace, nezaměstnanost, úroková míra, měnový kurz.

Národní hospodářství je systém ekonomických subjektů na území daného státu a vztahy
mezi nimi. Souhrn hospodářských činností na území daného státu, kterých se účastní tři
ekonomické subjekty: STÁT, PODNIKY, DOMÁCNOSTI

Subjekty národního hospodářství
1. Ziskový a neziskový sektor

2. Domácnosti, firma, stát

3. FO, PO

4. Právní formy subjektů

Sektory národního hospodářství

1. Primární sektor - prvovýroba (získávání surovin z přírody, např. zemědělství, rybolov,
těžební průmysl)

2. Sekundární sektor - zpracování toho co vyprodukuje prim. sektor, např. strojírenský,
textilní, stavební, potravinářský, výroba a rozvod plynu, elektřiny a vody

3. Terciální sektor - služby (obchod, doprava, pošta, školství, zdravotnictví, bydlení, obrana)
4. Kvartární sektor - věda a výzkum

Magický čtyřúhelník - co všechno je třeba sledovat při hodnocení národního hospodářství
jako celku



Magický čtyřůhelník







Velikost a růst HOP | | inflace, stabilita cen |













Zahraniční obchod, jeho bilance Nezaměstnanost, její velikost
a struktura









Hrubý domácí produkt (HDP)

= souhrn statků a služeb vyjádřený v penězích vytvořený za určité období výrobními faktory
(práce, přírodní zdroje, kapitál) na území státu, bez ohledu na to, zda jsou vlastněny občany
státu nebo cizinci.

59
\newpage
Hrubý národní produkt (HNP)

= souhrn statků a služeb vyjádřený v penězích vytvořený za určité období výrobními faktory
ve vlastnictví občanů příslušné země, bez ohledu na to, zda výroba probíhala na území státu
nebo v zahraničí.

/ / W

Český statistický úřad počítá HDP třemi rovnocennými metodami
1. Produkční metoda

2. Výdajová metoda

3. Důchodová metoda

Produkční metoda sčítá přidané hodnoty při výrobě (produkci statků a služeb. Je rozdílem
mezi produkcí a mezi spotřebou

Výdajová metoda sčítá výdaje všech subjektů státu za finální statky a služby, tj. spotřeba
domácností, výdaje vlády, tvorba kapitálu firem a Čisté vývozy

HDP= spotřeba domácnosti + investice soukromých firem + vládní nákupy + čistý vývoz
Důchodová metoda - vychází z fáze rozdělování, kdy každý z účastníků výroby získává svůj
podíl na vyrobených statcích a službách jako odměnu za vynaložení svých výrobních faktorů.
Zaměstnanci dostávají mzdy a platy, majitelé půdy rentu, majitelé kapitálu čistý úrok a zisky.
U této metody do výpočtů nemohou být zahrnovány platby firem jiným firmám. Pro úplnost
nesmíme zapomenout přičíst odpisy a nepřímé daně (především DPH)

HDP= mzdy + renty + zisky + úroky + opotřebení investic + nepřímé daně

HDP = národní důchod + opotřebení investic + nepřímé daně

Běžné ceny - HDP v cenách sledovaného roku neočištěné od vlivu znehodnocení peněz (od
inflace)

Stálé ceny - HDP přepočítané ve stálých cenách zvoleného roku, přepočítáním na stálé ceny
očistíme výsledky od inflačního znehodnocení peněz

Skutečný HDP - pomocí produkční nebo důchodové metody vypočítaný skutečně dosažený
HDP dané ekonomiky

Potenciální HDP - HDP, který by ekonomika dosáhla při využití všech svých výrobních
zdrojů

Čisté ekonomické bohatství - makroekonomický ukazatel čisté ekonomické bohatství (NEW
- Net Economic Welfare) vychází ze známého ukazatele hrubého národního produktu, který
je zvýšen o nelegálně produkované výrobky a služby (šedá a černá ekonomika), je zvýšen o
výrobky produkované ve volném čase pro svou potřebu a na druhé straně je snížen o negativní
dopady hospodářské činnosti na kvalitu našeho života (především dopady na životní prostředí
tzv. externality)

NEW= HNP + šedá a černá ekonomika, produkce pro vlastní potřebu apod. - negativní
dopady hospodářské činnosti

60
\newpage
Negativní externality - špatný vliv výrobců na Životní prostředí ( třetí poškozenou osobou
stranou, které výrobci nic nezaplatí za zničené prostředí, jsou lidé žijící v blízkosti továren,
průmyslových zón a dopravních komunikací.

Pozitivní externality - podnikatelskou činností můžeme někomu přinést 1 užitek, aniž by se o
to třetí osoba jakkoliv zasloužila nebo zaplatila - např. přivedením infrastruktury (dálnice,
elektrifikace, plynofikace)

Šedá ekonomika - je souhrnem ekonomických vztahů, které porušují běžné etické a morální
normy společnosti, ale většinou jsou na hranici zákona a jsou těžko právně postižitelné.
Nejrozšířenějším ekonomickým vztahem šedé ekonomiky je podplácení

Černá ekonomika - je souhrn ekonomických vztahů, které porušují zákony dané země, popř.
zákony mezinárodní. Jedná se o protiprávní ekonomické vztahy, které, pokud jsou odhaleny a
prokázány, jsou trestně postižitelné.

Zahrneme sem hospodářskou kriminalitu jednotlivců (krádeže, zpronevěry, padělání, daňové
úniky apod.), ale 1 hospodářskou trestnou činnost organizovaných zločinců - mafií.

Ekonomika a neziskový sektor

- neziskový sektor může ve společnosti fungovat pouze za předpokladu, že tato společnost je
schopna část svých prostředků (výrobních faktorů 1 hotových produktů - statků a služeb) na
tuto činnost vyčlenit, aniž by to vedlo k sebedestrukci

- zdroje pro neziskový sektor musí společnost hledat ve svém fungujícím ekonomickém
systému (v tržním hospodářství je to podnikatelský = ziskový sektor)

- v neziskovém sektoru jsou typické statky, u kterých nemůžeme jednotlivce vyloučit ze
spotřeby (policie chrání občany, vzdělání získají všechny děti, lékař zachrání život každému,
pouliční lampy svítí všem). Hovoříme o veřejných statcích, které získají příslušníci určité
společnosti bez přímé protihodnoty (neplatí za ně, nebo platí cenu netržní)

- většinu statků a služeb ale musíme zaplatit, chceme-li je spotřebovávat (když nebudeme mít
peníze na dovolenou v zahraničí, tak na ni nepojedeme a život půjde dál) - zde plně působí
tržní prostředí a hovoříme o ziskovém sektoru

Principy rozdělování:

- v ziskovém sektoru rozdělujeme výsledný produkt podle množství, kvality a tržní úspěšnosti
práce

- v neziskovém sektoru rozdělujeme podle potřeb

Státní neziskové organizace

- patří sem především státní školství, zdravotnictví, instituce na ochranu životního prostředí,
kulturních památek, celá oblast státní správy atd.

Nestátní neziskové organizace

- církevní organizace, spolky (dříve občanská sdružení), ústavy, fundace (nadace, nadační
fondy), politické strany

61
\newpage
EU - Evropa začala ve druhé polovině 20. Století zaostávat v dynamice ekonomického
vývoje za ostatními světovými hospodářskými centry (USA, Japonsko, jihovýchodní Asie).
To si uvědomovali čelní představitelé nejvýznamnějších evropských států a již od konce
druhé světové války začali vyvíjet politické aktivity k hospodářskému sjednocení Evropy
Integrace má dvě podoby

- federalistická - státy se sjednotí politicky a ekonomická integrace následuje po politické
integraci, tento proces byl typický pro USA

- funkcionalistická - státy se propojují nejdříve ekonomickými vazbami a úplná politická
integrace je završením tohoto ekonomického integračního procesu. (Evropa)





Integrace (spojování do větších celků) má velký ekonomický význam - umožňuje
zhromadnění výroby a efektivnější využívání zdrojů, působí jak na mikroekonomické úrovní
(spojování firem) tak na makroekonomické úrovni.

Stupně mezinárodní ekonomické integrace:
1. Pásmo volného obchodu

2. Celní unie

3. Společný trh

4. Hospodářská unie

5. Úplná ekonomická unie

EU je v současné době ve fázi přechodu od hospodářské unie k úplné ekonomické unii. EU má 28 členských států včetně CR. Evropská měnová unie (eurozóna) zavedla jednotnou měnu - Euro.

Evropská unie má 3 pilíře:

- hospodářská a měnová unie - zavedení společné měny eura
- spolupráce v oblasti justice a vnitřních věcí

- společná zahraniční a bezpečnostní politika

62
\newpage