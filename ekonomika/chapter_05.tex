\chapter{Obchodní společnosti a družstva}

\begin{description}
    \item[Společenská smlouva] je smlouva, již alespoň dva společníci zakládají obchodní korporaci.
    \item[Zakladatelská listina] je právní dokument mající formu notářského zápisu, kterým samostatná osoba zakládá obchodní korporaci.
\end{description}

\paragraph*{Charakteristické rysy}
\begin{enumerate}
    \item osobní společnosti (v.o.s, k.s.)
        \begin{itemize}
            \item neomezené solidární ručení společníků za závazky společnosti, solidární ručení neznamená, že všichni ručí stejným dílem
            \item osobní účast společníků na řízení společnosti, nemají předepsány statutární orgán
            \item nemají ze zákona předepsán minimální základní kapitál společnosti
        \end{itemize}
    \item kapitálové společnosti (s.r.o., a.s., e.s.)
        \begin{itemize}
            \item omezené nebo žádné ručení společníků za závazky společnosti
            \item osobní účast společníků na řízení není vyžadována
            \item kapitálové spol. může založit jediný zakladatel nebo může mít jediného společníka v důsledku soustředění všech podílů v jeho rukou
            \item Jsou stanoveny zásady tvorby statutárních orgánů společnosti
            \item zákon ukládá vložit minimální základní kapitál a tvořit rezervní fond
        \end{itemize}
\end{enumerate}

\begin{description}
    \item[Základní kapitál] peněžní vyjádření souhrnu peněžních i nepeněžních vkladů společníků
    \item[Podíl společníka] podíl je účast společníka ve společnosti a z ní plynoucí práva a povinnosti
    \item[Rezervní fond] je vytvářen povinně ke krytí ztrát, pouze u s.r.o., a.s.
    \item[Statutární orgán] jsou osoby oprávněné jednat za právnické osoby
    \item[Přeměny společnosti]
        \begin{enumerate}[label=(\alph*)]
            \item []
            \item fůzí (sloučení dvou či více firem do jedné)
            \item rozdělením (z jedné firmy vzniknou dvě či více právně samostatných firem)
            \item změna právní formy společnosti (pokud to umožňuje zákon)	
        \end{enumerate}
\end{description}

\section{Obchodní společnosti - členění}
    \subsection{Osobní společnosti}
        \subsubsection{Veřejná obchodní společnost (v.o.s.) - společenská smlouva}
            \begin{itemize}
                \item společnost, ve které alespoň dvě osoby podnikají pod společnou firmou a ručí za závazky společnosti společně a nerozdílně celým svým majetkem
                \item statutární orgán je každý ze společníků
                \item základní kapitál: není povinný
                \item rozdělení zisku a ztrát: rovným dílem, nestanoví-li společenská smlouva jinak
                \item zákaz konkurence v oboru - nesmí mít zároveň jinou firmu se stejným předmětem
            \end{itemize}
        \subsubsection{Komanditní společnost (k.s., kom. spol.) - společenská smlouva}
            \begin{itemize}
                \item společnost, v níž jeden nebo více společníků ručí za závazky společnosti do výše svého nesplaceného vkladu zapsaného do obchodního rejstříku (komandisté) a jeden nebo více společníků ručí celým majetkem (komplementáři)
                \item statutární orgán - komplementáři
                \item základní kapitál: je povinný
                \item účast na řízení společnosti: mají pouze komplementáři
                \item zákaz konkurence v oboru: platí pouze pro komplementáře
                \item rozdělení zisku: nejdříve rozdělené na část náležící komanditistům a na část náležící komplementářům
                \item rozdělení ztrát: nesou komplementáři rovným dílem
            \end{itemize}
    \subsection{Kapitálové společnosti}
        \subsubsection{Společnost s ručením omezeným (s.r.o., spol.s.r.o.) - SS nebo ZL}
            \begin{itemize}
                \item společnost, jejíž základní kapitál je tvořen vklady společníků. Společnost může být založena 1 jednou osobou. Může mít max. 50 společníků. Společnost jako celek odpovídá za porušení svých závazků celým svým majetkem.
                \item základní kapitál: je povinný, minimálně 1 Kč
                \item rozhodování per rollam: může probíhat rozhodování i mimo valnou hromadu
                \item rozdělení zisku: zásadně do výše vkladu, pouze společenská smlouva může jinak
                \item zákaz konkurence v oboru:platí pro jednatele a členy dozorčí rady					
                \item ORGÁNY:
                    \begin{itemize}
                        \item Valná hromada - nejvyšší orgán
                        \item Jednatelé - statutární orgán, jsou povinní zajistit vedení evidence a účetnic
                        \item Dozorčí rada - kontrolní orgán, dohlíží na činnost jednatelů, podává zprávy VH
                    \end{itemize}
            \end{itemize}
        \subsubsection{Akciová společnost (a.s., akc.spol.) - SS nebo ZL}
            \begin{itemize}
                \item společnost, jejíž základní kapitál je rozvržen na určitý počet akcií o určité jmenovité hodnotě
                \item základní kapitál: minimálně 2 000 000 Kč nebo 80 000 eur
            \end{itemize}
            \begin{description}
                \item[Emisní ážio] již při emisi může být akcie prodána za cenu vyšší než nominální. Rozdíl je emisní ážio a účtuje se odděleně.
                \item[Orgány]
                    \begin{itemize}
                        \item []
                        \item Valná hromada - nejvyšší orgán
                        \item Dualistický systém
                            \begin{itemize}
                                \item []
                                \item statutární orgán - představenstvo - řídí firmu po celý rok
                                \item kontrolní orgán - dozorčí rada - kontroluje práci představen
                            \end{itemize}
                        \item Monistický systém - místo představenstva je statutární ředitel a místo dozorčí rady vykonává kontrolní činnost správní rada
                    \end{itemize}
                \item[Akcie] dlouhodobý cenný papír jehož majitel je společníkem akciové společnosti \par \textbf{Akcie rozlišujeme}: 
                    \begin{itemize}
                        \item Podle formy:
                            \begin{itemize}
                                \item []
                                \item listinné jsou fyzicky vytisknuté na papíře a mají řadu ochranných prvků (jako bankovky), aby se nedaly padělat
                                \item pláště, kde jsou uvedeny předepsané náležitosti o nominální hodnotě akcie, emitentovi a emisi
                                \item kupónového archu-kupón stříháte, když jednou ročně jdete pro dividendy
                                \item talonu - až vám dojdou kupóny, za talon dostanete nový arch
                                \item zaknihované jsou modernější forma, kdy je akcie zaznamenána v počítači Centrálního depozitáře cenných papírů. Většina akcií v ČR má tuto podobu.								
                            \end{itemize}
                        \item Podle druhu akcie:
                            \begin{itemize}
                                \item na jméno - stanovy společnosti mohou omezit, nikoliv však vyloučit, převoditelnost akcií na jméno
                                \item na majitele
                                \item speciální
                            \end{itemize}
                    \end{itemize}
                \item[TANTIÉMY] odměny ze zisku po zdanění členům představenstva a dozorčí rady společ.
                \item[DIVIDENDY] právo akcionáře na podíl ze zisku
            \end{description}
        \subsubsection{Družstva}
            Společnost neuzavřeného počtu osob založeného za účelem podnikání nebo zajišťování hospodářských, sociálních nebo jiných potřeb svých členů.
            Pdružstvo má nejméně 3 členy (alespoň 3 zakladatelé (FO nebo PO):
            ORGÁNY:
            \begin{itemize}
                \item Členská schůze - nejvyšší orgán
                \item Představenstvo - statutární orgán
                \item Kontrolní emise - kontrolní orgán							
            \end{itemize}
            Další orgány družstva: zejména komise (mzdová, kulturní, rozhodčí, sociální\ldots)
            Příklady družstev: Zemědělská, Obchodní, Výrobní, Úvěrová, Bytová, Sociální
            Spojování podnikatelů bez vzniku PO:
            \begin{enumerate}
                \item Společnost: Podnikatelé se mohou sdružit a vytvořit společnost, která není PO. Upraveno OZ.
                \item Smlouva o tichém společenství: OZ umožňuje tichému společníkovi vložit do firmy kapitál a inkasovat podíl na zisku, aniž by o tom kdokoliv věděl (kromě finančního úřadu).
            \end{enumerate}


        \subsubsection{Státní podniky (s.p.)}
            PO, zapisuje se do obchodního rejstříku.
            ORGÁNY:
                \begin{itemize}
                    \item Ředitel jmenovaný zakladatelem - statutární orgán
                    \item Dozorčí rada - kontrolní orgán
                \end{itemize}

