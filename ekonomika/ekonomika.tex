\documentclass[11pt,a4paper,twoside]{book}
\usepackage[czech]{babel}
\usepackage[utf8]{inputenc}
\usepackage[T1]{fontenc}
\usepackage[cm]{fullpage}
\usepackage{enumitem}
\usepackage{caption} 
\usepackage{amsmath}

\captionsetup[table]{skip=10pt}

\title{Ekonomie}
\author{Šárka Válková}
\date{duben 2019}

\begin{document}
	
	\pagenumbering{gobble}
	\maketitle
	\newpage
	\pagenumbering{arabic}
	\frenchspacing
	
	\setlist[enumerate]{itemsep=0mm}
	\setlist[itemize]{itemsep=0mm}
	
	
	\chapter{Ekonomické pojmy}
	
	\begin{description}
		\item[Ekonomie] společenská věda, zabývající se ekonomickými vztahy mezi lidmi
		\begin{itemize}
			\item jako věda se zabývá společenskou realitou zvanou ekonomika, ekonomická praxe
			\item využívá poznatků psychologie, sociologie, demografie, historie, managementu, marketingu, práva, politiky, hospodářského zeměpisu, etiky a také filozofie
		\end{itemize}
		\item[Ekonomika] se zabývá procesem, chováním jednotlivých subjektů, ekonomická teorie
		\item[Makroekonomie] se zabývá ekonomii státu, zkoumáním ekonomického systému jako celku, sleduje vztahy mezi agregátními veličinami
		\item[Mikroekonomie] se zabývá ekonomií firem (malé, střední, velké), domácností apod.
		\item[Ekonomický systém]
		\begin{description}
			\item[]
			\item[ZVYKOVÝ] rozhodoval vůdce kmene-náčelník, podle svých schopností
			\item[PŘÍKAZOVÝ] rozhodovala politická špička
			\item[TRŽNÍ] rozhoduje trh a jeho zákony
		\end{description}
		\item[Tři základní otázky]
		\begin{enumerate}
			\item[]
			\item Co a kolik se má ve společnosti vyrábět?
			\item Jak vyrábět,jakou technologií a jakými výrobními faktory?(práce,přírodní zdroje, kapitál)
			\item Jak se rozdělí to, co bylo vyrobeno?
		\end{enumerate}
		\item[Zákon vzácnosti] Co nemůžeme mít kdykoliv v jakémkoliv množství, je vzácné. Naše potřeby jsou neomezené, ale zdroje jsou omezené.
		\item[Zákon ekonomie času] Říká, že lidé musí šetřit svůj čas - snažit se v kratším čase dosáhnout stejné produkce nebo ve stejném
		čase dosáhnout vyšší produkce. Tomu říkáme zvyšování produktivity práce. Obětovaná příležitost.
		\item[Teorie potřeb] Každý z nás má jiné potřeby. Potřeba je pociťovaný nedostatek a hnací motor ekonomiky. Dělíme:
		\begin{itemize}
			\item Hmotné (jíst, bydlet, oblékat se\ldots)
			\item Nehmotné (svoboda, přátelství, kulturní zážitek\ldots)
			\item Zbytné (kaviár, loď\ldots)
			\item Nezbytné (spánek, voda, vzduch\ldots)
			\item Individuální (poslech rádia)
			\item Současné (nynější), budoucí (pozdější)
		\end{itemize}
		Potřeby uspokojujeme pomocí statků a služeb:
		
		\begin{tabular}{| p{8cm} | p{8cm} |}
			\hline
			\textbf{STATKY - určité předměty} & \textbf{SLUŽBY - cizí činnosti} \\
			\hline
			Hmotné (jídlo, byt, oblečení) \newline Nehmotné (vlastnosti, dovednosti, znalosti) \newline Volné (sluneční světlo, vzduch, déšť) \newline Ekonomické (světlo žárovky) \newline Kapitálové (budovy, stroje) \newline Spotřební (potraviny, uhlí, auta) & Věcné (oprava obuvi, vymalování bytu) \newline Osobní (kosmetika, lékař, kadeřník) \newline Zvířecí (kosmetika, lékař, manikůra) \\
			\hline
		\end{tabular}
		\item[Hospodářský proces] Členíme jej na fáze:
		\begin{enumerate}
			\item výroba
			\item rozdělování a přerozdělování
			\item směna
			\item spotřeba
		\end{enumerate}
		
		\item[Výroba] je činnost, při které člověk přetváří přírodu ve statky. Výrobu může provádět jednotlivec nebo celé výrobní firmy. \newline K výrobě potřebují mít čtyři základní předpoklady - \emph{VÝROBNÍ FAKTORY}.
		\begin{enumerate}
			\item PRÁCE - cenou je mzda
			\item PŘÍRODNÍ ZDROJE - především půdu
			\item KAPITÁL - je zisk nebo úrok
			\item INFORMACE
		\end{enumerate}
		\item[Rozdělování a přerozdělování]
		\item[Směna]
		\item[Spotřeba]
		\item[Práce] je cílevědomá lidská činnost vytvářející statky a služby. Práce je vzácný výrobní faktor, což ovlivňuje jeho cenu na trhu práce, tuto cenu nazýváme MZDOU.
		\item[Mzda reálná] mzda, vypovídá o skutečné hodnotě výdělku, co si zaměstnanec může pořídit.
		\item[Mzda nominální] mzda, kterou si pracovník vydělal jako zaměstnanec, když vynaložil svou pracovní sílu, jeho příjem peněz
		\item[Faktory ovlivňující výši mzdy]
		\begin{itemize}
			\item[]
			\item kvalifikace
			\item poptávka na trhu práce
			\item tržní úspěšnost
		\end{itemize}
		\item[Dělba práce] jednotlivec nevyrábí všechno, soustředí se pouze na určitou výrobu, jednotlivci jsou na sobě závislí
		\item[Specializace] zaměření na určitý okruh činnosti
		\item[Kooperace] vzájemná spolupráce
		\item[Přírodní zdroje (půda)] Při prodeji dosahuje vlastník tržní cenu. Při nájmu dosahuje pozemkovou rentu.
		\item[Pozemková renta]
		\begin{enumerate}[label=(\alph*)]
			\item[]
			\item absolutní renta - vyplývá z monopolu vlastnictví půdy a mají ji veškeré pozemky
			\item diferenční renta - závislá na kvalitě půdy a její vhodnost pro zemědělství-bonita půdy
		\end{enumerate}
		
		\item[Kapitál] peníze přinášející další peníze. Má 2 ceny:
		\begin{description}
			\item[Úrok] je cena vloženého (např. do banky) kapitálu
			\item[Zisk] je cena, kterou očekává vlastník při aktivním podnikání
		\end{description}
		\item[Hranice produkčních možností] Pokud společnost nemůže vyrábět se svými zdroji více určitých výrobků, aniž by snížila výrobu jiných výrobků. Společnost vyrábí efektivně, pokud se pohybuje na hranici produkčních možností
		\item[Spotřeba] Spotřeba je závěrečná fáze hospodářského procesu. Spotřebou statků a služeb uspokojujeme naše potřeby.
		\item[Spotřeba výrobní] Je to spotřeba firem (stroje, kancelářský papír, materiál).
		\item[Spotřeba konečná] Je spotřeba lidí, občanů, domácností (jídlo, pití, bydlení).
	\end{description}

	\chapter{Trh a jeho zákony}
	
	Trh je místo, kde dochází ke směně zboží. Místo, kde se setkává nabídka s poptávkou.
	Zboží je statek nebo služby určená ke směně na trhu.
	
	\begin{table}[h]
		\centering
		\caption{Členění trhu}
		\begin{tabular}{| p{5cm} | p{5cm} | p{5cm} |}
			\hline
			Podle územního hlediska &
			Podle množství &
			Podle předmětu prodeje a koupě zboží \\
			\hline
			\begin{enumerate}[label=(\alph*)]
				\item místní (oblastní) - tvarůžky
				\item národní - italská pizza 
				\item nadnárodní (EU) - ovoce 
				\item světový - ropa
			\end{enumerate} &
			\begin{enumerate}[label=(\alph*)]
				\item dílčí - trh mléka
				\item agregátní - trh veškerého zboží
			\end{enumerate} &
			\begin{enumerate}[label=(\alph*)]
				\item trh výrobních faktorů (trh práce)
				\item finanční - kapitálový a peněžní
				\item trh produktů (výrobků a služeb)
			\end{enumerate} \\
			\hline
		\end{tabular}
	\end{table}
	
	\paragraph*{Zákony trhu}
	
	V tržním systému se ekonomické subjekty rozhodují v zásadě samy. Chování tržních subjektů je však ovlivněno zákony trhu.
	
	\begin{enumerate}
		\item Zákon nabídky
		\begin{itemize}
			\item s rostoucí cenou roste i nabídka zboží
			\item nabídkou rozumíme souhrn všech zamýšlených prodejů
		\end{itemize}
		\item Zákon poptávky
		\begin{itemize}
			\item s rostoucí cenou klesá poptávka po zboží
			\item poptávkou rozumíme souhrn všech zamýšlených nákupů
		\end{itemize}
	\end{enumerate}
	
	Poptávku můžeme rozlišit:
	\begin{itemize}
		\item Individuální (1 kupující, 1 statek či služba)
		\item Dílčí (poptávka všech lidí po určitém statku či službě)
		\item Agregátní (všech lidí po všech statcích a službách)
	\end{itemize}
	
	Nabídku obdobně:
	\begin{itemize}
		\item Individuální (1 výrobce určitého statku či služby)
		\item Dílčí (nabídka všech výrobců určitého zboží)
		\item Agregátní (nabídka všech výrobců, všeho druhu zboží)
	\end{itemize}
	
	\paragraph*{Faktory ovlivňující poptávku}
	\begin{itemize}
		\item Cena
		\item Demografické změny (v počtech a charakteristikách kupujících)
		\item Změny velikosti důchodů (mzdy, renty vlastníků půdy, úroky a zisky vlastníků kapitálu)
		\item Změny v preferencích (zvyky, móda, změny potřeb)
		\item Změny cen jiných zboží
		\begin{itemize}
			\item substituty - jiné zboží, které může nahradit při spotřebě zboží sledované (jablka - hrušky, čaj - káva)
			\item komplementy- jiné zboží, které doplňuje při použití zboží sledované (auto — benzin, DVD přehrávač - DVD filmy)
		\end{itemize}
	\end{itemize}
	
	\paragraph*{Faktory ovlivňující nabídku}
	\begin{itemize}
		\item Cena
		\item Náklady výroby a obchodu (dražší suroviny, energie, zvyšování mezd)
		\item Změny vnějších podmínek podnikání (organizace trhu, počasí pro zemědělce, daně apod.)
		\item Změny kapitálové výnosnosti
	\end{itemize}
	
	\paragraph*{Interakce nabídky a poptávky}
	\begin{itemize}
		\item Výrobci dopředu neznají rovnovážnou cenu a jí odpovídající ideální množství vyrobeného zboží pro trh
		\item Interakcí nabídky a poptávky získáváme rovnovážnou cenu a jí odpovídající množství statků
	\end{itemize}
	
	[Obr. 8 Interakce nabídky a poptávky]
	
	\textbf{Dokonalá konkurence}
	\begin{itemize}
		\item Výrobci mají rovné podmínky přístupu na trh
		\item Kupující mají stejný přístup a jediným kritériem pro koupi či nekoupi je cena
		\item V praxi skoro nenajdeme
	\end{itemize}
	
	\textbf{Nedokonalá konkurence}
	\begin{itemize}
		\item Mezi potencionálními prodávajícími má jeden výsadní postavení na trhu
		\begin{itemize}
			\item administrativní monopol - jediný výrobce či prodejce má povolení od státu (Česká pošta)
			\item absolutní monopol - jediný výrobce zná recept na výrobu (Coca-cola, Kofola)
		\end{itemize}
		\item Výsadní postavení mezi poptávajícími, je výhradním nakupujícím-stát - monopsony (speciální technologie, zbraně hromadného ničení)
	\end{itemize}
	
	\textbf{Monopson} - výsadní postavení určitého kupujícího na trhu.
	
	\paragraph*{Členění trhu podle míry konkurence}
	\begin{itemize}
		\item Dokonalá konkurence
		\item Monopolistická konkurence
		\item Oligopol
	\end{itemize}
	
	\paragraph*{Selhání trhu}
	\begin{itemize}
		\item Zneužití výsadního postavení monopoly
		\item Existence veřejných statků
		\item Externality trhu
		\begin{itemize}
			\item negativní/špatný vliv výrobců na životní prostředí
			\item pozitivní/podnikatelská činnost může přinést i užitek (včelař zajišťuje opylení sadů a polí)
		\end{itemize}
	\end{itemize}
	
	\paragraph*{Subjekty trhu}
	\begin{itemize}
		\item domácnosti - přichází na trh za účelem uspokojení potřeb
		\item firmy - subjekty vyrábějící za účelem prodeje
		\item stát (vláda) - vstupuje na trh s cílem ovlivnit jej
	\end{itemize}
	
	\begin{description}
		\item[TRŽNÍ ROVNOVÁHA] Když se vyrovná nabídka a poptávka, vzniká stabilní situace na trhu. Důsledkem je rovnovážná cena na trhu.
		\item[PŘEBYTEK] Vzniká soutěž prodávajících, snižování cen, výroby atd., pokles nabízeného množství
		\item[NEDOSTATEK] Zboží na trhu, typické pro ekonomiku plánovanou, vzniká soutěž kupujících, jsou ochotní zaplatit i cenu vyšší, zvyšování ceny i při nekvalitních výrobcích.
	\end{description}
	
	Tržní síly vedou k utváření rovnovážných cen, tím je určeno vyráběné množství výrobků (struktura výroby).
	
	\chapter{Firmy a jejich formy}
	
	\begin{description}
		\item[Právní osobnost] Vzniká narozením a zaniká smrtí. Schopnost být nositelem práv a povinností (právní subjektivita).
		\item[Svéprávnost] Způsobilost k právním úkonům - právně jednat (uzavírat smlouvy). Plná svéprávnost platí dovršením 18 let.
	\end{description}
	
	\paragraph{Fyzické osoby}
	
	\begin{itemize}
		\item je jeden konkrétní člověk, který platí daň z příjmu
		\item podnikatelé podnikající na základě živnostenského či jiného oprávnění, zaměstnanci
		\item způsobilost podnikat nabývá dovršením 18 let
	\end{itemize}
	
	\textbf{Členění fyzických osob}:
	\begin{enumerate}
		\item Podnikatelé
		\begin{itemize}
			\item Podnikající na základě živnostenského oprávnění dle živnostenského zákona
			\item Podnikající na základě jiného oprávnění dle jiných zákonů (lékaři, veterináři)
		\end{itemize}
		\item Zaměstnanci
		\begin{itemize}
			\item Osoby schopné vstupovat do pracovně právních vztahů se svými zaměstnavateli, zde se řídíme především zákoníkem práce
		\end{itemize}
		\item Spotřebitel
		\begin{itemize}
			\item Spotřebitel je každý člověk, který sám uzavírá smlouvu s podnikatelem nebo s ním jedná (NOZ)
		\end{itemize}
	\end{enumerate}
	
	\paragraph*{Právnické osoby}
	\begin{itemize}
		\item je společnost několika osob, která je oprávněna vstupovat do právních vztahů a jednat svým jménem
		\item jsou zapsány v obchodním rejstříku
		\item obchodní společnosti, družstva, státní podniky
	\end{itemize}
	
	\textbf{Členění právnických osob}:
	\begin{enumerate}
		\item Korporace (obchodní korporace a spolky, politické strany)
		\item Fundace (nadace a nadační fondy)
		\item Ústavy (propojení majetkové a osobní složky)
	\end{enumerate}
	
	\textbf{Členění obchodních společností}:
	\begin{enumerate}
		\item Osobní společnosti
		\begin{itemize}
			\item veřejná obchodní společnost (v.o.s.)
			\item komanditní společnost (k.s.)
		\end{itemize}
		\item Kapitálové společnosti
		\begin{itemize}
			\item společnost s ručením omezeným (s.r.o.)
			\item akciová společnost (a.s.)
		\end{itemize}
	\end{enumerate}
	
	
	\textbf{Živnost} je soustavná Činnost provozovaná samostatně, vlastním jménem, na vlastní
	odpovědnost, za účelem dosažení zisku a za podmínek stanovených tímto zákonem
	
	\textbf{Živnost není}
	\begin{itemize}
		\item činnosti vyhrazené zákonem státu nebo určené právnické osoby
		\item využívání výsledků duševní tvůrčí činnosti (autorský zákon, vynálezy)
		\item povolání lékařů, veterinářů, advokátů, notářů, znalců, daňových poradců, makléřů
		\item činnost bank, pojišťoven, penzijních fondů, burz, pořádání loterií, hornická činnost, výroba a rozvod elektřiny, plynu a tepla, zemědělství, provoz drah, telekomunikační sítě, výroba léčiv, zprostředkovávání zaměstnání, výchova a vzdělávání ve školách \ldots
		\item pronájem nemovitostí, bytu a nebytových prostor
	\end{itemize}
	
	\textbf{Všeobecné podmínky pro podnikání}:
	\begin{itemize}
		\item plná svéprávnost (18 let a způsobilost k právním úkonům)
		\item bezúhonnost (dle výpisu z TR, který není starší více než 3 měsíce)
	\end{itemize}
	
	\subparagraph{Zvláštní podmínky pro podnikání:}
	\begin{itemize}
		\item odborná a jiná způsobilost (u řemeslných živností se požaduje vyučení v oboru)
	\end{itemize}
	
	\subparagraph{Případy, kdy nelze provozovat činnost:}
	\begin{itemize}
		\item když soud uložil podnikateli zákaz činnosti v oboru
		\item když na majetek podnikatele byl prohlášen konkurz
		\item podnikat nelze v případě, pokud návrh na konkurz byl zrušen pro nedostatek majetku
	\end{itemize}
	
	\subparagraph{Emancipace}
	\begin{itemize}
		\item osvobození z podřízeného postavení, zrovnoprávnění
		\item boj za rovnoprávnost žen
	\end{itemize}
	
	\textbf{Odpovědný zástupce} je fyzická osoba ustanovená podnikatelem, která odpovídá za řádný provoz živnosti a dodržování živnostenskoprávních předpisů a která je obvykle v pracovně právním nebo jiném vztahu k podnikateli.

	Odpovědného zástupce je povinen ustanovit:
	\begin{enumerate}
		\item podnikatel - FO, který nesplňuje zvláštní podmínky provozování živnosti
		\item podnikatel - zahraniční FO, který nemá na území ČR povolen pobyt
		\item podnikatel - PO se sídlem v ČR.
		\item podnikatel - zahraniční PO
	\end{enumerate}

	\textbf{Rozdělení živností}:
	\begin{enumerate}
		\item ohlašovací
			\begin{itemize}
				\item Řemeslné - podmínkou je odborná způsobilost získaná vyučením v oboru
				\item Vázané - podmínkou je odborná způsobilost (autoškola)
				\item Volné - poskytování služeb pro zemědělství, zahradnictví, lesnictví, myslivost
			\end{itemize}				
		\item koncesované - lze je provozovat pouze na základě udělení koncese-
	\end{enumerate}	

	\textbf{Zánik Živnostenského oprávnění}:
	\begin{itemize}
		\item smrtí podnikatele
		\item zánikem právnické osoby, výmazem zahraniční osoby z obchodního rejstříku
		\item uplynutím doby, pokud bylo vydáno na dobu určitou
		\item rozhodnutím živnostenského úřadu o zrušení živnostenského oprávnění
	\end{itemize}

	\textbf{Živnostenský zákon dále stanovuje}:
	\begin{itemize}
		\item povinnosti podnikatelů	
			\begin{itemize}
				\item povinnost dokladovat kontrolním orgánům způsob nabytí prodávaného zboží
				\item povinnost zajistit, aby v provozovně, byla přítomna osoba znalá českého nebo sloven. jazyka
			\end{itemize}
		\item na požádání vydávat kupujícímu doklady o prodeji zboží
		\item každý podnikatel dostane své IČ (identifikační číslo)
		\item Živnostenské úřady vedou živnostenský rejstřík v elektronické podobě
		\item velká část Živnostenského rejstříku je veřejná - obsahuje údaje o podnikatelích
		\item DIČ = daňové identifikační číslo
		\item Živnostenský rejstřík - informační systém veřejné správy.
	\end{itemize}

	\chapter{Občanský zákoník a zákon o obchodních korporacích}

	\begin{description}
		\item[Podnikatel] Samostatně vykonává na vlastní účet a odpovědnost výdělečnou činnost živnostenským nebo obdobným způsobem se záměrem Činit tak soustavně za účelem dosažení zisku.
	\end{description}

	\section*{Zákon o obchodních korporacích}
	Upravuje obchodní společnosti a družstva.
		\begin{description}
			\item[Obchodní firma] Název, pod kterým je podnikatel zapsán v obchodním rejstříku. Podnikatel je povinen činit právní úkony pod svou firmou.
				\begin{enumerate}
					\item Fyzické osoby mají povinně jako obchodní firmu své jméno a příjmení. (Ivo Večerka, popřípadě dodatek Ivo Večerka - truhlář)
					\item Právnické osoby mají jméno, pod kterým jsou zapsána v obchodním rejstříku. Součástí je i dodatek označující jejich právní formu. (Kaufland, v.o.s., Globus ČR k.s., Jednota Kladno, družstvo)
				\end{enumerate}
			\item[Obchodní závod] Organizovaný soubor jmění, který vytvořil podnikatel a slouží k provozování jeho činnosti.
			\item[Sídlo podnikatele] Adresa zapsána do obchodního rejstříku. (místo, kde má obchodní závod nebo bydliště)
			\item[Jednání podnikatele]
				\begin{itemize}
					\item []
					\item FO - jednají osobně nebo prostřednictvím zástupce
					\item PO - jednají prostřednictvím statutárních orgánů (jednatel, představenstvo\ldots) nebo prostřednictvím zástupce
				\end{itemize}
			\item[Prokura] širší plná moc.
				\begin{itemize}
					\item zmocňuje prokuristu ke všem právním úkonům (kromě prodeje nemovitosti)
					\item lze ji udělit pouze FO (spolumajiteli nebo zaměstnanci firmy)
					\item prokurista musí být zapsán v obchodním rejstříku
					\item praxi bývá prokurista druhý nejdůležitější muž firmy
				\end{itemize}
			\item[Prokurista] zástupce firmy s velmi širokou plnou mocí (prokurou)
			\item[Obchodní tajemství] Tvoří konkurenčně významné, určitelné, ocenitelné a v příslušných obchodních kruzích běžně nedostupné skutečnosti související se závodem a jejichž vlastník zajišťuje ve svém zájmu odpovídajícím způsobem jejich utajení.
			\item[Obchodní rejstřík] Je veřejný rejstřík, do kterého se zapisují zákonem stanovené údaje týkající se podnikatelů nebo organizačních složek jejich obchodních závodů, o nichž to stanoví zákon.
			\item[Konstitutivní funkce OR] Zápisem do OR vznikají právnické osoby. \par Do obchodního rejstříku se zapisují:
				\begin{enumerate}
					\item obchodní společnosti, družstva a jiné právnické osoby-povinně
					\item zahraniční osoby a jejich závody-povinně
					\item FO s trvalým pobytem v ČR, se zapíše do OR povinně nebo na vlastní žádost (chtějí mít prokuristu)			
				\end{enumerate}
			\item[Účetnictví podnikatelů] Podnikatelé vedou účetnictví v rozsahu a způsobem stanoveným zákonem (zákon o účetnictví), který v paragrafu 1 říká:
				\begin{itemize}
					\item Podnikatelé zapsaní v OR vedou povinně účetnictví
					\item FOsobratem nad 25 milionů vedou povinně účetnictví
					\item Ostatní podnikatelé mohou vést účetnictví dobrovolně nebo vedou daň.evidenci
				\end{itemize}
			\item[Účetní období]
				\begin{itemize}
					\item []
					\item Účetním obdobím je nepřetržitě po sobě jdoucích 12 měsíců. Buď se shoduje s kalendářním rokem nebo je hospodářským rokem (zemědělci-od sklizně do sklizně)
					\item Při přechodu na jiné účetní období může být délka delší nebo kratší než 12 měsíců
					\item Při vzniku nebo zániku obchodních společností, může být delší až 15 měsíců nebo kratší
					\item V případě přeměn obchodních společností, může být kratší nebo neomezeně delší
				\end{itemize}
			\item[Hospodářská soutěž (konkurence)] Je souběžná snaha subjektů na trhu určitého druhu zboží nebo služeb, cílem je dosažení určitých výhod před ostatními v oblasti hospodářských užitků.\par Zneužití účasti v hospodářské soutěži:
				\begin{enumerate}
					\item nekalá soutěž - upravuje občanský zákoník
					\begin{itemize}
						\item Klamavá reklama
						\item Podplácení
						\item Zlehčování
						\item Srovnávací reklama
						\item Porušení obchodního tajemství
					\end{itemize}
					\item nedovolené omezování hospodářské soutěže - upravuje zákon o ochraně hosp.soutěže
					\begin{itemize}
						\item Dohody soutěžitelů omezující hospodářskou soutěž
						\item Dohody o sloučení soutěžitelů vedoucí k omezení konkurence
						\item Zneužití monopolního nebo dominantního postavení soutěžitelů
					\end{itemize}
				\end{enumerate}
			\item[Dvoustupňové zahájení podnikání PO]
				\begin{enumerate}
					\item []
					\item ustavení (založení) obchodní společnosti (sepsání smlouvy mezi společníky)
					\item vznik obchodní společnosti (zápis do obchodního rejstříku)			
				\end{enumerate}
				V mezičase si společnost vyřizuje živnostenské oprávnění
			\item[Dvoustupňové ukončení podnikání PO]
				\begin{enumerate}
					\item []
					\item zrušení obchodní společnosti (s likvidací, bez likvidace, insolvenční řízení a úpadek)
					\item zánik obchodní společnosti (výmaz z obchodního rejstříku)
				\end{enumerate}
			\item[Zrušení s likvidací]
				\begin{itemize}
					\item []
					\item neexistuje právní nástupce (firma ukončením likvidace přestane existovat)
					\item statutární orgány jsou nahrazeny likvidátorem
					\item zjistí-li likvidátor, že je společnost v úpadku (předlužená \ldots), podá insolvenční návrh
				\end{itemize}
			\item[Zrušení bez likvidace]
				\begin{itemize}
					\item []
					\item existuje právní nástupce (sloučení firem, prodej firmy, rozdělení firmy)
					\item majetek přechází na právního nástupce
					\item firma udělá účetní závěrku, tím vyčíslí svůj majetek, závazky a pohledávky, které převezme nástupnická firma- původní firma zaniká
				\end{itemize}
			\item[Řešení úpadku]
				\begin{itemize}
					\item []
					\item věřitelé firmy mohou podat k soudu návrh na řešení úpadku dlužníka
					\item v případě prohlášení konkurzu je majetek firmy rozprodán a z peněz jsou hrazeny dluhy
					\item o platbě dluhů rozhoduje insolvenční soud
				\end{itemize}
		\end{description}

	\section*{Ekonomika a neziskový sektor}
		\begin{description}
			\item[Neziskový sektor] Cílem subjektů neziskového sektoru není podnikat (dosahovat zisk), ale zajišťovat jiné funkce ve společnosti. V omezené míře však mohou subjekty NS i podnikat.
			\item[Principy rozdělování]
				\begin{itemize}
					\item []
					\item v ziskovém sektoru rozdělujeme výsledný produkt podle množství, kvality a tržní úspěšnosti práce
					\item v neziskovém sektoru rozdělujeme podle potřeb
				\end{itemize}
			\item[Státní neziskové organizace]
				\begin{itemize}
					\item []
					\item Státní školství
					\item Státní zdravotnictví
					\item Instituce na ochranu Životního prostředí, kulturních památek
					\item Celá oblast státní správy
				\end{itemize}
			\item[Nestátní neziskové organizace]
				\begin{itemize}
					\item []
					\item Církevní organizace (Centra pro rodinu, charity, semináře)
					\item Spolky (sdružení občanů)
					\item Ústavy
					\item Fundace (nadace, nadační fondy)
					\item Politické strany
				\end{itemize}
		\end{description}
	\chapter{Obchodní společnosti a družstva}

	\begin{description}
		\item[Společenská smlouva] je smlouva, již alespoň dva společníci zakládají obchodní korporaci.
		\item[Zakladatelská listina] je právní dokument mající formu notářského zápisu, kterým samostatná osoba zakládá obchodní korporaci.
	\end{description}

	\paragraph*{Charakteristické rysy}
	\begin{enumerate}
		\item osobní společnosti (v.o.s, k.s.)
			\begin{itemize}
				\item neomezené solidární ručení společníků za závazky společnosti, solidární ručení neznamená, že všichni ručí stejným dílem
				\item osobní účast společníků na řízení společnosti, nemají předepsány statutární orgán
				\item nemají ze zákona předepsán minimální základní kapitál společnosti
			\end{itemize}
		\item kapitálové společnosti (s.r.o., a.s., e.s.)
			\begin{itemize}
				\item omezené nebo žádné ručení společníků za závazky společnosti
				\item osobní účast společníků na řízení není vyžadována
				\item kapitálové spol. může založit jediný zakladatel nebo může mít jediného společníka v důsledku soustředění všech podílů v jeho rukou
				\item Jsou stanoveny zásady tvorby statutárních orgánů společnosti
				\item zákon ukládá vložit minimální základní kapitál a tvořit rezervní fond
			\end{itemize}
	\end{enumerate}

	\begin{description}
		\item[Základní kapitál] peněžní vyjádření souhrnu peněžních i nepeněžních vkladů společníků
		\item[Podíl společníka] podíl je účast společníka ve společnosti a z ní plynoucí práva a povinnosti
		\item[Rezervní fond] je vytvářen povinně ke krytí ztrát, pouze u s.r.o., a.s.
		\item[Statutární orgán] jsou osoby oprávněné jednat za právnické osoby
		\item[Přeměny společnosti]
			\begin{enumerate}[label=(\alph*)]
				\item []
				\item fůzí (sloučení dvou či více firem do jedné)
				\item rozdělením (z jedné firmy vzniknou dvě či více právně samostatných firem)
				\item změna právní formy společnosti (pokud to umožňuje zákon)	
			\end{enumerate}
	\end{description}

	\section{Obchodní společnosti - členění}
			\subsection{Osobní společnosti}
				\subsubsection{Veřejná obchodní společnost (v.o.s.) - společenská smlouva}
					\begin{itemize}
						\item společnost, ve které alespoň dvě osoby podnikají pod společnou firmou a ručí za závazky společnosti společně a nerozdílně celým svým majetkem
						\item statutární orgán je každý ze společníků
						\item základní kapitál: není povinný
						\item rozdělení zisku a ztrát: rovným dílem, nestanoví-li společenská smlouva jinak
						\item zákaz konkurence v oboru - nesmí mít zároveň jinou firmu se stejným předmětem
					\end{itemize}
				\subsubsection{Komanditní společnost (k.s., kom. spol.) - společenská smlouva}
					\begin{itemize}
						\item společnost, v níž jeden nebo více společníků ručí za závazky společnosti do výše svého nesplaceného vkladu zapsaného do obchodního rejstříku (komandisté) a jeden nebo více společníků ručí celým majetkem (komplementáři)
						\item statutární orgán - komplementáři
						\item základní kapitál: je povinný
						\item účast na řízení společnosti: mají pouze komplementáři
						\item zákaz konkurence v oboru: platí pouze pro komplementáře
						\item rozdělení zisku: nejdříve rozdělené na část náležící komanditistům a na část náležící komplementářům
						\item rozdělení ztrát: nesou komplementáři rovným dílem
					\end{itemize}
			\subsection{Kapitálové společnosti}
				\subsubsection{Společnost s ručením omezeným (s.r.o., spol.s.r.o.) - SS nebo ZL}
					\begin{itemize}
						\item společnost, jejíž základní kapitál je tvořen vklady společníků. Společnost může být založena 1 jednou osobou. Může mít max. 50 společníků. Společnost jako celek odpovídá za porušení svých závazků celým svým majetkem.
						\item základní kapitál: je povinný, minimálně 1 Kč
						\item rozhodování per rollam: může probíhat rozhodování i mimo valnou hromadu
						\item rozdělení zisku: zásadně do výše vkladu, pouze společenská smlouva může jinak
						\item zákaz konkurence v oboru:platí pro jednatele a členy dozorčí rady					
						\item ORGÁNY:
							\begin{itemize}
								\item Valná hromada - nejvyšší orgán
								\item Jednatelé - statutární orgán, jsou povinní zajistit vedení evidence a účetnic
								\item Dozorčí rada - kontrolní orgán, dohlíží na činnost jednatelů, podává zprávy VH
							\end{itemize}
					\end{itemize}
				\subsubsection{Akciová společnost (a.s., akc.spol.) - SS nebo ZL}
					\begin{itemize}
						\item společnost, jejíž základní kapitál je rozvržen na určitý počet akcií o určité jmenovité hodnotě
						\item základní kapitál: minimálně 2 000 000 Kč nebo 80 000 eur
					\end{itemize}
					\begin{description}
						\item[Emisní ážio] již při emisi může být akcie prodána za cenu vyšší než nominální. Rozdíl je emisní ážio a účtuje se odděleně.
						\item[Orgány]
							\begin{itemize}
								\item []
								\item Valná hromada - nejvyšší orgán
								\item Dualistický systém
									\begin{itemize}
										\item []
										\item statutární orgán - představenstvo - řídí firmu po celý rok
										\item kontrolní orgán - dozorčí rada - kontroluje práci představen
									\end{itemize}
								\item Monistický systém - místo představenstva je statutární ředitel a místo dozorčí rady vykonává kontrolní činnost správní rada
							\end{itemize}
						\item[Akcie] dlouhodobý cenný papír jehož majitel je společníkem akciové společnosti \par \textbf{Akcie rozlišujeme}: 
							\begin{itemize}
								\item Podle formy:
									\begin{itemize}
										\item []
										\item listinné jsou fyzicky vytisknuté na papíře a mají řadu ochranných prvků (jako bankovky), aby se nedaly padělat
										\item pláště, kde jsou uvedeny předepsané náležitosti o nominální hodnotě akcie, emitentovi a emisi
										\item kupónového archu-kupón stříháte, když jednou ročně jdete pro dividendy
										\item talonu - až vám dojdou kupóny, za talon dostanete nový arch
										\item zaknihované jsou modernější forma, kdy je akcie zaznamenána v počítači Centrálního depozitáře cenných papírů. Většina akcií v ČR má tuto podobu.								
									\end{itemize}
								\item Podle druhu akcie:
									\begin{itemize}
										\item na jméno - stanovy společnosti mohou omezit, nikoliv však vyloučit, převoditelnost akcií na jméno
										\item na majitele
										\item speciální
									\end{itemize}
							\end{itemize}
						\item[TANTIÉMY] odměny ze zisku po zdanění členům představenstva a dozorčí rady společ.
						\item[DIVIDENDY] právo akcionáře na podíl ze zisku
					\end{description}
				\subsubsection{Družstva}
					Společnost neuzavřeného počtu osob založeného za účelem podnikání nebo zajišťování hospodářských, sociálních nebo jiných potřeb svých členů.
					Pdružstvo má nejméně 3 členy (alespoň 3 zakladatelé (FO nebo PO):
					ORGÁNY:
					\begin{itemize}
						\item Členská schůze - nejvyšší orgán
						\item Představenstvo - statutární orgán
						\item Kontrolní emise - kontrolní orgán							
					\end{itemize}
					Další orgány družstva: zejména komise (mzdová, kulturní, rozhodčí, sociální\ldots)
					Příklady družstev: Zemědělská, Obchodní, Výrobní, Úvěrová, Bytová, Sociální
					Spojování podnikatelů bez vzniku PO:
					\begin{enumerate}
						\item Společnost: Podnikatelé se mohou sdružit a vytvořit společnost, která není PO. Upraveno OZ.
						\item Smlouva o tichém společenství: OZ umožňuje tichému společníkovi vložit do firmy kapitál a inkasovat podíl na zisku, aniž by o tom kdokoliv věděl (kromě finančního úřadu).
					\end{enumerate}


				\subsubsection{Státní podniky (s.p.)}
					PO, zapisuje se do obchodního rejstříku.
					ORGÁNY:
						\begin{itemize}
							\item Ředitel jmenovaný zakladatelem - statutární orgán
							\item Dozorčí rada - kontrolní orgán
						\end{itemize}

	\chapter{Majetková a kapitálová výstavba podniku}

	Členění majetku:
	[Obr. 2 Členění majetku firmy podle druhů (aktiva firmy)]

	Zdroje krytí majetku:
	[Obr. 3 Majetek firmy podle zdrojů jeho krytí (pasiva)]
	\begin{description}
		\item[ROZVAHA] je statický pohled na majetek, má počáteční a konečný stav \par
			\begin{center}
				{\Large AKTIVA = PASIVA}
			\end{center}
		\item[AKTIVA] Majetek firmy. To, co firma vlastní.
		\item[PASIVA] Zdroje financování. Z čeho byl majetek zaplacen.
	\end{description}

	\begin{table}[h]
		\caption{Rozvaha k 1.1.2016}
		\begin{tabular}{ p{8cm} | p{8cm} }
			Aktiva &
			Pasiva \\ \hline
			\begin{enumerate}
				\item Stálá aktiva
					\begin{itemize}
						\item dlouhodobý hmotný majetek
						\item dlouhodoby nehmotný majetek
						\item douhodoby finanční majetek
					\end{itemize}
				\item Oběžná aktiva
					\begin{itemize}
						\item zásoby
						\item bankovní účty
						\item penize v pokladně
						\item pohledávky
						\item []
					\end{itemize}
				\item Ostatní aktiva 
			\end{enumerate} &
			\begin{enumerate}
				\item Vlastní zdroje
					\begin{itemize}
						\item základní kapitál
						\item fondy
						\item zisky
					\end{itemize}
				\item Cizí zdroje
					\begin{itemize}
						\item dlouhodobé úvěry
						\item krátkodobé úvěry
						\item dodavatelé
						\item zaměstnanci
						\item státní rozpočet
					\end{itemize}
				\item Ostatní pasiva
			\end{enumerate} \\ \hline
			AKTIVA CELKEM & PASIVA CELKEM \\
		\end{tabular}
	\end{table}

	\textbf{Způsoby pořízení dlouhodobého majetku}:
	\begin{itemize}
		\item Nákup - nového nebo již použitého DM
		\item Vlastní výroba - stavební firma si postaví novou výrobní halu
		\item Darovánií - stát může darovat ekologické zařízení
		\item Převod z osobního majetku podnikatele - truhlář vloží do firmy svoji garáž
		\item Vklad majetku společníky - dceřina společnost - do které vloží budovu se sklady
		\item Novým zjištěním - jde o majetek, který v UCE nebyl dosud zachycen
		\item Finanční leasing
	\end{itemize}

	\textbf{Finanční leasing}: DM si pronajmeme a splácíme a po splacení za symbolickou cenu odkoupíme do vlastnictví.
	\textbf{Operativní leasing}: DM si zapůjčíme, platíme pronájem a po ukončení nájmu majetek vrátíme - nestane se našim vlastnictvím.
	\textbf{Výhody finančního leasingu}:
	\begin{itemize}
		\item Firma si může koupit DM, i když na něj nemá finance, pokud ví, že tato investice si bude sama vydělávat na své uhrazení
		\item Investice pořízená formou FL se rychleji dostane do nákladů
	\end{itemize}

	\textbf{Nevýhody finančního leasingu}:
	\begin{itemize}
		\item V případě, že firma se dostane do finančních potíží a přestane splácet řádně splátky, leasingová společnost si vezme DM zpět a již uhrazené splátky propadají
		\item V případě zcizení nebo zničení DM před konečným splacením hradí pojišťovna pojistné leasing. spol., ta pokryje své náklady, a teprve zbytek uhradí nájemci
	\end{itemize}

	\textbf{Způsoby vyřazení dlouhodobého majetku}:
	\begin{enumerate}
		\item likvidace
		\item prodej
		\item manko nebo škoda
		\item darování
		\item přeřazení do osobního užívání
	\end{enumerate}

	\textbf{Oceňování a odepisování majetku}:
	Majetek oceňuje firma vstupní cenou, která se liší podle způsobu pořízení majetku.
	Hmotný majetek:
	\begin{itemize}
		\item \textbf{Pořizovací cenou} - při nákupu od dodavatele \par PC = cena pořízení + náklady související s pořízením (doprava, montáž, inflace)
		\item \textbf{Reprodukční pořizovací cenou} - v případě, že nemá firma doklad o hodnotě majetku, určí cenu odhadce
		\item \textbf{Cenou ve vlastních nákladech} - v případě, že si firma sama vyrobí DM. \par Firma sečte všechny náklady, ale nesmí započítat zisk.
	\end{itemize}

	\textbf{Nehmotný majetek} oceňujeme stejně jako hmotný dlouhodobý majetek:
	\begin{itemize}
		\item pořizovací cenou
		\item reprodukční pořizovací cenou
		\item cenou ve vlastních nákladech
	\end{itemize}

	\textbf{Finanční majetek}:
	Oceňujeme pořizovací cenou včetně přímých nákladů souvisejících s pořízením

	\textbf{Odepisování majetku}:
	Odepisování znamená, že hodnotu majetku přenášíme do nákladů firmy postupně několik let prostřednictvím ročních odpisů. Odepisujeme hmotný a nehmotný dlouhodobý majetek.

	\textbf{Opotřebování majetku}:
	\begin{itemize}
		\item \textbf{Fyzické} používáním se součástky ničí, nepoužíváním součástky rezavějí\ldots
		\item \textbf{Morální} i fyzicky skvěle zachovalý stroj může být technicky zastaralý\ldots
	\end{itemize}

	\textbf{Druhy odpisů}:
	\begin{itemize}
		\item \textbf{Odpisy účetní} Jsou upraveny zákonem o účetnictví. Tyto odpisy mají vyjadřovat co nejobjektivněji skutečnou míru opotřebení toho kterého DM ve firmě. S jejich pomocí dosáhneme dobrého přehledu o skutečné výši majetku firmy.
		\item \textbf{Odpisy daňové} Tyto odpisy slouží jako daňový doklad a ovlivňují výši daní z příjmu podnikatele. Metody výpočtu těchto odpisů jsou závazně stanoveny státem a zakotveny v zákonu o daní z příjmu.
	\end{itemize}

	\begin{center}
	Zůstatková cena = vstupní cena - oprávky
	Oprávky = jsou součtem doposud provedených odpisů	
	\end{center}

	[Tabulka Odpisové třídy a doba odpisu]

	\textbf{Funkce odpisů}:
	\begin{itemize}
		\item \textbf{Funkce nákladová} - pomocí odpisů přenášíme hodnotu DM do nákladů
		\item \textbf{Funkce zdrojová} - odpisy jsou pro firmu zdrojem financí
		\item \textbf{Funkce fiskální} - odpisy ovlivňují výši příjmů státního rozpočtu z daně z příjmu
		\item \textbf{Funkce rozvojová} - umožní-li stát podnikatelům rychle odepisovat DM, stimuluje je tím k rychlejší obměně strojního vybavení a k zavádění moderních technologií, které umožní rozvoj firem a tím i celého hospodářství
	\end{itemize}

	\textbf{Evidence a reprodukce majetku}:
	Důvody evidence DM:
	\begin{itemize}
		\item Kontrola majetku-inventarizace
		\item Odepisování majetku-účetnictví a daně
		\item Přehled o finanční hodnotě firmy
		\item Úhrady škod na majetku pojišťovnou
	\end{itemize}

	Základní evidenci provádíme na \textbf{inventárních kartách}.

	Povinně karta musí obsahovat inventární číslo přidělené DM, zvolený způsob odepisování, vstupní cenu, jednotlivé roční daňové odpisy.

	Při pořízení DM navíc vypisujeme zápis o pořízení DM a při vyřazení pak zápis o vyřazení.

	\textbf{Reprodukce majetku}:
	Údržba, opravy a nákup dlouhodobé majetku.

	\textbf{Reprodukci členíme}:
	\begin{itemize}
		\item Reprodukce prostá - stroj nahradíme strojem stejného výkonu
		\item Reprodukce rozšířená - stroj nahradíme jedním strojem s vyšším výkonem nebo více stroji se stejným výkonem
		\item Reprodukce zůžená - stroj nahradíme strojem s menším výkonem nebo nenahradíme vůbec
	\end{itemize}


	\chapter{Zásobování a logistika}

	Zásobování je činnost podniku, při niž si podnik zajišťuje potřebné suroviny a materiál V požadovaném množství, kvalitě, druzích ve stanovené době a ve výhodných cenách. Tyto suroviny a materiál používá pro svou činnost.

	Oběžný majetek - členění:
	\begin{enumerate}
		\item ZÁSOBY
			\begin{itemize}
				\item Materiál
				\item Nedokončená výroba (neupečený rohlík)
				\item Polotovary (deska dřeva)
				\item Hotové výrobky (výrobky, které už firma dokončila)
				\item Zboží (vše, co je nakoupené za účelem dalšího prodeje)
				\item Zvířata
			\end{itemize}
		\item PENÍZE
			\begin{itemize}
				\item Peníze v hotovosti v pokladně
				\item Peníze na účtech peněžních ústavů
				\item Ceniny - kolky, stravenky, poukázky, dálniční známky, dopisní známky,mobilní karty
				\item Krátkodobé cenné papíry - směnky, depozitní certifikáty, vkladový list
				\item Pohledávky - peníze, které firmě dluží odběratelé, společníci, zaměstnanci, dlužníci
			\end{itemize}		
	\end{enumerate}

	Členění materiálu (základní suroviny - stavební hmoty, kov, dřevo, kůže\ldots)
	\begin{itemize}
		\item Pomocné materiály - barvy, mořidla, maziva\ldots
		\item Obaly - plechovky, kartóny, plasty\ldots
		\item Pohonné hmoty - nafta, benzin\ldots
		\item Drobné nářadí - šroubováky, klíče, vrtáky, přípravky\ldots
		\item Kancelářské potřeby - papíry, tužky, pásky do psacího stroje, šanony\ldots
		\item Čistící prostředky - pro hygienu zaměstnanců, úklid prostor\ldots
	\end{itemize}

	Obaly a jejich funkce:
	\begin{itemize}
		\item plechovky, kartóny, plasty, palety, kontejnery, láhve\ldots)
		\item slouží k ochraně a dopravě nakoupeného materiálu, zboží a výrobků
	\end{itemize}

	Koloběh oběžného majetku:
	[Obr. 1 Koloběh oběžného majetku firmy]

	Platí, že peníze na začátku koloběhu by měly být menší než na konci = zisk firmy.

	\paragraph*{Evidence a doklady při zásobování}
	\begin{enumerate}
		\item \textbf{Dodací list}
			\begin{itemize}
				\item vystavuje dodavatel pro kontrolu, co za zboží posílá
				\item fyzicky musí jít s dodávkou, aby odběratel mohl provést přejímku zboží
			\end{itemize}
		\item \textbf{Faktura - daňový doklad} - je dokladem, který slouží pro
			\begin{itemize}
				\item Zanesení do účetnictví
				\item Pro účely zůčtování daně z přidané hodnoty
				\item Vznik a uhrazení závazku
			\end{itemize}
			Fakturu smí dodavatel vystavit nejdřív v okamžiku zaplacení nebo v okamžiku zaplacení nebo v okamžiku předání zboží odběrateli nebo prvnímu veřejnému přepravci (České dráhy, kamiónová doprava, lodní doprava, letecká doprava\ldots)
		\item \textbf{Příjemka}
			\begin{itemize}
				\item je doklad vystavený ve skladu odběratele a spolu se skladní kartou a výdejkou se vztahuje ke skladovému hospodářství
				\item příjemka slouží jednorázově pro jedno přijetí
					\begin{itemize}
						\item Přejímka je proces kontroly a přijímání zboží na sklad
						\item Příjemka je doklad evidující přijaté zboží
					\end{itemize}
			\end{itemize}
		\item \textbf{Skladní karta}
			\begin{itemize}
				\item je doklad vystavený ve skladu, slouží pro evidenci pohybu zásoby určitého druhu v čase
			\end{itemize}
		\item \textbf{Výdejka}
			\begin{itemize}
				\item slouží k výdeji ze skladu do výroby
				\item jednorázový multidruhový doklad
				\item současně se při výdeji provede odepsání vydaného množství ze skladní karty
			\end{itemize}
		\item \textbf{Kniha došlých faktur}
			\begin{itemize}
				\item Je evidence sloužící v účtárně k přehledu o vzniklých závazcích firmy a o datu a způsobu uhrazování těchto faktur
				\item kniha je velmi důležitá při běžné práci i inventurách
			\end{itemize}
		\item \textbf{Příkaz k úhradě 1 výdajový pokladní doklad}
	\end{enumerate}

	\begin{description}
		\item[Skladování] Skladování je činnost, při níž se hmotné statky soustřeďují na určitém místě a ve stanoveném množství a připravují se pro další činnosti: výdej do spotřeby\ldots
		\item[Metoda JUST-IN-TIME] Materiál je přivážen v přesném čase přímo k výrobní lince (vůbec není skladován ve skladu zásobování). Zcela odpadají náklady na skladování.
		\item[Řízení zásob - metoda ABC]
			\begin{description}
				\item[Skupina A] - metoda normování zásob. Sem zařadíme především základní suroviny, které nezbytně firma potřebuje pro svou výrobu
				\item[Skupina B] - sem patří zásoby, které se relativně snadno a rychle objednávají a jejich spotřeba už pro firmu není tak nákladově významná
				- stanovit a hlídat minimální skladový limit
				\item[Skupina C] - tato skupina je počtem druhů zásob největší, ale objemem spotřeby ve finančním vyjádření je pro firmu nejméně významná
			\end{description}
		\item[Normování zásob]
			\begin{itemize}
				\item Časová norma zásob - udává čas, jak dlouho vydrží průměrná zásoba
				\item Normovaná zásoba v naturálních jednotkách - udává fyzický objem zásoby
				\item Normovaná zásoba ve finančním vyjádření - udává objem peněz v zásobách
			\end{itemize}
		\item[Plán zásobování formou bilance]
			\begin{equation*}
				\sum_{}^{} P = \sum_{}^{} Z
			\end{equation*}
			\begin{description}
				\item[$P$] Potřeby (co bychom potřebovali)
				\item[$Z$] Zdroje (vyrobime, nakoupíme hotové)			
			\end{description}
	\end{description}

	[Obr. SCHÉMA NORMOVÁNÍ ZÁSOB]

	\begin{description}
		\item[Zásoba běžná] zásoba, ze které se průběžně vydává podle požadavků výroby
		\item[Zásoba pojistná] zásoba pro případ, kdy se dodavatel opozdí s dodávkou
		\item[Zásoba technická] bývá pouze u některých zásob technickou zásobu nejsme schopni předčasně čerpat, protože tato zásoba ještě není technologicky připravená (dosušení zásoby - dřevo)
		\item[Dodávkový cyklus] čas mezi dvěma smluvními dodávkami od dodavatele
	\end{description}

	\section*{Právní stránka obchodních vztahů}
	\subsection*{Kupní smlouva}
	Kupní smlouvou se prodávající zavazuje dodat kupujícímu movitou věc (zboží) a převést na něho vlastnické právo k této věci a kupující se zavazuje zaplatit kupní cenu. K platnému vzniku smlouvy stačí dohoda o podstatných náležitostech smlouvy:
	\begin{itemize}
		\item Určení smluvních stran (prodávající a kupující)
		\item Určení předmětu
		\item Určení kupní ceny
	\end{itemize}

	\paragraph*{Forma uzavření smlouvy}
	Kupní smlouvu k movitým věcem lze uzavřít písemně, ústně nebo konkludentním jednáním.
	
	Povinnosti prodávajícího:
	\begin{itemize}
		\item Dodat řádně zboží
		\item Předat potřebné dokumenty
		\item Umožnit kupujícímu nabytí vlastnického práva
		\item Uchovávat zboží v případě, je-li kupující v prodlení s převzetím	
	\end{itemize}

	\paragraph*{Povinnosti kupujícího}
	\begin{itemize}
		\item Zaplatit kupní cenu
		\item Převzít a prohlédnout dodané zboží
	\end{itemize}
	Vlastnické právo přechází na kupujícího převzetím, není-li dohodnuto jinak.

	\subsection*{Smlouva o dílo}
	Smlouvou o dílo se zhotovitel zavazuje provést na svůj náklad a nebezpečí pro objednatele dílo a objednavatel se zavazuje dílo převzít a zaplatit cenu.

	Předmětem smlouvy je dílo:
	\begin{itemize}
		\item Zhotovení určité movité věci
		\item Oprava, údržba nebo úprava určité movité věci
		\item Stavba, její zhotovení, údržba, oprava nebo úprava
		\item Samostatnou úpravu má zhotovení díla s nehmotným výsledkem
	\end{itemize}

	\textbf{Forma uzavření smlouvy - písemná i ústní}

	Podstatné náležitosti:
	\begin{itemize}
		\item Určení stran - objednatel a zhotovitel
		\item Předmět smlouvy (popis díla)
		\item Cena nebo způsob jejího určení	
	\end{itemize}

	\subsection*{Reklamace}
	Pro uplatnění práv z odpovědnosti za vady musí poškozená strana oznámit vady.
	Reklamaci je dobré provádět písemně, např. formou dopisu.


	\chapter{Marketing}

	Marketing je nauka o trhu, podnikatelská koncepce. Marketing je proces řízení, jehož výsledkem je poznání, předvídání, ovlivňování a v konečné fázi uspokojení střeb a přání zákazníka efektivním a výhodným způsobem, zajišťujícím splnění cílů organizace.

	\paragraph*{Historie marketingu-fáze vývoje}
	\begin{description}
		\item[Výrobní koncepce] (co nejlevnější výrobek)
			\begin{itemize}
				\item představiteli této koncepce byli Henry Ford v USA a Tomáš Baťa u nás
			\end{itemize}
		\item[Výrobková koncepce] (kvalitní výrobek)
			\begin{itemize}
				\item výroba je menší, světová krize
				\item důležitá vysoká kvalita, velký důraz se klade na technický rozvoj a inovaci výrobků
			\end{itemize}
		\item[Prodejní koncepce] (přeceňování úlohy reklamy)
			\begin{itemize}
				\item po 2.světové válce, výrobek se lépe prodává s reklamou-tisk, tv, rozhlas
			\end{itemize}
		\item[Marketingová koncepce] (poznej potřeby svého zákazníka a teprve pak vyráběj)
			\begin{itemize}
				\item státy zničené válkou po obnově hospodářství
				\item hospodářství roste, trh je nasycený
				\item poznejme naše potencionální zákazníky, jejich potřeby, a teprve vyrábějme
			\end{itemize}
		\item[Sociální marketing] (zohlední nejen potřeby zákazníka, ale i celé společnosti)
			\begin{itemize}
				\item nejnovější vývojový stupeň marketingu
				\item bere se ohled na společenské zájmy
			\end{itemize}
	\end{description}

	\paragraph*{Tomáš Baťa}
		\begin{itemize}
			\item {Československý podnikatel}
			\item Vytvořil ve Zlíně obuvnickou firmu Baťa, postupně rozsáhlý komplex výroby, obchodu, dopravy, služeb a financí
			\item Jeden z největších podnikatelů své doby
		\end{itemize}

	\paragraph*{Prodej $\times$ Marketing}
	Prodej musíme chápat jako jednu z činností podniku. Prodej je podmnožinou ve velké množině marketingových činností.
	
	Marketing není samostatná podniková funkce, musí prolínat všemi činnostmi podniku.

	\paragraph*{Informační systém marketingu}
		Základní zdroje informací:
		\begin{itemize}
			\item Vnitřní zdroje-informace, které má firma sama k dispozici. Sem patří informace z podnikového účetnictví a statistické evidence, z ekonomických rozborů.
			\item Vnější zdroje-informace, které nejsou firmě běžně dostupné a zjistitelné. Je třeba informace získat z vnějších zdrojů (statistické přehledy, články v odbor.časopisech, odbor. konference nebo prostřednictvím \textbf{marketingového výzkumu}.
		\end{itemize}

	K metodám marketingového výzkumu patří:
	\begin{itemize}
		\item Pozorování (např.známé počítání aut u cest pro zjištění průjezdnosti)
		\item Experiment (zjišťování reakce zákazníků na změnu ceny, reklamy, obalu\ldots)
		\item Průzkum trhu (např.formou dotazníků v prodejnách,pohovorů se zákazníky)
	\end{itemize}

	Informace můžeme členit také:
	\begin{itemize}
		\item Primární - info byly získány za účelem marketingového využití
		\item Sekundární - info původně sloužily jinému účelu (uce,statistika), ale jsou využitelné 1 pro marketing
	\end{itemize}

	Informace jiných věd:
	Psychologie (chování kupujících), sociologie (chování skupin lidí), demografie (věkové rozložení).

	Tři základní subjekty, o kterých jsou zjišťovány informace:
	O zákaznících, o konkurenci, o vlastní firmě.
	
	\textbf{Segmentace trhu}:
	Souvisí s cíleným marketingem a znamená rozčlenění trhu na specifické segmenty zákazníků, na které se firma zaměří svým marketingovým mixem.

	\textbf{Základní marketingové strategie}
	\begin{enumerate}
		\item hromadný marketing - univerzální výrobek nabízíme všem zákazníkům (sůl)
		\item diferencovaný marketing - výrobek v několika obměnách nabízíme všem zákazníkům (cukrovinky, drogerie, auta)
		\item cílený marketing - výrobce nabízí konkrétnímu segmentu zákazníků specifický výrobek (automobil)
	\end{enumerate}

	Postup při aplikaci cíleného marketingu:
	\begin{itemize}
		\item Rozčleníme trh na tzv. \textbf{tržní segmenty}
		\item Zaměříme se na jeden nebo více z nich (tzv. tržní zacílení - segment rodinných automobilů)
		\item Hledáme a určujeme konkrétní nástroje a prostředky pro získání potencionálních zákazníků (tzv. tržní umístění)
	\end{itemize}

	Hlediska segmentace trhů spotřebních:
	\begin{itemize}
		\item Geografická (územní)
		\item Demografická (věk, vzdálenost, pohlaví, povolání, národnost)
		\item Psychografická (podle sociální třídy, osobnosti)
		\item Podle chování (věrnost značce)
	\end{itemize}

	Hlediska segmentace trhů průmyslových:
	\begin{itemize}
		\item Geografická (velikost firmy)
		\item Kritéria provozu (technologie, potřeby služeb a servisu)
		\item Nákupní kritéria (firmy, které chtějí nízkou cenu, kvalitu, servis)
	\end{itemize}

	\textbf{Výhody segmentace}
	\begin{itemize}
		\item Lepší uspokojení konkrétních potřeb zákazníka
		\item Efektivnější stimulace a distribuce výrobku nebo služby
		\item Získání konkurenční převahy ve vybraném segmentu trhu (mýdlo na hlavu)
	\end{itemize}

	\textbf{Charakteristiky určitého segmentu}
	\begin{itemize}
		\item Velikost a síla segmentu (kolik segment obsahuje zákazníků a jaká je jejich kupní síla)
		\item Vývojový trend segmentu
		\item Síla konkurence v daném segmentu, a to současná i potencionální, substituující výrobky (jiné výrobky uspokojí stejnou potřebu zákazníka, maso vepřové mohou nahradit hovězím)
	\end{itemize}

	\paragraph*{Potrfoliová matice růst - podíl $\times$ analýza BCG}
	
	Bostonská matice pochází od poradenské firmy Boston Consulting Group (BCG), odtud také její název \emph{BCG matice} nebo \emph{Bostonská matice}. Používá se pro hodnocení porfolia produktů podniku.

	[Obr. Bostonská matice]

	Kvadranty portfoliové matice
	\begin{description}
		\item[Otazníky] (Question marks) jsou takové SBU\footnote{strategická obchodní jednotka (Strategic Bussines Unit)} firmy, které se uskutečňují na trzích s vysokým tempem růstu, ale mají na trzích malý relativní tržní podíl. (elektro auta-TESLA)
		\item[Hvězdy] (Stars) pokud je SBU-otazník úspěšný stává se hvězdou a má vysoký podíl na trhu (Red Bull)
		\item[Peněžní krávy] (Cash cows) tyto SBU už nemají tak velké tempo růstu na trhu, proto už nemají, tak vysoké vklady a naopak samy jsou nejdůležitější zdroj příjmů pro firmu (Škoda-Octavia)
		\item[Bídní psi] (Dogs) jsou takové SBU, které mají slabý tržní podíl a nízké tempo růstu. Tyto SBU znamenají nízké zisky nebo dokonce ztrátu (Microsoft Lumia)
	\end{description}

	\paragraph*{SWOT analýzy}
	\begin{itemize}
		\item analýza vnitřního prostředí obchodní jednotky (silné a slabé stránky) a analýza okolí (příležitosti a ohrožení)
		\item při této analýze firma kriticky hodnotí sama sebe v porovnání s konkurencí
		\item musí zohledňovat 1 další vlivy o okolí í např. politický vývoj, (egislativní vývoj..
	\end{itemize}

	[Obr. SWOT analýzy]

	\paragraph*{Marketingový Mix (4P)}
	MM je jedním ze základních nástrojů marketingu. Pomáhá nám rozčlenit, na co se zaměřit při tvorbě marketingového plánu.

	MM se skládá z následujících prvků:
	\begin{itemize}
		\item Výrobek (product)
		\item Cena (price)
		\item Propagace (promotion)
		\item Distribuce (placement)
	\end{itemize}

	\begin{description}
		\item[P1 - VÝROBEK] Výrobek je statek či služba, který se stává předmětem směny na trhu a je určen k uspokojení potřeb zákazníka. Marketing hovoří o \uv{komplexním výrobku}.
		\item[P2 - CENA] Je jedním z nejdůležitějších nástrojů MM a její správná volba je velmi náročná.
		\item[P3 - PROPAGACE] Propagace je forma komunikace mezi prodávajícím a kupujícím, jejímž cílem je větší prodej výrobku nebo služby. (podpora prodeje, reklama, personál\ldots)
		\item[P4 - DISTRIBUCE] Distribuční cesty - prodejní cena = souhrn všeho co zajistí tok zboží od výrobce k zákazníkovi (MO, VO\ldots)
	\end{description}

	[Obr. Komplexní výrobek se skládá z]

	\paragraph*{Životní cyklus výrobku-fáze}
	\begin{enumerate}
		\item Uvedení na trh-v této fázi firma do výrobku a jeho prosazení na trhu hodně investuje a je ztrátová
		\item Růst-objem prodeje se zvětšuje a firma začíná dosahovat zisky
		\item Zralost-prodej a zisk již tolik nerostou, dosahují vrcholu a začínají stagnovat pro firmu je výhodné, aby tato fáze trvala, co nejdéle
		\item Pokles-zájem zákazníků o výrobek slábne a firma, chce-li přežít, musí přijít s výrobkem inovovaným
	\end{enumerate}

	\paragraph*{Průběh S-křivky}
	Je samozřejmě u různých výrobků odlišný a na její tvar má vliv celá řada faktorů, ovšem zákonité snížení zájmu zákazníků jednou určitě přijde a fáze poklesu je neodvratná.

	\paragraph*{Značka výrobku}
	Značka je nedílnou součástí komplexního výrobku. Značka odlišuje výrobek od ostatních obdobných výrobků na trhu. Značka by měla být registrována formou ochranné známky, aby byla chráněna před zneužitím.

	Typy značek:
	\begin{enumerate}
		\item značka výrobce (adidas, Coca Cola)
		\item značka obchodu (Ahold má značku Albert)
	\end{enumerate}

	\chapter{Výroba, jakost, inovace}

	Výroba je základní fází hospodářského procesu (výroba, rozdělování, směna, spotřeba). Výroba je hodnototvorný proces-při výrobě jsou vytvářeny za spoluúčasti všech výrobních faktorů (práce, přírodní zdroje, kapitál) = statky a služby, které mají uspokojit určité konkrétní lidské potřeby. Při výrobě přeměňujeme přírodní zdroje v hotové výrobky, pomocí práce a kapitálu.

	\paragraph*{Příprava výroby}
	Sem zařadíme celý výzkum a vývoj nového výrobku až po detailní zpracování technologického postupu pro výrobu. Proces přípravy nového výrobku nazveme inovací a záleží na míře změny nového výrobku oproti původnímu výrobku.

	Úkolem přípravy výroby je tedy vymyslet:
	\begin{itemize}
		\item Konstrukci samotného výrobku, včetně použitých materiálů
		\item Technologi výroby, včetně pracnosti
	\end{itemize}

	\paragraph*{Plán výroby a bilance}
	Podnikatel řeší především organizaci výroby - organizaci pracovišť, práce, kapacity zásobovací a skladové, výše finančních zdrojů včetně možnosti vnějšího financování.

	Plán výroby v podobě bilance:
	\begin{displaymath}
		\sum_{}^{} \text{Potřeba výroby} = \sum_{}^{} \text{Výrobní zdroje}
	\end{displaymath}

	\textbf{Vybilancovat plán výroby} je velmi složité. V plánu výroby potřebujeme sladit velmi mnoho a často protichůdných veličin.

	\textbf{Kapacita} je schopnost firmy vyrobit při optimálních podmínkách určitý počet výrobků v určitém čase. Využitelný časový fond = 365 dnů - svátky - so,ne = čas, kdy je stroj v provozu.
	
	Při nedostatečné kapacitě strojů může firma:
	\begin{itemize}
		\item Dočasně si potřebné stroje pronajmout (operativní leasing)
		\item Dohodnout výrobní kooperací s jinou firmou (často 1 konkurencí)
		\item Při prognóze dlouhodobého rozvoje prodeje určitého výrobku je možné připravit investiční výstavbu a pořídit nový DM do vlastnictví firmy, to je ovšem časově 1 finančně nejnáročnější varianta
	\end{itemize}

	Při nadbytečných strojních kapacitách může firma:
	\begin{itemize}
		\item Dočasně stroje pronajmout
		\item Dohodnout vytížení strojů výrobním programem pro jinou firmu
		\item Stroje odprodat
	\end{itemize}

	Stupně rozpracovanosti výroby:
	\begin{itemize}	
		\item Materiál
			\begin{itemize}
				\item zásoby nakoupené pro výrobu (suroviny, barvy, mazadla\ldots)
			\end{itemize}
		\item Nedokončená výroba
			\begin{itemize}
				\item materiál v různých fázích rozpracovanosti ,neprodej. (roztavené železo ve slévárně\ldots)
			\end{itemize}
		\item Polotovary
			\begin{itemize}
				\item produkt, který se bude ještě ve výrobě dále zpracovávat
				\item produkt už můžeme prodat jiné firmě (odlitek)
			\end{itemize}
		\item Hotové výrobky
			\begin{itemize}
				\item jsou výrobky vlastní produkce určené k prodeji
				\item jsou dokončeny všechny operace a prošly OTK-odd.t.kont. (soustruh,stojanová vrtačka\ldots) -- kontroluje jakost
			\end{itemize}
	\end{itemize}

	\paragraph*{Členění výrobního procesu}
	Skládá z jednotlivých operací, které na sebe navazují:
	\begin{enumerate}
		\item podle stupně mechanizace
			\begin{itemize}
				\item  Ruční výroba -- práci vykonává člověk
				\item  Mechanizovaná výroba -- práci vykonává stroj, řídí člověk
				\item  Automatizovaná výroba -- práci vykonává pouze stroj
			\end{itemize}
		\item podle počtu vyráběných výrobků jednoho druhu
			\begin{itemize}
				\item Kusová výroba -- jeden nebo málo kusů určitého druhu
				\item Sériová výroba -- větší množství výrobků jednoho druhu a méně druhů
				\item Hromadná výroba -- velké množství jednoho druhu výrobku typická pro spotřební materiál
			\end{itemize}
	\end{enumerate}

	\paragraph*{Průběh výroby}
	\textbf{Úsečkový diagram}, kde každá úsečka představuje dobu výroby a montáže jednotlivých součástí do vyšších celků až nám vznikne hotový výrobek.
	\textbf{Průběžnou dobu výroby jednoho výrobku} či jedné dávky výrobků, tj. celkovou dobu, kterou musíme počítat na výrobu tohoto výrobku. Potřebujeme vědět časy potřebné na zhotovení a montáž každé součástky -- k tomu nám dopomůže \textbf{stanovení výrobního postupu pro každou součástku}, ve kterém budou předepsány všechny výrobní operace -- část výrobního procesu. Pro jednotlivou operaci \textbf{stanovíme normu spotřeby času}. NSČ slouží pro účely plánování výroby a dále pro účely odměňování dělníků.

	\textbf{Henry Ford} -- přinesl na začátku 20. století normování práce dělníků - je potřeba k normování spotřeby času.

	\textbf{Normovaný čas se skládá z}:
	\begin{itemize}
		\item Času práce (slouží pro účely plánování a odměnu)
		\item Času nutných přestávek (dělník potřebuje obnovovat svou pracovní sílu)
	\end{itemize}

	Ostatní čas, který není pro práci nutný, jsou ztráty a do normy se nezahrnuje.

	\paragraph*{Mzda za operaci}
	Mzda je přímo úměrná délce vynaloženého normovaného času a kvalifikaci dělníka, který tuto práci vykonává.

	\paragraph*{Ergonomie}
	Ergonomie je nauka o zákonitostech vztahů mezi člověkem, strojem a pracovním prostředím. Informace, jak optimálně zatížit člověka při práci, aby pracoval v pohodě a zároveň přinesl maximální výkon, jak optimálně uspořádat pracoviště, aby měl pracovník vše \uv{po ruce}. Informace, které pomáhají firmě dosahovat vyšších výkonů, a to nikoliv na úkor zaměstnance. Tj. aby člověk pracoval v pohodě.

	[Obr. Základní toky ve výrobě]

	\textbf*{Při manipulaci s materiálem rozlišujeme}:
	\begin{itemize}
		\item Ruční manipulaci (hmotnostní limity)
		\item Manipulace pomocí jeřábů
		\item Manipulace pomocí dopravníků (dopravních pásů)
		\item Manipulace pomocí dopravních a zvedacích vozíků-ručních, elektrických\ldots
		\item Speciální zakládací a manipulační systémy
		\item Potrubní doprava-špinavé prádlo trubkou až do prádelny
	\end{itemize}

	\textbf{Obalové hospodářství}
	\begin{itemize}
		\item Obaly na jedno použití (kartóny plechovky atd.)
		\item Obaly kolovací (palety kontejnery láhve atd.)
	\end{itemize}

	\textbf{Odpadové hospodářství}
	\begin{itemize}
		\item Třídí odpady do skupin dle škodlivosti
		\item Rozpracovává na naše podmínky směrnici EU o obalech a obal. odpadech
		\item Ukládá povinnost vést evidenci odpadů a ohlašovat je
		\item Vybízí k recyklaci a energetickému využívání odpadů
		\item Stanovuje podmínky pro zneškodňování odpadů, dopravu, dovoz, vývoz a tranzit odpadů
	\end{itemize}

	\paragraph*{Ekologie}
	Člověk by měl vrátit přírodě vše, co si od ní vzal-třídění odpadu.

	\paragraph*{Jakost výroby}
	V naší republice se přiklání podniková praxe spíše k systému norem jakosti, tedy dobře hodnotitelných parametrů výroby a výsledného produktu. Firmy se snaží získat mezinárodně uznávaný certifikát, že splňují firmy ISO řady. \par Naše firmy mají zájmem do EU vyvážet svou produkci. Získáním certifikátu firmy:
	\begin{itemize}
		\item Mají písemný doklad o jakosti své produkce, což jim pomáhá při uzavírání obchodních smluv
		\item Výhodou je vybudovaný dobře kontrolovatelný a hodnotitelný systém řízení jakostiuvnitř firmy
	\end{itemize}

	Jakost kontroluje ŘKJ-řízení a kontrola jakosti.

	\paragraph*{Bezpečnost práce}
	Od roku 1993 jsou zaměstnavatelé povinni ze zákona (zákoník práce) platit zákonné pojištění pracovních úrazů a nemocí z povolání svých zaměstnanců u jedné z komerčních pojišťoven -- České pojišťovny nebo Kooperativy. \\ Bezpečnost práce je u nás řešena řadou zákonů a vyhlášek.

	\chapter{Personalistika}

	Výrobní faktory nezbytné k zajištění výroby -- práce, přírodní zdroje a kapitál.

	Personalistika je soubor činností podniku, jejichž cílem je zajistit, rozmístit a udržet pro podnik kvalitní zaměstnance - práce s lidmi.

	Zákoník práce -- nejdůležitější ustanovení: \textbf{Mezi zaměstnanci a zaměstnavateli vznikají pracovněprávní vztahy, které se řídí zákoníkem práce}

	Závislá práce -- není samostatná činnost, není podnikatel.

	\paragraph*{Parcovní poměr}
	\begin{itemize}
		\item způsobilost FO vstupovat do pracovněprávních vztahů jako zaměstnanec vzniká dnem dosažení 15 let
		\item každý zaměstnanec má v průběhu pracovního poměru obecnou zodpovědnost vůči zaměstnavateli za jím způsobené škody
	\end{itemize}

	\paragraph*{Pracovněprávní vztah}
	Může mít podobu:
	\begin{enumerate}
		\item dohody o práci konané mimo pracovní poměr:
			\begin{enumerate}
				\item Dohoda o provedení práce
					\begin{itemize} 
						\item Může být uzavřena na max. 300 hodin ročně u jednoho zaměstnavatele
						\item Musí být uzavřena písemně a musí obsahovat dobu,na jakou je uzavřena
						\item Vhodná především pro krátkodobé brigády
						\item U této formy nevzniká povinnost platit SaZ pojištění, pokud mzda nepřekročí 10000 Kč měsíčně
						\item Je povinnost uhradit daň z příjmu
					\end{itemize}
				\item Dohoda o pracovní činnosti
					\begin{itemize}
						\item Sjednává se na práci, která nesmí svým rozsahem převýšit v průměru, polovinu stanovené týdenní pracovní doby
						\item Musí být uzavřena písemně
						\item Vzniká povinnost uhradit SaZ pojištění, daň z příjmu
					\end{itemize}
			\end{enumerate}
		\item pracovní poměr
			\begin{enumerate}
				\item Vznik pracovního poměru:
					\begin{itemize}
						\item volbou -- zvolení poslanci, zvolení řídící pracovníci družstev
						\item jmenováním -- vedoucí pracovníci
						\item uzavřením pracovní smlouvy -- tato forma je nejběžnější
					\end{itemize}
				\item Změny pracovního poměru:
					\begin{itemize}
						\item převedení na jinou práci -- zdravotní potíže, těhotenství
						\item přeložení na jiné místo výkonu práce -- se souhlasem zaměstnance
					\end{itemize}
				\item Skončení pracovního poměru:
					\begin{itemize}
						\item dohodou -- je nutný souhlas obou stran, není stanovena výpověd.lhůta
						\item zrušením ve zkušební době -- nemusí uvádět důvody, je třeba písemně oznámit 3 dny předem
						\item okamžité zrušení:
							\begin{itemize}
								\item Zaměstnavatel:
									\begin{itemize}
										\item pokud zaměstnanec porušil pracovní kázeň
										\item odsouzen pro trestný čin k nepodmíněnému trestu odnětí svobody na dobu delší než 1 rok
									\end{itemize}
								\item Zaměstnanec:
									\begin{itemize}
										\item podle lékařského posudku nemůže déle konat práci bez vážného ohrožení svého zdraví
										\item zaměstnavatel nevyplatil mzdu do 15 dnů po uplynutí splatnosti
									\end{itemize}
							\end{itemize}
						\item výpověď -- je jednostranný právní akt, musí být dána písemně
							\begin{itemize}
								\item Výpověď ze strany zaměstnance -- výpovědní doba je 2 měsíce
								\item Výpověď ze strany zaměstnavatele
									\begin{itemize}
										\item stane-li se zaměstnanec nadbytečný má nárok na 2 měsíce výpovědní doby a odstupné ve výši jednoho měsíčního platu (0-1 rok), dvou platů (1-2 roky), tří platů (2-? roky)
										\item smrtí
									\end{itemize}
							\end{itemize}
					\end{itemize}
			\end{enumerate}
	\end{enumerate}

	Zaměstnavatel nesmí dát výpověď zaměstnanci pracovně neschopnému na nemocenské, povolanému k výkonu vojenské služby, uvolněnému pro výkon veřejné funkce, těhotné 	zaměstnankyni nebo trvale pečující alespoň o jedno dítě mladší než tři roky.

	\section*{Pracovní smlouva}
	\begin{itemize}
		\item Pracovní náležitosti pracovní smlouvy
			\begin{itemize}
				\item druh práce
				\item místo výkonu práce
				\item den nástupu do práce
				\item účastníci smlouvy
			\end{itemize}
		\item V pracovní smlouvě mohou být i další ujednání, nejsou však povinná
			\begin{itemize}
				\item zda se jedná o prac. poměr na dobu určitou či neurčitou, není-li udáno, jedná se o smlouvu na dobu neurčitou
				\item zkušební doba -- není-li zkušení doba ve smlouvě písemně sjednána, neplatí
					\begin{itemize}
						\item maximální zkušební doba je 3 měsíce
						\item nelze ji sjednat v případě pracovního poměru na dobu neurčitou
					\end{itemize}			
			\end{itemize}
	\end{itemize}

	Zaměstnavatel je povinen uzavřít pracovní smlouvu písemně a jedno vyhotovení smlouvy vydat zaměstnanci.

	\paragraph*{Povinnosti zaměstnavatele}
	\begin{itemize}
		\item přidělovat zaměstnanci práci podle sjednané smlouvy
		\item platit mu mzdu
		\item vytvářet podmínky pro plnění jeho pracovních úkolů
	\end{itemize}

	\paragraph*{Práva zaměstnavatele}
	\begin{itemize}
		\item pracovní řád, odebrat vzorek při alkoholu a drogách	
	\end{itemize}

	\paragraph*{Povinnosti zaměstnance}
	\begin{itemize}
		\item konat práci podle pokynů zaměstnavatele
		\item konat práci osobně a ve stanovené pracovní době
		\item dodržovat pracovní kázeň
	\end{itemize}

	\paragraph*{Práva zaměstnance}
	\begin{itemize}
		\item mzda, přestávky, ochranné pomůcky	
	\end{itemize}

	\paragraph*{Pracovní doba a doba odpočinku}
	Pracovní doba je nejvýše 40 hodin týdně, u mladistvých do 18 let maximálně 30 hodin týdně.

	\section*{Mzda}

	Zaměstnavatelé samostatně rozhodují o uplatnění formy základní mzdy:
	\begin{itemize}
		\item mzda úkolová -- mzda závisí na množství kvalitní práce
		\item mzda časová -- zaměstannec je odměňován podle času práce
		\item mzda podílová -- určitý podíl na dosažených výsledcích (obchodník)
	\end{itemize}

	\paragraph*{Zaručená mzda}
	Stanovuje nařízení vlády, a to tak že při týdenní pracovní době (40 hodin) stanoví 8 tarifních skupin podle namáhavosti, složitosti a odpovědnosti stanoví minimální mzdu. (12200-24400 Kč)

	Mzda nesmí být nižší než minimální mzda stanovená vládním nařízením.

	Hrubá minimální mzda platná pro rok 2018 je 12 200 Kč. Srovnáme-li s průměrnou nominální mzdou 29 504 Kč, zjistíme, že tato minimální mzda je velmi nízká.

	Stanovení minimální mzdy má mnoho souvislostí a dopadů:
	\begin{itemize}
		\item minimální mzda je nad hranicí životního minima -- 3410 Kč
		\item příspěvek na bydlení a existenční minimum (osoby bez trvalého bydliště)
	\end{itemize}

	\emph{Valorizace mezd} znamená zvyšování mezd při znehodnocení peněz (inflaci).

	\section*{Bezpečnost a ochrana zdraví při práci}

	Zaměstnavatelé jsou v rozsahu své působivosti povinni vytvářet podmínky pro bezpečnou a zdraví neohrožující práci v souladu s předpisy o bezpečnosti práce a bezpečnosti technických zařízení. Zaměstnavatelé mají povinnost odškodňovat pracovní úrazy a nemoci z povolání 	zaměstnanců.

	Každý rok musí být zaměstnanci proškolení.

	Zaměstnavatelé jsou ze zákona povinni platit zákonné úrazové pojištění zaměstnanců.

	\paragraph*{Zákoník práce}
	\begin{description}
		\item[Dovolená] nárok na 4 týdny (2 týdny vkuse), placené volno, někdo i 5 týdnů
		\item[Odbory] sdružení zaměstnanců, založené s cílem prosazovat jejich pracovní, hospodářské, politické a sociální zájmy
		\item[Stávka] je forma kolektivního protestu zaměstnanců
		\item[Výluka] je zamezení práce zaměstnavatelem (opak stávky)
		\item[Kolektivní smlouva] smlouva mezi odbory a zaměstnavatelem o mzdách a dalších ujednáních, týkajících se zaměstnanců
		\item[Odměňování zaměstnanců]
			\begin{itemize}
				\item Přímé -- plat,mzda
				\item Nepřímé -- zaměstnanecké výhody (podnikové půjčky, stravenky, příspěvek na dovolenou)
				\item Nefinanční -- pochvala, stáž, studijní dovolená				
			\end{itemize}
		\item[Náhrady mzdy]
			\begin{itemize}
				\item zákonem -- za dovolenou nebo svátek
				\item překážky v práci-svědectví u soudu, výpadek dodávky energie
			\end{itemize}
	\end{description}

	\section*{Složky mzdy}
	\begin{itemize}			
		\item Osobní ohodnocení -- vyjadřuje kvalitu práce zaměstnance
		\item Příplatky -- v noci(10\%), víkendy (10\%), ztížené prac. prostředí (10\% z min. mzdy), přesčas (25\% hod. mzdy), ve svátek
		\item Prémie a odměny -- vyplácejí se za určité výsledky práce
	\end{itemize}

	\paragraph*{Význam úřadu práce}
	Státní úřad, poskytuje informace z oblasti pracovního trhu ČR a EU, eviduje uchazeče o zaměstnání a volná pracovní místa. \\
	Pobočky v různých místech ČR. Nabízí rekvalifikační kurzy.

	\section*{Zdravotní a Sociální pojištění}
	\paragraph*{Zdravotní pojištění}
	Toto pojištění spravují zdravotní pojišťovny (největší VZP) a hradí z něj lékařům jejich práci a léky. Za děti, studenty, ženy na mateřské, vojáky, registrované nezaměstnané a důchodce platí toto pojištění stát ze státního rozpočtu.

	\paragraph*{Sociální pojištění}
	Tyto příjmy spravuje státní instituce správa sociálního zabezpečení
	\begin{itemize}
		\item Nemocenské pojištění -- hrazeny nemocenské dávky
		\item Důchodové pojištění -- jsou z něj vypláceny důchody (starobní, ,vdovské,sirotčí)
		\item Příspěvek na státní politiku zaměstnanosti -- z něj se vyplácí podpory v nezaměstnanosti (také rekvalifikace)
	\end{itemize}

	Zaměstnanci odvádějí ze své hrubé mzdy 11\% pro účely SaZP a zaměstnavatelé povinně odvádí navíc 34\% z této hrubé mzdy zaměstnance (superhrubá mzda).

	Podnikatelé vypočítávají své SaZP tak, že ze svého zisku (výnosy-náklady) vypočítají 50\% základ, ze kterého odvedou 45\% na účely SaZP.

\end{document}
11. MANAGEMENT

Management (věda o řízení) je ucelený soubor ověřených přístupů, názorů, zkušeností,
doporučení a metod, které vedoucí pracovníci (manažeři) užívají k zvládnutí specifických
činností (manažerských funkcí), jež jsou nezbytné k dosažení soustavy podnikatelských cílů
organizace.

Manažer je profese a její nositel je zodpovědný za dosahování cílů svěřených mu
organizačních jednotek (útvarů, kolektivů), včetně tvůrčí účasti na jejich tvorbě a zajištění.
Využívá při tom kolektiv spolupracovníků.

Manažer a podnikatel
U menších firem podnikatel= manažer firmy.
Manažeři nenesou riziko podnikání, maximálně své místo, podnikatel o peníze.

Role a funkce manažera
e | Role manažera-pohled statický, charakterizující samotnou osobu manažera a její postavení



v hierarchii organizace
e © Funkce manažera-pohled procesní, dynamický, zachycující manažera při jeho řídící
Činnosti v organizaci

Role manažera
1. Podle úrovně řízení
e © Vrcholový management
-sem zařadíme nejvýše postavené pracovníky (ředitele, náměstky ředitele, prezidenty
společnosti a jejich viceprezidenty), skupina nejlépe placených
-na jejich Ž umění řídit a znalostech závisí úspěch firmy
e | Střední management
-sem patří vedoucí útvarů firmy (vedoucí marketingového oddělení, vedoucí nákupu,
vedoucí kontroly jakosti)



e © Nejnižší management
-sem zařadíme manažery na nejnižším stupni řízení (mistr dílny, vedoucí závodní
jídelny), očekává se schopnost řešení každodenních problémů

Role manažera

Manažerská pyramida








Koncepční práce
Nejvyšší představitele fireny, penerální
ředitel, ředitel, představenstvo. jednatel,
Zodpovida;í za podnik jako celek,



Střední
Smiiddiej
managemtit

Manažeři závodů, vedcuci útvarů,
středisek, podřízení top managementu
Zodpovídají za psřiděloné úseky



Předáci, mistři, vadaucí tymů, dilovedouci,
podřízení středního managementu,
Zodpovídají za týmy





Prvni lime

fowermanegement PE En







Výkanní pracovní

39
\newpage
2. Podle stylu řízení
e © Autokratický styl
-vedoucí, který sám rozhoduje a přikazuje svým podřízeným
-autokrat detailně a systematicky kontroluje, zda byly splněny jeho příkazy
-armáda



e © Demokratický styl
-je vedoucí (demokrat), bere ohled na názory svých podřízených, o problémech
diskutuje, konečné rozhodnutí musí však udělat sám a kontroluje jeho splnění
(plánování, projektování)



e © Liberální styl



-1berální vedoucí již nepoužívá přímých řídících příkazů
-styl je vhodný v organizacích, kde pracují pracovníci s VŠ vzděláním, kteří mají
vysokou vnitřní motivaci k práci (výzkumná pracoviště, vysoké školy)

Manažerské funkce
-jsou typické úlohy, které vedoucí pracovníci řeší v procesu své řídící práce

Sekvenční funkce

-plánování-manažer stanovuje cíle a postupy k jejich dosažení

-organizování-stanoví a uspořádá role lidí, kterým přidělí konkrétní práci

-výběr a rozmisťování pracovníků-vybírá a získává konkrétní pracovníky

-vedení ldí-vznikají vzájemné vztahy nadřízenosti

-kontrola-kontrola hodnotí kvalitu a kvantitu průběžných a konečných výsledků
a vyvodí příslušné závěry

Průběžné funkce

-prostupují všemi sekvenčními funkcemi
-analyzování problémů

-rozhodování

-koordinace při realizaci (implementace)

PLANOVANÍ
Je proces stanovení cílů řízené činnosti a vhodných cest a prostředků k jejich dosažení ve
stanoveném čase.

Postup tvorby plánu:

1. stanovím cíle

2. vymezím cesty jejich dosažení (varianty)
3. jednu variantu zvolím = to je plán

40
\newpage
Plány členíme z časového hlediska:
e Strategické plány (dlouhodobé)
e © Taktické plány (střednědobé-roční)
e | Operativní (krátkodobé-každodenní)

Vztah marketingu a managementu

Obě vědy mají své specifika: management se zabývá řízením a marketing trhem. Však firemní
marketing 1 management mají shodné cíle = prosperita firmy a dosažení zisku, mají 1 mnoho
společného. Jeden z bodů je právě plánování.

ORGANIZOVÁNÍ
Posláním organizování je zajistit dosažení stanovených cílů pomocí procesů specializace
navázané a nezbytné koordinace prací a lidí.
Kroky v procesu organizování
-identifikace činností-co vše je potřeba zajistit |
-seskupení vymezených činností-přiřadíme konkrétním organizačním útvarům
firmy (nákup materiálu zajistí zásobování)

-stanovení a přiřazení rolí lidí-při této činnosti říkám, kolik lidí a s jakými

konkrétními úkoly bude zajišťovat práci

Organizační struktury
Je organizovaný systém, ve kterém je práce rozdělena, seskupena a koordinována. Jsou
graficky zobrazovány v organizačních schématech.
Třídění organizačních struktur
1. Hledisko formálnosti:
e © Formální struktura-nadřízenost a podřízenost, činnosti
e © Neformální struktura-vedoucí nejsou jmenováni
2. Hledisko druhu sdružování
e © Funkcionální struktura
e © Výrobková struktura
e © Ostatní účelové struktury
3. Hledisko rozhodovací pravomoci:
e © Liniové struktury
e © Štábní struktury
e © Liniově štábní struktury
e © Cílově programové struktury

41
\newpage
4. Hledisko míry centralizace:
e © Centralizované
e | Decentralizované
5. Hledisko počtu řídících úrovní:
e | Ploché struktury
e © Úzké struktury
6. Hledisko časového trvání:
e © Dočasné
e Trvalé
Faktory ovlivňující volbu organizační struktury
1. Vnitřní faktory
Sem patří faktory jako velikost firmy, výrobně-technická základna (strojní vybavení),
teritoriální rozmístění (jestli má firma pobočky či závody)
2. Vnější faktory
Jedná se o faktory, které firma sama není schopna ovlivnit (legislativní možnosti, stabilita
podnikatelského okolí)





VÝBĚR, ROZMISŤTOVÁNÍ A HODNOCENÍ PRACOVNÍKA
Získávání vhodných pracovníků
1. Definujeme naší potřebu
2. Realizace personálního zajištění
e | Fáze plánovací
e | Fáze náboru a výběru
Zvyšování kvalifikace, rekvalifikace
Ne vždy je potřeba s novou prací hned přijímat nového pracovníka. Stačí u stávajících
zaměstnanců pouze zvýšit nebo změnit kvalifikaci (znalosti, dovednosti a návyky, které
využívají při práci).
Proces zvyšování kvalifikace:
1. fáze- stanovení reálné potřeby zvýšení kvalifikace
2. fáze-vlastní zvýšení kvalifikace
3. fáze-vyhodnocení výsledků
Způsoby zvyšování kvalifikace:
a) školení v rámci pracovního procesu-je organizováno v rámci podniku
např. zaškolení mistrem k určité nové práci
b) školení mimo pracovní proces-kurzy, dálkové university, jazykové stáže,
školení (kurzy daňových poradců, svářečské kurzy)

Pracovní kariéra manažera

1. etapa - přípravná - doba studia (SŠ,VŠ)

2. etapa - zakotvení - zapracování se v prvním zaměstnání

3. etapa - rozvoj - má manažer pro firmu velký význam a přínos (35-50 let)

4. etapa - pozdní kariéra - dochází obvykle k úbytku energie, manažeři začínají mít sklon
k rutinním řešením

A2
\newpage
Hodnocení pracovníků

Má pomoci lépe využít profesní kvalifikaci zaměstnanců, rozvíjet jejich pracovní kariéru,
motivovat a spravedlivě odměňovat.

Obvykle se hodnotí:

a) plnění pracovních úkolů

b) chování v pracovním procesu a mimo něj

c) osobní a charakterové rysy

Hodnocení provádí:
1. vedoucí pracovníci
2. pracovníci personálních útvarů
3. externí nebo interní specialisté

Systémy odměňování
Na základě hodnocení pracovníka provádí vedoucí jeho odměňování či potrestání. Systémy
odměňování úzce souvisí s motivací pracovníků na jedné straně a finančními možnostmi
firmy na straně druhé. S odměňováním souvisí i daňová problematika.
1. hmotné odměny
e | Přímé odměny (mzda, prémie a odměny, podíly na zisku)-prochází zdaněním
e | Nepřímé odměny (příplatky na dovolenou, životní pojištění a důchodové, příspěvky na
stravování, na mateřské školy, poskytované zboží a služby
(některé podléhají zdanění daní z příjmu)
2. nehmotné odměny
e © Prestižní funkce, volná či individuální pracovní doba, možnost odborného růstu

Hodnocení nekvalitní práce
e | Ústní napomenutí
e © Písemné napomenutí
e | Finanční postih (odebrání prémií, osobní ohodnocení)

Výpověď (v extrémních případech okamžité zrušení pracovního poměru)

Vedení lidí-teorie X a Y

Je čtvrtou manažerskou funkcí. Manažer musí správně vést pracovníky, aby pracovali

v žádoucí kvalitě, kvantitě a směrem naplnění cílů.

Teorie X vychází z předpokladu, že průměrný pracovník nemá své zaměstnání rád, je mu
přítěží nutnou k zajištění obživy. Nemá zvláštní ambice. Základním rysem je lenost a snaha
práci se vyhnout.

Teorie Y vychází z opačných předpokladů: pracovník má přirozený sklon k práci, chce

V práci najít svou seberealizaci, má sklon k odpovědnosti, aktivně se účastní na práci a řízení.
Teorie X a Y představují vlastně dva extrémy chování podřízených a jim odpovídající řídící
působení vedoucího pracovníka. V praxi samozřejmě nepracujeme s průměrnými lidmi, každý
člověk je jiný, a proto u každého je třeba najít potřebnou míru skloubení teorie X a teorie Y
současně. Vedoucí tedy hledá určitý kompromis pro každého jednotlivého podřízeného.

43
\newpage
KONTROLA
Poslední manažerskou funkcí je kontrola=proces sledování, rozboru a přijetí závěrů
V souvislosti s odchylkami od záměru (cíle).

Fáze kontrolního procesu:

. stanovení cíle kontroly

. stanovení kontrolních kritérií

. rozbor kontrolovaných procesů a porovnání s kritérii kontroly

. vyhodnocení zjištěných odchylek a přijetí závěrů

. realizace závěrů (firma předepsala manko k úhradě skladníkovi)

U KB UO NN

44
\newpage
12. HOSPODÁŘSKÁ POLITIKA

Hospodářská politika se nás dotýká dnes a denně, přímo 1 nepřímo.
HP je souhrn cílů, nástrojů, rozhodovacích procesů a opatření státu zaměřených na kontrolu a
ovlivňování ekonomického vývoje.

Propojení ekonomiky a politiky:

Ekonomika a politika spolu úzce souvisí, hospodářská politika se dotýká denně všech lidí i
firem přímo 1 nepřímo.

Její součástí je např.: zdravotní péče a úhrady za ni, výše školného, regulace výše nájemného,
velikost daní, podmínky přístupu k cizím měnám

HP je ovlivňována tím, která z politických stran se při volbách dostala k moci (pravicová-
podnikatelé, levicová-zaměřuje se na sociálně slabší a na „normální“.

Subjekty hospodářské politiky:
e Parlament
-složený z 2 komor (poslanec. sněmovna-200 poslanců a senát-81 senátorů)
-volení zástupci občanů, kteří schvalují zákony
-státní rozpočet je zákonem, sestavuje se vždy na 1 rok, stát plánuje,
kolik vybere na daních od občanů a firem a co bude z těchto peněz
financovat, parlament schvaluje návrh rozpočtu
e Vláda
-nejvyšší výkonný orgán, je základním tvůrcem konkrétní aktuální HP
-připravuje návrh rozpočtu a předkládá ho ke schválení parlamentu, pokud
je rozpočet schválen, je poslanci vyjádřena důvěra
e Centrální banka (Česká národní banka)
-jejím hlavním úkolem je střežit a korigovat komerční bankovní trh
a množství peněz v oběhu tak, aby nedocházelo k jejich znehodnocování
-inflaci (vede účet státního rozpočtu, vydává peníze)
-řídí bankovní trh
-sídlí v Praze, řídí ji guvernér-Jiří Rusnok a 6 členů bankovní rady
-guvernéra jmenuje prezident

Funkce institucí státu

e © Funkce právní jistoty a bezpečí
Cílem je vytvořit právní podmínky a dodržování zákonů, jde o bezpečnost vnitrostátní
1 mezinárodní, ministerstvo obrany má na starost armádu a ministerstvo vnitra-polici1

e © Funkce sociální
Státní instituce provádějí přerozdělovací procesy (transfery obyvatelstvu=peníze
vybrané na daních a sociální pojištění používají na výplatu důchodů, sociálních a
nemocenských dávek.
Starají se p veřejné statky (školství, zdravotnictví, Životní prostředí, infrastruktura-
dálnice, železnice, energetika)
Fungující neziskový sektor je nutnou podmínkou pro fungující ekonomiku

45
\newpage
e © Funkce hospodářská
1.stát sám podniká-za účelem dosažení zisku a rozmnožení bohatství státu
2.stát vytváří podmínky pro úspěšné podnikání ostatních subjektů hospod.
-kdy stát určuje daňové zatížení
-pravidla hospodářské soutěže: konkurenční boj, zákaz klamavé reklamy

Nástroje hospodářské politiky

1. Právní systém, legislativní proces
2. Monetární systém

3. Státní rozpočet a fiskální politika
4. Důchodová a cenová politika

5. Zahraničně obchodní politika

1. PRÁVNÍ SYSTÉM A LEGISLATIVNÍ PROCES

Zákony v ČR prochází schválením Parlamentu a vyjadřuje se k nim prezident republiky.
Schválené normy jsou zveřejňovány ve sbírce zákonů a jsou závazné pro občany státu 1
všechny, kdo se zdržují na území naší republiky.

Porušení zákona je předmětem šetření a příslušné státní instituce podřízené vládě, mají
vymezeny své pravomoci trestat pachatele (krádež či dopravní nehoda-policie, daňové
nesrovnalosti-FÚ, nedovolené podnikání-ŽÚ). Sporné případy řeší soudy.

2. MONETÁRNÍ SYSTÉM

Základní úlohu hraje centrální banka dané země (u nás ČNB). Specifická je Evropská
centrální banka se sídlem ve Frankfurtu, která byla založena pro EU po zavedení jednotné
měny eura.

K udržení stabilní měny používá každá centrální banka následující nástroje:

e REPO SAZBA
-je úrok centrální banky pro terminované operace s komerčními bankami

e © POVINNÉ MINIMÁLNÍ REZERVY (PMR)
-každá komerční banka má povinnost složit u CB část svých depozit (vkladů) jako
rezervu, a tím tyto peníze „umrtví“a nemůže s nimi podnikat

e OPERACE NA VOLNÉM TRHU
-CB prodává a nakupuje cenné papíry, obvykle státní dluhopisy

* OSTATNÍ NÁSTROJE MĚNOVÉ POLITIKY
-CB může stanovovat úvěrové limity

3. STÁTNÍ ROZPOČET A FISKÁLNÍ POLITIKA

K plnění svých funkcí potřebuje stát peníze, jako řádný hospodář dopředu plánuje, kolik

v následujícím roce peněz potřebuje (výdajová stránka státního rozpočtu) a kolik jich získá
(příjmová stránka).

46
\newpage
Porovnáním těchto dvou stran pak státní rozpočet může být:

VYROVNANÝ- příjmy=výdaje

PŘEBYTKOVÝ- příjmy>výdaje

SCHODKOVÝ - příjmy<výdaje

Přebytkový rozpočet vytváří rezervu státu na dobu nepříznivého období.

Schodek znamená, že si stát na své výdaje musí někde půjčit. Většinou vydá státní dluhopisy, které si
koupí tuzemské banky a podniky nebo zahraniční investoři.

Návrh předkládá vláda ke schválení parlamentu a schválený rozpočet má podobu zákona.

RESTRIKTIVNÍ FISKÁLNÍ POLITIKA

=výsledek přebytkového rozpočtu, nepodporuje rozvoj ekonomiky

-stát nevytváří nové pracovní příležitosti, nedává zakázky soukromým firmám
EXPANZIVNÍ FISKÁLNÍ POLITIKA

=výsledek schodkového rozpočtu, podporuje rozvoj ekonomiky, ale na „dluh“

Schéma vyrovnaného státního rozpočtu







Příjmová stránka Výdajová stránka
e Daně e | Státní správa (úřady)
e Cla e | Obrana státu, školství, zdravotnictví
e | Sociální pojištění e | Transfery obyvatelstvu (důchody, dávky
e | Ostatní příjmy (poplatky atd.) atd.)
e | Příjmy z rozpočtu EU e | Státní zakázky (dálnice)
Investice do život.pojištění
Odvody do rozpočtu EU





Příjmy celkem = Výdaje celkem

Náš státní rozpočet na jeden rok je okolo 1 100 mld.Kč, což je necelá třetina našeho
ročního HDP(hrubý domácí produkt).

4. DŮCHODOVÁ A CENOVÁ POLITIKA

Regulace mezd u nás byla používána v roce 1990-1995, dnes stát mzdy v soukromém sektoru
nereguluje a ponechává je volnému působení trhu.

Regulace cen-ceny státem regulované jsou elektřina, plyn, voda, nájemné v bytech, dálkové
vytápění, telekomunikační a poštovní poplatky apod.

5. ZAHRANIČNĚ OBCHODNÍ POLITIKA
Stát řeší kurz koruny k zahraničním měnám, celní politiku, kvóty-dovozní, vývozní (množstevní
omezení dovozu a vývozu), embargo (úplný zákaz dovozu a vývozu), cla.

CLO
Celní poplatek, je dávka vybíraná státem při přechodu zboží přes celní hranici.
V EU jsou mezi státy zrušena cla.
Funkce cla:
e | Fiskální-příjem do státního rozpočtu-odvádíme do rozpočtu EU
e © Obchodně politická-nástroj hospodářské politiky
e | Cenotvorná-u dovozového zboží firmy započítají clo do prodejní ceny

47


\newpage
Druhy cla v EU:

Dovozní clo-nejběžnější
Vývozní clo-není používáno
Vyrovnávací

Odvetné

Antidumpingové

DOVOZNÍ A VÝVOZNÍ KVÓTY
Stanovené počet nebo poměr.

Dovozní kvóta je omezení množství určitého dováženého statku.
Vývozní kvóta je omezení množství určitého vyváženého statku.

MA ASTRICHTSKÁ KRITÉRIA

Jsou kritéria pro členské státy EU pro vstup do 3.fáze Evropské hospodářské a měnové unie

(EMU) a pro zavedení společné měny-eura.

48
\newpage
13. DAŇOVÝ SYSTÉM

Daňová soustava je složka finanční soustavy státu. DS souvisí s existencí státního rozpočtu-
daně jsou peníze, které tvoří podstatnou část příjmů státního rozpočtu- přerozdělují se a dávají
na ta místa, kde je jich potřeba.

Funkce daní
e | Fiskální-mají schopnost naplnit veřejný rozpočet
e © Alokační-stát může poskytovat zvýhodnění (daňové úlevy)-očkování\ldots
e © Redistribuční-zmírnění rozdílů v důchodech jednotlivých subjektů
e | Stabilizační-daně odčerpávají do rozpočtů vyšší díl a dělají rezervu

Daň je povinná a nevratná platba státu.
Daň může být:
e © Progresivní-přerozděluje důchod od subjektů s většími důchody k subjektům
s menšími důchody
e © Proporcionální-přerozděluje důchody mezi jedinci s různou úrovní důchodu
e © Degresivní-více dopadá na subjekty s menšími důchody (zdanění spotřeby)
Zájmy občanů a firem:
v“ Stát potřebuje vybrat co nejvíce peněz, aby mohl rozvíjet své aktivity
Občané a firmy naopak chtějí odvádět co nejméně daní
Stát musí respektovat sociální únosnost daní (to jej vede k systému úlev)
Při konstrukci daňového systému musí stát zohledňovat i technická a ekonomická
kritéria vybíratelnosti a vymahatelnosti daní
(čím více vyjímek a úlev, tím se administrativní náročnost zvyšuje)

SS

Daňová reforma v naší republice proběhla k 1.1.1993.Principy:
1.Spravedlnost zdanění-stejné podmínky pro různé typy subjektů (tuzemsko, zahraničí)
2.Všeobecnost zdanění-zdanění podléhají všechny typy vlastnictví

3.Účinnost zdanění-vhodným zdanění stimulovat žádoucí aktivity

4.Harmonizace- sbližování naší daňové soustavy se systémy EU

ZÁKLADNÍ POJMY

Vynětí a osvobození od daně
e © příjmy z prodeje bytu, pokud v něm měl majitel bytu bydliště alespoň 2 roky
e © příjmy z prodeje movitých věcí, kromě motorových vozidel, letadel, lodí, nepřesahuje-li doba
mezi nabytím a prodejem | rok
e © příjmy z prodeje nemovitostí - přesáhne-li doba mezi nabytím a prodejem 5 let
e | ceny z veřejné soutěže do 10000 Kč
e © příjmy ve formě dávek nemocenského pojištění, důchodového pojištění, státní sociální
podpory
Základ daně- částka, ze které se určitým procentem vypočítá odváděná daň
Sazba daně- procentem vyjádřený poměr daně k základu daně

Daně důchodové- platí poplatníci podle výše svého příjmu

49
\newpage
LMĎ



Částky se zaokrouhlují na celé Kč nahoru















Údeje jsou uvedeny v % Zaměstnavatel | Zaměstnanci | Celkem | OSVČ
Sociální pojištění 25,00 6,50 31,50 | 30,60
a) nemocenské pojištění 2,30 0 2,30 1,40
(nemoc, OCR, PMD)
b) důchodové zabezpečení 21,50 6,50 28,00 28,00
c) státní politika zaměstnanosti
(rekvalifikace, vytváření veřejně- | | © 1,20, | 0 1,20 1,20
prospěšných míst NN O
Zdravotní pojištění o k v
(lékař, rehabiliface, léčiva, pobyt 900 4,50" | 13,50 | "13,50
v nemocnici, lázně)
Celkem 34,00 11,00 45,00 44,10











Daň z příjmů fyzických osob 15 % ze superhrubé mzdy; u zaměstnanců, u OSVČ ze základu.

































Sleva z vypočítané daně | Rok Měsíc

Na poplatníka 24 840,-- | 2 070,--
Částečný invalidní-důchod 2520,--| 210,-
Invalidní důchod 5 040,-- |  420,--

Držitel průkazu ZTP-P 16 140,-- | 1 345,--
Manželka, manžel 24 840,- |-

Manželka, manžel ZTP/P < 149680, |

Student do 26 let 4 020,--| | 335.--

Dítě l.vřadě ©. 45,54. | 13-404--| -1117-- 1207 .
Dítě 2, vřadě.:. A ter | X7004 | 1617 =
Dítě 3. v řadě a další 27 Z 7 | 20-604 2017



bore Ta 4 0 P

Či


\newpage


Postup při výpočtu mzdy

1. Zjistíme o jaký druh mzdy se jedná (časová, úkolová a plat)
2. Pro jednolivé druhy vyhledáme potřebné údaje
a) časová - počet dnů v měsíci, denní úvazek v hodinách a hodinovou sazby,
(vynásobíme)
b) úkolová - počet vyrobených kusů x sazba na 1 kus
c) plat - je stanoven na celý měsíc
Požnámka - tyto údaje doplníme do základní mzdy
3. Vypočítáme pobídkové složky mzdy - prémie, odměny - přílušné procento
vypočítáme ze základní mzdy
4. Do hrubé mzdy doplníme součet základní mzdy a pobídkové složky - prémie
a odměny
o. Vypočítáme SZ a ZP placené zaměstnavatelem dle taháku
6. Sečteme hrubou mzdu a.SZ a ZP a vyjde nám superhy
7. Superhrubou mzdu zaokroulíme na 100 nahoru a vyjít ná
8. Vypočítáme 15 % daň z daňového základu n
9. Zjistíme slevy dle taháku a sečteme je
10. Vypočítáme daň po slevách. ©
11. Vypočítáme čistou mzdu = HM - SZ a ZP placené zaměstnancem - daň po
slevách,
12. Z tohoto důvodu si vypočítáme - také dle taháku SZ a ZP placené
zaměstnancem (z hrubé mzdy) .
13. Zjistíme srážky ze mzdy dle zadání včetně záloh m
14. Vypočítáme částku pro výplatu - čistá mzda mínus srážky a zálohy






\newpage
LISTINA



Měsíc leden znak
22 dnů >Oelkém“.
UKALOV
o.. \ldots 1
úkolová

8



mzda

Hrubá mzda
SZ zaměstnavatelem
ZP zaměstnavatelem

mzda
vavke.cL

SZ né zaměstnancem
„ZP '
Daň ze -
„Daňové
Daň

mzda

včetně záloh

k



















Opravena /

ČH= HW- 4 a 2Ť oamůslremum) - 5 po oetpocle k


\newpage
5
A
=
CD

-a

ZU
Měsíc leden
o 22 dnů

VACÍ VYPLA

SOUKALOV
3

8

ištění
ks

Základní mzda
Hrubá mzda

SZ zaměstnavatelem
ZP zaměstnavatelem

9 mzda
u X
SŽ zaměstnancem
ZP zaměstnancem
Daň ze
Daňové
Daň
istá mzda
včetně záloh
k výplatě

LISTINA

KOUKAL
2
úkolová
8

320
Á f
, -5%

4 000,-



RAM oa Mikkola,

OV
1..
časová
6
2 000
180,-

nn íÁlrny 7
W m [XC 1

16,10%
700,-

3500-

500
\newpage
















ZÚČTOVACÍ VÝPLATNÍ LISTINA

Měsíc leden
odpracováno 21 dnů



Ť

|
|

W


Jméno ČECHOVÁ | SCHLEZINGER | NĚMCOVÁ
děti | : 2 3
mzda - druh úkolová časová plat
denní úvazek 8 6 8
záloha 1500,- 3 000,- 4 000.-
hodinová sazba „ 180,-
kusová sazba 56
lat JE 6 AA SO 22 000,-

Prémie, odměny RAO 25 % 8% 8 000
Srážky - manka a škody 600,-

- výživné 3000,-

- spoření 3 000,-

- půjčky 4500,- | 700,-
vyrobeno ks 315

X

Základní mzda ACH Ž
Hrubá mzda + prěvně ZLO ,
SZ placené zaměstnavatelem /7% S A |
ZP placené zaměstnavatelem 7'- P „E
Superhrubá mzda LÝ Cr VFU
Já s 4; 2, . PV né 4 > Ó 3 9 A s Mě
SZ placené zaměstnancem še fe PED A
ZP placené zaměstnancem 7 7, - T |
Daň ze mzdy P 5 40 L | Z
Daňové odpočty SUI |- Aozns Le | one “
Daň po odpočtech +252 if m E Š
Cistá mzda PDE M ++ be 7 Z 677
Srážky včetně záloh ÓA7+% 4" xff0o E? če +
Částka k výplatě „RKO 3 C


\newpage
ZU VACI VYPLA LISTINA
4611, 10] „
23/dnů „BA |

Jméno B KLUS BARTOSOV
děti l 3 | 2
mzda - druh úkolová časová lat
denní úvazek 8 6. * 8
záloha | 1500,- 5 000,- 4 000,-
hodinová sazba m 240,-
kusová sazba 85
Sv 28 000,-
Prémi 8% | 25% 8 000
S - manka a 600,-
- výži 2000,-
- / 3 000,-
- půj 4500-
ks 320 /
mzda | DP“ C
mzda 22.764 AMG M . f 4co
Z placené zaměstnavatelem /**" +3 :
lacené zaměstnavatelem ** -

mzda 36

zaměstnancem |- 14M1Ď
lacené zaměstnancem
ze

istá mzda
včetně záloh -“ '-
k výplatě


\newpage
14. PŘÍMÉ A NEPŘÍMÉ DANĚ

DAŇ Z PŘIDANÉ HODNOTY
Je nepřímou univerzální daní, tvoří část ceny výrobků a služeb.
Je vybíraná při každém prodeji.

Před vstupem do EU 1.5.2004 ČR harmonizovala svůj zákon o DPH s Šestou směrnicí
Evropského společenství tak, aby ihned po našem vstupu mohl fungovat volný trh. (byly
zrušeny pohraniční celní kontroly-tím pádem se začalo přiznávat DPH při pohybu zboží
z EU).
Přesuny zboží v rámci EU:
e © Dodání (případné zaslání) zboží do jiného členského státu (náš export z EU)
e © Pořízení zboží z jiného členského státu (náš import z EU)
Zahraniční obchod s nečlenskými státy EU:
e | Vývoz (náš export mimo EU)
e © Dovoz (náš import ze zemí mimo EU)
Předmětem daně je:
a) Prodej zboží nebo nemovitosti osobou povinnou k dani s místem plnění v tuzemsku
b) Poskytnutí služby plátce DPH v tuzemsku
c) Pořízení zboží z jiného členského státu EU za úplatu plátcem DPH a pořízení nového
dopravního prostředku z jiného státu EU soukromou osobou-občanem
d) Dovoz z nečlenských států EU s místem plnění v tuzemsku

Sazby daně:
e © Základní sazba-21% ze základu daně-uplatňuje se na zboží a služby
e Snížená sazba-15% ze základu daně-uplatňuje se na zboží a služby
Zboží-(noviny živá zvířata, voda, dětské pleny, knihy,potraviny)
Služby-(úprava a rozvod vody, hromadná doprava, kulturní činnosti)
DPH vybírají:
e Finanční úřady-v tuzemsku a při obchodech v rámci EU
e © Celnice-při dovozu z nečlenských zemí EU

Poplatníkem daně je kupující a plátcem daně je prodávající.

Osoba povinná k dani

Ne každý prodávající v ČR musí být plátcem daně.

Z osoby povinné k dani, která má sídlo nebo místo podnikání v tuzemsku, se stane plátce
DPH, pokud její obrat za nejbližších 12 předcházejících měsíců přesáhne částku 750 000 Kč.
Musí se registrovat na finančním úřadě.

Firma registrovaná v ČR k placení daní dostává daňové identifikační číslo (DIČ).

Uskutečnění zdanitelného plnění a vznik daňové povinnosti je dnem dodání zboží či
poskytnutí služby. Plátce je povinen přiznat daň ke dni zdanitelného plnění nebo ke dni
platby, podle toho, který den nastane dříve. [o znamená, že přijatá záloha na zboží či služby
podléhá povinnosti zaplatit DPH.

53
\newpage
Daňový doklad

Prodávající při prodeji vystavuje DD, který má své závazné náležitosti. Tento doklad je
povinen archivovat 10 let pro účely daňové kontroly. Při prodeji v hotovosti, platební kartou
nebo šekem v celkové hodnotě menší než 10 000 Kč včetně DPH vystavuje plátce
zjednodušený daňový doklad.

Základ daně je peněžní částka snížená o daň.
a. Pokud známe základ daně (částku bez DPH), pak vypočítáme DPH jako součin
základu daně krát sazba daně v procentech (20*0,15=23 Kč)
b. Pokud známe cenu celkem, pak daň vypočítáme jako součin ceny celkem krát
koeficient, kde činiteli je sazba daně a ve jmenovateli součet 100+sazba daně.
(100*21/121=17,36 Kč) Cena bez DPH je 82,64 Kč.

Osvobození od daně bez nároku na odpočet daně
e Poštovní služby
e © Rozhlasové a televizní vysílání
e © Finanční činnosti (poskytování úvěrů)
e © Penzijní Činnosti
e © Pojišťovací činnosti
e © Převod a nájem pozemků, staveb, bytů a nebytových prostor
e © Výchova a vzdělávání
e | Zdravotnické služby a zboží
e | Sociální pomoc
e © Loterie a podobné hry
Zdaňovací období
Základní zdaňovací období je I měsíc. Pokud plátce za předcházející kalendářní rok
nepřekročil obrat 10 mil.Kč, může požádat o zdaňovací období čtvrtletní.

Výhody DPH
1. Všeobecnost zdanění
2.. Zdanění spotřeby
3.. Z hlediska plátce je zatížena jen ta hodnota, kterou sám k výrobku přidal
4. Daň je lehce sledovatelná na cestě od prvovýroby ke spotřebiteli,
omezuje možnosti daňových úniků
5. Sbližujeme se s daňovými systémy v EU
Nevýhody DPH
1. Značná administrativní náročnost na straně podnikatelů i FÚ

54
\newpage
Postup přiznání a odvodu DPH

Po ukončení zdaňovacího období (čtvrt roku,měsíce) musí plátce podat na FÚ daňové
přiznání, kde uvede zdanitelná plnění přijatá (nakoupené zboží) a součet daně na vstupu,
kterou má nárok nechat si od FÚ vrátit.

Pak uvede zdanitelná plnění uskutečněná (prodané zboží) a součet daně na výstupu, který je
povinen odvést FÚ.

Rozdílem daně na vstupu a daně na výstupu vypočítáme skutečně placenou částku.

Uplatňování DPH při obchodu se zeměmi EU (neprochází celním režimem)
a) Firma z ČR zasílá zboží do EU firmě (je plátce DPH), musí na FA uvést její DIČ a
nechat si od ní průkazně potvrdit, že zboží ve své zemi obdržela.
b) Firma z ČR pořizuje zboží z jiné země EU od plátce DPH, nakoupí ho bez DPH a je
povinna v ČR přiznat DPH dle českého zákona o DPH.
Uplatňování DPH při obchodu s nečlenskými státy EU
DPH při dovozu vybírají celnice. Základ DPH při dovozu se vypočítá:
Celní hodnota zboží+clotspotřební daň = základ DPH
Při dovozu je DPH počítáno 1 ze cla a popřípadě 1 spotřební daně (u alkoholu, tabáku a
ropných produktů). Čím vyšší clo, tím více naroste DPH.

DPH je významným cenotvorným činitelem.

SPOTŘEBNÍ DANĚ
Spotřební daň je nepřímá selektivní daň, co znamená, že je vybírána prostřednictvím prodeje
vybraných druhů zboží.
Předmětem spotřební daně jsou:
e © Minerální oleje (uhlovodíková paliva a maziva)
e Lihalihoviny
e Pivo
e Víno
e | Tabákové výrobky

Zdaňovací období je jeden měsíc. Pokud v určitém měsíci nevznikne daňová povinnost,
daňové přiznání se nepodává.

Plátci spotřební daně jsou tedy nejen výrobci u tuzemské produkce a dovozci při dovozu
z nečlenských zemí EU, ale 1 provozovatelé daňových skladů.

Poplatníci jsou všichni, kdo tyto výrobky kupují pro vlastní spotřebu.

Sazby daně jsou stanoveny samostatně pro každý druh výrobku v závislosti na měrných
jednotkách. Velikost spotřební daně není závislá na ceně výrobku.

55
\newpage
DANĚ PRO ŽIVOTNÍ PROSTŘEDÍ (nepřímé daně)
e | Daň ze zemního plynu
e | Daň z pevných paliv
e | Daň z elektřiny
Zdaňovací období je kalendářní měsíc.
Sazby u jednotlivých daní jsou pevně dány zákonem ve vztahu k fyzikálním jednotkám
dodaných energií.

DAŇ Z PŘÍJMU
Zákon o dani z příjmu má tři části:
e | Daň z příjmu FO
e | Daň z příjmu PO
e | Společná část (odepisování HM a NDM)

Daň z příjmu FO
Poplatníci daně: FO které mají na území ČR bydliště. Jejich daňová povinnost se vztahuje
na příjmy plynoucí ze zdrojů na území ČR i na příjmy plynoucí ze zahraničí.
Plátci daně: Podnikatelé platí daň sami za sebe (plátci je shodný s poplatníkem)
U zaměstnanců odvádí zálohy na daň zaměstnavatel

Zdaňovací období je kalendářní rok, popřípadě hospodářský rok.

e © Podnikatel sám za sebe může podat přiznání do 3 měsíců

e © Daňový poradce, může požádat prodloužení až do konce června

e | Zasvé zaměstnance zpracovává firma přiznání do 15.února

Osvobození od daně:

e | Příjmy z prodeje bytů a rodinných domů

e | Příjmy z prodeje nemovitých věcí

e | Příjmy z prodeje movitých věcí

e | Ceny z veřejné soutěže, reklamní soutěže a ze sportovní (do 10 000 Kč)

e © Příjmy ve formě dávek nemocenského pojištění, důchodového, státní sociální podpory
apod.

e | Dotace od státu, granty z EU na pořízení hmotného majetku

e | Příjmy v naturální formě v podobě reklamních předmětů do hodnoty 500Kč

e © Dary mezi nepříbuznými osobami jsou osvobozeny do souhrnné hodnoty

e © Odroku 2014 již příjmy z dědictví nepodléhají dani dědické, ale jsou osvobozeny
v zákoně o dani z příjmu

56
\newpage
Předmět daně:
Příjmy v peněžní 1 nepeněžní podobě (naturálie, hmotné odměny, využívání služebního
automobilu pro osobní potřeby) kromě příjmů osvobozených od daně.
Členění příjmů:
a) Příjmy ze závislé činnosti (ze zaměstnání)
b) Příjmy ze samostatné činnosti (Z podnikání)
c) Příjmy z kapitálového majetku (dividendy, podíly na zisku)
d) Příjmy z nájmu
e) Ostatní příjmy (výhry, příjem z prodeje majetku (bez osvobození)







Základ daně:

U příjmů z podnikání zjistíme základ daně
z účetnictví firmy Výnosy-Náklady=základ daně
z daňové evidence Příjmy- Výdaje=základ daně





Vedení UCE 1 DE je složitá a nákladná věc, proto podnikatelé mají možnost zjednodušení:
e © Prokazovat pouze příjmy a výdaje v daňovém přiznání uplatnit paušálem
(80% zemědělci, 60% živnostníci, 40% nebo 30%)
e © Stanovení daně paušální částkou. Zákon stanoví přesné podmínky, podnikatel může
zažádat FÚ o stanovení paušální částky

Slevy na dani:

e | Snižuje konečnou vypočítanou daň (např.sleva na dani 24 840 Kč znamená, že od
vypočítané daně odečteme celou částku)

e | Sleva na poplatníka, na manželku s příjmem nižším než 68 000 Kč, invalidní
důchodce 1.,2.stupně, invalidní důchodce 3.stupně, student prezenčního studia,
daňové zvýhodnění na dítě

Nezdanitelná část základu daně je snížením základu daně:
e © Po snížení základu vypočítáme procentem daň (15% ze základu daně)
Daňový bonus:

e © Jsou zavedeny daňové bonusy v případě daňového zvýhodnění za každé dítě žijící ve
společné domácnosti ve výši 15 204 Kč- má-li poplatník v daném roce nižší daňovou
povinnost než je daňové zvýhodnění ve formě slevy na dani, je tento rozdíl daňovým
bonusem a bude mu FÚ proplacen

Sazba daně:

Pokud vypočítáme základ daně a upravíme ho o nezdanitelné a odčitatelné položky, je čas
vypočítat samotnou daň. [a činí 15%. Novinkou od ledna 2013 je solidární zvýšení daně 7%
z částky na 48násobek průměrné mzdy.

57


\newpage
DAŇ Z PŘÍJMŮ PRÁVNICKÝCH OSOB
Poplatníci daně jsou osoby, které nejsou FO (PO). Od daně se osvobozuje centrální banka
ČR. Poplatníci daně jsou současně plátci daně.

Zdaňovací období je kalendářní rok nebo hospodářský rok.

Předmětem daně jsou výnosy z veškeré činnosti a nakládání s majetkem, kromě
jmenovitých výjimek.

Základ daně je zisk, který zjistíme z účetnictví tak, že vypočítáme.
Výnosy - náklady = výsledek hospodaření

DAŇ SILNIČNÍ
Předmětem daně jsou silniční motorová vozidla používaná k podnikatelské činnosti.
Poplatníkem i plátcem je majitel vozidla uvedený v technickém průkazu vozu.

Sazba daně:
e © U osobních aut závisí na objemu válců v motoru
© © U nákladních aut závisí na hmotnosti vozidla a počtu náprav

Zdaňovací období je kalendářní rok, přičemž podnikatel může platit zálohy a do konce ledna
následujícího roku je vyúčtovat.

DAŇ Z NABYTÍ NEMOVITÝCH VĚCÍ

Jedná se o daň přímou, poplatník je zároveň plátce-v případě nabytí nemovité věci převodce.
Je to daň jednorázová (prodej nemovitosti, darování, dědictví). Sazba daně z nabytí
nemovitých věcí činí 4% z nabývací ceny, kterou je:

a) sjednaná cena nebo srovnávací daňová hodnota

b) zjištěná hodnota

c) zvláštní cena u obchodních korporací

DAŇ Z NEMOVITÝCH VĚCÍ

Daň z nemovitých věcí je spojena s vlastnictvím nemovité věci, jejím plátcem i poplatníkem
je majitel nemovité věci evidovaný v katastrálním úřadě k prvnímu dni kalendářního roku. U
pronajatých pozemků he id roku 2005 poplatníkem v určených případech nájemce.

Zdaňovací období je kalendářní rok. Tato daň má 2 části.
e © Daň z pozemků - základem je cena pozemku dle vyhlášky a forma využívání
pozemku u zemědělské půdy a lesů, u stavebních pozemků je to rozloha
e | Daň ze staveb a jednotek - základem je půdorys nadzemní části stavby v m2 a sazba
závisí 1 na poloze stavby (většinou se jedná o byt)

58
\newpage
CELNICTVÍ

V EU jsou mezi státy zrušena cla.

EU má jednotnou celní politiku vůči nečlenským státům.

-pokud firma z ČR vyváží zboží do USA, musí vyplnit celní dokumenty a zboží
prochází celní kontrolou, clo pro vývoz ale v EU není vyměřováno, takže firma
clo neplatí

-firma dováží zboží z USA, musí ho proclít-je uvaleno clo podle celního sazebníku
platného pro celou EU

Clo je celní poplatek, dávka vybíraná státem při přechodu zboží přes celní hranici.

Funkce cla:

e | Fiskální-příjem do státního rozpočtu-odvádíme do rozpočtu EU

e © Obchodně politická-nástroj hospodářské politiky

e © Cenotvorná-u dovozového zboží firmy započítají clo do prodejní ceny
Druhy cla v EU:

e © Dovozní clo-nejběžnější

e © Vývozní clo-není používáno

e © Vyrovnávací

e | Odvetné

e © Antidumpingové

Celní sazebník
Celní sazebník je jednotný pro všechny státy EU, sazebník obsahuje všechny druhy
dováženého zboží a jejich celní sazby (% cla z celní hodnoty).
Celní sazby obsažené v celním sazebníku můžeme členit:

e © Všeobecné celní sazby (většinou sazby nejvyšší)

e © Smluvní celní sazby (na základě mezinárodních dohod dohodnuté nižší)

e © Preferenční sazební opatření (vyplývající z mnohostranných mezinárod. dohod)
Všeobecná dohoda o clech a obchodu GATT
Nejdůležitější mezinárodní organizace GATT (1947) má 135 členských zemí (ČR-1993).
Základním cílem GATT je odbourávání překážek světového obchodu.
Usnadňuje vstup zboží na trhy smluvních stran.

52
\newpage
Rozdíly ve zdanění spotřeby a důchodu
1.. Zdanění příjmů přihlíží ke konkrétním poměrům poplatníka, zdanění spotřeby nikoliv.
Tím se přímé daně stávají adresným nástrojem regulace důchodů poplatníka, nepřímé

daně tuto schopnost nemají.

D3

Nepřímé zdanění má méně nepříznivý vliv na pracovní motivaci a výrobní aktivitu než

přímé.

V)

Vzhledem k tomu. že nepřímé daně jsou součástí konečné ceny, přispívají více k růstu
Inflace než daně přímé.
4.. Zdanění příjmů vyvolává u poplatníků větší odpor než zdanění spotřeby. To vede 1 ke

snaze obejít zdanění a daňovým únikům.
Lafferova křivka
/obrazuje závislost celkového objemu vybraných daní na míře zdanění.

(respektive na daňové sazbě)

A :Lafferův bod





prohibitivní
zóna

danovy příjem

t:




VO WWW W

0% 77 daňová sazba. 100%

Lafferova křivka

Aktuálnost daňové soustavy

Velikost příjmů veřejných rozpočtů z jednotlivých druhů daní

51
\newpage
Daně majetkové- platí poplatníci podle velikosti svého nemovitého majetku,
při majetkových převodech (darování, dědictví, prodej či převod) a při
využívání vozidel pro podnikání
Daně univerzální (DPH)- vybírány při prodeji téměř všech druhů zboží a služeb
Daně selektivní-spotřební daň, daně pro životní prostředí
-jsou vybírány pouze u vybraných druhů zboží
(cigarety, alkohol, benzin, nafta, pevná paliva, zemní plyn, elektřina)
Poplatník-je FO či PO, z jejíchž peněz je daň placena (ten, z jehož kapsy peníze)
Plátce-je FO či PO, která má ze zákona povinnost peníze odvádět státu
Daňové přiznání- podoba formuláře-doklad potřebný pro kontrolu správnosti
Daňový únik-je situace, kdy se plátce či poplatník vyhýbá úhradě daně, únik
může být úmyslný nebo neůúmyslný také legální nebo nelegální
Daňová kvóta-vyjadřuje celkovou úroveň daňové zátěže v dané zemi
Daňový ráj-se označují země s velmi nízkými daněmi a ekonomikou orientovanou
na zahraniční kapitál

Den daňové svobody

-hranice, která rozděluje kalendářní rok na 2 období

-v 1. vydělávají daňoví poplatníci na pokrytí výdajů vlády a institucí státu

-V 2. s1 až 0 penězích rozhodují svobodně sami
Slevy na dani-snižuje konečnou vypočítanou daň (na dítě, poplatníka, manželku)
Zdaňovací období

-je kalendářní rok nebo hospodářský rok

-rozhodné období, časový úsek, za který se počítá příslušná daň

Struktura daňové soustavy
1.. daně přímé (poplatník podává přímo na finančním úřadu daňové přiznání)
= důchodové:
= -© daň z příjmů fyzických osob a právnických osob
= © majetkové:
= © daň z nemovitostí (tj. daň z pozemků, daň ze staveb)
=- daň silniční
= daně převodové (tj. daň dědická, darovací a z převodu nemovitostí)
2. daně nepřímé (ze spotřeby-platíme při každém nákupu zboží a služeb)
=- univerzální:
= -© daň z přidané hodnoty,
« selektivní:
= © daně spotřební (daň z minerálních olejů, daň z alkoholu, cigaret a tabák.
výrobků)

« © ekologické (k ochraně životního prostředí)
= © daň ze zemního plynu
daň z pevných paliv

= daň z elektřiny

50
\newpage
15. Finanční trhy

STRUKTURA FINANČNÍHO TRHU A NA ČEM JE ZALOŽEN

Finanční trh je založený na nabídce relativně volných peněz (které jejich majitelé nechtějí nechat ležet ladem a ztrácet
na hodnotě díky inflaci, ale chtějí je zhodnotit) a poptávce po penězích (kdy podnikatelé vědí jak vydělat, ale nemají
dostatek kapitálu). Střetem této nabídky a poptávky se vytváří cena peněz (výše úroků, tržní cena cenných papírů apod.)
Členění finančního trhu: peněžní trh, kapitálový trh, trh drahých kovů, devizový trh

ÚROK A ÚROKOVÁ MÍRA

Úrok je peněžitá odměna za půjčení peněz. Velikost úroku se obvykle vyjadřuje pomocí úrokové míry (sazby), která je
procentním vyjádřením zvýšení půjčené částky za určité časové období.

Úrokové sazby zjednodušeně říkají, jak velkou část zaplatí dlužník navíc, zjednodušeně jde o měřítko ceny peněz. U
spoření říkají, kolik peněz dostanete navíc, u půjček zase, kolik musíte zaplatit navíc k půjčené částce. Vyhlašují se na
určité období (den, měsíc, rok). Úrok je peněžitá odměna za půjčení peněz.

MOTIVY POPTÁVKY PO PENĚZÍCH

Poptávka po penězích je představována množstvím peněz v držbě jednotlivých ekonomických subjektů (domácností,
podniků, vlády) Hlavním motivem, proč člověk drží peníze v hotovosti a ztrácí tak úrok, je možnost disponovat těmito
penězi.

Druhy poptávky po penězích

1) Transakční poptávka - vyplývá z funkce peněz jako prostředku směny - jednotlivé subjekty potřebují peníze na
běžné nákupy výrobků a služeb, podniky na úhradu svých nákladů.

2) Majetková poptávka - vyplývá z funkce uchovatele hodnoty. Znamená, že ekonomické subjekty drží určitou část
svého majetku v podobě peněz jako vysoce likvidní formy majetku. Formy struktury jsou různé a působí na ní celá řada
faktorů (výnos, splatnost, riziko, atd.). Úspory => investice.

Úrok je nákladem na držbu peněz - budou-li úrokové sazby vysoké, je držba peněz nákladná, jelikož znamená ztrátu
úroku, které by přinesly.

FINANČNÍ INSTITUCE
Instituce finančního trhu jsou subjekty, které podnikají s penězi.

Instituce finančního trhu se liší charakterem poskytovaných služeb a také, zda mohou být zakládány bez povolení státu
(např. nebankovní instituce poskytující úvěry) či zda pro svojí činnost potřebují licenci (např. banky, pojišťovny,
makléřské společnosti, burzy apod.).

Česká národní banka - Komerční banky - Družstevní záložny (spořitelní a úvěrová družstva) - Stavební spořitelny -
Investiční společnosti, investiční fondy a podílové fondy - Pojišťovny - aj.

Na mezinárodní úrovni působí finanční instituce, které plní regulační, dohledovou a stabilizační funkci. Jsou to např.
Evropská centrální banka (pravidla finančního trhu), Mezinárodní měnový fond (stabilizace) nebo Evropská bank pro
obnovu a rozvoj (pomoc zemím s přechodem na tržní ekonomiku).

CENNÝ PAPÍR

Cenné papír je písemnou formou zachycený právní vztah mezi dvěma (i více) subjekty, tato písemnost má určité
náležitosti stanovené zákonem, a pak může být samostatně obchodovatelná.

Cenné papíry peněžního trhu zachycují vztah dlužnický (věřitel x dlužník)
\newpage
PENĚŽNÍ TRH - CHARAKTERISTIKA

Finanční trh vzniká na základě střetu poptávky a nabídky, kde je poptávkou investování a nabídkou spoření. Jinými slovy
poptávajícími jsou ti, kteří poptávají peníze a nabízejícími jsou ti, kteří peníze půjčují, tedy nabízí. Na peněžním trhu se
obchoduje s krátkodobými vklady a úvěry ve formě krátkodobých CP se splatností do 1 roku

ŠEK

je CP, kterým výstavce šeku dává příkaz bance, aby osobě uvedené na šeku nebo doručiteli zaplatila z jeho účtu částku,
na kterou je šek vystavený. Šek je splatný na viděnou (po předložení) do 8 dnů v téže zemi.

Podstatné náležitosti šeku:

- označení CP slovem šek

- částka (číslicí i slovy)

- jméno toho, kdo má platit (banka)

- místo, kde se má platit

- datum a místo vystavení šeku

- podpis výstavce

Použití šeku:

- proplacení v hotovosti - zaplacení šekem - zúčtování šeku - proplacení či jeho vklad na účet

SMĚNKY
Směnka je obchodovatelný cenný papír, který slouží k platbě nebo jako zajišťovací nástroj. Jedná se o nejjednodušší a
často nejrychlejší formu úvěru. Jasně z ní vyplývají závazkové vztahy a tím je ideálním prostředkem pro obchodování.

Druhy směnek

Základním členění směnek je na vlastní a cizí. Jednoduše řečeno, pokud osoba, která směnku vystavila, se zaváže
zaplatit v daném termínu a místě, jde o směnku vlastní. Směnka vlastní obsahuje slovo „zaplatím“. Za směnku cizí musí



























. bh .. , .. . .-
zaplatit někdo jiný, než ten kdo jí vystavil a vyskytuje se termín
1 “
Véřitel 2. osoba „zaplaťte .
„prožá zboží 3 osobša
W , „prodává s 2. vsoké VÝSAD VĚ VĚMŘU 5 PP 20
Věňitel Dhé nk Z Ao tací obě ase -------- P | zoplozit důsníkoví vé
prospěch věširelh
- prodávě zboži * upsdavý zoněníku ve i
- majitel směnku (| pospěch věřitele
Dhůni:

- BÍV OBE ŘLONÍCH SDĚKÝ
věřiteli







Náležitosti směnky:

* označení, že se jedná o směnku (doslovné označení v textu listiny)
* příkaz zaplatit danou finanční sumu (bez dalších podmínek)

* jméno a adresa toho, kdo má směnku zaplatit

* informace o splatnosti

* informace o místě zaplacení

* jméno věřitele, tedy toho, jemuž má být zaplaceno

+ datum a místo vystavení směnky

* podpis výstavce směnky
\newpage
POKLADNIČNÍ POUKÁZKY -
emituje stát prostřednictvím ČNB, je to CP, který slouží ke krytí přechodného schodku (deficitu) státního rozpočtu

- určen pro obchodování na mezibankovním trhu pro velké investory

- nominální hodnota jedné poukázky bývá 10mil. Kč, objem jedné emise se pohybuje v řádech miliard korun

- o tyto CP je velký zájem, protože stát v pozici dlužníka je velmi solidní zárukou navrácení peněz včetně úrokového výnosu

DEPOZITNÍ CERTIFIKÁT
- vydává banka (dlužník) a potvrzuje jím přijetí jednorázového termínovaného vkladu od klienta (věřitele). Certifikáty nakupují
podniky, občané, banky a jiné subjekty. Jsou vystavovány na částku 10 tisíc Kč. Čím vyšší je jejich nominální hodnota tím vyšší je úrok.

Kapitálový trh

- obchoduje se s dlouhodobými termínovanými vklady, úvěry a zdroji financí ve formě dlouhodobých CP se splatností nad 1 rok

Podoby CP : 1. Listinné - je vydán v podobě listiny
2. Zaknihované - neexistují fyzicky, jsou evidovány v elektronické podobě na účtu majitelů

ČLENĚNÍ CP
1 Podle právního nároku

dlužnické- představují nárok na splacení dluhu ( např. dluhopisy, směnky),
7 majetkové- představují podíl na majetku (např. akcie, podílové listy),

dispoziční - představují právo nakládat s určitým majetkem (např. konosament, skladištní list).
2 Podle doby splatnosti

peněžní - krátkodobé, splatné do 1 roku,

kapitálové - dlouhodobé, splatné za dobu delší než 1 rok.
3 Podle podoby

zaknihované - mají jen podobu zápisu v určitém registru (nyní Centrální depozitář cenných papírů, a. s.),
listinné - mají podobu speciální listiny.
4 Podle formy

Cenné papíry na doručitele - jméno majitele není na cenném papíru uvedeno, převádějí se pouhým předáním novému majiteli.

Cenné papíry na řad - jméno majitele cenného papíru je na něm uvedeno, někdy s doložkou „na řad“ (zákonné ji obsahovat nemusí),
převádějí se rubopisem a předáním. Uvedením doložky "nikoli na řad" se cenný papír na řad změní v cenný papír na jméno.

Cenné papíry na jméno - jméno majitele cenného papíru je na něm uvedeno, někdy s doložkou „nikoli na řad“, převádějí se smlouvou o
postoupení pohledávky a předáním.

5 Podle obchodovatelnosti



Obchodovatelní - nakupují se a prodávají na sekundárním trhu. Jsou převoditelné.

Neobchodovatelné - emitent zakazuje jejich nákup a prodej na sekundárním trhu. Prodávají se jen na trhu primárním, např. Vkladní
knížky.

6 Podle emise:
CP hromadně vydávané - např. Akcie, dluhopisy

CP individuálně vydávané - např. Směnka, šek
\newpage
Akcie



Akcie je majetkovým cenným papírem a její vlastník se stává spolumajitelem firmy. Souček akcií tvoří základní kapitál.
Práva akcionáře, jako společníka:
podílet se na zisku (výplata dividend)
podílet se na řízení společnosti (hlasovat na valné hromadě akcionářů)
podílet se na likvidačním zůstatku společnosti, pokud jde firma do likvidace
Akcie může být emitována:
1. Přímo za nominální hodnotu,
2. S emisním ážiem (za vyšší cenu než nominální),
3. S emisním disážiem (za nižší cenu než nominální) - současný zákon o obchodních korporacích neumožňuje.

Rozlišujeme:
1.Kmenové akcie

© Na jméno - převoditelné rubopisem neboli indosamentem,

© Na majitele - volně převoditelné, na doručitele, od 2017 pouze jako zaknihovaný CP nebo imobilizovaný CP.
2. Speciální

© Zaměstnanecké (pro zaměstnance firmy, akcie vystaveny na jméno, max. 5% objemu emise akcií),

© Prioritní akcie (přednostní výplata dividendy, ovšem majitel nemá právo hlasovat na valné hromadě).

Členění dle fyzické podoby:

© Akcie materializované (listinné - fyzicky vytištěné na papíře - mají plášť, kuponový arch pro výplaty dividend a talón pro
získání nového kuponového archu, když je předchozí vyčerpán)

© Akcie dematerializované (zaknihované - většinou jako údaje v paměti počítače).

© Dividenda je příjmem z kapitálového majetku a jste povinni dle zákona o dani z příjmu fyzických osob ji zdanit srážkovou
daní 15%.

Skládá se ze dvou částí :

a) plášť

- obchodní jméno a sídlo společnosti, číselné označení akcie, jmenovitá hodnota akcie, označení, zda je akcie na jméno nebo na
majitele, výše ZK, počet akcií v době vydání akcie, datum vydání akcie, podpisy dvou členů představenstva

b) kupónový arch s talónem

- jednotlivé kupóny se odstřihávají (detašují) a vyplácí se podíl na zisku, který se nazývá dividenda (vyjadřuje se V %). Dividenda by
měla být vyšší než úrok v bankách, v opačném případě by občané ukládali svůj kapitál do bank - poslední ústřižek
kupónového archu je talón - poukázka na nový kupónový arch.

Podílové listy

jsou vydávány podílovými fondy, tyto fondy zakládají investiční společnosti nebo banky, které tak shromažďují peníze od investorů
(občanů, firem) formou prodeje svých podílových listů. Takto získané prostředky se investují do nákupu různých druhů CP, tzv.
portfolio a tím rozloží možné riziko. V současnosti existují v ČR pouze tzv. otevřené podílové fondy. To znamená že podílník má právo
kdykoliv svůj podílový list prodat zpět podílovému fondu.

Jaké potencionální zhodnocení fond nabízí, závisí především na aktivech, do kterých investuje.

Rozeznáváme fondy: a) akciové d) smíšené
b) dluhopisové e) zajištění - minimální zhodnocení investic
c) peněžního trhu
\newpage
Obligace

- dlouhodobý úvěrový CP, v němž se vydavatel zavazuje jeho majiteli splatit dlužnou nominální částku a vyplácet
výnosy k určitému datu. Splatnost je zpravidla pevně stanovena. Vydavatel si emisí obligací opatřuje finanční zdroje
převážně na uskutečnění svých investičních záměrů. Obligace podléhají při emisi schválení ministerstvem financí-
emise obligací se pohybují v řádek stovek milionů a jejich emitenti bývají největšími ekonomickými subjekty v
národním hospodářství.

Členění obligací z hlediska emitentů:

1) státní - emituje je vláda

2) komunální - emitují územně samosprávné celky, např. obce
3) bankovní

4) podnikové - vyšší úročení, vyšší riziko

Hypoteční zástavní listy

- emituje banka, která má od ČNB povolení k této činnosti. Jejich prodejem banka získá prostředky, které dále používá.
Tyto listy jsou kryty (ručeny) splátkami hypotečních úvěrů a zastavenými nemovitostmi. Vzhledem k dokonalému jištění
je nízké úročení. Bývají veřejné obchodovatelné, doba splatnosti 5 let.

Hypotéční zástavní listy mají podobu:
© listinnou (cenný papír materializovaný)
© zaknihovanou (cenný papír dematerializovaný)

© Mohou býti neveřejně obchodovatelné a veřejně obchodovatelné

Deriváty

- využívají se nejen u CP kapitálového trhu, ale i u deviz. Předmětem koupě nebo prodeje je pouze určité právo, nikoliv
věc hmatatelná. Jejich podstatou je forma termínovaného obchodu, tzn. že dochází k určitému zpoždění mezi
sjednáním obchodu a jeho plněním. V době sjednání jsou jasně definovány podmínky obchodu, tedy cena a datum
plnění. V den splatnosti dochází k dodání finančních instrumentů a jejich proplácení.

Mají funkci: 1) zajišťovací - proti rizikům výkyvu kurzu
2) spekulativní - spekuluji, abych vydělal. V praxi běžnější

Druhy derivátů:
Forwardy, Futures - pevné termínované obchody k určitému budoucímu datu za předem dohodnutou cenu

Swapy - prodej CP či deviz v aktuálním kurzu se současným podpisem smlouvy o budoucím odkoupení zpět za předem
dohodnutou cenu

Opce - je právo k budoucímu datu uskutečnit obchod s CP nebo devizami za předem sjednanou cenu
\newpage
Investování volných peněžních prostředků
Kam uložit peníze? 13 základních investičních možností

1. Bankovní účty

2. Podílové fondy

3. ETF (ETF jsou fondy obchodované na burzách. Jejich výhodou oproti klasickým podílovým fondům je, že můžete fond koupit a
prodat kdykoli a vždy vidíte jeho aktuální cenu (u klasických podílových fondů je hodnota podílového listu přeceňována nejčastěji
jednou denně, občas i jednou týdně). Zároveň jsou zpravidla levnější na pořízení i na správu (mají nižší poplatky).

4. Investiční certifikáty

5. Penzijní připojištění

6. Stavební spoření

7. Životní pojištění

8. Akcie

9. Dluhopisy

10. Nemovitosti

11. Zlato

12. Ropa a další komodity

13. FOREX (Na Forexu se obchoduje s měnovými páry - tedy dvojicemi měn. Posiluje-li jedna měna, cena měnového páru roste.
Posiluje-li druhá z páru, cena klesá. I na tomto trhu se obchoduje s pákovým efektem a investice jsou tak značně rizikové. "Vyčistit"
účet je mnohem snadnější než ho zhodnotit do závratných výšek)

Trh drahých kovů
je pro nás jen doplňkovým trhem, protože prostřednictvím drahých kovů probíhá jen min. finančních transakcí.

Za nejdůležitější trhy drahých kovů jsou všeobecně považovány trhy zlata a stříbra, přičemž sem bývají zahrnovány i
trhy platiny a palladia.

Nejdůležitější institucí je burza drahých kovů v Londýně.

Devizový trh

- Na devizovém trhu je stejně jako na ostatních trzích cena určována poptávkou a nabídkou po dané komoditě.
Obchoduje se zde s měnami různých zemí. Střetávají se zde zájmy kupujících a prodávajících různé měnové jednotky.
Na základě poptávky a nabídky je určen poměr v jakém se dané množství měny smění. Devizové trhy tvoří banky,
finanční firmy atd.

Poptávka a nabídka po měně dané země bývá zpravidla určena vývojem mezinárodního obchodu, vývojem úrokových
měr v zemi, inflačním očekáváním, platební bilancí země, celkovým stavem ekonomiky. Naše měna je volně směnitelná
od roku 1995.
\newpage
16. NÁRODNÍ HOSPODÁŘSTVÍ

Makroekonomie - obor ekonomické teorie, který se zabývá zkoumáním ekonomického
systému jako celku, sleduje vztahy mezi agregátními veličinami např. HDP, agregátní nabídka
a poptávka, inflace, nezaměstnanost, úroková míra, měnový kurz.

Národní hospodářství je systém ekonomických subjektů na území daného státu a vztahy
mezi nimi. Souhrn hospodářských činností na území daného státu, kterých se účastní tři
ekonomické subjekty: STÁT, PODNIKY, DOMÁCNOSTI

Subjekty národního hospodářství
1. Ziskový a neziskový sektor

2. Domácnosti, firma, stát

3. FO, PO

4. Právní formy subjektů

Sektory národního hospodářství

1. Primární sektor - prvovýroba (získávání surovin z přírody, např. zemědělství, rybolov,
těžební průmysl)

2. Sekundární sektor - zpracování toho co vyprodukuje prim. sektor, např. strojírenský,
textilní, stavební, potravinářský, výroba a rozvod plynu, elektřiny a vody

3. Terciální sektor - služby (obchod, doprava, pošta, školství, zdravotnictví, bydlení, obrana)
4. Kvartární sektor - věda a výzkum

Magický čtyřúhelník - co všechno je třeba sledovat při hodnocení národního hospodářství
jako celku



Magický čtyřůhelník







Velikost a růst HOP | | inflace, stabilita cen |













Zahraniční obchod, jeho bilance Nezaměstnanost, její velikost
a struktura









Hrubý domácí produkt (HDP)

= souhrn statků a služeb vyjádřený v penězích vytvořený za určité období výrobními faktory
(práce, přírodní zdroje, kapitál) na území státu, bez ohledu na to, zda jsou vlastněny občany
státu nebo cizinci.

59
\newpage
Hrubý národní produkt (HNP)

= souhrn statků a služeb vyjádřený v penězích vytvořený za určité období výrobními faktory
ve vlastnictví občanů příslušné země, bez ohledu na to, zda výroba probíhala na území státu
nebo v zahraničí.

/ / W

Český statistický úřad počítá HDP třemi rovnocennými metodami
1. Produkční metoda

2. Výdajová metoda

3. Důchodová metoda

Produkční metoda sčítá přidané hodnoty při výrobě (produkci statků a služeb. Je rozdílem
mezi produkcí a mezi spotřebou

Výdajová metoda sčítá výdaje všech subjektů státu za finální statky a služby, tj. spotřeba
domácností, výdaje vlády, tvorba kapitálu firem a Čisté vývozy

HDP= spotřeba domácnosti + investice soukromých firem + vládní nákupy + čistý vývoz
Důchodová metoda - vychází z fáze rozdělování, kdy každý z účastníků výroby získává svůj
podíl na vyrobených statcích a službách jako odměnu za vynaložení svých výrobních faktorů.
Zaměstnanci dostávají mzdy a platy, majitelé půdy rentu, majitelé kapitálu čistý úrok a zisky.
U této metody do výpočtů nemohou být zahrnovány platby firem jiným firmám. Pro úplnost
nesmíme zapomenout přičíst odpisy a nepřímé daně (především DPH)

HDP= mzdy + renty + zisky + úroky + opotřebení investic + nepřímé daně

HDP = národní důchod + opotřebení investic + nepřímé daně

Běžné ceny - HDP v cenách sledovaného roku neočištěné od vlivu znehodnocení peněz (od
inflace)

Stálé ceny - HDP přepočítané ve stálých cenách zvoleného roku, přepočítáním na stálé ceny
očistíme výsledky od inflačního znehodnocení peněz

Skutečný HDP - pomocí produkční nebo důchodové metody vypočítaný skutečně dosažený
HDP dané ekonomiky

Potenciální HDP - HDP, který by ekonomika dosáhla při využití všech svých výrobních
zdrojů

Čisté ekonomické bohatství - makroekonomický ukazatel čisté ekonomické bohatství (NEW
- Net Economic Welfare) vychází ze známého ukazatele hrubého národního produktu, který
je zvýšen o nelegálně produkované výrobky a služby (šedá a černá ekonomika), je zvýšen o
výrobky produkované ve volném čase pro svou potřebu a na druhé straně je snížen o negativní
dopady hospodářské činnosti na kvalitu našeho života (především dopady na životní prostředí
tzv. externality)

NEW= HNP + šedá a černá ekonomika, produkce pro vlastní potřebu apod. - negativní
dopady hospodářské činnosti

60
\newpage
Negativní externality - špatný vliv výrobců na Životní prostředí ( třetí poškozenou osobou
stranou, které výrobci nic nezaplatí za zničené prostředí, jsou lidé žijící v blízkosti továren,
průmyslových zón a dopravních komunikací.

Pozitivní externality - podnikatelskou činností můžeme někomu přinést 1 užitek, aniž by se o
to třetí osoba jakkoliv zasloužila nebo zaplatila - např. přivedením infrastruktury (dálnice,
elektrifikace, plynofikace)

Šedá ekonomika - je souhrnem ekonomických vztahů, které porušují běžné etické a morální
normy společnosti, ale většinou jsou na hranici zákona a jsou těžko právně postižitelné.
Nejrozšířenějším ekonomickým vztahem šedé ekonomiky je podplácení

Černá ekonomika - je souhrn ekonomických vztahů, které porušují zákony dané země, popř.
zákony mezinárodní. Jedná se o protiprávní ekonomické vztahy, které, pokud jsou odhaleny a
prokázány, jsou trestně postižitelné.

Zahrneme sem hospodářskou kriminalitu jednotlivců (krádeže, zpronevěry, padělání, daňové
úniky apod.), ale 1 hospodářskou trestnou činnost organizovaných zločinců - mafií.

Ekonomika a neziskový sektor

- neziskový sektor může ve společnosti fungovat pouze za předpokladu, že tato společnost je
schopna část svých prostředků (výrobních faktorů 1 hotových produktů - statků a služeb) na
tuto činnost vyčlenit, aniž by to vedlo k sebedestrukci

- zdroje pro neziskový sektor musí společnost hledat ve svém fungujícím ekonomickém
systému (v tržním hospodářství je to podnikatelský = ziskový sektor)

- v neziskovém sektoru jsou typické statky, u kterých nemůžeme jednotlivce vyloučit ze
spotřeby (policie chrání občany, vzdělání získají všechny děti, lékař zachrání život každému,
pouliční lampy svítí všem). Hovoříme o veřejných statcích, které získají příslušníci určité
společnosti bez přímé protihodnoty (neplatí za ně, nebo platí cenu netržní)

- většinu statků a služeb ale musíme zaplatit, chceme-li je spotřebovávat (když nebudeme mít
peníze na dovolenou v zahraničí, tak na ni nepojedeme a život půjde dál) - zde plně působí
tržní prostředí a hovoříme o ziskovém sektoru

Principy rozdělování:

- v ziskovém sektoru rozdělujeme výsledný produkt podle množství, kvality a tržní úspěšnosti
práce

- v neziskovém sektoru rozdělujeme podle potřeb

Státní neziskové organizace

- patří sem především státní školství, zdravotnictví, instituce na ochranu životního prostředí,
kulturních památek, celá oblast státní správy atd.

Nestátní neziskové organizace

- církevní organizace, spolky (dříve občanská sdružení), ústavy, fundace (nadace, nadační
fondy), politické strany

61
\newpage
EU - Evropa začala ve druhé polovině 20. Století zaostávat v dynamice ekonomického
vývoje za ostatními světovými hospodářskými centry (USA, Japonsko, jihovýchodní Asie).
To si uvědomovali čelní představitelé nejvýznamnějších evropských států a již od konce
druhé světové války začali vyvíjet politické aktivity k hospodářskému sjednocení Evropy
Integrace má dvě podoby

- federalistická - státy se sjednotí politicky a ekonomická integrace následuje po politické
integraci, tento proces byl typický pro USA

- funkcionalistická - státy se propojují nejdříve ekonomickými vazbami a úplná politická
integrace je završením tohoto ekonomického integračního procesu. (Evropa)





Integrace (spojování do větších celků) má velký ekonomický význam - umožňuje
zhromadnění výroby a efektivnější využívání zdrojů, působí jak na mikroekonomické úrovní
(spojování firem) tak na makroekonomické úrovni.

Stupně mezinárodní ekonomické integrace:
1. Pásmo volného obchodu

2. Celní unie

3. Společný trh

4. Hospodářská unie

5. Úplná ekonomická unie

EU je v současné době ve fázi přechodu od hospodářské unie k úplné ekonomické unii. EU
má 28 členských států včetně CR. Evropská měnová unie (eurozóna) zavedla jednotnou
měnu - €uro.



Evropská unie má 3 pilíře:

- hospodářská a měnová unie - zavedení společné měny eura
- spolupráce v oblasti justice a vnitřních věcí

- společná zahraniční a bezpečnostní politika

62
\newpage
17. MAKROEKONOMICKÉ VELIČINY

Hrubý domácí produkt (HDP)

= souhrn statků a služeb vyjádřený v penězích vytvořený za určité období výrobními faktory
(práce, přírodní zdroje, kapitál) na území státu, bez ohledu na to, zda jsou vlastněny občany
státu nebo cizinci.

Ekonomická rovnováha - bod, ve kterém jsou si rovny množství nabízeného a poptávaného
zboží/ služby

Agregátní poptávka - poptávka všech subjektů v celém hospodářství

Agregátní nabídka - nabídka všech subjektů v celém hospodářství

- pokud agregátní poptávka převyšuje agregátní nabídku -je to stav podezřelý, protože strana



poptávky strana poptávky má více peněz než je na trhu zboží a toho lze dosáhnout buď
inflačním znehodnocením peněz nebo tak, že poptávající si peníze půjčili a chtějí
spotřebovávat na dluh. Tento stav je založen na nezdravém principu, má negativní dopady na
hospodářství

- převis agregátní poptávky nad agregátní poptávkou - pokud hospodářství vyprodukovalo



více zboží než je samo schopno spotřebovat, má jedinou šanci - jestli firmy vyrobily
produkci natolik efektivně, že je konkurence schopná na zahraničních trzích, mohou zboží
vyvézt a vydělat, pokud však své zboží nerealizují - krachují, propouštějí zaměstnance, stát
má nižší příjmy z daní - hovoříme o hospodářské krizi z nadvýroby

- ekonomické dění ve společnosti je trvalým pohybem, trvalou změnou, stálým kolísáním.
Základní tendence dosažení ekonomické rovnováhy jako nejefektivnějšího stavu je tak
dosahována posloupností dílčích nerovnováh => cyklický vývoj hospodářství



Fáze hospodářského cyklu:

1) expanze (rozvoj, konjunktura, rozmach nabídky 1 poptávky) - domácnostem 1 firmám se
daří, rostou zisky 1 platy, roste spotřeba statků a služeb, stát dostává více daní, takže může
více investovat

2) vrchol (převis nabídky nad poptávkou) - firmy pořád produkují vysokým tempem nové
statky a služby,ovšem strana poptávky už začíná zaostávat

3) krize (recese, deprese, pokles nabídky 1 poptávky) - poptávka se zcela zabrzdila, firmy
mají problémy s prodejem, krachují, zvyšuje se nezaměstnanost, lidé mají existenční
problémy, 1 ti co mají práci raději spoří a neutrácí, což krizi prohlubuje

4) sedlo (dno, vyrovnání, oživení nabídky a poptávky) - žít se musí, takže poptávka nikdy
neklesne na nulu. Firmy minimalizovali náklady a ceny, ty které přežijí recesi se dostanou se
svou nabídkou do souladu s poptávkou.

63
\newpage
Inflace

Ceny na trhu nejsou stabilní. Jeden z významných faktorů, který je ovlivňuje je hodnota
peněz, kterými se ceny měří. Pokud peníze ztrácejí svou hodnotu, hovoříme o inflaci
(znehodnocení peněz, růst cenové hladiny).

Příčiny inflace:

Inflace tažená poptávkou - na trhu je více poptávky při stejném objemu nabídky, příčina
růstu cenové hladiny je na straně poptávky (schodek státního rozpočtu, nákupy na dluh)
Inflace tažená nabídkou - zvyšování cenové hladiny je na straně výrobců (růst nákladů na
výrobu)





Inflace je makroekonomický pojem, je to velmi důležitý ukazatel nejen pro národní
ekonomiku, její ekonomy a politiky, ale 1 pro mezinárodní srovnávání a rozhodování, proto
existují mezinárodně platné postupy, podle kterých se inflace počítá.

Míra inflace se dá vyjádřit ukazatelem:

Míra inflace = cenová hladina (+ - cenová hladina -1 x 100 (%)
cenová hladina (ti



t = určité období (měsíc. rok)

Indexy k vyjádření cenové hladiny:
1. Index spotřebitelských cen (CPI) - cenová hladina je průměrem úrovně cen spotřebních



výrobků a služeb ( hovoříme o spotřebním koši, ve kterém má každá skupina
spotřebovávaných výrobků a služeb určitou váhu, podíl). U daných druhů zboží se sleduje po
celém území státu v pravidelných intervalech pohyb cen

2. Index cen výrobců (PPI) - sleduje se pro různá odvětví a obory, všeobecně se má za to, že



vývoj PPI signalizuje nadcházející změny CPI
3. Deflátor HDP - cenový deflátor HDP se vytvoří jako poměr HDP v běžných cenách k HDP
ve stálých cenách určitého roku. Změnu cenové hladiny tak získáme zprostředkovaně



(implicitně). Protože se jedná o komplexnější zobrazení vývoje cen všech statků a služeb
v ekonomice, je tento ukazatel přesnější než CPI. Na druhou stranu má však jednu nevýhodu -
můžeme ho spočítat pouze zpětně, až když statistický úřad vyjádřil HDP za předcházející rok.

Míry inflace:



- inflace mírná (plíživá) - jednociferná. Lidé nepřestávají věřit penězům, ekonomika běžně
funguje, tempo růstu cen odpovídá tempu růstu a výroby
- inflace pádivá - dvojciferná. Lidé přestávají věřit domácí měně, preferují stabilnější cizí



měny nebo jiné trvalejší hodnoty (zlato, nemovitost atd.). Chod ekonomiky už je narušován,
ekonomická výkonnost klesá
- hyperinflace -trojciferná a větší. Ceny se zvyšují natolik rychle, že peníze přestávají plnit



svou funkci uchovatele hodnot a zprostředkovatele směny, lidé preferují naturální směnu,
ekonomický systém společnosti se úplně rozpadá, nastává chaos a anarchie.

64
\newpage
Deflace - opak inflace, absolutní meziroční pokles cenové hladiny v ekonomice
Desinflace - opakem akcelerující inflace (zpomalující inflace), je pokles tempa růstu
všeobecné cenové hladiny

Nezaměstnanost - vzniká, pokud na trhu práce převyšuje nabídka práce zaměstnanců
poptávku firem.

a) ekonomicky aktivní obyvatelstvo (t1 co pracují nebo aktivně hledají práci) tvoří trh práce
b) ekonomicky neaktivní obyvatelstvo (děti do 15ti let, invalidé, důchodci) mimo trh práce

Nezaměstnanost členíme :
- nezaměstnanost dobrovolná - lidé mají pracovní sílu, ale za nabízenou mzdu nejsou ochotni



pracovat
- nezaměstnanost nedobrovolná - lidé chtějí a potřebují pracovat, aby si zajistili obživu, ale



nemohou odpovídající práci sehnat => stát jim pomáhá situaci řešit :

a) aktivní opatření - rekvalifikace, podpora vzniku nových pracovních míst, zaměstnávání
absolventů škol, daňové úlevy při zaměstnání postižených občanů apod.

b) pasivní opatření - podpora nezaměstnanosti ze státního rozpočtu

Míra přerozdělování:
- dostatečná sociální pomoc, ale zároveň ne tak vysoká, aby nesváděla k jejímu zneužívání
- míra přerozdělování nezatížila neúměrně pracující část populace

Příčiny nezaměstnanosti :
- frikční nezaměstnanost - lidé běžně mění svou práci - přirozený, krátkodobý jev



- strukturální nezaměstnanost - některé odvětví se dostává do útlumu, lidé přicházejí o práci.
Jiné odvětví v národním hospodářství jsou naopak ve fázi rozvoje a nové pracovní síly
potřebují -řešením je rekvalifikace

- cyklická nezaměstnanost - v období krize a sedla dochází k nárůstu nezaměstnanosti a



snižování objemu mezd, naopak v období konjunktury dochází k nárůstu zaměstnanosti a
zvyšování mezd

Měření míry nezaměstnanosti

Míra nezaměstnanosti = nezaměstnaní, aktivně hledající práci x 100 (%)



ekonomicky aktivní obyvatelstvo

P „W

Obecné míry nezaměstnanosti počítá Český statistický úřad nebo Eurostat, také ještě



Ministerstvo práce a sociálních věcí ČR (MPSV)
Na makroekonomické úrovní probíhají jednání v Radě pro sociální dialog (tripartita), kdy



účastníky jsou stát, podnikatelé a odborové organizace (zástupci zaměstnanců). Dlouhodobou

Zo w?

a vysokou nezaměstnanost stát musí řešit.

65
\newpage
Mezinárodní obchod

- jsme malou zemí a je pro nás životně důležité zapojit se do mezinárodního obchodu.

- jsme členy WTO (dříve GATT), což je světová organice působící směrem k odstranění
ochranářských opatření jednotlivých států (cla, dovozní a vývozní kvóty zákazy obchodu
apod.)

- teoretické vysvětlení ekonomických výhod mezinárodního obchodu spočívá v objasnění
absolutních a komparativních výhod :

a) absolutní výhoda - nižší náklady, vyšší produktivita práce vedou k výrobě určitého
výrobku za nižší cenu než v ostatních státech, pro ostatní státy je ekonomičtější zboží dovést
než ho vyrábět doma dráž

b) komparativní výhoda - souvisí s omezeností zdrojů každé země, pro zemi je výhodné
vyrábět produkci, u které dosahuje ve srovnání s ostatními zeměmi co největší absolutní
výhodou a ostatní výrobky dovážet.

66
\newpage
18. BANKOVNICTVÍ

Bankovní soustava - historie do roku 1990)

Do roku 1990 byl jednoúrovňový bankovní systém s výrazným monopolem státní banky
československé.

Bankovní systém od roku 1990- charakteristika, licence, bankovní dohled

Od roku 1990 je bankovní systém dvouúrovňový

1. Centrální banka ČNB - státní instituce, nepodnikatelský subjekt

2. obchodní banky - podnikatelské subjekty

V roce 2004 se bankovní sektor stabilizoval a vstupem do EU v květnu 2004 lze využít
princip jednotné licence, která vychází ze svobody poskytování služeb a svobody usazování,
jakožto jedněch ze základních zásad, na nichž stojí EU.

Jednotná licence představuje další možnosti pro podnikání zahraničních bank z EU v ČR, ale i
našich bank na území členských států EU a ESVO a nemusí procházet licenčním řízením v
hostitelském státě. Princip také mění postavení bankovního dohledu CNB.

ČNB - základní charakteristika

ČNB je centrální bankou českého státu. Má postavení ústředního orgánu státní správy v
oblasti měny, bankovnictví a vydávání obecně závazných předpisů. Je právnickou osobou,
která usměrňuje peněžní trh z měnových hledisek, reguluje činnost bank a spořitelen
bankovními ekonomickými nástroji, emituje peníze a hospodaří podle zásad stanovených
vládou. Její postavení a funkce jsou především měnově řídící a nikoliv podnikatelské, CNB
nepracuje na komerčních principech.

Nejvyšší řídící orgán, cuvernér

Nejvyšším řídícím orgánem je bankovní rada; v čele stojí guvernér CNB - jmenuje jej a
odvolává prezident. V současné době je guvernérem Jiří Rusnok

Základní úkoly centrální banky a její nezávislost



Hlavním cílem ČNB je zabezpečovat stabilitu české měny. Za tímto účelem plní tyto
funkce:

- určuje a prosazuje vnitřní a vnější měnovou politiku

- sleduje množství peněz v oběhu, emituje (vydává) nové peníze a opotřebované nebo
neplatné peníze stahuje z oběhu

- dohlíží nad činností obchodních bank, poskytuje bankám úvěry a ukládá jejich depozita
(banka bank)

67
\newpage
- vede účty státního rozpočtu

- spravuje měnové rezervy ve zlatě a devizách
- obchoduje s cennými papíry

- je vrcholnou institucí bankovního dozoru

Učinnost měnové politiky centrální banky je přímo úměrná její nezávislosti (především na
vládě)

Přímé nástroje - základní charakteristika

CNB disponuje řadou nástrojů, pomocí kterých prosazuje své cíle a měnovou politiku. Tyto
nástroje můžeme rozdělit na přímé (administrativní, omezující volné tržní hospodářství) a
nepřímé, které využívají tržních zákonů a plošně působí na ostatní subjekty finančního trhu.

Pravidla likvidity, úvěrové kontingenty, povinné vklady, doporučení, výzvy, dohody

Přímé nástroje mají velký vliv na finanční hospodářství, proto jich centrální banka využívá
jen výjimečně a na přechodnou dobu. K těmto nástrojům patří:

e | Pravidla likvidity - centrální banka určuje obchodním bankám, jaký mají mít vztah
mezi aktivy a pasivy. Patří sem například ukazatel kapitálové přiměřenosti. Aktivní
operace banky mohou činit maximálně 125% vlastního kapitálu, 8% z hlediska
pokladní hotovosti k aktivům.

« | Povinné vklady - povinné vedení běžných účtů státních institucí u centrální banky.

« | Úvěrové kontingenty - určení limitních úvěrů a úvěrových stropů. Patří mezi velmi
razantní přímé nástroje.

Nepřímé nástroje - základní charakteristika

využívají tržních zákonů a plošně působí na ostatní subjekty finančního trhu. Využívají
tržních zákonů a plošně působí na ostatní subjekty finančního trhu. využívají tržních zákonů a
plošně působí na ostatní subjekty finančního trhu

Povinné minimální rezervy, operace na volném trhu, diskontní sazba, lombardní úvěr,
konverze měny, swapové obchody

Mezi nepřímé nástroje centrální banky patří:

e © Diskontní sazba - úroková sazba, za kterou si mohou komerční banky půjčit peníze
od centrální banky. Centrální banka výší této sazby ovlivňuje peněžní zásobu
komerčních bank, podle které určují banky výši poskytovaných úvěrů. Diskontní
sazba představuje dolní mez krátkodobých úrokových sazeb na peněžním trhu.
Zvýšení diskontní sazby pomáhá snižovat inflaci, její snížení naopak vede k

expanzivnímu navyšování zásoby peněz.

68
\newpage
« © Repo sazba - Při repo operacích centrální banka přijímá od bank přebytečnou
likviditu a na oplátku jim předává dohodnuté cenné papíry. Obě strany se zároveň
zavazují, že po uplynutí doby splatnosti centrální banka jako dlužník vrátí věřitelské
bance zapůjčenou jistinu zvýšenou o dohodnutý úrok a věřitelská banka vrátí
poskytnuté cenné papíry. Základní doba trvání těchto operací je 14 dní, úrok při této
operaci je nazýván repo sazbou (refinanční sazbou). ČNB podle americké aukční
procedury přijímá přednostně nabídky bank požadující nejnižší úrokovou sazbu. Při
vyšší repo sazbě dochází ke zdražení peněz, banky si půjčují méně, naopak jsou
ochotné poskytnout samy své prostředky centrální bance, dochází ke stahování peněz
z oběhu a zmírňování inflace.

« © Lombardní sazba - představuje úrokovou sazbu při operacích, kdy si banky vypůjčují
likviditu oproti zástavě cenných papírů. V současné době je vzhledem k trvalému
přebytku likvidity bank tato možnost využívána minimálně. Lombardní sazba
představuje horní mez krátkodobých úrokových sazeb na peněžním trhu - je vyšší než
diskontní sazba nebo repo sazba. Zvýšení lombardní sazby má za následek menší
půjčky obchodních bank, peníze v oběhu jsou omezeny, což vede ke snižování inflace.

« | Operace na otevřeném trhu - Centrální banka nakupuje a prodává na volném
peněžním trhu státní cenné papíry (státní pokladniční poukázky, popř. státní
dluhopisy). Obchodní banky si mohou půjčit u centrální banky peníze tím, že jí prodají
své cenné papíry a dohodnou se na budoucím zpětném odkupu (repo obchodě), nebo
může jít o obchod bez budoucích ujednání (tzv. promptní obchod). Pokud centrální
banka prodává státní cenné papíry, dochází k odčerpávání peněz z obchodních bank a
zpomalení oběhu peněz. Tato restriktivní monetární politika vede ke snížení inflace.
Naopak při expanzivní politice, tj. nákupu cenných papírů centrální bankou, dochází k
nárůstu peněžní zásoby v oběhu. Emise státních cenných papírů slouží 1 ke krytí
přechodného nedostatku peněz ve státní pokladně nebo ke krytí schodku státního
rozpočtu.

e | Povinné minimální rezervy - Centrální banka předepisuje obchodním bankám určité
procento z vkladů, které si u ní musí uložit ve formě neúročené povinné minimální
rezervy. Tyto peníze jsou mimo oběh a působí protiinflačně. Výše povinných
minimálních rezerv ovlivňuje úvěrovou kapacitu obchodních bank, důsledkem toho i
výši úrokových sazeb.

e« | Konverze a swapy - Centrální banka nakupuje a prodává cizí měny za koruny
obchodním bankám. Tyto operace mají vliv na měnové kurzy. Dochází ke:

« © konverzi - promptnímu obchodu v aktuálním kurzu bez následných zpětných operací

e © swapu - kombinaci promptního obchodu s následnou zpětnou operací -
prodává/nakupuje se za aktuální kurz a budoucí zpětný odkup/prodej se odehrává za
předem dohodnutého kurzu

Při prodeji deviz centrální bankou dochází ke stahování českých korun z oběhu a zpomaluje
se oběh peněz, při nákupu deviz se české koruny přilévají do oběhu.

69
\newpage
Banky obchodní - základní charakteristika

Obchodní banky jsou podnikatelskými subjekty, které podnikají za účelem dosažení zisku.
Poskytují tyto služby: depozitní (vkladové) a úvěrové operace, převody peněz a další služby
zprostředkovatelské povahy.

Zisk banky je tvořen čistými bankovními úroky (získané úroky mínus úroky vydané) a
poplatky za služby. Současnou tendencí je poskytování stále většího rozpětí bankovních
služeb.

Operace obchodních bank

Bilance banky - aktiva a pasiva

VKLADY

1. pasivní operace - banka přijímá peníze, je v dlužnické pozici

- netermínované vklady (úročeny velmi nízkým procentem)

- termínované vklady (termínované účty, vkladní knížky)

nebo z hlediska měny:

- korunové (v KČ)

- devizové (V cizí měně)

ÚVĚRY

Banka zde sleduje dva cíle: výnosnost a návratnost úvěru

2. aktivní operace - banka poskytuje úvěry, vystupuje v roli věřitele

- úvěry od ČNB - refinanční operace za repo sazbu, lombardní úvěr, Nouzový úvěr

- úvěry od ostatních bank - korunové úvěry např za sazbu Pribor

- emise bankovních obligací - emise bývá v mld. Objemech a je dlouhodobým zdrojem.

- emise hypotéčních zástavních listů

70
\newpage
Pojišťovnictví

Pojišťovnictví můžeme charakterizovat jako specifický ekonomický obr řešící minimalizaci
rizik ekonomických i neekonomických činností člověka.

Smyslem pojišťovnictví je zabezpečit pojištěného pro případ nahodilých nepříznivých
událostí. Samotné pojištění nemůže zabránit ztrátám, ale může zmírnit jejich následky

Pojišťovny

Dnes působí na českém trhu kolem 50 pojišťovacích ústavů, VIG RE zajišťovna, a.s. jako
první česká zajišťovna.

Právní formy - akciová společnost (nejčastější forma)

- družstevní organizace

Zajišťovny

Je právnická osoba, která přebírá na základě smlouvy jistou část rizik pojištění pojišťoven a
zajišťoven. Tento typ pojištění sjednávají pojišťovny pro určitá pojištění nebo celé portfolio
pojištění, většinou formou podílové spoluúčasti na pojistném 1 škodách.

Státní dozor nad pojišťovnictvím vykonává Ministerstvo financí

Nemůže dělat každý. Je třeba vysoký kapitál v minimální výši 1 000 000 000 Kč

Pojištění, riziková událost - koho se týká

pro jednotlivce: pří úrazu, při dožití určitého věku, při léčení v cizině, \ldots

pro kolektiv: při požáru, při odcizení majetku organizace, \ldots

pro hospodářství: pomáhá zajišťovat plynulý chod ekonomiky, omezuje počet bankrotů, \ldots
Členění pojištění z hlediska povinnosti uzavření

a) povinné pojištění - jsou zákonem uložena a to firmám i osobám, sledují zajištění
sociálních jistot lidí a zabezpeční proti škodám způsobenými jiným osobami na provozu
motorových vozidel

1. zákonné sociální pojištění osob (dle zákona o sociálním pojištění)

- správcem tohoto pojištění je správa sociálního zabezpečení

- z tohoto pojištění jsou vypláceny nemocenské dávky, důchody, podpory v nezaměstnanosti
2. Zákonné zdravotní pojištění osob (dle zákona o zdravotním pojištění)

- toto pojištění zpracovávají zdravotní pojišťovny

Ji
\newpage
3. Zákonné pojištění odpovědnosti za škodu z provozu motorového vozidla
- od roku 2000 toto pojištění poskytují vybrané největší pojišťovny
4. Zákonné pojištění pracovních úrazů a nemocí z povolání zaměstnanců

- tato pojištění jsou povinní uzavírat zaměstnavatelé pro své zaměstnance (Česká pojišťovna,
Kooperativa)

b) Dobrovolná pojištění
- uzavírá se na komerční bázi
- nejsou povinná

- klient, který má zájem pojistit se proti určitým rizikům si vybírá z pestré nabídky
komerčních pojišťoven.

Členění pojistných služeb

1. Životní pojištění - toto pojištění fyzických osob pomáhá chránit tyto osoby a jejich rodiny
proti rizikům těžkých úrazů, jejich trvalých následků, vážných nemocí a následné ztráty
příjmu, případně při úmrtí pojištěného pomáhá nahradit zdroj příjmu jeho pozůstalým

- rizikové (za nižší pojistné poskytuje vysokou pojistnou ochranu, nedojde-li k pojistné
události zamká bez náhrady)

- rezervotvorné (pojistné je vyšší, protože obsahuje spořící složku. Pojistná částka plus podíly
na zisku jsou vypláceny při pojistné události nebo na konci sjednané pojistné doby

2. Neživotní pojištění - zahrnuje především pojištění movitostí a nemovitostí.

Pojmy - pojištěný, pojistné plnění, pojistná událost, pojistitel, pojistné, pojistník,
pojistná částka, pojistka, pojistná smlouva, pojistný zájem, obmyšlená osoba

Pojistitel - pojišťovna, má právo pojistné a povinnost vyplatit pojistné plnění v případě
pojistné události

Pojistník - subjekt, který uzavřel s pojistitelem pojistnou smlouvu. Má povinnost platit
pojistné

Pojištěný - subjekt, na jehož majetek, život, zdraví nebo odpovědnost za škodu se pojištění
vztahuje. Má právo na pojistné plnění

Obmyšlená osoba - subjekt, kterému v případě pojištění ve prospěch jiné osoby vznikne v
případě pojistné události právo na pojistné plnění Pojistná událost - je nahodilá událost, při
které vzniká nárok na pojistné plnění. Její nahlášení (a případné doložení, že se jedná
skutečně o pojistnou událost) je povinností pojištěného

72
\newpage
Pojistné - představuje cenu za poskytnutí pojistné ochrany.

Pojistné plnění - je částka, kterou pojišťovna při pojištění vyplácí v případě pojistné události.
Pojistná částka - je nejvyšší finanční částka, jež může být vyplacena, dojde-li k pojistné
události. Hodnota této částky je sjednána při uzavírání pojistné smlouvy a je v této smlouvě
uvedena.

Pojistka je písemným potvrzení pojistitele o uzavření pojistné smlouvy. U některých
pojišťoven je vystavováno tzv. potvrzení o akceptaci pojištění, tedy o přijetí rizika do
pojištění.

Pojistná smlouva - je smlouvou o finančních službách, ve které se pojistitel zavazuje v
případě vzniku pojistné události poskytnout pojistníkovi nebo třetí osobě ve sjednaném
rozsahu pojistné plnění a pojistník se zavazuje platit pojistiteli pojistné.

Pojistný zájem - je oprávněná potřeba ochrany před následky pojistné události. Pojistník má

pojistný zájem na vlastním životě a zdraví. Pojistník má pojistný zájem na vlastním
majetku.

Náležitosti pojistné smlouvy

Základní náležitosti pojistné smlouvy

- smluvní strany (pojistník, pojistitel, pojištěný)

- předmět pojištění (na co smlouva je - na úraz, na majetek\ldots)

- začátek platnosti pojistné smlouvy, někdy 1 konec

- výše pojistného (kolik a jak často platí klient)

- výše pojistného plnění (kolik pojišťovna zaplatí a podmínky, za kterých zaplatí)

- všeobecné podmínky

73
\newpage
19. BANKOVNÍ SLUŽBY

Úvěr - základní charakteristika

Úvěr je podobně jako zápůjčka formou dočasného postoupení peněžních prostředků věřitelem
na principu návratnosti dlužníkovi, který je ochoten za tuto půjčku po uplynutí nebo ještě v
průběhu doby splatnosti zaplatit určitý úrok.

Pojmy - úrok, úroková sazba, jistina, úrokové období, RPSN, likvidita, zadlužení,
předlužení, dluhová past, konsolidace úvěrů, bonita, registr dlužníků, oddlužení,
exekuce

Urok je peněžitá odměna za půjčení peněz.

Uroková sazba je procentní vyjádření zvýšení půjčené částky za určité časové období.
Uroková sazba určuje kolik z jistiny musí dlužník za předem smluvně stanovenou dobu
věřiteli za půjčku či úvěr zaplatit.

Jistina je základní peněžní částka, která byla půjčena nebo která tvořila vklad.
Úrokové období - Období, během kterého jsou připisovány úroky.

RPSN (roční procentní sazba nákladů) je číslo, které má umožnit spotřebiteli lépe vyhodnotit
výhodnost nebo nevýhodnost poskytovaného úvěru.

Likvidita - Likviditu lze definovat také jako míru schopnosti podniku přeměnit svá aktiva na
peněžní prostředky a těmi krýt včas, v požadované podobě a na požadovaném místě všechny
své splatné závazky, a to při minimálních nákladech.

Zadlužení Zadluženost je ekonomický pojem, který označuje skutečnost, že podnik používá
pro financování svých aktiv cizí kapitál.

Předlužení O předlužení jde tehdy, jestliže osoba má více věřitelů a jestliže její splatné
závazky jsou vyšší než její majetek; do ocenění dlužníkova majetku se zahrne i očekávaný
výnos z pokračující podnikatelské činnosti, lze-li příjem převyšující náklady při pokračování
podnikatelské činnosti důvodně předpokládat.

Zadlužení není předlužení. Zatímco zadlužení je normální, zdravé, umožňuje si pořídit nové
věci, vybavení a pomáhá ekonomice, předlužení je negativní patologická forma zadlužení,
která vede k ekonomickému zhroucení dlužníka.

Dluhová past je známější a používanější termín pro dluhovou spirálu. Jedná se o stav, kdy
není jednotlivec anebo 1 celá rodina schopna splácet závazky věřitelům. Což vede k penále,
anebo dalším závazkům formou braní si dalších úvěrů.

Konsolidace úvěrů, též konsolidace půjček, označuje sloučení více půjček do jedné.

Bonita je převrácená hodnota úvěrového rizika a vyjadřuje důvěryhodnost ekonomického
subjektu (firmy, jednotlivce, ale 1 obce nebo státu) na finančním trhu.

74
\newpage
Registr dlužníků (úvěrový registr) je databáze evidující zejména osoby a organizace, které
Jsou pozadu ve splácení svých závazků.

Oddlužení je zákonným prostředkem, kterým mohou dlužníci především fyzické osoby řešit
situaci, kdy mají více věřitelů a nedokáží plnit své splatné závazky. Oddlužením se řeší
úpadek fyzické či právnické osoby nepodnikatele.

Exekuce je nástroj, kterým je možné po někom, kdo dluží, domoci zaplacení dluhu (nebo i
jiných povinností, které neplní - nejen těch finančních), který dlužník dobrovolně neuhradil,
ačkoli měl a ačkoli mu takovou povinnost uložil (nejčastěji) soud v soudním rozhodnutí.

Druhy úvěrů podle splatnosti - kontokorentní, eskontní, akceptační, revolvingový,
lombardní, hypoteční úvěr, emisní úvěr, spotřební půjčky občanům

1. krátkodobé se splatností do 1 roku:

- kontokorentní úvěr - kombinace běžného účtu s možností čerpat krátkodobý úvěr do výše
úvěrového limitu stanoveného ve smlouvě o zřízení kontokorentního účtu

- eskontní úvěr - souvisí s eskontem (odkoupením) směnky klienta před dobou splatnosti
bankou. Nesplatí-l1 dlužník bance směnku v termínu splatnosti, žádá banka úhradu od
posledního majitele směnky (klienta, který směnku prodal)

- akceptační úvěr - banka neposkytne klientovi přímo peníze, ale akceptuje cizí směnku
vystavenou klientem - příjemcem akceptačního úvěru a tím se stává hlavním směnečným
dlužníkem. Akceptační úvěr dává banka jen nejlepším a ověřeným klientům

- revolvingový úvěr - banka umožňuje klientovi opakované čerpání úvěru. Základní
podmínkou je předešlé splacení úvěru.

- lombardní úvěr - úvěr jištěný zástavou movité věci (cenné papíry, zboží \ldots)

2. střednědobé a dlouhodobé - se splatností od 1 do 10 let

- hypoteční úvěr - jištěný hypotékou (zástavou nemovitostí)
- emisní úvěr - spojená s emisí dlouhodobých cenných papírů, např podnikových obligací
- spotřební půjčky občanům

Speciální formy úvěru - faktoring, forfaiting

Faktoring - odkup krátkodobých pohledávek (do 1 roku)
Forfaiting - odkup dlouhodobých pohledávek (nad 1 rok)

Druhy úvěru podle poskytované měny
Jsou úvěry v domácí nebo cizí měně (devizové)

Devizový úvěr

- Jde o požadavek komitenta o úvěr u zahraniční banky. Tyto zahraniční banky mají většinou
nižší úrokové sazby, na druhé straně však hrozí kursové riziko. Nehledě k tomu, že zahraniční
banky vyžadují záruku od českých bank.

- Také naše banky dávají účelové devizové úvěry. Vlastní postup spočívá v tom, že
zahraničnímu výstavci zaplatí česká banka z úvěru, který získá u cizí banky v tzv. úvěrová
linka. Zahraniční banky tak podporují vlastní ekonomiku.

7
\newpage
Druhy úvěru podle zajištěnosti

WWT?

Nezajištěný úvěr je z pohledu bamky velmi rizikový. Proto je tedy i dražší. Tento typ úvěru
se vyskytuje velmi vzácně a je poskytován pouze těm nejspolehlivějším klientům (VIP
klienti).

Zajištěný úvěr

Většina věřitelů chce mít 100 % jistotu, že své peníze dostane zpět. Proto požadují nějaké
zajištění. Tím se znatelně sníží riziko toho, že by se o půjčené peníze přišlo. Jinak věřitel
finance neposkytne. Banka zajišťuje své úvěry zástavou majetku dlužníka (movitý i
nemovitý). V případě, že pak dlužník řádně nesplácí, zástava bude bankou zabavena. Ručí se
formou zástavního práva k věci, popřípadě jde o zajištění další osobou. Zajištěné úvěry bývají
levnější než nezajištěné (nižší úroky). Navíc si věřitel může nechat prověřit klientovu bonitu
nebo stanovit pro úvěr limit. Mezi zajištěné úvěry může patřit úvěr hypotéční, spotřebitelský
nebo nebankovní úvěr na vyšší částku. Stejně tak sem spadají 1 neúčelové půjčky nebo
krátkodobý úvěr (zajištěný zástavou movitých věcí). Méně známé jsou lombardní a eskontní
úvěry, kdy je úvěr zajištěn cennými papíry, šperky či drahými kovy.

Druhy úvěrů podle způsobu čerpání

- jednorázové
- postupné
- před zápisem zástavního práva

Druhy úvěrů podle účelu a subjektů
1. úvěry pro podnikatelské účely

a. účelově - v úvěr. smlouvě je uvedeno, na co je úvěr poskytován = zásoby, pohledávky
b. neúčelově - otevření úvěr. Linky

2. úvěry občanům

- na nákup nemovitého 1 movitého majetku, užívají se zde krátko-, dlouho- i střednědobé
úvěry

- účelově zaměřené - na dům, rekonstrukci, nákup auta = dlohodobá spotřeba

- osobní - na překlenutí dočasného nedostatku financí, převodem na BŮ nebo v hotovosti
- kontokorentní - stanoven úvěrový rámec, podle kterého můžou občané čerpat úvěr na
bězném, na sporožirovém účtě \ldots

- úvěrové karty

3. mezibankovní úvěry

Banka může mít krátkodobě nevyrovnanou peněžní pozici z těchto důvodů:
- větší výběr vkladů z účtů zákazníků banky
- zákazníci banky včas nesplácí bance úvěry

76
\newpage
Dlouhodobě nevyrovnaná pozice:
- klienti u dané banky převážně čerpají úvěry
- klienti převážně ukládají vklady a banka to půjčuje jiným bankám

4. další úvěry
Hlavně obcím a městům, jedná se o středně a dlouhodobé investiční úvěry, krátkodobé
provozní úvěry na překlenutí nedostatku finančních prostředků.

Cíle banky při poskytování úvěru

1. Výnosnost - banka musí stanovit takové úroky z úvěru, aby vydělala. Úrok může být
stanoven pevný nebo pohyblivý.
2. Návratnost - banka se musí zajistit proti případnému nesplacení závazku dlužníka.

Postup při poskytnutí úvěru

Před poskytnutím úvěru si banka ověřuje Bonitu klienta, likviditu jeho majetku, podnikatelský
záměr.

Toto však ještě nestačí, ve většině případů chce banka určité záruky. jištění úvěru, a to ve
formě:

- zástavy nemovitosti (hypotéční úvěr)

- zástavy movitosti - cenné papíry, stroje, zásoby, \ldots (lombardní úvěr)

- ručitelé - ručit mohou 1 podnikatelské subjekty za podnikatelské úvěry, ale takové subjekty
se těžko shánějí

- vinkulace vkladu - zablokování vkladu dlužníka na účtu či na knížce ve prospěch věřitele
pro případ nesplacení

- postoupení pohledávek apod.

Uvěrová smlouva

obsah: název, jméno klienta, název banky, účel a výše úvěru, způsob čerpání, zajištění úvěru,
postup v případě neschopnosti klienta platit splátky, práva a povinnosti banky 1 klienta,
datum, místo, rizika a platné podpisy obou stran + přílohy

Doba splatnosti úvěru se počítá ode dne prvního čerpání úvěru

Kontrola dodržování podmínek úvěrové smlouvy
- sleduje se dodržování výše a termínů splátek úvěrů a úroků, pokud se nedodržují je dlužná
částka převedena na účet úvěrů neuhrazených ve lhůtě a je úročena vyšší sankční sazbou.

Další služby, které poskytují banky
Založení a vedení účtu - Běžné účty, termínované účty, devizové účty atd.

Bezhotovostní platební styk:

v tuzemsku - příkaz k úhradě - trvalý příkaz k úhradě
- příkaz k inkasu - trvalý příkaz k inkasu

JI
\newpage