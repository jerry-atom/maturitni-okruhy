\chapter{Bankovnictví}

\section*{Bankovní soustava}

\paragraph{Historie}
Do roku 1990 byl jednoúrovňový bankovní systém s výrazným monopolem státní banky československé.

Od roku 1990 je bankovní systém dvouúrovňový
\begin{enumerate}
    \item centrální banka ČNB -- státní instituce, nepodnikatelský subjekt
    \item obchodní banky -- podnikatelské subjekty
\end{enumerate}

V roce 2004 se bankovní sektor stabilizoval a vstupem do EU v květnu 2004 lze využít princip jednotné licence, která vychází ze svobody poskytování služeb a svobody usazování, jakožto jedněch ze základních zásad, na nichž stojí EU.

Jednotná licence představuje další možnosti pro podnikání zahraničních bank z EU v ČR, ale i našich bank na území členských států EU a ESVO a nemusí procházet licenčním řízením v hostitelském státě. Princip také mění postavení bankovního dohledu CNB.

\section*{Česká Národní Banka}

\paragraph{Základní charakteristika}
ČNB je centrální bankou českého státu. Má postavení ústředního orgánu státní správy v oblasti měny, bankovnictví a vydávání obecně závazných předpisů. Je právnickou osobou, která usměrňuje peněžní trh z měnových hledisek, reguluje činnost bank a spořitelen bankovními ekonomickými nástroji, emituje peníze a hospodaří podle zásad stanovených vládou. Její postavení a funkce jsou především měnově řídící a nikoliv podnikatelské, CNB nepracuje na komerčních principech.

\paragraph{Nejvyšší řídící orgán, guvernér}
Nejvyšším řídícím orgánem je bankovní rada; v čele stojí guvernér CNB - jmenuje jej a odvolává prezident. V současné době je guvernérem Jiří Rusnok.

\paragraph{Základní úkoly centrální banky a její nezávislost}
Hlavním cílem ČNB je zabezpečovat stabilitu české měny. Za tímto účelem plní tyto funkce:
\begin{itemize}
    \item určuje a prosazuje vnitřní a vnější měnovou politiku
    \item sleduje množství peněz v oběhu, emituje (vydává) nové peníze a opotřebované nebo neplatné peníze stahuje z oběhu
    \item dohlíží nad činností obchodních bank, poskytuje bankám úvěry a ukládá jejich depozita (banka bank)
    \item vede účty státního rozpočtu
    \item spravuje měnové rezervy ve zlatě a devizách
    \item obchoduje s cennými papíry
    \item je vrcholnou institucí bankovního dozoru
\end{itemize}

Učinnost měnové politiky centrální banky je přímo úměrná její nezávislosti (především na vládě)

\paragraph{Přímé nástroje - základní charakteristika}
CNB disponuje řadou nástrojů, pomocí kterých prosazuje své cíle a měnovou politiku. Tyto nástroje můžeme rozdělit na přímé (administrativní, omezující volné tržní hospodářství) a nepřímé, které využívají tržních zákonů a plošně působí na ostatní subjekty finančního trhu.

Pravidla likvidity, úvěrové kontingenty, povinné vklady, doporučení, výzvy, dohody.

Přímé nástroje mají velký vliv na finanční hospodářství, proto jich centrální banka využívá jen výjimečně a na přechodnou dobu. K těmto nástrojům patří:
\begin{description}
    \item[Pravidla likvidity] -- centrální banka určuje obchodním bankám, jaký mají mít vztah mezi aktivy a pasivy. Patří sem například ukazatel kapitálové přiměřenosti. Aktivní operace banky mohou činit maximálně 125\% vlastního kapitálu, 8\% z hlediska pokladní hotovosti k aktivům.
    \item[Povinné vklady] -- povinné vedení běžných účtů státních institucí u centrální banky.
    \item[Úvěrové kontingenty] -- určení limitních úvěrů a úvěrových stropů. Patří mezi velmi razantní přímé nástroje.
\end{description}

\paragraph{Nepřímé nástroje - základní charakteristika}

Využívají tržních zákonů a plošně působí na ostatní subjekty finančního trhu. Využívají tržních zákonů a plošně působí na ostatní subjekty finančního trhu. využívají tržních zákonů a plošně působí na ostatní subjekty finančního trhu

Povinné minimální rezervy, operace na volném trhu, diskontní sazba, lombardní úvěr, konverze měny, swapové obchody

Mezi nepřímé nástroje centrální banky patří:
\begin{description}
    \item[Diskontní sazba] -- úroková sazba, za kterou si mohou komerční banky půjčit peníze od centrální banky. Centrální banka výší této sazby ovlivňuje peněžní zásobu komerčních bank, podle které určují banky výši poskytovaných úvěrů. Diskontní sazba představuje dolní mez krátkodobých úrokových sazeb na peněžním trhu. Zvýšení diskontní sazby pomáhá snižovat inflaci, její snížení naopak vede k expanzivnímu navyšování zásoby peněz.
    \item[Repo sazba] -- Při repo operacích centrální banka přijímá od bank přebytečnou likviditu a na oplátku jim předává dohodnuté cenné papíry. Obě strany se zároveň zavazují, že po uplynutí doby splatnosti centrální banka jako dlužník vrátí věřitelské bance zapůjčenou jistinu zvýšenou o dohodnutý úrok a věřitelská banka vrátí poskytnuté cenné papíry. Základní doba trvání těchto operací je 14 dní, úrok při této operaci je nazýván repo sazbou (refinanční sazbou). ČNB podle americké aukční procedury přijímá přednostně nabídky bank požadující nejnižší úrokovou sazbu. Při vyšší repo sazbě dochází ke zdražení peněz, banky si půjčují méně, naopak jsou ochotné poskytnout samy své prostředky centrální bance, dochází ke stahování peněz z oběhu a zmírňování inflace.
    \item[Lombardní sazba] -- představuje úrokovou sazbu při operacích, kdy si banky vypůjčují likviditu oproti zástavě cenných papírů. V současné době je vzhledem k trvalému přebytku likvidity bank tato možnost využívána minimálně. Lombardní sazba představuje horní mez krátkodobých úrokových sazeb na peněžním trhu - je vyšší než diskontní sazba nebo repo sazba. Zvýšení lombardní sazby má za následek menší půjčky obchodních bank, peníze v oběhu jsou omezeny, což vede ke snižování inflace.
    \item[Operace na otevřeném trhu] -- Centrální banka nakupuje a prodává na volném peněžním trhu státní cenné papíry (státní pokladniční poukázky, popř. státní dluhopisy). Obchodní banky si mohou půjčit u centrální banky peníze tím, že jí prodají své cenné papíry a dohodnou se na budoucím zpětném odkupu (repo obchodě), nebo může jít o obchod bez budoucích ujednání (tzv. promptní obchod). Pokud centrální banka prodává státní cenné papíry, dochází k odčerpávání peněz z obchodních bank a zpomalení oběhu peněz. Tato restriktivní monetární politika vede ke snížení inflace. Naopak při expanzivní politice, tj. nákupu cenných papírů centrální bankou, dochází k nárůstu peněžní zásoby v oběhu. Emise státních cenných papírů slouží 1 ke krytí přechodného nedostatku peněz ve státní pokladně nebo ke krytí schodku státního rozpočtu.
    \item[Povinné minimální rezervy] -- Centrální banka předepisuje obchodním bankám určité procento z vkladů, které si u ní musí uložit ve formě neúročené povinné minimální rezervy. Tyto peníze jsou mimo oběh a působí protiinflačně. Výše povinných minimálních rezerv ovlivňuje úvěrovou kapacitu obchodních bank, důsledkem toho i výši úrokových sazeb.
    \item[Konverze a swapy] -- Centrální banka nakupuje a prodává cizí měny za koruny obchodním bankám. Tyto operace mají vliv na měnové kurzy. Dochází ke:
        \begin{itemize}
            \item konverzi -- promptnímu obchodu v aktuálním kurzu bez následných zpětných operací
            \item swapu -- kombinaci promptního obchodu s následnou zpětnou operací - prodává/nakupuje se za aktuální kurz a budoucí zpětný odkup/prodej se odehrává za předem dohodnutého kurzu
        \end{itemize}
\end{description}        

Při prodeji deviz centrální bankou dochází ke stahování českých korun z oběhu a zpomaluje se oběh peněz, při nákupu deviz se české koruny přilévají do oběhu.

\section*{Banky obchodní}

Obchodní banky jsou podnikatelskými subjekty, které podnikají za účelem dosažení zisku. Poskytují tyto služby: depozitní (vkladové) a úvěrové operace, převody peněz a další služby zprostředkovatelské povahy.

Zisk banky je tvořen čistými bankovními úroky (získané úroky mínus úroky vydané) a poplatky za služby. Současnou tendencí je poskytování stále většího rozpětí bankovních služeb.

\subsection*{Operace obchodních bank}

\textbf{Bilance banky - aktiva a pasiva}

\paragraph{VKLADY}
\begin{enumerate}
    \item pasivní operace - banka přijímá peníze, je v dlužnické pozici
        \begin{itemize}
            \item netermínované vklady (úročeny velmi nízkým procentem)
            \item termínované vklady (termínované účty, vkladní knížky)
        \end{itemize}
        nebo z hlediska měny:
        \begin{itemize}
            \item korunové (v Kč)
            \item devizové (V cizí měně)
        \end{itemize}
\end{enumerate}

\paragraph{ÚVĚRY}
Banka zde sleduje dva cíle: výnosnost a návratnost úvěru
\begin{enumerate}
    \setcounter{enumi}{1}
    \item aktivní operace - banka poskytuje úvěry, vystupuje v roli věřitele
        \begin{itemize}
            \item úvěry od ČNB - refinanční operace za repo sazbu, lombardní úvěr, Nouzový úvěr
            \item úvěry od ostatních bank - korunové úvěry např za sazbu Pribor
            \item emise bankovních obligací - emise bývá v mld. Objemech a je dlouhodobým zdrojem.
            \item emise hypotéčních zástavních listů
        \end{itemize}
\end{enumerate}

\section*{Pojišťovnictví}

Pojišťovnictví můžeme charakterizovat jako specifický ekonomický obr řešící minimalizaci rizik ekonomických i neekonomických činností člověka.

Smyslem pojišťovnictví je zabezpečit pojištěného pro případ nahodilých nepříznivých událostí. Samotné pojištění nemůže zabránit ztrátám, ale může zmírnit jejich následky

\subsection*{Pojišťovny}

Dnes působí na českém trhu kolem 50 pojišťovacích ústavů, VIG RE zajišťovna, a.s. jako první česká zajišťovna.

Právní formy:
\begin{itemize}
    \item akciová společnost (nejčastější forma)
    \item družstevní organizace
\end{itemize}

\subsection*{Zajišťovny}

Je právnická osoba, která přebírá na základě smlouvy jistou část rizik pojištění pojišťoven a zajišťoven. Tento typ pojištění sjednávají pojišťovny pro určitá pojištění nebo celé portfolio pojištění, většinou formou podílové spoluúčasti na pojistném i škodách.

Státní dozor nad pojišťovnictvím vykonává Ministerstvo financí.

Nemůže dělat každý. Je třeba vysoký kapitál v minimální výši 1000 000 000 Kč

\subsection*{Pojištění, riziková událost - koho se týká}

Pro jednotlivce: pří úrazu, při dožití určitého věku, při léčení v cizině, \ldots

pro kolektiv: při požáru, při odcizení majetku organizace, \ldots

pro hospodářství: pomáhá zajišťovat plynulý chod ekonomiky, omezuje počet bankrotů, \ldots

\subsection*{Členění pojištění z hlediska povinnosti uzavření}
\begin{enumerate}
    \item povinné pojištění -- jsou zákonem uložena a to firmám i osobám, sledují zajištění sociálních jistot lidí a zabezpeční proti škodám způsobenými jiným osobami na provozu motorových vozidel
        \begin{enumerate}
            \item zákonné sociální pojištění osob (dle zákona o sociálním pojištění)
                \begin{itemize}
                    \item správcem tohoto pojištění je správa sociálního zabezpečení                    
                    \item z tohoto pojištění jsou vypláceny nemocenské dávky, důchody, podpory v nezaměstnanosti
                \end{itemize}
            \item Zákonné zdravotní pojištění osob (dle zákona o zdravotním pojištění)
                \begin{itemize}
                    \item toto pojištění zpracovávají zdravotní pojišťovny
                \end{itemize}
            \item Zákonné pojištění odpovědnosti za škodu z provozu motorového vozidla
                \begin{itemize}
                    \item od roku 2000 toto pojištění poskytují vybrané největší pojišťovny
                \end{itemize}            
            \item Zákonné pojištění pracovních úrazů a nemocí z povolání zaměstnanců
                \begin{itemize}
                    \item tato pojištění jsou povinní uzavírat zaměstnavatelé pro své zaměstnance (Česká pojišťovna, Kooperativa)
                \end{itemize}            
        \end{enumerate}
    \item Dobrovolná pojištění
        \begin{itemize}
            \item uzavírá se na komerční bázi
            \item nejsou povinná
            \item klient, který má zájem pojistit se proti určitým rizikům si vybírá z pestré nabídky komerčních pojišťoven.
        \end{itemize}
\end{enumerate}

\subsection*{Členění pojistných služeb}

\begin{enumerate}
    \item \textbf{Životní pojištění} -- toto pojištění fyzických osob pomáhá chránit tyto osoby a jejich rodiny proti rizikům těžkých úrazů, jejich trvalých následků, vážných nemocí a následné ztráty příjmu, případně při úmrtí pojištěného pomáhá nahradit zdroj příjmu jeho pozůstalým 
        \begin{itemize}
            \item rizikové (za nižší pojistné poskytuje vysokou pojistnou ochranu, nedojde-li k pojistné události zamká bez náhrady)
            \item rezervotvorné (pojistné je vyšší, protože obsahuje spořící složku. Pojistná částka plus podíly na zisku jsou vypláceny při pojistné události nebo na konci sjednané pojistné doby    
        \end{itemize}
    \item \textbf{Neživotní pojištění} -- zahrnuje především pojištění movitostí a nemovitostí.
\end{enumerate}
        
\subsection*{Pojmy}
\begin{description}
    \item[Pojistitel] - pojišťovna, má právo pojistné a povinnost vyplatit pojistné plnění v případě pojistné události
    \item[Pojistník] - subjekt, který uzavřel s pojistitelem pojistnou smlouvu. Má povinnost platit pojistné
    \item[Pojištěný] - subjekt, na jehož majetek, život, zdraví nebo odpovědnost za škodu se pojištění vztahuje. Má právo na pojistné plnění
    \item[Obmyšlená osoba] - subjekt, kterému v případě pojištění ve prospěch jiné osoby vznikne v případě pojistné události právo na pojistné plnění Pojistná událost - je nahodilá událost, při které vzniká nárok na pojistné plnění. Její nahlášení (a případné doložení, že se jedná skutečně o pojistnou událost) je povinností pojištěného
    \item[Pojistné] - představuje cenu za poskytnutí pojistné ochrany.
    \item[Pojistné plnění] - je částka, kterou pojišťovna při pojištění vyplácí v případě pojistné události.
    \item[Pojistná částka] - je nejvyšší finanční částka, jež může být vyplacena, dojde-li k pojistné události. Hodnota této částky je sjednána při uzavírání pojistné smlouvy a je v této smlouvě uvedena.
    \item[Pojistka] je písemným potvrzení pojistitele o uzavření pojistné smlouvy. U některých pojišťoven je vystavováno tzv. potvrzení o akceptaci pojištění, tedy o přijetí rizika do pojištění.
    \item[Pojistná smlouva] - je smlouvou o finančních službách, ve které se pojistitel zavazuje v případě vzniku pojistné události poskytnout pojistníkovi nebo třetí osobě ve sjednaném rozsahu pojistné plnění a pojistník se zavazuje platit pojistiteli pojistné.
    \item[Pojistný zájem] - je oprávněná potřeba ochrany před následky pojistné události. Pojistník má pojistný zájem na vlastním životě a zdraví. Pojistník má pojistný zájem na vlastním majetku.
\end{description}

\subsection*{Náležitosti pojistné smlouvy}

Základní náležitosti pojistné smlouvy:
\begin{itemize}
    \item smluvní strany (pojistník, pojistitel, pojištěný)
    \item předmět pojištění (na co smlouva je - na úraz, na majetek\ldots)
    \item začátek platnosti pojistné smlouvy, někdy 1 konec
    \item výše pojistného (kolik a jak často platí klient)
    \item výše pojistného plnění (kolik pojišťovna zaplatí a podmínky, za kterých zaplatí)
    \item všeobecné podmínky
\end{itemize}
