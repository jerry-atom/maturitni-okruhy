\chapter{Výroba, jakost, inovace}

Výroba je základní fází hospodářského procesu (výroba, rozdělování, směna, spotřeba). Výroba je hodnototvorný proces-při výrobě jsou vytvářeny za spoluúčasti všech výrobních faktorů (práce, přírodní zdroje, kapitál) = statky a služby, které mají uspokojit určité konkrétní lidské potřeby. Při výrobě přeměňujeme přírodní zdroje v hotové výrobky, pomocí práce a kapitálu.

\paragraph*{Příprava výroby}
Sem zařadíme celý výzkum a vývoj nového výrobku až po detailní zpracování technologického postupu pro výrobu. Proces přípravy nového výrobku nazveme inovací a záleží na míře změny nového výrobku oproti původnímu výrobku.

Úkolem přípravy výroby je tedy vymyslet:
\begin{itemize}
    \item Konstrukci samotného výrobku, včetně použitých materiálů
    \item Technologi výroby, včetně pracnosti
\end{itemize}

\paragraph*{Plán výroby a bilance}
Podnikatel řeší především organizaci výroby - organizaci pracovišť, práce, kapacity zásobovací a skladové, výše finančních zdrojů včetně možnosti vnějšího financování.

Plán výroby v podobě bilance:
\begin{displaymath}
    \sum_{}^{} \text{Potřeba výroby} = \sum_{}^{} \text{Výrobní zdroje}
\end{displaymath}

\textbf{Vybilancovat plán výroby} je velmi složité. V plánu výroby potřebujeme sladit velmi mnoho a často protichůdných veličin.

\textbf{Kapacita} je schopnost firmy vyrobit při optimálních podmínkách určitý počet výrobků v určitém čase. Využitelný časový fond = 365 dnů - svátky - so,ne = čas, kdy je stroj v provozu.

Při nedostatečné kapacitě strojů může firma:
\begin{itemize}
    \item Dočasně si potřebné stroje pronajmout (operativní leasing)
    \item Dohodnout výrobní kooperací s jinou firmou (často 1 konkurencí)
    \item Při prognóze dlouhodobého rozvoje prodeje určitého výrobku je možné připravit investiční výstavbu a pořídit nový DM do vlastnictví firmy, to je ovšem časově 1 finančně nejnáročnější varianta
\end{itemize}

Při nadbytečných strojních kapacitách může firma:
\begin{itemize}
    \item Dočasně stroje pronajmout
    \item Dohodnout vytížení strojů výrobním programem pro jinou firmu
    \item Stroje odprodat
\end{itemize}

Stupně rozpracovanosti výroby:
\begin{itemize}	
    \item Materiál
        \begin{itemize}
            \item zásoby nakoupené pro výrobu (suroviny, barvy, mazadla\ldots)
        \end{itemize}
    \item Nedokončená výroba
        \begin{itemize}
            \item materiál v různých fázích rozpracovanosti ,neprodej. (roztavené železo ve slévárně\ldots)
        \end{itemize}
    \item Polotovary
        \begin{itemize}
            \item produkt, který se bude ještě ve výrobě dále zpracovávat
            \item produkt už můžeme prodat jiné firmě (odlitek)
        \end{itemize}
    \item Hotové výrobky
        \begin{itemize}
            \item jsou výrobky vlastní produkce určené k prodeji
            \item jsou dokončeny všechny operace a prošly OTK-odd.t.kont. (soustruh,stojanová vrtačka\ldots) -- kontroluje jakost
        \end{itemize}
\end{itemize}

\paragraph*{Členění výrobního procesu}
Skládá z jednotlivých operací, které na sebe navazují:
\begin{enumerate}
    \item podle stupně mechanizace
        \begin{itemize}
            \item  Ruční výroba -- práci vykonává člověk
            \item  Mechanizovaná výroba -- práci vykonává stroj, řídí člověk
            \item  Automatizovaná výroba -- práci vykonává pouze stroj
        \end{itemize}
    \item podle počtu vyráběných výrobků jednoho druhu
        \begin{itemize}
            \item Kusová výroba -- jeden nebo málo kusů určitého druhu
            \item Sériová výroba -- větší množství výrobků jednoho druhu a méně druhů
            \item Hromadná výroba -- velké množství jednoho druhu výrobku typická pro spotřební materiál
        \end{itemize}
\end{enumerate}

\paragraph*{Průběh výroby}
\textbf{Úsečkový diagram}, kde každá úsečka představuje dobu výroby a montáže jednotlivých součástí do vyšších celků až nám vznikne hotový výrobek.
\textbf{Průběžnou dobu výroby jednoho výrobku} či jedné dávky výrobků, tj. celkovou dobu, kterou musíme počítat na výrobu tohoto výrobku. Potřebujeme vědět časy potřebné na zhotovení a montáž každé součástky -- k tomu nám dopomůže \textbf{stanovení výrobního postupu pro každou součástku}, ve kterém budou předepsány všechny výrobní operace -- část výrobního procesu. Pro jednotlivou operaci \textbf{stanovíme normu spotřeby času}. NSČ slouží pro účely plánování výroby a dále pro účely odměňování dělníků.

\textbf{Henry Ford} -- přinesl na začátku 20. století normování práce dělníků - je potřeba k normování spotřeby času.

\textbf{Normovaný čas se skládá z}:
\begin{itemize}
    \item Času práce (slouží pro účely plánování a odměnu)
    \item Času nutných přestávek (dělník potřebuje obnovovat svou pracovní sílu)
\end{itemize}

Ostatní čas, který není pro práci nutný, jsou ztráty a do normy se nezahrnuje.

\paragraph*{Mzda za operaci}
Mzda je přímo úměrná délce vynaloženého normovaného času a kvalifikaci dělníka, který tuto práci vykonává.

\paragraph*{Ergonomie}
Ergonomie je nauka o zákonitostech vztahů mezi člověkem, strojem a pracovním prostředím. Informace, jak optimálně zatížit člověka při práci, aby pracoval v pohodě a zároveň přinesl maximální výkon, jak optimálně uspořádat pracoviště, aby měl pracovník vše \uv{po ruce}. Informace, které pomáhají firmě dosahovat vyšších výkonů, a to nikoliv na úkor zaměstnance. Tj. aby člověk pracoval v pohodě.

[Obr. Základní toky ve výrobě]

\textbf*{Při manipulaci s materiálem rozlišujeme}:
\begin{itemize}
    \item Ruční manipulaci (hmotnostní limity)
    \item Manipulace pomocí jeřábů
    \item Manipulace pomocí dopravníků (dopravních pásů)
    \item Manipulace pomocí dopravních a zvedacích vozíků-ručních, elektrických\ldots
    \item Speciální zakládací a manipulační systémy
    \item Potrubní doprava-špinavé prádlo trubkou až do prádelny
\end{itemize}

\textbf{Obalové hospodářství}
\begin{itemize}
    \item Obaly na jedno použití (kartóny plechovky atd.)
    \item Obaly kolovací (palety kontejnery láhve atd.)
\end{itemize}

\textbf{Odpadové hospodářství}
\begin{itemize}
    \item Třídí odpady do skupin dle škodlivosti
    \item Rozpracovává na naše podmínky směrnici EU o obalech a obal. odpadech
    \item Ukládá povinnost vést evidenci odpadů a ohlašovat je
    \item Vybízí k recyklaci a energetickému využívání odpadů
    \item Stanovuje podmínky pro zneškodňování odpadů, dopravu, dovoz, vývoz a tranzit odpadů
\end{itemize}

\paragraph*{Ekologie}
Člověk by měl vrátit přírodě vše, co si od ní vzal-třídění odpadu.

\paragraph*{Jakost výroby}
V naší republice se přiklání podniková praxe spíše k systému norem jakosti, tedy dobře hodnotitelných parametrů výroby a výsledného produktu. Firmy se snaží získat mezinárodně uznávaný certifikát, že splňují firmy ISO řady. \par Naše firmy mají zájmem do EU vyvážet svou produkci. Získáním certifikátu firmy:
\begin{itemize}
    \item Mají písemný doklad o jakosti své produkce, což jim pomáhá při uzavírání obchodních smluv
    \item Výhodou je vybudovaný dobře kontrolovatelný a hodnotitelný systém řízení jakostiuvnitř firmy
\end{itemize}

Jakost kontroluje ŘKJ-řízení a kontrola jakosti.

\paragraph*{Bezpečnost práce}
Od roku 1993 jsou zaměstnavatelé povinni ze zákona (zákoník práce) platit zákonné pojištění pracovních úrazů a nemocí z povolání svých zaměstnanců u jedné z komerčních pojišťoven -- České pojišťovny nebo Kooperativy. \\ Bezpečnost práce je u nás řešena řadou zákonů a vyhlášek.

