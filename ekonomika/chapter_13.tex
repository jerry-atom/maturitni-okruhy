\chapter{Daňový systém}

Daňová soustava je složka finanční soustavy státu. DS souvisí s existencí státního rozpočtu-
daně jsou peníze, které tvoří podstatnou část příjmů státního rozpočtu- přerozdělují se a dávají
na ta místa, kde je jich potřeba.

Funkce daní
e | Fiskální-mají schopnost naplnit veřejný rozpočet
e © Alokační-stát může poskytovat zvýhodnění (daňové úlevy)-očkování\ldots
e © Redistribuční-zmírnění rozdílů v důchodech jednotlivých subjektů
e | Stabilizační-daně odčerpávají do rozpočtů vyšší díl a dělají rezervu

Daň je povinná a nevratná platba státu.
Daň může být:
e © Progresivní-přerozděluje důchod od subjektů s většími důchody k subjektům
s menšími důchody
e © Proporcionální-přerozděluje důchody mezi jedinci s různou úrovní důchodu
e © Degresivní-více dopadá na subjekty s menšími důchody (zdanění spotřeby)
Zájmy občanů a firem:
v“ Stát potřebuje vybrat co nejvíce peněz, aby mohl rozvíjet své aktivity
Občané a firmy naopak chtějí odvádět co nejméně daní
Stát musí respektovat sociální únosnost daní (to jej vede k systému úlev)
Při konstrukci daňového systému musí stát zohledňovat i technická a ekonomická
kritéria vybíratelnosti a vymahatelnosti daní
(čím více vyjímek a úlev, tím se administrativní náročnost zvyšuje)

SS

Daňová reforma v naší republice proběhla k 1.1.1993.Principy:
1.Spravedlnost zdanění-stejné podmínky pro různé typy subjektů (tuzemsko, zahraničí)
2.Všeobecnost zdanění-zdanění podléhají všechny typy vlastnictví

3.Účinnost zdanění-vhodným zdanění stimulovat žádoucí aktivity

4.Harmonizace- sbližování naší daňové soustavy se systémy EU

ZÁKLADNÍ POJMY

Vynětí a osvobození od daně
e © příjmy z prodeje bytu, pokud v něm měl majitel bytu bydliště alespoň 2 roky
e © příjmy z prodeje movitých věcí, kromě motorových vozidel, letadel, lodí, nepřesahuje-li doba
mezi nabytím a prodejem | rok
e © příjmy z prodeje nemovitostí - přesáhne-li doba mezi nabytím a prodejem 5 let
e | ceny z veřejné soutěže do 10000 Kč
e © příjmy ve formě dávek nemocenského pojištění, důchodového pojištění, státní sociální
podpory
Základ daně- částka, ze které se určitým procentem vypočítá odváděná daň
Sazba daně- procentem vyjádřený poměr daně k základu daně

Daně důchodové- platí poplatníci podle výše svého příjmu

49
\newpage
LMĎ



Částky se zaokrouhlují na celé Kč nahoru















Údeje jsou uvedeny v % Zaměstnavatel | Zaměstnanci | Celkem | OSVČ
Sociální pojištění 25,00 6,50 31,50 | 30,60
a) nemocenské pojištění 2,30 0 2,30 1,40
(nemoc, OCR, PMD)
b) důchodové zabezpečení 21,50 6,50 28,00 28,00
c) státní politika zaměstnanosti
(rekvalifikace, vytváření veřejně- | | © 1,20, | 0 1,20 1,20
prospěšných míst NN O
Zdravotní pojištění o k v
(lékař, rehabiliface, léčiva, pobyt 900 4,50" | 13,50 | "13,50
v nemocnici, lázně)
Celkem 34,00 11,00 45,00 44,10











Daň z příjmů fyzických osob 15 % ze superhrubé mzdy; u zaměstnanců, u OSVČ ze základu.

































Sleva z vypočítané daně | Rok Měsíc

Na poplatníka 24 840,-- | 2 070,--
Částečný invalidní-důchod 2520,--| 210,-
Invalidní důchod 5 040,-- |  420,--

Držitel průkazu ZTP-P 16 140,-- | 1 345,--
Manželka, manžel 24 840,- |-

Manželka, manžel ZTP/P < 149680, |

Student do 26 let 4 020,--| | 335.--

Dítě l.vřadě ©. 45,54. | 13-404--| -1117-- 1207 .
Dítě 2, vřadě.:. A ter | X7004 | 1617 =
Dítě 3. v řadě a další 27 Z 7 | 20-604 2017



bore Ta 4 0 P

Či


\newpage


Postup při výpočtu mzdy

1. Zjistíme o jaký druh mzdy se jedná (časová, úkolová a plat)
2. Pro jednolivé druhy vyhledáme potřebné údaje
a) časová - počet dnů v měsíci, denní úvazek v hodinách a hodinovou sazby,
(vynásobíme)
b) úkolová - počet vyrobených kusů x sazba na 1 kus
c) plat - je stanoven na celý měsíc
Požnámka - tyto údaje doplníme do základní mzdy
3. Vypočítáme pobídkové složky mzdy - prémie, odměny - přílušné procento
vypočítáme ze základní mzdy
4. Do hrubé mzdy doplníme součet základní mzdy a pobídkové složky - prémie
a odměny
o. Vypočítáme SZ a ZP placené zaměstnavatelem dle taháku
6. Sečteme hrubou mzdu a.SZ a ZP a vyjde nám superhy
7. Superhrubou mzdu zaokroulíme na 100 nahoru a vyjít ná
8. Vypočítáme 15 % daň z daňového základu n
9. Zjistíme slevy dle taháku a sečteme je
10. Vypočítáme daň po slevách. ©
11. Vypočítáme čistou mzdu = HM - SZ a ZP placené zaměstnancem - daň po
slevách,
12. Z tohoto důvodu si vypočítáme - také dle taháku SZ a ZP placené
zaměstnancem (z hrubé mzdy) .
13. Zjistíme srážky ze mzdy dle zadání včetně záloh m
14. Vypočítáme částku pro výplatu - čistá mzda mínus srážky a zálohy






\newpage
LISTINA



Měsíc leden znak
22 dnů >Oelkém“.
UKALOV
o.. \ldots 1
úkolová

8



mzda

Hrubá mzda
SZ zaměstnavatelem
ZP zaměstnavatelem

mzda
vavke.cL

SZ né zaměstnancem
„ZP '
Daň ze -
„Daňové
Daň

mzda

včetně záloh

k



















Opravena /

ČH= HW- 4 a 2Ť oamůslremum) - 5 po oetpocle k


\newpage
5
A
=
CD

-a

ZU
Měsíc leden
o 22 dnů

VACÍ VYPLA

SOUKALOV
3

8

ištění
ks

Základní mzda
Hrubá mzda

SZ zaměstnavatelem
ZP zaměstnavatelem

9 mzda
u X
SŽ zaměstnancem
ZP zaměstnancem
Daň ze
Daňové
Daň
istá mzda
včetně záloh
k výplatě

LISTINA

KOUKAL
2
úkolová
8

320
Á f
, -5%

4 000,-



RAM oa Mikkola,

OV
1..
časová
6
2 000
180,-

nn íÁlrny 7
W m [XC 1

16,10%
700,-

3500-

500
\newpage
















ZÚČTOVACÍ VÝPLATNÍ LISTINA

Měsíc leden
odpracováno 21 dnů



Ť

|
|

W


Jméno ČECHOVÁ | SCHLEZINGER | NĚMCOVÁ
děti | : 2 3
mzda - druh úkolová časová plat
denní úvazek 8 6 8
záloha 1500,- 3 000,- 4 000.-
hodinová sazba „ 180,-
kusová sazba 56
lat JE 6 AA SO 22 000,-

Prémie, odměny RAO 25 % 8% 8 000
Srážky - manka a škody 600,-

- výživné 3000,-

- spoření 3 000,-

- půjčky 4500,- | 700,-
vyrobeno ks 315

X

Základní mzda ACH Ž
Hrubá mzda + prěvně ZLO ,
SZ placené zaměstnavatelem /7% S A |
ZP placené zaměstnavatelem 7'- P „E
Superhrubá mzda LÝ Cr VFU
Já s 4; 2, . PV né 4 > Ó 3 9 A s Mě
SZ placené zaměstnancem še fe PED A
ZP placené zaměstnancem 7 7, - T |
Daň ze mzdy P 5 40 L | Z
Daňové odpočty SUI |- Aozns Le | one “
Daň po odpočtech +252 if m E Š
Cistá mzda PDE M ++ be 7 Z 677
Srážky včetně záloh ÓA7+% 4" xff0o E? če +
Částka k výplatě „RKO 3 C


\newpage
ZU VACI VYPLA LISTINA
4611, 10] „
23/dnů „BA |

Jméno B KLUS BARTOSOV
děti l 3 | 2
mzda - druh úkolová časová lat
denní úvazek 8 6. * 8
záloha | 1500,- 5 000,- 4 000,-
hodinová sazba m 240,-
kusová sazba 85
Sv 28 000,-
Prémi 8% | 25% 8 000
S - manka a 600,-
- výži 2000,-
- / 3 000,-
- půj 4500-
ks 320 /
mzda | DP“ C
mzda 22.764 AMG M . f 4co
Z placené zaměstnavatelem /**" +3 :
lacené zaměstnavatelem ** -

mzda 36

zaměstnancem |- 14M1Ď
lacené zaměstnancem
ze

istá mzda
včetně záloh -“ '-
k výplatě


\newpage