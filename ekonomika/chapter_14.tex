\chapter{Přímé a nepřímé daně}

\section*{Daň z přidané hodnoty}

Je nepřímou univerzální daní, tvoří část ceny výrobků a služeb. \\
Je vybíraná při každém prodeji.

Před vstupem do EU 1.5.2004 ČR harmonizovala svůj zákon o DPH s Šestou směrnicí Evropského společenství tak, aby ihned po našem vstupu mohl fungovat volný trh. (byly zrušeny pohraniční celní kontroly-tím pádem se začalo přiznávat DPH při pohybu zboží z EU).

Přesuny zboží v rámci EU:
\begin{itemize}
    \item Dodání (případné zaslání) zboží do jiného členského státu (náš export z EU)
    \item Pořízení zboží z jiného členského státu (náš import z EU)
\end{itemize}

Zahraniční obchod s nečlenskými státy EU:
\begin{itemize}
    \item Vývoz (náš export mimo EU)
    \item Dovoz (náš import ze zemí mimo EU)
\end{itemize}

Předmětem daně je:
\begin{enumerate}
    \item Prodej zboží nebo nemovitosti osobou povinnou k dani s místem plnění v tuzemsku
    \item Poskytnutí služby plátce DPH v tuzemsku
    \item Pořízení zboží z jiného členského státu EU za úplatu plátcem DPH a pořízení nového dopravního prostředku z jiného státu EU soukromou osobou-občanem
    \item Dovoz z nečlenských států EU s místem plnění v tuzemsku
\end{enumerate}

\textbf{Sazby daně}:
\begin{itemize}
    \item \textbf{Základní sazba} 21\% ze základu daně -- uplatňuje se na zboží a služby
    \item \textbf{Snížená sazba} 15\% ze základu daně -- uplatňuje se na zboží a služby
\end{itemize}

\paragraph{DPH vybírají}
\begin{itemize}
    \item Finanční úřady-v tuzemsku a při obchodech v rámci EU
    \item Celnice-při dovozu z nečlenských zemí EU
\end{itemize}

\textbf{Poplatníkem daně je kupující a plátcem daně je prodávající}.

\paragraph{Osoba povinná k dani}
Ne každý prodávající v ČR musí být plátcem daně. \\
Z osoby povinné k dani, která má sídlo nebo místo podnikání v tuzemsku, se stane plátce DPH, pokud její obrat za nejbližších 12 předcházejících měsíců přesáhne částku 750.000 Kč. \\
Musí se registrovat na finančním úřadě. \\
Firma registrovaná v ČR k placení daní dostává daňové identifikační číslo (DIČ).

\paragraph{Uskutečnění zdanitelného plnění a vznik daňové povinnosti}
Je dnem dodání zboží či poskytnutí služby. Plátce je povinen přiznat daň ke dni zdanitelného plnění nebo ke dni platby, podle toho, který den nastane dříve. To znamená, že přijatá záloha na zboží či služby podléhá povinnosti zaplatit DPH.

\paragraph{Daňový doklad}
Prodávající při prodeji vystavuje DD, který má své závazné náležitosti. Tento doklad je povinen archivovat 10 let pro účely daňové kontroly. Při prodeji v hotovosti, platební kartou nebo šekem v celkové hodnotě menší než 10 000 Kč včetně DPH vystavuje plátce zjednodušený daňový doklad.

\paragraph{Základ daně je peněžní částka snížená o daň.}
\begin{itemize}
    \item Pokud známe základ daně (částku bez DPH), pak vypočítáme DPH jako součin základu daně krát sazba daně v procentech (20*0,15=23 Kč)
    \item Pokud známe cenu celkem, pak daň vypočítáme jako součin ceny celkem krát koeficient, kde činiteli je sazba daně a ve jmenovateli součet 100+sazba daně. (100*21/121=17,36 Kč) Cena bez DPH je 82,64 Kč.
\end{itemize}

\paragraph{Osvobození od daně bez nároku na odpočet daně}
\begin{itemize}
    \item Poštovní služby
    \item Rozhlasové a televizní vysílání
    \item Finanční činnosti (poskytování úvěrů)
    \item Penzijní Činnosti
    \item Pojišťovací činnosti
    \item Převod a nájem pozemků, staveb, bytů a nebytových prostor
    \item Výchova a vzdělávání
    \item Zdravotnické služby a zboží
    \item Sociální pomoc
    \item Loterie a podobné hry
\end{itemize}

\paragraph{Zdaňovací období}
Základní zdaňovací období je I měsíc. Pokud plátce za předcházející kalendářní rok nepřekročil obrat 10 mil.Kč, může požádat o zdaňovací období čtvrtletní.

\paragraph{Výhody DPH}
\begin{itemize}   
    \item Všeobecnost zdanění
    \item Zdanění spotřeby
    \item Z hlediska plátce je zatížena jen ta hodnota, kterou sám k výrobku přidal
    \item Daň je lehce sledovatelná na cestě od prvovýroby ke spotřebiteli, omezuje možnosti daňových úniků
    \item Sbližujeme se s daňovými systémy v EU
\end{itemize}

\paragraph{Nevýhody DPH}
\begin{itemize}
    \item Značná administrativní náročnost na straně podnikatelů i FÚ
\end{itemize}

\paragraph{Postup přiznání a odvodu DPH}
Po ukončení zdaňovacího období (čtvrt roku,měsíce) musí plátce podat na FÚ daňové přiznání, kde uvede zdanitelná plnění přijatá (nakoupené zboží) a součet daně na vstupu, kterou má nárok nechat si od FÚ vrátit. \\
Pak uvede zdanitelná plnění uskutečněná (prodané zboží) a součet daně na výstupu, který je povinen odvést FÚ. \\
Rozdílem daně na vstupu a daně na výstupu vypočítáme skutečně placenou částku.

\paragraph{Uplatňování DPH při obchodu se zeměmi EU (neprochází celním režimem)}
\begin{enumerate}
    \item Firma z ČR zasílá zboží do EU firmě (je plátce DPH), musí na FA uvést její DIČ a nechat si od ní průkazně potvrdit, že zboží ve své zemi obdržela.
    \item Firma z ČR pořizuje zboží z jiné země EU od plátce DPH, nakoupí ho bez DPH a je povinna v ČR přiznat DPH dle českého zákona o DPH.
\end{enumerate}

\paragraph{Uplatňování DPH při obchodu s nečlenskými státy EU}
DPH při dovozu vybírají celnice. Základ DPH při dovozu se vypočítá:
\begin{center}
    \textbf{Celní hodnota zboží + clotspotřební daň = základ DPH}
\end{center}

Při dovozu je DPH počítáno 1 ze cla a popřípadě 1 spotřební daně (u alkoholu, tabáku a ropných produktů). Čím vyšší clo, tím více naroste DPH.

\textbf{DPH je významným cenotvorným činitelem.}

\section*{Spotřební daně}
Spotřební daň je nepřímá selektivní daň, co znamená, že je vybírána prostřednictvím prodeje vybraných druhů zboží.

\paragraph{Předmětem spotřební daně jsou}
\begin{itemize}
    \item Minerální oleje (uhlovodíková paliva a maziva)
    \item Lihalihoviny
    \item Pivo
    \item Víno
    \item Tabákové výrobky
\end{itemize}

\paragraph{Zdaňovací období}
Je jeden měsíc. Pokud v určitém měsíci nevznikne daňová povinnost, daňové přiznání se nepodává.

\paragraph{Plátci spotřební daně}
Jsou tedy nejen výrobci u tuzemské produkce a dovozci při dovozu z nečlenských zemí EU, ale 1 provozovatelé daňových skladů.

\paragraph{Poplatníci}
Jsou všichni, kdo tyto výrobky kupují pro vlastní spotřebu.

\paragraph{Sazby daně}
Jsou stanoveny samostatně pro každý druh výrobku v závislosti na měrných jednotkách. Velikost spotřební daně není závislá na ceně výrobku.

\section*{Daně pro životní prostředí (nepřímé daně)}

\begin{itemize}
    \item Daň ze zemního plynu
    \item Daň z pevných paliv
    \item Daň z elektřiny
\end{itemize}

\textbf{Zdaňovací období} je kalendářní měsíc.

\textbf{Sazby daně} u jednotlivých daní jsou pevně dány zákonem ve vztahu k fyzikálním jednotkám dodaných energií.

\section*{Daň z příjmu}

\paragraph{Zákon o dani z příjmu má tři části}
\begin{itemize}
    \item Daň z příjmu FO
    \item Daň z příjmu PO
    \item Společná část (odepisování HM a NDM)
\end{itemize}

\subsection*{Daň z příjmu FO}

\paragraph{Poplatníci daně} FO které mají na území ČR bydliště. Jejich daňová povinnost se vztahuje na příjmy plynoucí ze zdrojů na území ČR i na příjmy plynoucí ze zahraničí.

\paragraph{Plátci daně} Podnikatelé platí daň sami za sebe (plátci je shodný s poplatníkem) U zaměstnanců odvádí zálohy na daň zaměstnavatel.

\paragraph{Zdaňovací období}
Je kalendářní rok, popřípadě hospodářský rok.

\begin{itemize}
    \item Podnikatel sám za sebe může podat přiznání do 3 měsíců
    \item Daňový poradce, může požádat prodloužení až do konce června
    \item Zasvé zaměstnance zpracovává firma přiznání do 15.února
\end{itemize}

\paragraph{Osvobození od daně}
\begin{itemize}
    \item Příjmy z prodeje bytů a rodinných domů
    \item Příjmy z prodeje nemovitých věcí
    \item Příjmy z prodeje movitých věcí
    \item Ceny z veřejné soutěže, reklamní soutěže a ze sportovní (do 10 000 Kč)
    \item Příjmy ve formě dávek nemocenského pojištění, důchodového, státní sociální podpory apod.
    \item Dotace od státu, granty z EU na pořízení hmotného majetku
    \item Příjmy v naturální formě v podobě reklamních předmětů do hodnoty 500Kč
    \item Dary mezi nepříbuznými osobami jsou osvobozeny do souhrnné hodnoty
    \item Odroku 2014 již příjmy z dědictví nepodléhají dani dědické, ale jsou osvobozeny v zákoně o dani z příjmu
\end{itemize}

\paragraph{Předmět daně} Příjmy v peněžní i nepeněžní podobě (naturálie, hmotné odměny, využívání služebního automobilu pro osobní potřeby) kromě příjmů osvobozených od daně.

\paragraph{Členění příjmů}
\begin{enumerate}
    \item Příjmy ze závislé činnosti (ze zaměstnání)
    \item Příjmy ze samostatné činnosti (Z podnikání)
    \item Příjmy z kapitálového majetku (dividendy, podíly na zisku)
    \item Příjmy z nájmu
    \item Ostatní příjmy (výhry, příjem z prodeje majetku (bez osvobození)
\end{enumerate}

\paragraph{Základ daně}
U příjmů z podnikání zjistíme základ daně
\begin{itemize}
    \item z účetnictví firmy: Výnosy - Náklady = základ daně
    \item z daňové evidence Příjmy: Výdaje = základ daně
\end{itemize}

Vedení UCE i DE je složitá a nákladná věc, proto podnikatelé mají možnost zjednodušení:
\begin{itemize}
    \item Prokazovat pouze příjmy a výdaje v daňovém přiznání uplatnit paušálem (80\% zemědělci, 60\% živnostníci, 40\% nebo 30\%)
    \item Stanovení daně paušální částkou. Zákon stanoví přesné podmínky, podnikatel může zažádat FÚ o stanovení paušální částky
\end{itemize}

\paragraph{Slevy na dani}
\begin{itemize}
    \item Snižuje konečnou vypočítanou daň (např.sleva na dani 24 840 Kč znamená, že od vypočítané daně odečteme celou částku)
    \item Sleva na poplatníka, na manželku s příjmem nižším než 68 000 Kč, invalidní důchodce 1., 2. stupně, invalidní důchodce 3. stupně, student prezenčního studia,
daňové zvýhodnění na dítě
\end{itemize}

\paragraph{Nezdanitelná část základu daně je snížením základu daně}
\begin{itemize}
    \item Po snížení základu vypočítáme procentem daň (15\% ze základu daně)
\end{itemize}

\paragraph{Daňový bonus}
Jsou zavedeny daňové bonusy v případě daňového zvýhodnění za každé dítě žijící ve společné domácnosti ve výši 15 204 Kč- má-li poplatník v daném roce nižší daňovou povinnost než je daňové zvýhodnění ve formě slevy na dani, je tento rozdíl daňovým bonusem a bude mu FÚ proplacen

\paragraph{Sazba daně}
Pokud vypočítáme základ daně a upravíme ho o nezdanitelné a odčitatelné položky, je čas vypočítat samotnou daň. Ta činí 15\%. Novinkou od ledna 2013 je solidární zvýšení daně 7\% z částky na 48násobek průměrné mzdy.

\section*{Daň z příjmů právnických osob}

\paragraph{Poplatníci daně}
Jsou osoby, které nejsou FO (PO). Od daně se osvobozuje centrální banka ČR. Poplatníci daně jsou současně plátci daně.

\paragraph{Zdaňovací období}
Je kalendářní rok nebo hospodářský rok.

\paragraph{Předmětem daně}
Jsou výnosy z veškeré činnosti a nakládání s majetkem, kromě jmenovitých výjimek.

\paragraph{Základ daně}
Je zisk, který zjistíme z účetnictví tak, že vypočítáme: \textbf{Výnosy - náklady = výsledek hospodaření}.

\section*{Daň silniční}

\paragraph{Předmětem daně}
Jsou silniční motorová vozidla používaná k podnikatelské činnosti.

\paragraph{Poplatníkem i plátcem}
Je majitel vozidla uvedený v technickém průkazu vozu.

\paragraph{Sazba daně}
\begin{itemize}
    \item U osobních aut závisí na objemu válců v motoru
    \item U nákladních aut závisí na hmotnosti vozidla a počtu náprav
\end{itemize}

\paragraph{Zdaňovací období}
Je kalendářní rok, přičemž podnikatel může platit zálohy a do konce ledna následujícího roku je vyúčtovat.

\section*{Daň z nabytí nemovitých věcí}

Jedná se o daň přímou, poplatník je zároveň plátce-v případě nabytí nemovité věci převodce. Je to daň jednorázová (prodej nemovitosti, darování, dědictví). Sazba daně z nabytí nemovitých věcí činí 4\% z nabývací ceny, kterou je:
\begin{enumerate}
    \item sjednaná cena nebo srovnávací daňová hodnota
    \item zjištěná hodnota
    \item zvláštní cena u obchodních korporací
\end{enumerate}

\section*{Daň z nemovitých věcí}

Daň z nemovitých věcí je spojena s vlastnictvím nemovité věci, jejím plátcem i poplatníkem je majitel nemovité věci evidovaný v katastrálním úřadě k prvnímu dni kalendářního roku. U pronajatých pozemků he id roku 2005 poplatníkem v určených případech nájemce.

\paragraph{Zdaňovací období} je kalendářní rok.
Tato daň má 2 části.
\begin{itemize}
    \item Daň z pozemků - základem je cena pozemku dle vyhlášky a forma využívání pozemku u zemědělské půdy a lesů, u stavebních pozemků je to rozloha
    \item Daň ze staveb a jednotek - základem je půdorys nadzemní části stavby v m2 a sazba závisí i na poloze stavby (většinou se jedná o byt)
\end{itemize}
