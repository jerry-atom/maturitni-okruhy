\chapter{Daňnový systém}

\textbf{Daňová soustava} je složka finanční soustavy státu. DS souvisí s existencí státního rozpočtu daně jsou peníze, které tvoří podstatnou část příjmů státního rozpočtu- přerozdělují se a dávají na ta místa, kde je jich potřeba.

\paragraph{Funkce daní}
\begin{itemize}
    \item Fiskální -- mají schopnost naplnit veřejný rozpočet
    \item Alokační -- stát může poskytovat zvýhodnění (daňové úlevy) -- očkování\ldots
    \item Redistribuční -- zmírnění rozdílů v důchodech jednotlivých subjektů
    \item Stabilizační -- daně odčerpávají do rozpočtů vyšší díl a dělají rezervu
\end{itemize}

\textbf{Daň je povinná a nevratná platba státu.}

\paragraph{Daň může být}
\begin{description}
    \item Progresivní -- přerozděluje důchod od subjektů s většími důchody k subjektům s menšími důchody
    \item Proporcionální -- přerozděluje důchody mezi jedinci s různou úrovní důchodu
    \item Degresivní -- více dopadá na subjekty s menšími důchody (zdanění spotřeby)
\end{description}

\paragraph{Zájmy občanů a firem}
\begin{itemize}
    \item Stát potřebuje vybrat co nejvíce peněz, aby mohl rozvíjet své aktivity
    \item Občané a firmy naopak chtějí odvádět co nejméně daní
    \item Stát musí respektovat sociální únosnost daní (to jej vede k systému úlev)
    \item Při konstrukci daňového systému musí stát zohledňovat i technická a ekonomická kritéria vybíratelnosti a vymahatelnosti daní (čím více vyjímek a úlev, tím se administrativní náročnost zvyšuje)
\end{itemize}

Daňová reforma v naší republice proběhla k 1.1.1993.

Principy:
\begin{enumerate}
    \item Spravedlnost zdanění -- stejné podmínky pro různé typy subjektů (tuzemsko, zahraničí)
    \item Všeobecnost zdanění -- zdanění podléhají všechny typy vlastnictví
    \item Účinnost zdanění -- vhodným zdanění stimulovat žádoucí aktivity
    \item Harmonizace -- sbližování naší daňové soustavy se systémy EU
\end{enumerate}

\section*{Základní pojmy}
\begin{description}
    \item[Vynětí a osvobození od daně]
        \begin{itemize}
            \item []
            \item příjmy z prodeje bytu, pokud v něm měl majitel bytu bydliště alespoň 2 roky
            \item příjmy z prodeje movitých věcí, kromě motorových vozidel, letadel, lodí, nepřesahuje-li doba mezi nabytím a prodejem | rok
            \item příjmy z prodeje nemovitostí -- přesáhne-li doba mezi nabytím a prodejem 5 let
            \item ceny z veřejné soutěže do 10000 Kč
            \item příjmy ve formě dávek nemocenského pojištění, důchodového pojištění, státní sociální podpory
        \end{itemize}
    \item[Základ daně] částka, ze které se určitým procentem vypočítá odváděná daň
    \item[Sazba daně] procentem vyjádřený poměr daně k základu daně
    \item[Daně důchodové] platí poplatníci podle výše svého příjmu
    \item[Daně majetkové] platí poplatníci podle velikosti svého nemovitého majetku, při majetkových převodech (darování, dědictví, prodej či převod) a při využívání vozidel pro podnikání
    \item[Daně univerzální (DPH)] vybírány při prodeji téměř všech druhů zboží a služeb
    \item[Daně selektivní] spotřební daň, daně pro životní prostředí -- jsou vybírány pouze u vybraných druhů zboží (cigarety, alkohol, benzin, nafta, pevná paliva, zemní plyn, elektřina)
    \item[Poplatník] je FO či PO, z jejíchž peněz je daň placena (ten, z jehož kapsy peníze)
    \item[Plátce] je FO či PO, která má ze zákona povinnost peníze odvádět státu
    \item[Daňové přiznání] podoba formuláře -- doklad potřebný pro kontrolu správnosti
    \item[Daňový únik] je situace, kdy se plátce či poplatník vyhýbá úhradě daně, únik může být úmyslný nebo neůúmyslný také legální nebo nelegální
    \item[Daňová kvóta] vyjadřuje celkovou úroveň daňové zátěže v dané zemi
    \item[Daňový ráj] se označují země s velmi nízkými daněmi a ekonomikou orientovanou na zahraniční kapitál
    \item[Den daňové svobody]
        \begin{itemize}
            \item []
            \item hranice, která rozděluje kalendářní rok na 2 období
            \item v 1. vydělávají daňoví poplatníci na pokrytí výdajů vlády a institucí státu
            \item V 2. si až o penězích rozhodují svobodně sami
        \end{itemize}
    \item[Slevy na dani] snižuje konečnou vypočítanou daň (na dítě, poplatníka, manželku)
    \item[Zdaňovací období]
        \begin{itemize}
            \item []
            \item je kalendářní rok nebo hospodářský rok
            \item rozhodné období, časový úsek, za který se počítá příslušná daň
        \end{itemize}
    \item[Struktura daňové soustavy]
        \begin{enumerate}
            \item []
            \item \textbf{daně přímé} (poplatník podává přímo na finančním úřadu daňové přiznání)
                \begin{itemize}
                    \item důchodové:
                        \begin{itemize}
                            \item daň z příjmů fyzických osob a právnických osob
                        \end{itemize}
                    \item majetkové:
                        \begin{itemize}
                            \item daň z nemovitostí (tj. daň z pozemků, daň ze staveb)
                            \item daň silniční
                            \item daně převodové (tj. daň dědická, darovací a z převodu nemovitostí)
                        \end{itemize}
                \end{itemize}
            \item \textbf{daně nepřímé} (ze spotřeby-platíme při každém nákupu zboží a služeb)
                \begin{itemize}
                    \item univerzální:
                        \begin{itemize}
                            \item daň z přidané hodnoty,
                        \end{itemize}
                    \item selektivní:
                        \begin{itemize}
                            \item daně spotřební (daň z minerálních olejů, daň z alkoholu, cigaret a tabák. výrobků)
                            \item ekologické (k ochraně životního prostředí)
                            \item daň ze zemního plynu
                            \item daň z pevných paliv
                            \item daň z elektřiny
                        \end{itemize}
                \end{itemize}
        \end{enumerate}
\end{description}

\paragraph{Rozdíly ve zdanění spotřeby a důchodu}
\begin{enumerate}
    \item Zdanění příjmů přihlíží ke konkrétním poměrům poplatníka, zdanění spotřeby nikoliv. Tím se přímé daně stávají adresným nástrojem regulace důchodů poplatníka, nepřímé daně tuto schopnost nemají.
    \item Nepřímé zdanění má méně nepříznivý vliv na pracovní motivaci a výrobní aktivitu než přímé.
    \item Vzhledem k tomu, že nepřímé daně jsou součástí konečné ceny, přispívají více k růstu inflace než daně přímé.
    \item Zdanění příjmů vyvolává u poplatníků větší odpor než zdanění spotřeby. To vede i ke snaze obejít zdanění a daňovým únikům.
\end{enumerate}

\paragraph{Lafferova křivka}
Zobrazuje závislost celkového objemu vybraných daní na míře zdanění (resp. na daňové sazbě)

[Obr. Lafferova křivka]

Aktuálnost daňové soustavy ???

Velikost příjmů veřejných rozpočtů z jednotlivých druhů daní ???

\section*{Celnictví}

V EU jsou mezi státy zrušena cla. EU má jednotnou celní politiku vůči nečlenským státům.
\begin{itemize}
    \item pokud firma z ČR vyváží zboží do USA, musí vyplnit celní dokumenty a zboží prochází celní kontrolou, clo pro vývoz ale v EU není vyměřováno, takže firma clo neplatí
    \item firma dováží zboží z USA, musí ho proclít-je uvaleno clo podle celního sazebníku platného pro celou EU
\end{itemize}

\textbf{Clo je celní poplatek}, dávka vybíraná státem při přechodu zboží přes celní hranici.

\paragraph{Funkce cla}
\begin{itemize}
    \item Fiskální-příjem do státního rozpočtu-odvádíme do rozpočtu EU
    \item Obchodně politická-nástroj hospodářské politiky
    \item Cenotvorná-u dovozového zboží firmy započítají clo do prodejní ceny
\end{itemize}

\paragraph{Druhy cla v EU}
\begin{itemize}
    \item Dovozní clo-nejběžnější
    \item Vývozní clo-není používáno
    \item Vyrovnávací
    \item Odvetné
    \item Antidumpingové
\end{itemize}

\paragraph{Celní sazebník}
Celní sazebník je jednotný pro všechny státy EU, sazebník obsahuje všechny druhy dováženého zboží a jejich celní sazby (\% cla z celní hodnoty).

Celní sazby obsažené v celním sazebníku můžeme členit:
\begin{itemize}
    \item Všeobecné celní sazby (většinou sazby nejvyšší)
    \item Smluvní celní sazby (na základě mezinárodních dohod dohodnuté nižší)
    \item Preferenční sazební opatření (vyplývající z mnohostranných mezinárod. dohod)   
\end{itemize}

\paragraph{Všeobecná dohoda o clech a obchodu GATT}
\begin{itemize}
    \item Nejdůležitější mezinárodní organizace GATT (1947) má 135 členských zemí (ČR-1993).
    \item Základním cílem GATT je odbourávání překážek světového obchodu. \par
    \item Usnadňuje vstup zboží na trhy smluvních stran.
\end{itemize}