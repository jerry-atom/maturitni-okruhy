\chapter{Bankovní služby}

\paragraph{Úvěr}
Úvěr je podobně jako zápůjčka formou dočasného postoupení peněžních prostředků věřitelem na principu návratnosti dlužníkovi, který je ochoten za tuto půjčku po uplynutí nebo ještě v průběhu doby splatnosti zaplatit určitý úrok.

\paragraph{Pojmy}
\begin{description}
    \item[Urok] je peněžitá odměna za půjčení peněz.
    \item[Uroková sazba] je procentní vyjádření zvýšení půjčené částky za určité časové období. Uroková sazba určuje kolik z jistiny musí dlužník za předem smluvně stanovenou dobu věřiteli za půjčku či úvěr zaplatit.
    \item[Jistina] je základní peněžní částka, která byla půjčena nebo která tvořila vklad.
    \item[Úrokové období] -- Období, během kterého jsou připisovány úroky.
    \item[RPSN] (roční procentní sazba nákladů) je číslo, které má umožnit spotřebiteli lépe vyhodnotit výhodnost nebo nevýhodnost poskytovaného úvěru.
    \item[Likvidita] - Likviditu lze definovat také jako míru schopnosti podniku přeměnit svá aktiva na peněžní prostředky a těmi krýt včas, v požadované podobě a na požadovaném místě všechny své splatné závazky, a to při minimálních nákladech.
    \item[Zadlužení] Zadluženost je ekonomický pojem, který označuje skutečnost, že podnik používá pro financování svých aktiv cizí kapitál.
    \item[Předlužení] O předlužení jde tehdy, jestliže osoba má více věřitelů a jestliže její splatné závazky jsou vyšší než její majetek; do ocenění dlužníkova majetku se zahrne i očekávaný výnos z pokračující podnikatelské činnosti, lze-li příjem převyšující náklady při pokračování podnikatelské činnosti důvodně předpokládat. Zadlužení není předlužení. Zatímco zadlužení je normální, zdravé, umožňuje si pořídit nové věci, vybavení a pomáhá ekonomice, předlužení je negativní patologická forma zadlužení, která vede k ekonomickému zhroucení dlužníka.
    \item[Dluhová past] je známější a používanější termín pro dluhovou spirálu. Jedná se o stav, kdy není jednotlivec anebo 1 celá rodina schopna splácet závazky věřitelům. Což vede k penále, anebo dalším závazkům formou braní si dalších úvěrů.
    \item[Konsolidace úvěrů], též konsolidace půjček, označuje sloučení více půjček do jedné.
    \item[Bonita] je převrácená hodnota úvěrového rizika a vyjadřuje důvěryhodnost ekonomického subjektu (firmy, jednotlivce, ale 1 obce nebo státu) na finančním trhu.
    \item[Registr dlužníků] (úvěrový registr) je databáze evidující zejména osoby a organizace, které jsou pozadu ve splácení svých závazků.
    \item[Oddlužení] je zákonným prostředkem, kterým mohou dlužníci především fyzické osoby řešit situaci, kdy mají více věřitelů a nedokáží plnit své splatné závazky. Oddlužením se řeší úpadek fyzické či právnické osoby nepodnikatele.
    \item[Exekuce] je nástroj, kterým je možné po někom, kdo dluží, domoci zaplacení dluhu (nebo i jiných povinností, které neplní - nejen těch finančních), který dlužník dobrovolně neuhradil, ačkoli měl a ačkoli mu takovou povinnost uložil (nejčastěji) soud v soudním rozhodnutí.
\end{description}

\subparagraph{Druhy úvěrů podle splatnosti}
\begin{enumerate}
    \item krátkodobé se splatností do 1 roku:
        \begin{itemize}
            \item kontokorentní úvěr - kombinace běžného účtu s možností čerpat krátkodobý úvěr do výše úvěrového limitu stanoveného ve smlouvě o zřízení kontokorentního účtu
            \item eskontní úvěr - souvisí s eskontem (odkoupením) směnky klienta před dobou splatnosti bankou. Nesplatí-l1 dlužník bance směnku v termínu splatnosti, žádá banka úhradu od posledního majitele směnky (klienta, který směnku prodal)
            \item akceptační úvěr - banka neposkytne klientovi přímo peníze, ale akceptuje cizí směnku vystavenou klientem - příjemcem akceptačního úvěru a tím se stává hlavním směnečným dlužníkem. Akceptační úvěr dává banka jen nejlepším a ověřeným klientům
            \item revolvingový úvěr - banka umožňuje klientovi opakované čerpání úvěru. Základní podmínkou je předešlé splacení úvěru.
            \item lombardní úvěr - úvěr jištěný zástavou movité věci (cenné papíry, zboží \ldots)
        \end{itemize}
    \item střednědobé a dlouhodobé - se splatností od 1 do 10 let
        \begin{itemize}
            \item hypoteční úvěr - jištěný hypotékou (zástavou nemovitostí)
            \item emisní úvěr - spojená s emisí dlouhodobých cenných papírů, např podnikových obligací
            \item spotřební půjčky občanům
        \end{itemize}
\end{enumerate}

\subparagraph{Speciální formy úvěru}
\begin{description}
    \item[Faktoring] -- odkup krátkodobých pohledávek (do 1 roku)
    \item[Forfaiting] -- odkup dlouhodobých pohledávek (nad 1 rok)
\end{description}

\subparagraph{Druhy úvěru podle poskytované měny}
Jsou úvěry v domácí nebo cizí měně (devizové)

Devizový úvěr
\begin{itemize}
    \item Jde o požadavek komitenta o úvěr u zahraniční banky. Tyto zahraniční banky mají většinou nižší úrokové sazby, na druhé straně však hrozí kursové riziko. Nehledě k tomu, že zahraniční banky vyžadují záruku od českých bank.
    \item Také naše banky dávají účelové devizové úvěry. Vlastní postup spočívá v tom, že zahraničnímu výstavci zaplatí česká banka z úvěru, který získá u cizí banky v tzv. úvěrová linka. Zahraniční banky tak podporují vlastní ekonomiku.
\end{itemize}

\subparagraph{Druhy úvěru podle zajištěnosti}
\begin{description}
    \item[Nezajištěný úvěr] je z pohledu bamky velmi rizikový. Proto je tedy i dražší. Tento typ úvěru se vyskytuje velmi vzácně a je poskytován pouze těm nejspolehlivějším klientům (VIP klienti).
    \item[Zajištěný úvěr] Většina věřitelů chce mít 100\% jistotu, že své peníze dostane zpět. Proto požadují nějaké zajištění. Tím se znatelně sníží riziko toho, že by se o půjčené peníze přišlo. Jinak věřitel finance neposkytne. Banka zajišťuje své úvěry zástavou majetku dlužníka (movitý i nemovitý). V případě, že pak dlužník řádně nesplácí, zástava bude bankou zabavena. Ručí se formou zástavního práva k věci, popřípadě jde o zajištění další osobou. Zajištěné úvěry bývají levnější než nezajištěné (nižší úroky). Navíc si věřitel může nechat prověřit klientovu bonitu nebo stanovit pro úvěr limit. Mezi zajištěné úvěry může patřit úvěr hypotéční, spotřebitelský nebo nebankovní úvěr na vyšší částku. Stejně tak sem spadají 1 neúčelové půjčky nebo krátkodobý úvěr (zajištěný zástavou movitých věcí). Méně známé jsou lombardní a eskontní úvěry, kdy je úvěr zajištěn cennými papíry, šperky či drahými kovy.
\end{description}

\subparagraph{Druhy úvěrů podle způsobu čerpání}
\begin{itemize}
    \item jednorázové
    \item postupné
    \item před zápisem zástavního práva
\end{itemize}

\subparagraph{Druhy úvěrů podle účelu a subjektů}
\begin{enumerate}
    \item úvěry pro podnikatelské účely
        \begin{enumerate}
            \item účelově - v úvěr. smlouvě je uvedeno, na co je úvěr poskytován = zásoby, pohledávky
            \item neúčelově - otevření úvěr. Linky
        \end{enumerate}
    \item úvěry občanům
        \begin{enumerate}
            \item na nákup nemovitého 1 movitého majetku, užívají se zde krátko-, dlouho- i střednědobé úvěry
            \item účelově zaměřené - na dům, rekonstrukci, nákup auta = dlohodobá spotřeba
            \item osobní - na překlenutí dočasného nedostatku financí, převodem na BŮ nebo v hotovosti
            \item kontokorentní - stanoven úvěrový rámec, podle kterého můžou občané čerpat úvěr na bězném, na sporožirovém účtě \ldots
            \item úvěrové karty     
        \end{enumerate}
    \item mezibankovní úvěry \\
        Banka může mít krátkodobě nevyrovnanou peněžní pozici z těchto důvodů:
        \begin{enumerate}        
            \item větší výběr vkladů z účtů zákazníků banky
            \item zákazníci banky včas nesplácí bance úvěry
        \end{enumerate}
        Dlouhodobě nevyrovnaná pozice:
        \begin{enumerate}        
            \item klienti u dané banky převážně čerpají úvěry
            \item klienti převážně ukládají vklady a banka to půjčuje jiným bankám
        \end{enumerate}
    \item další úvěry - Hlavně obcím a městům, jedná se o středně a dlouhodobé investiční úvěry, krátkodobé provozní úvěry na překlenutí nedostatku finančních prostředků.
\end{enumerate}

\section{Cíle banky při poskytování úvěru}
\begin{enumerate}
    \item \textbf{Výnosnost} -- banka musí stanovit takové úroky z úvěru, aby vydělala. Úrok může být stanoven pevný nebo pohyblivý.
    \item \textbf{Návratnost} -- banka se musí zajistit proti případnému nesplacení závazku dlužníka.    
\end{enumerate}

\paragraph{Postup při poskytnutí úvěru}
Před poskytnutím úvěru si banka ověřuje Bonitu klienta, likviditu jeho majetku, podnikatelský záměr.

Toto však ještě nestačí, ve většině případů chce banka určité záruky. jištění úvěru, a to ve formě:
\begin{itemize}
    \item zástavy nemovitosti (hypotéční úvěr)
    \item zástavy movitosti - cenné papíry, stroje, zásoby, \ldots (lombardní úvěr)
    \item ručitelé - ručit mohou 1 podnikatelské subjekty za podnikatelské úvěry, ale takové subjekty se těžko shánějí
    \item vinkulace vkladu - zablokování vkladu dlužníka na účtu či na knížce ve prospěch věřitele pro případ nesplacení
    \item postoupení pohledávek apod.
\end{itemize}

\paragraph{Uvěrová smlouva}
\textbf{Obsahuje}: název, jméno klienta, název banky, účel a výše úvěru, způsob čerpání, zajištění úvěru, postup v případě neschopnosti klienta platit splátky, práva a povinnosti banky i klienta, datum, místo, rizika a platné podpisy obou stran + přílohy

Doba splatnosti úvěru se počítá ode dne prvního čerpání úvěru

\paragraph{Kontrola dodržování podmínek úvěrové smlouvy}
Sleduje se dodržování výše a termínů splátek úvěrů a úroků, pokud se nedodržují je dlužná částka převedena na účet úvěrů neuhrazených ve lhůtě a je úročena vyšší sankční sazbou.

\paragraph{Další služby, které poskytují banky}
Založení a vedení účtu - Běžné účty, termínované účty, devizové účty atd.

\paragraph{Bezhotovostní platební styk}
\begin{itemize}
    \item V tuzemsku:
        \begin{itemize}
            \item příkaz k úhradě
            \item trvalý příkaz k úhradě
            \item příkaz k inkasu
            \item trvalý příkaz k inkasu
        \end{itemize}
    \item se zahraničím:
        \begin{itemize}
            \item platba formou \textbf{dokumentárního inkasa} nebo \textbf{dokumentárního akreditivu}
        \end{itemize}
\end{itemize}

Při dokumentárním inkasu se zavazuje inkasní banka (banka dodavatele) vydat dlužníkovi (odběrateli) dokumenty opravňující nakládat se zbožím, bude-li při jejich vydání zaplacena sjednaná částka.

Bezpečnější je dokumentární akreditiv, tedy písemný závazek banky odběratele, že poskytne dodavateli plnění (peníze, směnku), budou-li do určité doby splněné podmínky. Splnění podmínek doloží dodavatel většinou dokumenty o zaslání zboží.

\textbf{Platební karty} -- vydávají se k běžným nebo žirovým účtům a umožňují majiteli vybírat hotovost a provádět bezhotovostní platby. Karty máme debetní (normální) a kreditní (úvěrová karta).

\textbf{Šeky} -- dělíme na soukromé (vystavované nebankovními subjekty) a bankovní. V ČR je zaručený šekový systém, který stanoví jednotná pravidla.

\textbf{Homebanking} -- domácí internetové bankovnictví

\textbf{Směnárenská činnost} -- výměna valut a deviz (včetně cestovních šeků) v aktuálním kurzu

\textbf{Devizové operace} -- banky provádí bezhotovostní platební styk i v cizích měnách a nabízejí svým klientům možnost zajistit se proti kurzovým rizikům bankovními devizovými deriváty -- swapy, forwardy, opce.

\textbf{Obchody s cennými papíry} -- většina bank má své makléře a za úplatu zajišťují pro klienty zprostředkování nákupu či prodeje cenných papírů. Velkým klientům banky nabízí i tzv. správu portfolia, tzn. že banka po dohodě s klientem nakoupí pro něj nevhodnější cenné papíry a stará se o udržování a rozmnožování hodnoty této investice.

\textbf{Poradenská a informační činnost} -- Vysvětlení základních služeb a pomoc se správou peněz.

\textbf{Bezpečnostní schránky a úschova cenností}