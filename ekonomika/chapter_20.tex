\chapter{Peníze a platební styk}

\textbf{Peníze} jsou speciální druh zboží, který se vyčlenil a je směňován za jiný druh zboží či služby.

\textbf{Směnný obchod} (též barterový obchod či jen barter) je výměna zboží nebo služeb za jiné zboží nebo služby bez použití peněz, nebo při alespoň částečném započtení hodnoty zboží nebo služby jiným zbožím nebo službou. Barterové obchody jsou považovány za nejzákladnější formu kompenzačních obchodů.

\textbf{Směna přímá a nepřímá} -- Směna probíhající ve dvou krocích -- první krok představuje směnu určitého
statku nebo služby za -- prostředek směny, obvykle za -- peníze, a druhý krok směnu takto získaného
prostředku směny za požadovaný statek nebo službu. Nepřímá směna je opakem -- přímé směny, tj.
barteru, při níž člověk obchoduje se svými statky nebo službami přímo za požadované statky nebo
služby, bez zprostředkování pomocí prostředku směny, obvykle peněz. Potřeba nepřímé směny vzrůstá
s prohlubováním dělby práce a růstem objemu statků a služeb, jež jsou k dispozici pro směnu.

\textbf{Barter} je směna zboží za zboží a je historicky nejstarší formou směny.

\textbf{Historie peněz}:
\begin{enumerate}
    \item drahé kovy
    \item mince z drahých kovů
    \item papírové peníze a mince z jiných než drahých kovů
    \item bezhotovostní platby-formou platebních karet, šeků, plateb prostřednictvím BŮ apod.
\end{enumerate}

\textbf{Funkce peněz}:
\begin{itemize}
    \item prostředek směny -- zprostředkovávají výměnu zboží-peníze-zboží)
    \item míra hodnot (pomocí peněz měříme hodnotu jinak nepoměřitelných statků)
    \item prostředek akumulace (peníze se dají hromadit, schraňovat, akumulovat)
\end{itemize}

\textbf{Formy peněz}: \\
\textbf{Hotové peníze, cizí měny, pohledávky, cenné papíry krátkodobé a dlouhodobé.} \\
\textbf{Hotovostní peníze}: bankovky, mince a státovky. Státovky jsou vydávány pouze v době válek a krizí ke krytí ztrát státního rozpočtu. Hotovostní peníze jsou snadno dělitelné, snadno přenosné a trvanlivé. \\
Obsahují ochranné znaky -- vodoznak, proužek, speciální tisk.

\textbf{Bezhotovostní peníze}: Skoro peníze -- jsou to likvidní prostředky a snadno se dají přeměnit na peníze. \\
Patří sem např. akcie nebo termínovaný vklad.

\textbf{Plastikové peníze} -- platební karty.

\textbf{Hodnota peněz}: neboli jejich kupní síla, je závislá na tom, kolik peněz v dané ekonomice obíhá. Kupní síla znamená, kolik jednotek zboží a služeb (jak velkou část spotřebního koše) lze koupit za jednu peněžní jednotku. Růst cen (cenové hladiny), a tedy zmenšování hodnoty (kupní síly) peněz, se nazývá inflace. Vzniká, je-li tok peněz do ekonomiky silnější než objem zboží a služeb. V takovém případě převyšuje poptávka po zboží a službách nabídku a zvyšuje se cenová hladina.

\textbf{Zlatý standard}: je způsob vyjádření hodnoty měny v měnovém systému, kde standardním ekonomickým měřítkem je zlato. Hodnota měny, která je používána jako jednotka zúčtování, je odvozována od definovaného množství zlata

\textbf{Platební styk}: Platební styk v České republice metodicky řídí Česká národní banka. Nezastupitelný a zvláštní význam v něm mají dvě instituce a sice banky a pošty.

Platební styk můžeme posuzovat jednak z hlediska formy použitých platebních prostředků, a jednak z hlediska určení teritoria, ve kterém platební styk probíhá. Na základě uvedených hledisek rozlišujeme platební styk:
\begin{itemize}
    \item hotovostní a bezhotovostní
    \item tuzemský a zahraniční
\end{itemize}

\textbf{Platební prostředky}: Peněžní prostředky tvoří buď peníze v hotovosti nebo peníze uložené na bankovních účtech.

\textbf{Ochranné prvky bankovek}:
\begin{description}
    \item[Vodoznak] -- obrazce, obrázky nebo znaky, které jsou viditelné, když se bankovka přidrží proti světlu.
    \item[Okénkový proužek s mikrotextem] -- 3 mm široký proužek z uměle metalizované hmoty vsazený do papíru, který na lícní straně vystupuje v intervalech na povrch papíru.
    \item[Barevná vlákna] -- 6 mm dlouhá, běžně viditelná vlákna oranžové barvy zapuštěná do papíru.
    \item[Soutisková značka] -- kruhová značka tvořena písmeny „CR“. Na každé straně bankovky je vidět pouze část celé značky.
    \item[Skrytý obrazec] -- tento obrazec tvoří číslo označující nominální hodnotu bankovky a je viditelný pouze tehdy, když bankovku sklopíme ve výši očí do vodorovné polohy proti zdroji světla.
    \item[Proměnlivá barva] -- obrazec vytištěný speciální tiskovou barvou, která mění své zbarvení v závislosti na otáčení bankovky v různém úhlu proti dopadajícímu světlu.
    \item[Iridiscentní pruh] -- duhově proměnlivý ochranný pruh, 20 mm široký, umístěný na pravém okraji lícní straně bankovky.
    \item[Mikrotext] -- text ve formě slovního označení hodnoty bankovky konturuje různá pole bankovky.
    \item[UV světlo] -- neboli fluorescenční tisk, který je vidět pouze pod speciálním osvětlením tzv. UV světlo
    \item[Infračervené světlo] -- některé ochranné prvky jsou viditelné pouze pod speciálním osvětlením tzv. IR světlem
\end{description}

[Obr. Bankovka s ochrannými prvky]

\paragraph{Kdo provádí emisi peněz}
V ČR pouze Česká národní banka má výhradní právo vydávat bankovky a mince, jakož i mince pamětní.

\textbf{Padělané a pozměněné peníze}

\begin{description}
    \item[Měna] peněžní soustava určitého státu, základem každé měny je peněžní jednotka
    \item[Valuty] bankovky a mince cizího státu
    \item[Devizy] pohledávky znějící na cizí měnu (např.cenné papíry v cizí měně, bezhotovostní peníze)
    \item[Kurzovní lístek] soupis kurzů měn nejdůležitějších států k určitému datu. U nás vydává KL centrální banka každý pracovní den, směnárny se mohou od KL lišit.
    \item[Měnový kurz] cena měnové jednotky jedné země vyjádřená v peněžní jednotce jiné země
    \item[Devalvace] oficiální snížení kurzu měny určitého státu vůči všem ostatním měnám (za jednu cizí měnovou jednotku zaplatíme více domácí měny)
    \item[Revalvace] je opak devalvace, tedy administrativní zvýšení kurzu vůči všem cizím měnám.
\end{description}

Devalvaci a revalvaci měny provádí centrální banka státu, u nás CNB.

\section*{Učet}

Druhy, založení, smlouva o běžném účtu, podpisový vzor jaké jsou druhy účtů?
\begin{itemize}
    \item Základní bankovní účet). Na váš účet můžete vkládat peníze, vybírat hotovost z bankomatu a platit pravidelné účty.
    \item Běžný účet. Nejrozšířenější typ účtu. Má stejné vlastnosti jako účet základní, ale navíc umožňuje majiteli účtu používat k platbám v obchodech platební kartu. Kromě toho umožňuje i sjednání dočasného úvěru s vaší bankou.
    \item Spořící účet). U tohoto typu účtu vám platí banka vyšší úrok. Produkt je vytvořen proto, aby vám pomohl naspořit větší hotovost.
\end{itemize}

\paragraph{Založení}

Základní podmínkou k založení vlastního účtu je věk 18 let, pokud se nejedná o specifické produkty. Příkladem mohou být studentské účty, které lze uzavírat už od 15 let, ale s podmínkou přítomnosti zákonného zástupce. V případě cizinců je nezbytnou podmínkou také potvrzení pobytu na území ČR. Při zakládání účtu požadují banky předložit dva identifikační doklady.

Specifické podmínky panují při zakládání podnikatelského účtu. Pokud chcete založit účet jako právnická osoba, bude po vás banka požadovat standardní doklady, které byly zmíněny výše. Navíc je obvykle zapotřebí předložit doklad o založení společnosti, výpis z obchodního rejstříku v ČR nebo vydaný v jiném státě (odkud právnická osoba pochází). Pokud firma ještě nebyla zapsána v obchodním rejstříku, je třeba dodat zakladatelskou listinu nebo společenskou smlouvu. Všechny doklady v cizím jazyce musí zájemce o účet nechat přeložit do češtiny, úředně ověřit a předložit při žádosti o založení účtu.

\textbf{Smlouva o běžném účtu} -- Účelem smlouvy o běžném účtu je umožnit majiteli používat prostředky na jeho běžném účtu v \textbf{bezhotovostním platebním styku}. Na druhé straně banka dostává od majitele účtu k dispozicí peněžní prostředky, které může používat při své podnikatelské činnosti.

\textbf{Podpisový vzor} se dnes již téměř nepoužívá, neboť 99\% platebních příkazů je zadáváno přes internetbanking, kde jako autorizace slouží místo podpisu SMS kód. V případě potřeby podávat platební příkazy osobně na pobočce je možné podpisový vzor změnit na něco jednoduššího. Nicméně ani to nebývá v praxi třeba, neboť pobočkoví pracovníci přistupují ke starším lidem s tolerancí.

\section*{Příkaz k úhradě, trvalý příkaz k úhradě, příkaz k inkasu, trvalý příkaz k inkasu, platební karty}

\textbf{Příkaz k úhradě} je příkaz, kterým majitel účtu požaduje po bance provedení určité platby ze svého účtu ve prospěch účtu jiného subjektu. Od tohoto příkazu je odvozen tzv. \textbf{trvalý příkaz k úhradě}, kterým se z účtu převádí opakovaně stejná částka.

\textbf{Příkaz k inkasu} je pokyn k platbě z účtu, který dává věřitel.

\textbf{Platební karty} jsou nástroje určené k bezhotovostním platbám, které jsou nejčastěji vydávány fyzickým I právnickým osobám bankou. \ldots Za výběry hotovosti z bankomatu platí poplatky klient banky, transakce klienta u obchodních partnerů banky jsou strhávány z částky, kterou nakonec obdrží obchodní partner.

\section*{Rešení nedostatku peněz -- úvěr}
Podobně jako zápůjčka formou dočasného postoupení peněžních prostředků věřitelem na principu návratnosti dlužníkovi, který je ochoten za tuto půjčku po uplynutí nebo ještě v průběhu doby splatnosti zaplatit určitý úrok.

\paragraph{Jednoduché úrokování}
Při jednoduchém úrokování se úrok za stejné úrokovací doby nemění a počítá se z téže původní jistiny, kterou nazýváme počáteční jistinou JO. Jednoduché úrokování se používá tehdy, je-li úroková doba kratší než úrokovací doba nebo je maximálně rovna úrokovacímu období.

\textbf{Uročení polhůtní} -- úrok je vyplacen na konci úrokového období
\textbf{Uročení předlhůtní} -- úrok je vyplacen na začátku úrokového období.

[Vzorec pro jednoduché úrokování -- ???]

\paragraph{Složené úrokování -- základní charakteristika}

\textbf{Složené úročení} se používá v případech, kdy úrokovací doba tvoří několik celých úrokovacích
období. Nejběžnější příklad je v praxi několik celých let

\textbf{Úročitel}

\textbf{Odúročitel}
