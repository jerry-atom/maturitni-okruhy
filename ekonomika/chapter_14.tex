\chapter{Přímé a nepřímé daně}

DAŇ Z PŘIDANÉ HODNOTY
Je nepřímou univerzální daní, tvoří část ceny výrobků a služeb.
Je vybíraná při každém prodeji.

Před vstupem do EU 1.5.2004 ČR harmonizovala svůj zákon o DPH s Šestou směrnicí
Evropského společenství tak, aby ihned po našem vstupu mohl fungovat volný trh. (byly
zrušeny pohraniční celní kontroly-tím pádem se začalo přiznávat DPH při pohybu zboží
z EU).
Přesuny zboží v rámci EU:
e © Dodání (případné zaslání) zboží do jiného členského státu (náš export z EU)
e © Pořízení zboží z jiného členského státu (náš import z EU)
Zahraniční obchod s nečlenskými státy EU:
e | Vývoz (náš export mimo EU)
e © Dovoz (náš import ze zemí mimo EU)
Předmětem daně je:
a) Prodej zboží nebo nemovitosti osobou povinnou k dani s místem plnění v tuzemsku
b) Poskytnutí služby plátce DPH v tuzemsku
c) Pořízení zboží z jiného členského státu EU za úplatu plátcem DPH a pořízení nového
dopravního prostředku z jiného státu EU soukromou osobou-občanem
d) Dovoz z nečlenských států EU s místem plnění v tuzemsku

Sazby daně:
e © Základní sazba-21% ze základu daně-uplatňuje se na zboží a služby
e Snížená sazba-15% ze základu daně-uplatňuje se na zboží a služby
Zboží-(noviny živá zvířata, voda, dětské pleny, knihy,potraviny)
Služby-(úprava a rozvod vody, hromadná doprava, kulturní činnosti)
DPH vybírají:
e Finanční úřady-v tuzemsku a při obchodech v rámci EU
e © Celnice-při dovozu z nečlenských zemí EU

Poplatníkem daně je kupující a plátcem daně je prodávající.

Osoba povinná k dani

Ne každý prodávající v ČR musí být plátcem daně.

Z osoby povinné k dani, která má sídlo nebo místo podnikání v tuzemsku, se stane plátce
DPH, pokud její obrat za nejbližších 12 předcházejících měsíců přesáhne částku 750 000 Kč.
Musí se registrovat na finančním úřadě.

Firma registrovaná v ČR k placení daní dostává daňové identifikační číslo (DIČ).

Uskutečnění zdanitelného plnění a vznik daňové povinnosti je dnem dodání zboží či
poskytnutí služby. Plátce je povinen přiznat daň ke dni zdanitelného plnění nebo ke dni
platby, podle toho, který den nastane dříve. [o znamená, že přijatá záloha na zboží či služby
podléhá povinnosti zaplatit DPH.

53
\newpage
Daňový doklad

Prodávající při prodeji vystavuje DD, který má své závazné náležitosti. Tento doklad je
povinen archivovat 10 let pro účely daňové kontroly. Při prodeji v hotovosti, platební kartou
nebo šekem v celkové hodnotě menší než 10 000 Kč včetně DPH vystavuje plátce
zjednodušený daňový doklad.

Základ daně je peněžní částka snížená o daň.
a. Pokud známe základ daně (částku bez DPH), pak vypočítáme DPH jako součin
základu daně krát sazba daně v procentech (20*0,15=23 Kč)
b. Pokud známe cenu celkem, pak daň vypočítáme jako součin ceny celkem krát
koeficient, kde činiteli je sazba daně a ve jmenovateli součet 100+sazba daně.
(100*21/121=17,36 Kč) Cena bez DPH je 82,64 Kč.

Osvobození od daně bez nároku na odpočet daně
e Poštovní služby
e © Rozhlasové a televizní vysílání
e © Finanční činnosti (poskytování úvěrů)
e © Penzijní Činnosti
e © Pojišťovací činnosti
e © Převod a nájem pozemků, staveb, bytů a nebytových prostor
e © Výchova a vzdělávání
e | Zdravotnické služby a zboží
e | Sociální pomoc
e © Loterie a podobné hry
Zdaňovací období
Základní zdaňovací období je I měsíc. Pokud plátce za předcházející kalendářní rok
nepřekročil obrat 10 mil.Kč, může požádat o zdaňovací období čtvrtletní.

Výhody DPH
1. Všeobecnost zdanění
2.. Zdanění spotřeby
3.. Z hlediska plátce je zatížena jen ta hodnota, kterou sám k výrobku přidal
4. Daň je lehce sledovatelná na cestě od prvovýroby ke spotřebiteli,
omezuje možnosti daňových úniků
5. Sbližujeme se s daňovými systémy v EU
Nevýhody DPH
1. Značná administrativní náročnost na straně podnikatelů i FÚ

54
\newpage
Postup přiznání a odvodu DPH

Po ukončení zdaňovacího období (čtvrt roku,měsíce) musí plátce podat na FÚ daňové
přiznání, kde uvede zdanitelná plnění přijatá (nakoupené zboží) a součet daně na vstupu,
kterou má nárok nechat si od FÚ vrátit.

Pak uvede zdanitelná plnění uskutečněná (prodané zboží) a součet daně na výstupu, který je
povinen odvést FÚ.

Rozdílem daně na vstupu a daně na výstupu vypočítáme skutečně placenou částku.

Uplatňování DPH při obchodu se zeměmi EU (neprochází celním režimem)
a) Firma z ČR zasílá zboží do EU firmě (je plátce DPH), musí na FA uvést její DIČ a
nechat si od ní průkazně potvrdit, že zboží ve své zemi obdržela.
b) Firma z ČR pořizuje zboží z jiné země EU od plátce DPH, nakoupí ho bez DPH a je
povinna v ČR přiznat DPH dle českého zákona o DPH.
Uplatňování DPH při obchodu s nečlenskými státy EU
DPH při dovozu vybírají celnice. Základ DPH při dovozu se vypočítá:
Celní hodnota zboží+clotspotřební daň = základ DPH
Při dovozu je DPH počítáno 1 ze cla a popřípadě 1 spotřební daně (u alkoholu, tabáku a
ropných produktů). Čím vyšší clo, tím více naroste DPH.

DPH je významným cenotvorným činitelem.

SPOTŘEBNÍ DANĚ
Spotřební daň je nepřímá selektivní daň, co znamená, že je vybírána prostřednictvím prodeje
vybraných druhů zboží.
Předmětem spotřební daně jsou:
e © Minerální oleje (uhlovodíková paliva a maziva)
e Lihalihoviny
e Pivo
e Víno
e | Tabákové výrobky

Zdaňovací období je jeden měsíc. Pokud v určitém měsíci nevznikne daňová povinnost,
daňové přiznání se nepodává.

Plátci spotřební daně jsou tedy nejen výrobci u tuzemské produkce a dovozci při dovozu
z nečlenských zemí EU, ale 1 provozovatelé daňových skladů.

Poplatníci jsou všichni, kdo tyto výrobky kupují pro vlastní spotřebu.

Sazby daně jsou stanoveny samostatně pro každý druh výrobku v závislosti na měrných
jednotkách. Velikost spotřební daně není závislá na ceně výrobku.

55
\newpage
DANĚ PRO ŽIVOTNÍ PROSTŘEDÍ (nepřímé daně)
e | Daň ze zemního plynu
e | Daň z pevných paliv
e | Daň z elektřiny
Zdaňovací období je kalendářní měsíc.
Sazby u jednotlivých daní jsou pevně dány zákonem ve vztahu k fyzikálním jednotkám
dodaných energií.

DAŇ Z PŘÍJMU
Zákon o dani z příjmu má tři části:
e | Daň z příjmu FO
e | Daň z příjmu PO
e | Společná část (odepisování HM a NDM)

Daň z příjmu FO
Poplatníci daně: FO které mají na území ČR bydliště. Jejich daňová povinnost se vztahuje
na příjmy plynoucí ze zdrojů na území ČR i na příjmy plynoucí ze zahraničí.
Plátci daně: Podnikatelé platí daň sami za sebe (plátci je shodný s poplatníkem)
U zaměstnanců odvádí zálohy na daň zaměstnavatel

Zdaňovací období je kalendářní rok, popřípadě hospodářský rok.

e © Podnikatel sám za sebe může podat přiznání do 3 měsíců

e © Daňový poradce, může požádat prodloužení až do konce června

e | Zasvé zaměstnance zpracovává firma přiznání do 15.února

Osvobození od daně:

e | Příjmy z prodeje bytů a rodinných domů

e | Příjmy z prodeje nemovitých věcí

e | Příjmy z prodeje movitých věcí

e | Ceny z veřejné soutěže, reklamní soutěže a ze sportovní (do 10 000 Kč)

e © Příjmy ve formě dávek nemocenského pojištění, důchodového, státní sociální podpory
apod.

e | Dotace od státu, granty z EU na pořízení hmotného majetku

e | Příjmy v naturální formě v podobě reklamních předmětů do hodnoty 500Kč

e © Dary mezi nepříbuznými osobami jsou osvobozeny do souhrnné hodnoty

e © Odroku 2014 již příjmy z dědictví nepodléhají dani dědické, ale jsou osvobozeny
v zákoně o dani z příjmu

56
\newpage
Předmět daně:
Příjmy v peněžní 1 nepeněžní podobě (naturálie, hmotné odměny, využívání služebního
automobilu pro osobní potřeby) kromě příjmů osvobozených od daně.
Členění příjmů:
a) Příjmy ze závislé činnosti (ze zaměstnání)
b) Příjmy ze samostatné činnosti (Z podnikání)
c) Příjmy z kapitálového majetku (dividendy, podíly na zisku)
d) Příjmy z nájmu
e) Ostatní příjmy (výhry, příjem z prodeje majetku (bez osvobození)







Základ daně:

U příjmů z podnikání zjistíme základ daně
z účetnictví firmy Výnosy-Náklady=základ daně
z daňové evidence Příjmy- Výdaje=základ daně





Vedení UCE 1 DE je složitá a nákladná věc, proto podnikatelé mají možnost zjednodušení:
e © Prokazovat pouze příjmy a výdaje v daňovém přiznání uplatnit paušálem
(80% zemědělci, 60% živnostníci, 40% nebo 30%)
e © Stanovení daně paušální částkou. Zákon stanoví přesné podmínky, podnikatel může
zažádat FÚ o stanovení paušální částky

Slevy na dani:

e | Snižuje konečnou vypočítanou daň (např.sleva na dani 24 840 Kč znamená, že od
vypočítané daně odečteme celou částku)

e | Sleva na poplatníka, na manželku s příjmem nižším než 68 000 Kč, invalidní
důchodce 1.,2.stupně, invalidní důchodce 3.stupně, student prezenčního studia,
daňové zvýhodnění na dítě

Nezdanitelná část základu daně je snížením základu daně:
e © Po snížení základu vypočítáme procentem daň (15% ze základu daně)
Daňový bonus:

e © Jsou zavedeny daňové bonusy v případě daňového zvýhodnění za každé dítě žijící ve
společné domácnosti ve výši 15 204 Kč- má-li poplatník v daném roce nižší daňovou
povinnost než je daňové zvýhodnění ve formě slevy na dani, je tento rozdíl daňovým
bonusem a bude mu FÚ proplacen

Sazba daně:

Pokud vypočítáme základ daně a upravíme ho o nezdanitelné a odčitatelné položky, je čas
vypočítat samotnou daň. [a činí 15%. Novinkou od ledna 2013 je solidární zvýšení daně 7%
z částky na 48násobek průměrné mzdy.

57


\newpage
DAŇ Z PŘÍJMŮ PRÁVNICKÝCH OSOB
Poplatníci daně jsou osoby, které nejsou FO (PO). Od daně se osvobozuje centrální banka
ČR. Poplatníci daně jsou současně plátci daně.

Zdaňovací období je kalendářní rok nebo hospodářský rok.

Předmětem daně jsou výnosy z veškeré činnosti a nakládání s majetkem, kromě
jmenovitých výjimek.

Základ daně je zisk, který zjistíme z účetnictví tak, že vypočítáme.
Výnosy - náklady = výsledek hospodaření

DAŇ SILNIČNÍ
Předmětem daně jsou silniční motorová vozidla používaná k podnikatelské činnosti.
Poplatníkem i plátcem je majitel vozidla uvedený v technickém průkazu vozu.

Sazba daně:
e © U osobních aut závisí na objemu válců v motoru
© © U nákladních aut závisí na hmotnosti vozidla a počtu náprav

Zdaňovací období je kalendářní rok, přičemž podnikatel může platit zálohy a do konce ledna
následujícího roku je vyúčtovat.

DAŇ Z NABYTÍ NEMOVITÝCH VĚCÍ

Jedná se o daň přímou, poplatník je zároveň plátce-v případě nabytí nemovité věci převodce.
Je to daň jednorázová (prodej nemovitosti, darování, dědictví). Sazba daně z nabytí
nemovitých věcí činí 4% z nabývací ceny, kterou je:

a) sjednaná cena nebo srovnávací daňová hodnota

b) zjištěná hodnota

c) zvláštní cena u obchodních korporací

DAŇ Z NEMOVITÝCH VĚCÍ

Daň z nemovitých věcí je spojena s vlastnictvím nemovité věci, jejím plátcem i poplatníkem
je majitel nemovité věci evidovaný v katastrálním úřadě k prvnímu dni kalendářního roku. U
pronajatých pozemků he id roku 2005 poplatníkem v určených případech nájemce.

Zdaňovací období je kalendářní rok. Tato daň má 2 části.
e © Daň z pozemků - základem je cena pozemku dle vyhlášky a forma využívání
pozemku u zemědělské půdy a lesů, u stavebních pozemků je to rozloha
e | Daň ze staveb a jednotek - základem je půdorys nadzemní části stavby v m2 a sazba
závisí 1 na poloze stavby (většinou se jedná o byt)

58
\newpage
CELNICTVÍ

V EU jsou mezi státy zrušena cla.

EU má jednotnou celní politiku vůči nečlenským státům.

-pokud firma z ČR vyváží zboží do USA, musí vyplnit celní dokumenty a zboží
prochází celní kontrolou, clo pro vývoz ale v EU není vyměřováno, takže firma
clo neplatí

-firma dováží zboží z USA, musí ho proclít-je uvaleno clo podle celního sazebníku
platného pro celou EU

Clo je celní poplatek, dávka vybíraná státem při přechodu zboží přes celní hranici.

Funkce cla:

e | Fiskální-příjem do státního rozpočtu-odvádíme do rozpočtu EU

e © Obchodně politická-nástroj hospodářské politiky

e © Cenotvorná-u dovozového zboží firmy započítají clo do prodejní ceny
Druhy cla v EU:

e © Dovozní clo-nejběžnější

e © Vývozní clo-není používáno

e © Vyrovnávací

e | Odvetné

e © Antidumpingové

Celní sazebník
Celní sazebník je jednotný pro všechny státy EU, sazebník obsahuje všechny druhy
dováženého zboží a jejich celní sazby (% cla z celní hodnoty).
Celní sazby obsažené v celním sazebníku můžeme členit:

e © Všeobecné celní sazby (většinou sazby nejvyšší)

e © Smluvní celní sazby (na základě mezinárodních dohod dohodnuté nižší)

e © Preferenční sazební opatření (vyplývající z mnohostranných mezinárod. dohod)
Všeobecná dohoda o clech a obchodu GATT
Nejdůležitější mezinárodní organizace GATT (1947) má 135 členských zemí (ČR-1993).
Základním cílem GATT je odbourávání překážek světového obchodu.
Usnadňuje vstup zboží na trhy smluvních stran.

52
\newpage
Rozdíly ve zdanění spotřeby a důchodu
1.. Zdanění příjmů přihlíží ke konkrétním poměrům poplatníka, zdanění spotřeby nikoliv.
Tím se přímé daně stávají adresným nástrojem regulace důchodů poplatníka, nepřímé

daně tuto schopnost nemají.

D3

Nepřímé zdanění má méně nepříznivý vliv na pracovní motivaci a výrobní aktivitu než

přímé.

V)

Vzhledem k tomu. že nepřímé daně jsou součástí konečné ceny, přispívají více k růstu
Inflace než daně přímé.
4.. Zdanění příjmů vyvolává u poplatníků větší odpor než zdanění spotřeby. To vede 1 ke

snaze obejít zdanění a daňovým únikům.
Lafferova křivka
/obrazuje závislost celkového objemu vybraných daní na míře zdanění.

(respektive na daňové sazbě)

A :Lafferův bod





prohibitivní
zóna

danovy příjem

t:




VO WWW W

0% 77 daňová sazba. 100%

Lafferova křivka

Aktuálnost daňové soustavy

Velikost příjmů veřejných rozpočtů z jednotlivých druhů daní

51
\newpage
Daně majetkové- platí poplatníci podle velikosti svého nemovitého majetku,
při majetkových převodech (darování, dědictví, prodej či převod) a při
využívání vozidel pro podnikání
Daně univerzální (DPH)- vybírány při prodeji téměř všech druhů zboží a služeb
Daně selektivní-spotřební daň, daně pro životní prostředí
-jsou vybírány pouze u vybraných druhů zboží
(cigarety, alkohol, benzin, nafta, pevná paliva, zemní plyn, elektřina)
Poplatník-je FO či PO, z jejíchž peněz je daň placena (ten, z jehož kapsy peníze)
Plátce-je FO či PO, která má ze zákona povinnost peníze odvádět státu
Daňové přiznání- podoba formuláře-doklad potřebný pro kontrolu správnosti
Daňový únik-je situace, kdy se plátce či poplatník vyhýbá úhradě daně, únik
může být úmyslný nebo neůúmyslný také legální nebo nelegální
Daňová kvóta-vyjadřuje celkovou úroveň daňové zátěže v dané zemi
Daňový ráj-se označují země s velmi nízkými daněmi a ekonomikou orientovanou
na zahraniční kapitál

Den daňové svobody

-hranice, která rozděluje kalendářní rok na 2 období

-v 1. vydělávají daňoví poplatníci na pokrytí výdajů vlády a institucí státu

-V 2. s1 až 0 penězích rozhodují svobodně sami
Slevy na dani-snižuje konečnou vypočítanou daň (na dítě, poplatníka, manželku)
Zdaňovací období

-je kalendářní rok nebo hospodářský rok

-rozhodné období, časový úsek, za který se počítá příslušná daň

Struktura daňové soustavy
1.. daně přímé (poplatník podává přímo na finančním úřadu daňové přiznání)
= důchodové:
= -© daň z příjmů fyzických osob a právnických osob
= © majetkové:
= © daň z nemovitostí (tj. daň z pozemků, daň ze staveb)
=- daň silniční
= daně převodové (tj. daň dědická, darovací a z převodu nemovitostí)
2. daně nepřímé (ze spotřeby-platíme při každém nákupu zboží a služeb)
=- univerzální:
= -© daň z přidané hodnoty,
« selektivní:
= © daně spotřební (daň z minerálních olejů, daň z alkoholu, cigaret a tabák.
výrobků)

« © ekologické (k ochraně životního prostředí)
= © daň ze zemního plynu
daň z pevných paliv

= daň z elektřiny

50
\newpage