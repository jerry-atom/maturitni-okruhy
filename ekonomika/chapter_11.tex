\chapter{Management}

Management (věda o řízení) je ucelený soubor ověřených přístupů, názorů, zkušeností,
doporučení a metod, které vedoucí pracovníci (manažeři) užívají k zvládnutí specifických
činností (manažerských funkcí), jež jsou nezbytné k dosažení soustavy podnikatelských cílů
organizace.

Manažer je profese a její nositel je zodpovědný za dosahování cílů svěřených mu
organizačních jednotek (útvarů, kolektivů), včetně tvůrčí účasti na jejich tvorbě a zajištění.
Využívá při tom kolektiv spolupracovníků.

Manažer a podnikatel
U menších firem podnikatel= manažer firmy.
Manažeři nenesou riziko podnikání, maximálně své místo, podnikatel o peníze.

Role a funkce manažera
e | Role manažera-pohled statický, charakterizující samotnou osobu manažera a její postavení



v hierarchii organizace
e © Funkce manažera-pohled procesní, dynamický, zachycující manažera při jeho řídící
Činnosti v organizaci

Role manažera
1. Podle úrovně řízení
e © Vrcholový management
-sem zařadíme nejvýše postavené pracovníky (ředitele, náměstky ředitele, prezidenty
společnosti a jejich viceprezidenty), skupina nejlépe placených
-na jejich Ž umění řídit a znalostech závisí úspěch firmy
e | Střední management
-sem patří vedoucí útvarů firmy (vedoucí marketingového oddělení, vedoucí nákupu,
vedoucí kontroly jakosti)



e © Nejnižší management
-sem zařadíme manažery na nejnižším stupni řízení (mistr dílny, vedoucí závodní
jídelny), očekává se schopnost řešení každodenních problémů

Role manažera

Manažerská pyramida








Koncepční práce
Nejvyšší představitele fireny, penerální
ředitel, ředitel, představenstvo. jednatel,
Zodpovida;í za podnik jako celek,



Střední
Smiiddiej
managemtit

Manažeři závodů, vedcuci útvarů,
středisek, podřízení top managementu
Zodpovídají za psřiděloné úseky



Předáci, mistři, vadaucí tymů, dilovedouci,
podřízení středního managementu,
Zodpovídají za týmy





Prvni lime

fowermanegement PE En







Výkanní pracovní

39
\newpage
2. Podle stylu řízení
e © Autokratický styl
-vedoucí, který sám rozhoduje a přikazuje svým podřízeným
-autokrat detailně a systematicky kontroluje, zda byly splněny jeho příkazy
-armáda



e © Demokratický styl
-je vedoucí (demokrat), bere ohled na názory svých podřízených, o problémech
diskutuje, konečné rozhodnutí musí však udělat sám a kontroluje jeho splnění
(plánování, projektování)



e © Liberální styl



-1berální vedoucí již nepoužívá přímých řídících příkazů
-styl je vhodný v organizacích, kde pracují pracovníci s VŠ vzděláním, kteří mají
vysokou vnitřní motivaci k práci (výzkumná pracoviště, vysoké školy)

Manažerské funkce
-jsou typické úlohy, které vedoucí pracovníci řeší v procesu své řídící práce

Sekvenční funkce

-plánování-manažer stanovuje cíle a postupy k jejich dosažení

-organizování-stanoví a uspořádá role lidí, kterým přidělí konkrétní práci

-výběr a rozmisťování pracovníků-vybírá a získává konkrétní pracovníky

-vedení ldí-vznikají vzájemné vztahy nadřízenosti

-kontrola-kontrola hodnotí kvalitu a kvantitu průběžných a konečných výsledků
a vyvodí příslušné závěry

Průběžné funkce

-prostupují všemi sekvenčními funkcemi
-analyzování problémů

-rozhodování

-koordinace při realizaci (implementace)

PLANOVANÍ
Je proces stanovení cílů řízené činnosti a vhodných cest a prostředků k jejich dosažení ve
stanoveném čase.

Postup tvorby plánu:

1. stanovím cíle

2. vymezím cesty jejich dosažení (varianty)
3. jednu variantu zvolím = to je plán

40
\newpage
Plány členíme z časového hlediska:
e Strategické plány (dlouhodobé)
e © Taktické plány (střednědobé-roční)
e | Operativní (krátkodobé-každodenní)

Vztah marketingu a managementu

Obě vědy mají své specifika: management se zabývá řízením a marketing trhem. Však firemní
marketing 1 management mají shodné cíle = prosperita firmy a dosažení zisku, mají 1 mnoho
společného. Jeden z bodů je právě plánování.

ORGANIZOVÁNÍ
Posláním organizování je zajistit dosažení stanovených cílů pomocí procesů specializace
navázané a nezbytné koordinace prací a lidí.
Kroky v procesu organizování
-identifikace činností-co vše je potřeba zajistit |
-seskupení vymezených činností-přiřadíme konkrétním organizačním útvarům
firmy (nákup materiálu zajistí zásobování)

-stanovení a přiřazení rolí lidí-při této činnosti říkám, kolik lidí a s jakými

konkrétními úkoly bude zajišťovat práci

Organizační struktury
Je organizovaný systém, ve kterém je práce rozdělena, seskupena a koordinována. Jsou
graficky zobrazovány v organizačních schématech.
Třídění organizačních struktur
1. Hledisko formálnosti:
e © Formální struktura-nadřízenost a podřízenost, činnosti
e © Neformální struktura-vedoucí nejsou jmenováni
2. Hledisko druhu sdružování
e © Funkcionální struktura
e © Výrobková struktura
e © Ostatní účelové struktury
3. Hledisko rozhodovací pravomoci:
e © Liniové struktury
e © Štábní struktury
e © Liniově štábní struktury
e © Cílově programové struktury

41
\newpage
4. Hledisko míry centralizace:
e © Centralizované
e | Decentralizované
5. Hledisko počtu řídících úrovní:
e | Ploché struktury
e © Úzké struktury
6. Hledisko časového trvání:
e © Dočasné
e Trvalé
Faktory ovlivňující volbu organizační struktury
1. Vnitřní faktory
Sem patří faktory jako velikost firmy, výrobně-technická základna (strojní vybavení),
teritoriální rozmístění (jestli má firma pobočky či závody)
2. Vnější faktory
Jedná se o faktory, které firma sama není schopna ovlivnit (legislativní možnosti, stabilita
podnikatelského okolí)





VÝBĚR, ROZMISŤTOVÁNÍ A HODNOCENÍ PRACOVNÍKA
Získávání vhodných pracovníků
1. Definujeme naší potřebu
2. Realizace personálního zajištění
e | Fáze plánovací
e | Fáze náboru a výběru
Zvyšování kvalifikace, rekvalifikace
Ne vždy je potřeba s novou prací hned přijímat nového pracovníka. Stačí u stávajících
zaměstnanců pouze zvýšit nebo změnit kvalifikaci (znalosti, dovednosti a návyky, které
využívají při práci).
Proces zvyšování kvalifikace:
1. fáze- stanovení reálné potřeby zvýšení kvalifikace
2. fáze-vlastní zvýšení kvalifikace
3. fáze-vyhodnocení výsledků
Způsoby zvyšování kvalifikace:
a) školení v rámci pracovního procesu-je organizováno v rámci podniku
např. zaškolení mistrem k určité nové práci
b) školení mimo pracovní proces-kurzy, dálkové university, jazykové stáže,
školení (kurzy daňových poradců, svářečské kurzy)

Pracovní kariéra manažera

1. etapa - přípravná - doba studia (SŠ,VŠ)

2. etapa - zakotvení - zapracování se v prvním zaměstnání

3. etapa - rozvoj - má manažer pro firmu velký význam a přínos (35-50 let)

4. etapa - pozdní kariéra - dochází obvykle k úbytku energie, manažeři začínají mít sklon
k rutinním řešením

A2
\newpage
Hodnocení pracovníků

Má pomoci lépe využít profesní kvalifikaci zaměstnanců, rozvíjet jejich pracovní kariéru,
motivovat a spravedlivě odměňovat.

Obvykle se hodnotí:

a) plnění pracovních úkolů

b) chování v pracovním procesu a mimo něj

c) osobní a charakterové rysy

Hodnocení provádí:
1. vedoucí pracovníci
2. pracovníci personálních útvarů
3. externí nebo interní specialisté

Systémy odměňování
Na základě hodnocení pracovníka provádí vedoucí jeho odměňování či potrestání. Systémy
odměňování úzce souvisí s motivací pracovníků na jedné straně a finančními možnostmi
firmy na straně druhé. S odměňováním souvisí i daňová problematika.
1. hmotné odměny
e | Přímé odměny (mzda, prémie a odměny, podíly na zisku)-prochází zdaněním
e | Nepřímé odměny (příplatky na dovolenou, životní pojištění a důchodové, příspěvky na
stravování, na mateřské školy, poskytované zboží a služby
(některé podléhají zdanění daní z příjmu)
2. nehmotné odměny
e © Prestižní funkce, volná či individuální pracovní doba, možnost odborného růstu

Hodnocení nekvalitní práce
e | Ústní napomenutí
e © Písemné napomenutí
e | Finanční postih (odebrání prémií, osobní ohodnocení)

Výpověď (v extrémních případech okamžité zrušení pracovního poměru)

Vedení lidí-teorie X a Y

Je čtvrtou manažerskou funkcí. Manažer musí správně vést pracovníky, aby pracovali

v žádoucí kvalitě, kvantitě a směrem naplnění cílů.

Teorie X vychází z předpokladu, že průměrný pracovník nemá své zaměstnání rád, je mu
přítěží nutnou k zajištění obživy. Nemá zvláštní ambice. Základním rysem je lenost a snaha
práci se vyhnout.

Teorie Y vychází z opačných předpokladů: pracovník má přirozený sklon k práci, chce

V práci najít svou seberealizaci, má sklon k odpovědnosti, aktivně se účastní na práci a řízení.
Teorie X a Y představují vlastně dva extrémy chování podřízených a jim odpovídající řídící
působení vedoucího pracovníka. V praxi samozřejmě nepracujeme s průměrnými lidmi, každý
člověk je jiný, a proto u každého je třeba najít potřebnou míru skloubení teorie X a teorie Y
současně. Vedoucí tedy hledá určitý kompromis pro každého jednotlivého podřízeného.

43
\newpage
KONTROLA
Poslední manažerskou funkcí je kontrola=proces sledování, rozboru a přijetí závěrů
V souvislosti s odchylkami od záměru (cíle).

Fáze kontrolního procesu:

. stanovení cíle kontroly

. stanovení kontrolních kritérií

. rozbor kontrolovaných procesů a porovnání s kritérii kontroly

. vyhodnocení zjištěných odchylek a přijetí závěrů

. realizace závěrů (firma předepsala manko k úhradě skladníkovi)

U KB UO NN