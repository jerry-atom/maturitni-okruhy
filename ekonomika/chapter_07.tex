\chapter{Zásobování a logistika}

Zásobování je činnost podniku, při niž si podnik zajišťuje potřebné suroviny a materiál V požadovaném množství, kvalitě, druzích ve stanovené době a ve výhodných cenách. Tyto suroviny a materiál používá pro svou činnost.

Oběžný majetek - členění:
\begin{enumerate}
    \item ZÁSOBY
        \begin{itemize}
            \item Materiál
            \item Nedokončená výroba (neupečený rohlík)
            \item Polotovary (deska dřeva)
            \item Hotové výrobky (výrobky, které už firma dokončila)
            \item Zboží (vše, co je nakoupené za účelem dalšího prodeje)
            \item Zvířata
        \end{itemize}
    \item PENÍZE
        \begin{itemize}
            \item Peníze v hotovosti v pokladně
            \item Peníze na účtech peněžních ústavů
            \item Ceniny - kolky, stravenky, poukázky, dálniční známky, dopisní známky,mobilní karty
            \item Krátkodobé cenné papíry - směnky, depozitní certifikáty, vkladový list
            \item Pohledávky - peníze, které firmě dluží odběratelé, společníci, zaměstnanci, dlužníci
        \end{itemize}		
\end{enumerate}

Členění materiálu (základní suroviny - stavební hmoty, kov, dřevo, kůže\ldots)
\begin{itemize}
    \item Pomocné materiály - barvy, mořidla, maziva\ldots
    \item Obaly - plechovky, kartóny, plasty\ldots
    \item Pohonné hmoty - nafta, benzin\ldots
    \item Drobné nářadí - šroubováky, klíče, vrtáky, přípravky\ldots
    \item Kancelářské potřeby - papíry, tužky, pásky do psacího stroje, šanony\ldots
    \item Čistící prostředky - pro hygienu zaměstnanců, úklid prostor\ldots
\end{itemize}

Obaly a jejich funkce:
\begin{itemize}
    \item plechovky, kartóny, plasty, palety, kontejnery, láhve\ldots)
    \item slouží k ochraně a dopravě nakoupeného materiálu, zboží a výrobků
\end{itemize}

Koloběh oběžného majetku:
[Obr. 1 Koloběh oběžného majetku firmy]

Platí, že peníze na začátku koloběhu by měly být menší než na konci = zisk firmy.

\paragraph*{Evidence a doklady při zásobování}
\begin{enumerate}
    \item \textbf{Dodací list}
        \begin{itemize}
            \item vystavuje dodavatel pro kontrolu, co za zboží posílá
            \item fyzicky musí jít s dodávkou, aby odběratel mohl provést přejímku zboží
        \end{itemize}
    \item \textbf{Faktura - daňový doklad} - je dokladem, který slouží pro
        \begin{itemize}
            \item Zanesení do účetnictví
            \item Pro účely zůčtování daně z přidané hodnoty
            \item Vznik a uhrazení závazku
        \end{itemize}
        Fakturu smí dodavatel vystavit nejdřív v okamžiku zaplacení nebo v okamžiku zaplacení nebo v okamžiku předání zboží odběrateli nebo prvnímu veřejnému přepravci (České dráhy, kamiónová doprava, lodní doprava, letecká doprava\ldots)
    \item \textbf{Příjemka}
        \begin{itemize}
            \item je doklad vystavený ve skladu odběratele a spolu se skladní kartou a výdejkou se vztahuje ke skladovému hospodářství
            \item příjemka slouží jednorázově pro jedno přijetí
                \begin{itemize}
                    \item Přejímka je proces kontroly a přijímání zboží na sklad
                    \item Příjemka je doklad evidující přijaté zboží
                \end{itemize}
        \end{itemize}
    \item \textbf{Skladní karta}
        \begin{itemize}
            \item je doklad vystavený ve skladu, slouží pro evidenci pohybu zásoby určitého druhu v čase
        \end{itemize}
    \item \textbf{Výdejka}
        \begin{itemize}
            \item slouží k výdeji ze skladu do výroby
            \item jednorázový multidruhový doklad
            \item současně se při výdeji provede odepsání vydaného množství ze skladní karty
        \end{itemize}
    \item \textbf{Kniha došlých faktur}
        \begin{itemize}
            \item Je evidence sloužící v účtárně k přehledu o vzniklých závazcích firmy a o datu a způsobu uhrazování těchto faktur
            \item kniha je velmi důležitá při běžné práci i inventurách
        \end{itemize}
    \item \textbf{Příkaz k úhradě 1 výdajový pokladní doklad}
\end{enumerate}

\begin{description}
    \item[Skladování] Skladování je činnost, při níž se hmotné statky soustřeďují na určitém místě a ve stanoveném množství a připravují se pro další činnosti: výdej do spotřeby\ldots
    \item[Metoda JUST-IN-TIME] Materiál je přivážen v přesném čase přímo k výrobní lince (vůbec není skladován ve skladu zásobování). Zcela odpadají náklady na skladování.
    \item[Řízení zásob - metoda ABC]
        \begin{description}
            \item[Skupina A] - metoda normování zásob. Sem zařadíme především základní suroviny, které nezbytně firma potřebuje pro svou výrobu
            \item[Skupina B] - sem patří zásoby, které se relativně snadno a rychle objednávají a jejich spotřeba už pro firmu není tak nákladově významná
            - stanovit a hlídat minimální skladový limit
            \item[Skupina C] - tato skupina je počtem druhů zásob největší, ale objemem spotřeby ve finančním vyjádření je pro firmu nejméně významná
        \end{description}
    \item[Normování zásob]
        \begin{itemize}
            \item Časová norma zásob - udává čas, jak dlouho vydrží průměrná zásoba
            \item Normovaná zásoba v naturálních jednotkách - udává fyzický objem zásoby
            \item Normovaná zásoba ve finančním vyjádření - udává objem peněz v zásobách
        \end{itemize}
    \item[Plán zásobování formou bilance]
        \begin{equation*}
            \sum_{}^{} P = \sum_{}^{} Z
        \end{equation*}
        \begin{description}
            \item[$P$] Potřeby (co bychom potřebovali)
            \item[$Z$] Zdroje (vyrobime, nakoupíme hotové)			
        \end{description}
\end{description}

[Obr. SCHÉMA NORMOVÁNÍ ZÁSOB]

\begin{description}
    \item[Zásoba běžná] zásoba, ze které se průběžně vydává podle požadavků výroby
    \item[Zásoba pojistná] zásoba pro případ, kdy se dodavatel opozdí s dodávkou
    \item[Zásoba technická] bývá pouze u některých zásob technickou zásobu nejsme schopni předčasně čerpat, protože tato zásoba ještě není technologicky připravená (dosušení zásoby - dřevo)
    \item[Dodávkový cyklus] čas mezi dvěma smluvními dodávkami od dodavatele
\end{description}

\section*{Právní stránka obchodních vztahů}
\subsection*{Kupní smlouva}
Kupní smlouvou se prodávající zavazuje dodat kupujícímu movitou věc (zboží) a převést na něho vlastnické právo k této věci a kupující se zavazuje zaplatit kupní cenu. K platnému vzniku smlouvy stačí dohoda o podstatných náležitostech smlouvy:
\begin{itemize}
    \item Určení smluvních stran (prodávající a kupující)
    \item Určení předmětu
    \item Určení kupní ceny
\end{itemize}

\paragraph*{Forma uzavření smlouvy}
Kupní smlouvu k movitým věcem lze uzavřít písemně, ústně nebo konkludentním jednáním.

Povinnosti prodávajícího:
\begin{itemize}
    \item Dodat řádně zboží
    \item Předat potřebné dokumenty
    \item Umožnit kupujícímu nabytí vlastnického práva
    \item Uchovávat zboží v případě, je-li kupující v prodlení s převzetím	
\end{itemize}

\paragraph*{Povinnosti kupujícího}
\begin{itemize}
    \item Zaplatit kupní cenu
    \item Převzít a prohlédnout dodané zboží
\end{itemize}
Vlastnické právo přechází na kupujícího převzetím, není-li dohodnuto jinak.

\subsection*{Smlouva o dílo}
Smlouvou o dílo se zhotovitel zavazuje provést na svůj náklad a nebezpečí pro objednatele dílo a objednavatel se zavazuje dílo převzít a zaplatit cenu.

Předmětem smlouvy je dílo:
\begin{itemize}
    \item Zhotovení určité movité věci
    \item Oprava, údržba nebo úprava určité movité věci
    \item Stavba, její zhotovení, údržba, oprava nebo úprava
    \item Samostatnou úpravu má zhotovení díla s nehmotným výsledkem
\end{itemize}

\textbf{Forma uzavření smlouvy - písemná i ústní}

Podstatné náležitosti:
\begin{itemize}
    \item Určení stran - objednatel a zhotovitel
    \item Předmět smlouvy (popis díla)
    \item Cena nebo způsob jejího určení	
\end{itemize}

\subsection*{Reklamace}
Pro uplatnění práv z odpovědnosti za vady musí poškozená strana oznámit vady.
Reklamaci je dobré provádět písemně, např. formou dopisu.


