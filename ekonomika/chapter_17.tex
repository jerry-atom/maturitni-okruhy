\chapter{Makroekonomické veličiny}

\textbf{Hrubý domácí produkt (HDP)} je souhrn statků a služeb vyjádřený v penězích vytvořený za určité období výrobními faktory (práce, přírodní zdroje, kapitál) na území státu, bez ohledu na to, zda jsou vlastněny občany státu nebo cizinci.

\textbf{Ekonomická rovnováha} je bod, ve kterém jsou si rovny množství nabízeného a poptávaného zboží/služby

\textbf{Agregátní poptávka} -- poptávka všech subjektů v celém hospodářství

\textbf{Agregátní nabídka} -- nabídka všech subjektů v celém hospodářství
\begin{itemize}
    \item pokud \textit{agregátní poptávka převyšuje agregátní nabídku} je to stav podezřelý, protože strana poptávky strana poptávky má více peněz než je na trhu zboží a toho lze dosáhnout buď inflačním znehodnocením peněz nebo tak, že poptávající si peníze půjčili a chtějí spotřebovávat na dluh. Tento stav je založen na nezdravém principu, má negativní dopady na hospodářství
    \item \textit{převis agregátní poptávky nad agregátní poptávkou} -- pokud hospodářství vyprodukovalo více zboží než je samo schopno spotřebovat, má jedinou šanci -- jestli firmy vyrobily produkci natolik efektivně, že je konkurence schopná na zahraničních trzích, mohou zboží vyvézt a vydělat, pokud však své zboží nerealizují -- krachují, propouštějí zaměstnance, stát má nižší příjmy z daní -- hovoříme o hospodářské krizi z nadvýroby
    \item ekonomické dění ve společnosti je trvalým pohybem, trvalou změnou, stálým kolísáním. Základní tendence dosažení ekonomické rovnováhy jako nejefektivnějšího stavu je tak dosahována posloupností dílčích nerovnováh => cyklický vývoj hospodářství
\end{itemize}

\paragraph{Fáze hospodářského cyklu}
\begin{enumerate}
    \item \textbf{expanze} (rozvoj, konjunktura, rozmach nabídky 1 poptávky) -- domácnostem i firmám se daří, rostou zisky i platy, roste spotřeba statků a služeb, stát dostává více daní, takže může více investovat
    \item \textbf{vrchol} (převis nabídky nad poptávkou) -- firmy pořád produkují vysokým tempem nové statky a služby,ovšem strana poptávky už začíná zaostávat
    \item \textbf{krize} (recese, deprese, pokles nabídky 1 poptávky) -- poptávka se zcela zabrzdila, firmy mají problémy s prodejem, krachují, zvyšuje se nezaměstnanost, lidé mají existenční problémy, 1 ti co mají práci raději spoří a neutrácí, což krizi prohlubuje
    \item \textbf{sedlo} (dno, vyrovnání, oživení nabídky a poptávky) -- žít se musí, takže poptávka nikdy neklesne na nulu. Firmy minimalizovali náklady a ceny, ty které přežijí recesi se dostanou se svou nabídkou do souladu s poptávkou.
\end{enumerate}

\paragraph{Inflace}
Ceny na trhu nejsou stabilní. Jeden z významných faktorů, který je ovlivňuje je hodnota peněz, kterými se ceny měří. Pokud peníze ztrácejí svou hodnotu, hovoříme o inflaci (znehodnocení peněz, růst cenové hladiny).

\paragraph{Příčiny inflace}
\textbf{Inflace tažená poptávkou} -- na trhu je více poptávky při stejném objemu nabídky, příčina růstu cenové hladiny je na straně poptávky (schodek státního rozpočtu, nákupy na dluh) \\
\textbf{Inflace tažená nabídkou} -- zvyšování cenové hladiny je na straně výrobců (růst nákladů na výrobu)

Inflace je makroekonomický pojem, je to velmi důležitý ukazatel nejen pro národní ekonomiku, její ekonomy a politiky, ale i pro mezinárodní srovnávání a rozhodování, proto existují mezinárodně platné postupy, podle kterých se inflace počítá.

Míra inflace se dá vyjádřit ukazatelem:
\begin{displaymath}
    \text{Míra inflace v \%} = \frac{\text{cenová hladina}_t - \text{cenová hladina}_{t - 1}}{\text{cenová hladina}_{t - 1}}
\end{displaymath}
kde $t$ je určité období (měsíc. rok).

\paragraph{Indexy k vyjádření cenové hladiny}
\begin{enumerate}
    \item \textbf{Index spotřebitelských cen (CPI)} -- cenová hladina je průměrem úrovně cen spotřebních výrobků a služeb ( hovoříme o spotřebním koši, ve kterém má každá skupina spotřebovávaných výrobků a služeb určitou váhu, podíl). U daných druhů zboží se sleduje po celém území státu v pravidelných intervalech pohyb cen
    \item \textbf{Index cen výrobců (PPI)} -- sleduje se pro různá odvětví a obory, všeobecně se má za to, že vývoj PPI signalizuje nadcházející změny CPI
    \item \textbf{Deflátor HDP} -- cenový deflátor HDP se vytvoří jako poměr HDP v běžných cenách k HDP ve stálých cenách určitého roku. Změnu cenové hladiny tak získáme zprostředkovaně (implicitně). Protože se jedná o komplexnější zobrazení vývoje cen všech statků a služeb v ekonomice, je tento ukazatel přesnější než CPI. Na druhou stranu má však jednu nevýhodu -- můžeme ho spočítat pouze zpětně, až když statistický úřad vyjádřil HDP za předcházející rok.
\end{enumerate}

\paragraph{Míry inflace}
\begin{itemize}
    \item \textbf{inflace mírná (plíživá)} -- jednociferná. Lidé nepřestávají věřit penězům, ekonomika běžně funguje, tempo růstu cen odpovídá tempu růstu a výroby
    \item \textbf{inflace pádivá} -- dvojciferná. Lidé přestávají věřit domácí měně, preferují stabilnější cizí měny nebo jiné trvalejší hodnoty (zlato, nemovitost atd.). Chod ekonomiky už je narušován, ekonomická výkonnost klesá
    \item \textbf{hyperinflace} -- trojciferná a větší. Ceny se zvyšují natolik rychle, že peníze přestávají plnit svou funkci uchovatele hodnot a zprostředkovatele směny, lidé preferují naturální směnu, ekonomický systém společnosti se úplně rozpadá, nastává chaos a anarchie.
\end{itemize}

\paragraph{Deflace} je opak inflace, absolutní meziroční pokles cenové hladiny v ekonomice.

\paragraph{Desinflace} je opakem akcelerující inflace (zpomalující inflace), je pokles tempa růstu všeobecné cenové hladiny

\subsection{Nezaměstnanost} -- vzniká, pokud na trhu práce převyšuje nabídka práce zaměstnanců poptávku firem.
\begin{enumerate}
    \item ekonomicky aktivní obyvatelstvo (t1 co pracují nebo aktivně hledají práci) tvoří trh práce
    \item ekonomicky neaktivní obyvatelstvo (děti do 15ti let, invalidé, důchodci) mimo trh práce
\end{enumerate}

\paragraph{Nezaměstnanost členíme}
\begin{itemize}
    \item nezaměstnanost dobrovolná -- lidé mají pracovní sílu, ale za nabízenou mzdu nejsou ochotni pracovat
    \item nezaměstnanost nedobrovolná -- lidé chtějí a potřebují pracovat, aby si zajistili obživu, ale nemohou odpovídající práci sehnat => stát jim pomáhá situaci řešit:
        \begin{enumerate}
            \item aktivní opatření -- rekvalifikace, podpora vzniku nových pracovních míst, zaměstnávání absolventů škol, daňové úlevy při zaměstnání postižených občanů apod.
            \item pasivní opatření -- podpora nezaměstnanosti ze státního rozpočtu
        \end{enumerate}
\end{itemize}

\paragraph{Míra přerozdělování}
\begin{itemize}
    \item dostatečná sociální pomoc, ale zároveň ne tak vysoká, aby nesváděla k jejímu zneužívání
    \item míra přerozdělování nezatížila neúměrně pracující část populace
\end{itemize}

\paragraph{Příčiny nezaměstnanosti}
\begin{itemize}
    \item \textbf{frikční nezaměstnanost} -- lidé běžně mění svou práci -- přirozený, krátkodobý jev
    \item \textbf{strukturální nezaměstnanost} -- některé odvětví se dostává do útlumu, lidé přicházejí o práci. Jiné odvětví v národním hospodářství jsou naopak ve fázi rozvoje a nové pracovní síly potřebují -- řešením je rekvalifikace
    \item \textbf{cyklická nezaměstnanost} -- v období krize a sedla dochází k nárůstu nezaměstnanosti a snižování objemu mezd, naopak v období konjunktury dochází k nárůstu zaměstnanosti a zvyšování mezd
\end{itemize}

\paragraph{Měření míry nezaměstnanosti}
\begin{displaymath}
    \text{Míra nezaměstnanosti v \%} = \frac{\text{nezaměstnaní, aktivně hledající práci}}{\text{ekonomicky aktivní obyvatelstvo}} 100
\end{displaymath}

Obecné míry nezaměstnanosti počítá \textit{Český statistický úřad} nebo \textit{Eurostat}, také ještě \textit{Ministerstvo práce a sociálních věcí ČR (MPSV)}.

Na makroekonomické úrovní probíhají jednání v Radě pro sociální dialog (tripartita), kdy účastníky jsou stát, podnikatelé a odborové organizace (zástupci zaměstnanců). Dlouhodobou a vysokou nezaměstnanost stát musí řešit.

\section*{Mezinárodní obchod}
\begin{itemize}
    \item jsme malou zemí a je pro nás životně důležité zapojit se do mezinárodního obchodu
    \item jsme členy WTO (dříve GATT), což je světová organice působící směrem k odstranění ochranářských opatření jednotlivých států (cla, dovozní a vývozní kvóty zákazy obchodu apod.)
    \item teoretické vysvětlení ekonomických výhod mezinárodního obchodu spočívá v objasnění absolutních a komparativních výhod:
        \begin{itemize}
            \item \textbf{absolutní výhoda} -- nižší náklady, vyšší produktivita práce vedou k výrobě určitého výrobku za nižší cenu než v ostatních státech, pro ostatní státy je ekonomičtější zboží dovést než ho vyrábět doma dráž
            \item \textbf{komparativní výhoda} -- souvisí s omezeností zdrojů každé země, pro zemi je výhodné vyrábět produkci, u které dosahuje ve srovnání s ostatními zeměmi co největší absolutní výhodou a ostatní výrobky dovážet.
        \end{itemize}
\end{itemize}
