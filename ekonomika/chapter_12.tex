\chapter{Hospodářská politika}

Hospodářská politika se nás dotýká dnes a denně, přímo 1 nepřímo.
HP je souhrn cílů, nástrojů, rozhodovacích procesů a opatření státu zaměřených na kontrolu a
ovlivňování ekonomického vývoje.

Propojení ekonomiky a politiky:

Ekonomika a politika spolu úzce souvisí, hospodářská politika se dotýká denně všech lidí i
firem přímo 1 nepřímo.

Její součástí je např.: zdravotní péče a úhrady za ni, výše školného, regulace výše nájemného,
velikost daní, podmínky přístupu k cizím měnám

HP je ovlivňována tím, která z politických stran se při volbách dostala k moci (pravicová-
podnikatelé, levicová-zaměřuje se na sociálně slabší a na „normální“.

Subjekty hospodářské politiky:
e Parlament
-složený z 2 komor (poslanec. sněmovna-200 poslanců a senát-81 senátorů)
-volení zástupci občanů, kteří schvalují zákony
-státní rozpočet je zákonem, sestavuje se vždy na 1 rok, stát plánuje,
kolik vybere na daních od občanů a firem a co bude z těchto peněz
financovat, parlament schvaluje návrh rozpočtu
e Vláda
-nejvyšší výkonný orgán, je základním tvůrcem konkrétní aktuální HP
-připravuje návrh rozpočtu a předkládá ho ke schválení parlamentu, pokud
je rozpočet schválen, je poslanci vyjádřena důvěra
e Centrální banka (Česká národní banka)
-jejím hlavním úkolem je střežit a korigovat komerční bankovní trh
a množství peněz v oběhu tak, aby nedocházelo k jejich znehodnocování
-inflaci (vede účet státního rozpočtu, vydává peníze)
-řídí bankovní trh
-sídlí v Praze, řídí ji guvernér-Jiří Rusnok a 6 členů bankovní rady
-guvernéra jmenuje prezident

Funkce institucí státu

e © Funkce právní jistoty a bezpečí
Cílem je vytvořit právní podmínky a dodržování zákonů, jde o bezpečnost vnitrostátní
1 mezinárodní, ministerstvo obrany má na starost armádu a ministerstvo vnitra-polici1

e © Funkce sociální
Státní instituce provádějí přerozdělovací procesy (transfery obyvatelstvu=peníze
vybrané na daních a sociální pojištění používají na výplatu důchodů, sociálních a
nemocenských dávek.
Starají se p veřejné statky (školství, zdravotnictví, Životní prostředí, infrastruktura-
dálnice, železnice, energetika)
Fungující neziskový sektor je nutnou podmínkou pro fungující ekonomiku

45
\newpage
e © Funkce hospodářská
1.stát sám podniká-za účelem dosažení zisku a rozmnožení bohatství státu
2.stát vytváří podmínky pro úspěšné podnikání ostatních subjektů hospod.
-kdy stát určuje daňové zatížení
-pravidla hospodářské soutěže: konkurenční boj, zákaz klamavé reklamy

Nástroje hospodářské politiky

1. Právní systém, legislativní proces
2. Monetární systém

3. Státní rozpočet a fiskální politika
4. Důchodová a cenová politika

5. Zahraničně obchodní politika

1. PRÁVNÍ SYSTÉM A LEGISLATIVNÍ PROCES

Zákony v ČR prochází schválením Parlamentu a vyjadřuje se k nim prezident republiky.
Schválené normy jsou zveřejňovány ve sbírce zákonů a jsou závazné pro občany státu 1
všechny, kdo se zdržují na území naší republiky.

Porušení zákona je předmětem šetření a příslušné státní instituce podřízené vládě, mají
vymezeny své pravomoci trestat pachatele (krádež či dopravní nehoda-policie, daňové
nesrovnalosti-FÚ, nedovolené podnikání-ŽÚ). Sporné případy řeší soudy.

2. MONETÁRNÍ SYSTÉM

Základní úlohu hraje centrální banka dané země (u nás ČNB). Specifická je Evropská
centrální banka se sídlem ve Frankfurtu, která byla založena pro EU po zavedení jednotné
měny eura.

K udržení stabilní měny používá každá centrální banka následující nástroje:

e REPO SAZBA
-je úrok centrální banky pro terminované operace s komerčními bankami

e © POVINNÉ MINIMÁLNÍ REZERVY (PMR)
-každá komerční banka má povinnost složit u CB část svých depozit (vkladů) jako
rezervu, a tím tyto peníze „umrtví“a nemůže s nimi podnikat

e OPERACE NA VOLNÉM TRHU
-CB prodává a nakupuje cenné papíry, obvykle státní dluhopisy

* OSTATNÍ NÁSTROJE MĚNOVÉ POLITIKY
-CB může stanovovat úvěrové limity

3. STÁTNÍ ROZPOČET A FISKÁLNÍ POLITIKA

K plnění svých funkcí potřebuje stát peníze, jako řádný hospodář dopředu plánuje, kolik

v následujícím roce peněz potřebuje (výdajová stránka státního rozpočtu) a kolik jich získá
(příjmová stránka).

46
\newpage
Porovnáním těchto dvou stran pak státní rozpočet může být:

VYROVNANÝ- příjmy=výdaje

PŘEBYTKOVÝ- příjmy>výdaje

SCHODKOVÝ - příjmy<výdaje

Přebytkový rozpočet vytváří rezervu státu na dobu nepříznivého období.

Schodek znamená, že si stát na své výdaje musí někde půjčit. Většinou vydá státní dluhopisy, které si
koupí tuzemské banky a podniky nebo zahraniční investoři.

Návrh předkládá vláda ke schválení parlamentu a schválený rozpočet má podobu zákona.

RESTRIKTIVNÍ FISKÁLNÍ POLITIKA

=výsledek přebytkového rozpočtu, nepodporuje rozvoj ekonomiky

-stát nevytváří nové pracovní příležitosti, nedává zakázky soukromým firmám
EXPANZIVNÍ FISKÁLNÍ POLITIKA

=výsledek schodkového rozpočtu, podporuje rozvoj ekonomiky, ale na „dluh“

Schéma vyrovnaného státního rozpočtu







Příjmová stránka Výdajová stránka
e Daně e | Státní správa (úřady)
e Cla e | Obrana státu, školství, zdravotnictví
e | Sociální pojištění e | Transfery obyvatelstvu (důchody, dávky
e | Ostatní příjmy (poplatky atd.) atd.)
e | Příjmy z rozpočtu EU e | Státní zakázky (dálnice)
Investice do život.pojištění
Odvody do rozpočtu EU





Příjmy celkem = Výdaje celkem

Náš státní rozpočet na jeden rok je okolo 1 100 mld.Kč, což je necelá třetina našeho
ročního HDP(hrubý domácí produkt).

4. DŮCHODOVÁ A CENOVÁ POLITIKA

Regulace mezd u nás byla používána v roce 1990-1995, dnes stát mzdy v soukromém sektoru
nereguluje a ponechává je volnému působení trhu.

Regulace cen-ceny státem regulované jsou elektřina, plyn, voda, nájemné v bytech, dálkové
vytápění, telekomunikační a poštovní poplatky apod.

5. ZAHRANIČNĚ OBCHODNÍ POLITIKA
Stát řeší kurz koruny k zahraničním měnám, celní politiku, kvóty-dovozní, vývozní (množstevní
omezení dovozu a vývozu), embargo (úplný zákaz dovozu a vývozu), cla.

CLO
Celní poplatek, je dávka vybíraná státem při přechodu zboží přes celní hranici.
V EU jsou mezi státy zrušena cla.
Funkce cla:
e | Fiskální-příjem do státního rozpočtu-odvádíme do rozpočtu EU
e © Obchodně politická-nástroj hospodářské politiky
e | Cenotvorná-u dovozového zboží firmy započítají clo do prodejní ceny

47


\newpage
Druhy cla v EU:

Dovozní clo-nejběžnější
Vývozní clo-není používáno
Vyrovnávací

Odvetné

Antidumpingové

DOVOZNÍ A VÝVOZNÍ KVÓTY
Stanovené počet nebo poměr.

Dovozní kvóta je omezení množství určitého dováženého statku.
Vývozní kvóta je omezení množství určitého vyváženého statku.

MA ASTRICHTSKÁ KRITÉRIA

Jsou kritéria pro členské státy EU pro vstup do 3.fáze Evropské hospodářské a měnové unie

(EMU) a pro zavedení společné měny-eura.

48
\newpage