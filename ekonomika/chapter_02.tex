\chapter{Trh a jeho zákony}

Trh je místo, kde dochází ke směně zboží. Místo, kde se setkává nabídka s poptávkou.
Zboží je statek nebo služby určená ke směně na trhu.

\begin{table}[h]
    \centering
    \caption{Členění trhu}
    \begin{tabular}{| p{5cm} | p{5cm} | p{5cm} |}
        \hline
        Podle územního hlediska &
        Podle množství &
        Podle předmětu prodeje a koupě zboží \\
        \hline
        \begin{enumerate}[label=(\alph*)]
            \item místní (oblastní) - tvarůžky
            \item národní - italská pizza 
            \item nadnárodní (EU) - ovoce 
            \item světový - ropa
        \end{enumerate} &
        \begin{enumerate}[label=(\alph*)]
            \item dílčí - trh mléka
            \item agregátní - trh veškerého zboží
        \end{enumerate} &
        \begin{enumerate}[label=(\alph*)]
            \item trh výrobních faktorů (trh práce)
            \item finanční - kapitálový a peněžní
            \item trh produktů (výrobků a služeb)
        \end{enumerate} \\
        \hline
    \end{tabular}
\end{table}

\paragraph*{Zákony trhu}

V tržním systému se ekonomické subjekty rozhodují v zásadě samy. Chování tržních subjektů je však ovlivněno zákony trhu.

\begin{enumerate}
    \item Zákon nabídky
    \begin{itemize}
        \item s rostoucí cenou roste i nabídka zboží
        \item nabídkou rozumíme souhrn všech zamýšlených prodejů
    \end{itemize}
    \item Zákon poptávky
    \begin{itemize}
        \item s rostoucí cenou klesá poptávka po zboží
        \item poptávkou rozumíme souhrn všech zamýšlených nákupů
    \end{itemize}
\end{enumerate}

Poptávku můžeme rozlišit:
\begin{itemize}
    \item Individuální (1 kupující, 1 statek či služba)
    \item Dílčí (poptávka všech lidí po určitém statku či službě)
    \item Agregátní (všech lidí po všech statcích a službách)
\end{itemize}

Nabídku obdobně:
\begin{itemize}
    \item Individuální (1 výrobce určitého statku či služby)
    \item Dílčí (nabídka všech výrobců určitého zboží)
    \item Agregátní (nabídka všech výrobců, všeho druhu zboží)
\end{itemize}

\paragraph*{Faktory ovlivňující poptávku}
\begin{itemize}
    \item Cena
    \item Demografické změny (v počtech a charakteristikách kupujících)
    \item Změny velikosti důchodů (mzdy, renty vlastníků půdy, úroky a zisky vlastníků kapitálu)
    \item Změny v preferencích (zvyky, móda, změny potřeb)
    \item Změny cen jiných zboží
    \begin{itemize}
        \item substituty - jiné zboží, které může nahradit při spotřebě zboží sledované (jablka - hrušky, čaj - káva)
        \item komplementy- jiné zboží, které doplňuje při použití zboží sledované (auto — benzin, DVD přehrávač - DVD filmy)
    \end{itemize}
\end{itemize}

\paragraph*{Faktory ovlivňující nabídku}
\begin{itemize}
    \item Cena
    \item Náklady výroby a obchodu (dražší suroviny, energie, zvyšování mezd)
    \item Změny vnějších podmínek podnikání (organizace trhu, počasí pro zemědělce, daně apod.)
    \item Změny kapitálové výnosnosti
\end{itemize}

\paragraph*{Interakce nabídky a poptávky}
\begin{itemize}
    \item Výrobci dopředu neznají rovnovážnou cenu a jí odpovídající ideální množství vyrobeného zboží pro trh
    \item Interakcí nabídky a poptávky získáváme rovnovážnou cenu a jí odpovídající množství statků
\end{itemize}

[Obr. 8 Interakce nabídky a poptávky]

\textbf{Dokonalá konkurence}
\begin{itemize}
    \item Výrobci mají rovné podmínky přístupu na trh
    \item Kupující mají stejný přístup a jediným kritériem pro koupi či nekoupi je cena
    \item V praxi skoro nenajdeme
\end{itemize}

\textbf{Nedokonalá konkurence}
\begin{itemize}
    \item Mezi potencionálními prodávajícími má jeden výsadní postavení na trhu
    \begin{itemize}
        \item administrativní monopol - jediný výrobce či prodejce má povolení od státu (Česká pošta)
        \item absolutní monopol - jediný výrobce zná recept na výrobu (Coca-cola, Kofola)
    \end{itemize}
    \item Výsadní postavení mezi poptávajícími, je výhradním nakupujícím-stát - monopsony (speciální technologie, zbraně hromadného ničení)
\end{itemize}

\textbf{Monopson} - výsadní postavení určitého kupujícího na trhu.

\paragraph*{Členění trhu podle míry konkurence}
\begin{itemize}
    \item Dokonalá konkurence
    \item Monopolistická konkurence
    \item Oligopol
\end{itemize}

\paragraph*{Selhání trhu}
\begin{itemize}
    \item Zneužití výsadního postavení monopoly
    \item Existence veřejných statků
    \item Externality trhu
    \begin{itemize}
        \item negativní/špatný vliv výrobců na životní prostředí
        \item pozitivní/podnikatelská činnost může přinést i užitek (včelař zajišťuje opylení sadů a polí)
    \end{itemize}
\end{itemize}

\paragraph*{Subjekty trhu}
\begin{itemize}
    \item domácnosti - přichází na trh za účelem uspokojení potřeb
    \item firmy - subjekty vyrábějící za účelem prodeje
    \item stát (vláda) - vstupuje na trh s cílem ovlivnit jej
\end{itemize}

\begin{description}
    \item[TRŽNÍ ROVNOVÁHA] Když se vyrovná nabídka a poptávka, vzniká stabilní situace na trhu. Důsledkem je rovnovážná cena na trhu.
    \item[PŘEBYTEK] Vzniká soutěž prodávajících, snižování cen, výroby atd., pokles nabízeného množství
    \item[NEDOSTATEK] Zboží na trhu, typické pro ekonomiku plánovanou, vzniká soutěž kupujících, jsou ochotní zaplatit i cenu vyšší, zvyšování ceny i při nekvalitních výrobcích.
\end{description}

Tržní síly vedou k utváření rovnovážných cen, tím je určeno vyráběné množství výrobků (struktura výroby).

