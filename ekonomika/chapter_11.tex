\chapter{Management}

\paragraph{Management} (věda o řízení) je ucelený soubor ověřených přístupů, názorů, zkušeností, doporučení a metod, které vedoucí pracovníci (manažeři) užívají k zvládnutí specifických činností (manažerských funkcí), jež jsou nezbytné k dosažení soustavy podnikatelských cílů organizace.

\paragraph{Manažer} je profese a její nositel je zodpovědný za dosahování cílů svěřených mu organizačních jednotek (útvarů, kolektivů), včetně tvůrčí účasti na jejich tvorbě a zajištění. Využívá při tom kolektiv spolupracovníků.

\paragraph{Manažer a podnikatel} U menších firem podnikatel = manažer firmy. Manažeři nenesou riziko podnikání, maximálně své místo, podnikatel o peníze.

\paragraph{Role a funkce manažera}
\begin{itemize}
    \item Role manažera -- pohled statický, charakterizující samotnou osobu manažera a její postavení v hierarchii organizace
    \item Funkce manažera -- pohled procesní, dynamický, zachycující manažera při jeho řídící činnosti v organizaci
\end{itemize}

\paragraph{Role manažera}
\begin{enumerate}
    \item Podle úrovně řízení
        \begin{itemize}
            \item Vrcholový management
                \begin{itemize}
                    \item sem zařadíme nejvýše postavené pracovníky (ředitele, náměstky ředitele, prezidenty společnosti a jejich viceprezidenty), skupina nejlépe placených
                    \item na jejich Ž umění řídit a znalostech závisí úspěch firmy
                \end{itemize}
            \item Střední management
                \begin{itemize}
                    \item sem patří vedoucí útvarů firmy (vedoucí marketingového oddělení, vedoucí nákupu, vedoucí kontroly jakosti)
                \end{itemize}
            \item Nejnižší management
                \begin{itemize}
                    \item sem zařadíme manažery na nejnižším stupni řízení (mistr dílny, vedoucí závodní jídelny), očekává se schopnost řešení každodenních problémů
                \end{itemize}
        \end{itemize}
    \item Podle stylu řízení
        \begin{itemize}
            \item Autokratický styl
                \begin{itemize}
                    \item vedoucí, který sám rozhoduje a přikazuje svým podřízeným
                    \item autokrat detailně a systematicky kontroluje, zda byly splněny jeho příkazy
                    \item armáda
                \end{itemize}
            \item Demokratický styl
                \begin{itemize}
                    \item je vedoucí (demokrat), bere ohled na názory svých podřízených, o problémech diskutuje, konečné rozhodnutí musí však udělat sám a kontroluje jeho splnění (plánování, projektování)
                \end{itemize}
            \item Liberální styl
                \begin{itemize}
                    \item lberální vedoucí již nepoužívá přímých řídících příkazů
                    \item styl je vhodný v organizacích, kde pracují pracovníci s VŠ vzděláním, kteří mají vysokou vnitřní motivaci k práci (výzkumná pracoviště, vysoké školy)
                \end{itemize}
        \end{itemize}
\end{enumerate}

\section*{Manažerské funkce}
Jsou typické úlohy, které vedoucí pracovníci řeší v procesu své řídící práce

\paragraph*{Sekvenční funkce}
\begin{itemize}
    \item plánování -- manažer stanovuje cíle a postupy k jejich dosažení
    \item organizování -- stanoví a uspořádá role lidí, kterým přidělí konkrétní práci
    \item výběr a rozmisťování pracovníků -- vybírá a získává konkrétní pracovníky
    \item vedení ldí -- vznikají vzájemné vztahy nadřízenosti
    \item kontrola -- kontrola hodnotí kvalitu a kvantitu průběžných a konečných výsledků a vyvodí příslušné závěry
\end{itemize}

\paragraph{Průběžné funkce}
\begin{itemize}
    \item prostupují všemi sekvenčními funkcemi
    \item analyzování problémů
    \item rozhodování
    \item koordinace při realizaci (implementace)
\end{itemize}

\paragraph{PLANOVANÍ}
Je proces stanovení cílů řízené činnosti a vhodných cest a prostředků k jejich dosažení ve stanoveném čase.

\textbf{Postup tvorby plánu:}
\begin{enumerate}
    \item stanovím cíle
    \item vymezím cesty jejich dosažení (varianty)
    \item jednu variantu zvolím = to je plán    
\end{enumerate}

Plány členíme z časového hlediska:
\begin{itemize}
    \item Strategické plány (dlouhodobé)
    \item Taktické plány (střednědobé-roční)
    \item Operativní (krátkodobé -- každodenní)    
\end{itemize}

\paragraph{Vztah marketingu a managementu}
Obě vědy mají své specifika: management se zabývá řízením a marketing trhem. Však firemní marketing i management mají shodné cíle = prosperita firmy a dosažení zisku, mají i mnoho společného. Jeden z bodů je právě plánování.

\section*{Organizování}
Posláním organizování je zajistit dosažení stanovených cílů pomocí procesů specializace navázané a nezbytné koordinace prací a lidí.

\textbf{Kroky v procesu organizování}
\begin{itemize}
    \item identifikace činností -- co vše je potřeba zajistit
    \item seskupení vymezených činností -- přiřadíme konkrétním organizačním útvarům firmy (nákup materiálu zajistí zásobování)
    \item stanovení a přiřazení rolí lidí -- při této činnosti říkám, kolik lidí a s jakými konkrétními úkoly bude zajišťovat práci
\end{itemize}

\paragraph*{Organizační struktury}
Je organizovaný systém, ve kterém je práce rozdělena, seskupena a koordinována. Jsou graficky zobrazovány v organizačních schématech.

\paragraph*{Třídění organizačních struktur}
\begin{enumerate}
    \item Hledisko formálnosti:
        \begin{itemize}
            \item Formální struktura -- nadřízenost a podřízenost, činnosti
            \item Neformální struktura -- vedoucí nejsou jmenováni
        \end{itemize}
    \item Hledisko druhu sdružování
        \begin{itemize}
            \item Funkcionální struktura
            \item Výrobková struktura
            \item Ostatní účelové struktury
        \end{itemize}
    \item Hledisko rozhodovací pravomoci:
        \begin{itemize}
            \item Liniové struktury
            \item Štábní struktury
            \item Liniově štábní struktury
            \item Cílově programové struktury
        \end{itemize}
    \item Hledisko míry centralizace:
        \begin{itemize}
            \item Centralizované
            \item Decentralizované
        \end{itemize}
    \item Hledisko počtu řídících úrovní:
        \begin{itemize}
            \item Ploché struktury
            \item Úzké struktury
        \end{itemize}
    \item Hledisko časového trvání:
        \begin{itemize}
            \item Dočasné
            \item Trvalé
        \end{itemize}
\end{enumerate}

\paragraph*{Faktory ovlivňující volbu organizační struktury}
\begin{enumerate}
   \item Vnitřní faktory -- Sem patří faktory jako velikost firmy, výrobně-technická základna (strojní vybavení), teritoriální rozmístění (jestli má firma pobočky či závody)
   \item Vnější faktory -- Jedná se o faktory, které firma sama není schopna ovlivnit (legislativní možnosti, stabilita podnikatelského okolí)
\end{enumerate}

\section*{Výběr, rozmisťtování a hodnocení pracovníka}

\paragraph*{Získávání vhodných pracovníků}
\begin{enumerate}
    \item Definujeme naší potřebu
    \item Realizace personálního zajištění
        \begin{itemize}
            \item Fáze plánovací
            \item Fáze náboru a výběru
        \end{itemize}
\end{enumerate}

\paragraph*{Zvyšování kvalifikace, rekvalifikace}
Ne vždy je potřeba s novou prací hned přijímat nového pracovníka. Stačí u stávajících zaměstnanců pouze zvýšit nebo změnit kvalifikaci (znalosti, dovednosti a návyky, které využívají při práci).

\paragraph*{Proces zvyšování kvalifikace}
\begin{itemize}
    \item 1. fáze -- stanovení reálné potřeby zvýšení kvalifikace
    \item 2. fáze -- vlastní zvýšení kvalifikace
    \item 3. fáze -- vyhodnocení výsledků
\end{itemize}

\paragraph*{Způsoby zvyšování kvalifikace}
\begin{itemize}
    \item školení v rámci pracovního procesu-je organizováno v rámci podniku např. zaškolení mistrem k určité nové práci
    \item školení mimo pracovní proces-kurzy, dálkové university, jazykové stáže, školení (kurzy daňových poradců, svářečské kurzy)
\end{itemize}

\paragraph*{Pracovní kariéra manažera}
\begin{itemize}
    \item 1. etapa - přípravná - doba studia (SŠ,VŠ)
    \item 2. etapa - zakotvení - zapracování se v prvním zaměstnání
    \item 3. etapa - rozvoj - má manažer pro firmu velký význam a přínos (35-50 let)
    \item 4. etapa - pozdní kariéra - dochází obvykle k úbytku energie, manažeři začínají mít sklon k rutinním řešením
\end{itemize}

\paragraph*{Hodnocení pracovníků}
Má pomoci lépe využít profesní kvalifikaci zaměstnanců, rozvíjet jejich pracovní kariéru, motivovat a spravedlivě odměňovat.

Obvykle se hodnotí:
\begin{enumerate}
    \item plnění pracovních úkolů
    \item chování v pracovním procesu a mimo něj
    \item osobní a charakterové rysy
\end{enumerate}

Hodnocení provádí:
\begin{enumerate}
    \item vedoucí pracovníci
    \item pracovníci personálních útvarů
    \item externí nebo interní specialisté
\end{enumerate}

\paragraph*{Systémy odměňování}
Na základě hodnocení pracovníka provádí vedoucí jeho odměňování či potrestání. Systémy odměňování úzce souvisí s motivací pracovníků na jedné straně a finančními možnostmi firmy na straně druhé. S odměňováním souvisí i daňová problematika.
\begin{enumerate}
    \item hmotné odměny
        \begin{enumerate}
            \item Přímé odměny (mzda, prémie a odměny, podíly na zisku) -- prochází zdaněním
            \item Nepřímé odměny (příplatky na dovolenou, životní pojištění a důchodové, příspěvky na stravování, na mateřské školy, poskytované zboží a služby (některé podléhají zdanění daní z příjmu)
        \end{enumerate}
    \item nehmotné odměny
        \begin{enumerate}
            \item Prestižní funkce, volná či individuální pracovní doba, možnost odborného růstu
        \end{enumerate}
\end{enumerate}

\paragraph*{Hodnocení nekvalitní práce}
\begin{itemize}
    \item Ústní napomenutí
    \item Písemné napomenutí
    \item Finanční postih (odebrání prémií, osobní ohodnocení)
    \item Výpověď (v extrémních případech okamžité zrušení pracovního poměru)
\end{itemize}

\section*{Vedení lidí - teorie X a Y}
Je čtvrtou manažerskou funkcí. Manažer musí správně vést pracovníky, aby pracovali v žádoucí kvalitě, kvantitě a směrem naplnění cílů.

\emph{Teorie X} vychází z předpokladu, že průměrný pracovník nemá své zaměstnání rád, je mu přítěží nutnou k zajištění obživy. Nemá zvláštní ambice. Základním rysem je lenost a snaha práci se vyhnout.

\emph{Teorie Y} vychází z opačných předpokladů: pracovník má přirozený sklon k práci, chce V práci najít svou seberealizaci, má sklon k odpovědnosti, aktivně se účastní na práci a řízení.

Teorie X a Y představují vlastně dva extrémy chování podřízených a jim odpovídající řídící působení vedoucího pracovníka. V praxi samozřejmě nepracujeme s průměrnými lidmi, každý člověk je jiný, a proto u každého je třeba najít potřebnou míru skloubení teorie X a teorie Y současně. Vedoucí tedy hledá určitý kompromis pro každého jednotlivého podřízeného.

\section*{Kontrola}
Poslední manažerskou funkcí je kontrola=proces sledování, rozboru a přijetí závěrů V souvislosti s odchylkami od záměru (cíle).

Fáze kontrolního procesu:
\begin{itemize}
    \item stanovení cíle kontroly
    \item stanovení kontrolních kritérií
    \item rozbor kontrolovaných procesů a porovnání s kritérii kontroly
    \item vyhodnocení zjištěných odchylek a přijetí závěrů
    \item realizace závěrů (firma předepsala manko k úhradě skladníkovi)
\end{itemize}
