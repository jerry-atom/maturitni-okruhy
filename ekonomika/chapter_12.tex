\chapter{Hospodářská politika}

Hospodářská politika se nás dotýká dnes a denně, přímo 1 nepřímo. \\
HP je souhrn cílů, nástrojů, rozhodovacích procesů a opatření státu zaměřených na kontrolu a ovlivňování ekonomického vývoje.

\paragraph*{Propojení ekonomiky a politiky}
Ekonomika a politika spolu úzce souvisí, hospodářská politika se dotýká denně všech lidí i firem přímo i nepřímo. Její součástí je např.: zdravotní péče a úhrady za ni, výše školného, regulace výše nájemného,
velikost daní, podmínky přístupu k cizím měnám. HP je ovlivňována tím, která z politických stran se při volbách dostala k moci (pravicová-
podnikatelé, levicová-zaměřuje se na sociálně slabší a na \uv{normální}.

\paragraph*{Subjekty hospodářské politiky}
\begin{itemize}
    \item Parlament
        \begin{itemize}
            \item složený z 2 komor (poslanec. sněmovna-200 poslanců a senát-81 senátorů)
            \item volení zástupci občanů, kteří schvalují zákony
            \item státní rozpočet je zákonem, sestavuje se vždy na 1 rok, stát plánuje, kolik vybere na daních od občanů a firem a co bude z těchto peněz financovat, parlament schvaluje návrh rozpočtu
        \end{itemize}
    \item Vláda
        \begin{itemize}
            \item nejvyšší výkonný orgán, je základním tvůrcem konkrétní aktuální HP
            \item připravuje návrh rozpočtu a předkládá ho ke schválení parlamentu, pokud je rozpočet schválen, je poslanci vyjádřena důvěra
        \end{itemize}
    \item Centrální banka (Česká národní banka)
        \begin{itemize}
            \item jejím hlavním úkolem je střežit a korigovat komerční bankovní trh a množství peněz v oběhu tak, aby nedocházelo k jejich znehodnocování
            \item inflaci (vede účet státního rozpočtu, vydává peníze)
            \item řídí bankovní trh
            \item sídlí v Praze, řídí ji guvernér-Jiří Rusnok a 6 členů bankovní rady
            \item guvernéra jmenuje prezident
        \end{itemize}
\end{itemize}

\paragraph*{Funkce institucí státu}
\begin{description}
    \item[Funkce právní jistoty a bezpečí] Cílem je vytvořit právní podmínky a dodržování zákonů, jde o bezpečnost vnitrostátní i mezinárodní, ministerstvo obrany má na starost armádu a ministerstvo vnitra -- policii
    \item[Funkce sociální] Státní instituce provádějí přerozdělovací procesy (transfery obyvatelstvu = peníze vybrané na daních a sociální pojištění používají na výplatu důchodů, sociálních a nemocenských dávek. Starají se p veřejné statky (školství, zdravotnictví, Životní prostředí, infrastruktura-
dálnice, železnice, energetika) Fungující neziskový sektor je nutnou podmínkou pro fungující ekonomiku.
    \item[Funkce hospodářská]
        \begin{enumerate}
            \item stát sám podniká-za účelem dosažení zisku a rozmnožení bohatství státu
            \item stát vytváří podmínky pro úspěšné podnikání ostatních subjektů hospod.
                \begin{itemize}
                    \item kdy stát určuje daňové zatížení
                    \item pravidla hospodářské soutěže: konkurenční boj, zákaz klamavé reklamy
                \end{itemize}
        \end{enumerate}
\end{description}

\paragraph*{Nástroje hospodářské politiky}
\begin{enumerate}
    \item Právní systém, legislativní proces
    \item Monetární systém
    \item Státní rozpočet a fiskální politika
    \item Důchodová a cenová politika
    \item Zahraničně obchodní politika
\end{enumerate}

\section*{Právní systém a legislativní proces}

Zákony v ČR prochází schválením Parlamentu a vyjadřuje se k nim prezident republiky. Schválené normy jsou zveřejňovány ve sbírce zákonů a jsou závazné pro občany státu 1
všechny, kdo se zdržují na území naší republiky.

Porušení zákona je předmětem šetření a příslušné státní instituce podřízené vládě, mají vymezeny své pravomoci trestat pachatele (krádež či dopravní nehoda-policie, daňové nesrovnalosti -- FÚ, nedovolené podnikání -- ŽÚ). Sporné případy řeší soudy.

\section*{Monetární systém}

Základní úlohu hraje centrální banka dané země (u nás ČNB). Specifická je Evropská centrální banka se sídlem ve Frankfurtu, která byla založena pro EU po zavedení jednotné měny eura.

K udržení stabilní měny používá každá centrální banka následující nástroje:
\begin{description}
    \item[REPO SAZBA] je úrok centrální banky pro terminované operace s komerčními bankami
    \item[POVINNÉ MINIMÁLNÍ REZERVY (PMR)] každá komerční banka má povinnost složit u CB část svých depozit (vkladů) jako rezervu, a tím tyto peníze „umrtví“a nemůže s nimi podnikat
    \item[OPERACE NA VOLNÉM TRHU] CB prodává a nakupuje cenné papíry, obvykle státní dluhopisy
    \item[OSTATNÍ NÁSTROJE MĚNOVÉ POLITIKY] CB může stanovovat úvěrové limity
\end{description}

\section*{Státní rozpočet a fiskální politika}

K plnění svých funkcí potřebuje stát peníze, jako řádný hospodář dopředu plánuje, kolik v následujícím roce peněz potřebuje (výdajová stránka státního rozpočtu) a kolik jich získá (příjmová stránka).

Porovnáním těchto dvou stran pak státní rozpočet může být:
\begin{description}
    \item [VYROVNANÝ] příjmy = výdaje   
    \item [SCHODKOVÝ] příjmy < výdaje
    \item [PŘEBYTKOVÝ] příjmy > výdaje
\end{description}

Přebytkový rozpočet vytváří rezervu státu na dobu nepříznivého období. \\
Schodek znamená, že si stát na své výdaje musí někde půjčit. Většinou vydá státní dluhopisy, které si koupí tuzemské banky a podniky nebo zahraniční investoři. \\
Návrh předkládá vláda ke schválení parlamentu a schválený rozpočet má podobu zákona.

\paragraph{Restriktivní fiskální politika}
Výsledek přebytkového rozpočtu, nepodporuje rozvoj ekonomiky. Stát nevytváří nové pracovní příležitosti, nedává zakázky soukromým firmám.

\paragraph{Expanzivní fiskální politika}
Výsledek schodkového rozpočtu, podporuje rozvoj ekonomiky, ale na \uv{dluh}.

\paragraph{Schéma vyrovnaného státního rozpočtu}
\begin{table}[h]
    \centering
    \begin{tabular}{| p{8cm} | p{8cm} |} \hline
        Příjmová stránka & Výdajová stránka \\ \hline
        \begin{itemize}
            \item Daně
            \item Cla
            \item Sociální pojištění
            \item Ostatní příjmy (poplatky atd.)
            \item Příjmy z rozpočtu EU
        \end{itemize} &
        \begin{itemize}
            \item Státní správa (úřady)
            \item Obrana státu, školství, zdravotnictví
            \item Transfery obyvatelstvu (důchody, dávky atd.)
            \item Státní zakázky (dálnice)
            \item Investice do život. pojištění
            \item Odvody do rozpočtu EU
        \end{itemize} \\ \hline
    \end{tabular}
\end{table}

\begin{center}
    \textbf{Příjmy celkem = Výdaje celkem}
\end{center}

Náš státní rozpočet na jeden rok je okolo 1 100 mld.Kč, což je necelá třetina našeho ročního HDP(hrubý domácí produkt).

\section*{DŮCHODOVÁ A CENOVÁ POLITIKA}

Regulace mezd u nás byla používána v roce 1990-1995, dnes stát mzdy v soukromém sektoru nereguluje a ponechává je volnému působení trhu.

Regulace cen-ceny státem regulované jsou elektřina, plyn, voda, nájemné v bytech, dálkové vytápění, telekomunikační a poštovní poplatky apod.

\section*{ZAHRANIČNĚ OBCHODNÍ POLITIKA}

Stát řeší kurz koruny k zahraničním měnám, celní politiku, kvóty-dovozní, vývozní (množstevní omezení dovozu a vývozu), embargo (úplný zákaz dovozu a vývozu), cla.

\subsection*{CLO}

Celní poplatek, je dávka vybíraná státem při přechodu zboží přes celní hranici. \\
V EU jsou mezi státy zrušena cla. \\
\textbf{Funkce cla}:
\begin{itemize}
    \item Fiskální -- příjem do státního rozpočtu -- odvádíme do rozpočtu EU
    \item Obchodně politická -- nástroj hospodářské politiky
    \item Cenotvorná -- u dovozového zboží firmy započítají clo do prodejní ceny
\end{itemize}

\textbf{Druhy cla v EU}
\begin{itemize}
    \item Dovozní clo -- nejběžnější
    \item Vývozní clo -- není používáno
    \item Vyrovnávací
    \item Odvetné
    \item Antidumpingové
\end{itemize}

\paragraph*{DOVOZNÍ A VÝVOZNÍ KVÓTY}
Stanovené počet nebo poměr. \\
Dovozní kvóta je omezení množství určitého dováženého statku. \\
Vývozní kvóta je omezení množství určitého vyváženého statku.

\paragraph*{MAASTRICHTSKÁ KRITÉRIA}
Jsou kritéria pro členské státy EU pro vstup do 3. fáze Evropské hospodářské a měnové unie (EMU) a pro zavedení společné měny -- eura.
