\chapter{Finanční trhy}

\section*{Struktura finančního trhu a na čem je založen}

\textbf{Finanční trh} je založený na nabídce relativně volných peněz (které jejich majitelé nechtějí nechat ležet ladem a ztrácet na hodnotě díky inflaci, ale chtějí je zhodnotit) a poptávce po penězích (kdy podnikatelé vědí jak vydělat, ale nemají dostatek kapitálu). Střetem této nabídky a poptávky se vytváří cena peněz (výše úroků, tržní cena cenných papírů apod.)

Členění finančního trhu:
\begin{itemize}
    \item peněžní trh
    \item kapitálový trh
    \item trh drahých kovů
    \item devizový trh
\end{itemize}

\section*{Úrok a úroková míra}

\textbf{Úrok} je peněžitá odměna za půjčení peněz. Velikost úroku se obvykle vyjadřuje pomocí úrokové míry (sazby), která je procentním vyjádřením zvýšení půjčené částky za určité časové období.

\textbf{Úrokové sazby} zjednodušeně říkají, jak velkou část zaplatí dlužník navíc, zjednodušeně jde o měřítko ceny peněz. U spoření říkají, kolik peněz dostanete navíc, u půjček zase, kolik musíte zaplatit navíc k půjčené částce. Vyhlašují se na určité období (den, měsíc, rok). Úrok je peněžitá odměna za půjčení peněz.

\section*{Motivy poptávky po penězích}

Poptávka po penězích je představována množstvím peněz v držbě jednotlivých ekonomických subjektů (domácností, podniků, vlády) Hlavním motivem, proč člověk drží peníze v hotovosti a ztrácí tak úrok, je možnost disponovat těmito penězi.

\paragraph{Druhy poptávky po penězích}
\begin{enumerate}   
    \item \textbf{Transakční poptávka} - vyplývá z funkce peněz jako prostředku směny - jednotlivé subjekty potřebují peníze na běžné nákupy výrobků a služeb, podniky na úhradu svých nákladů.
    \item \textbf{Majetková poptávka} - vyplývá z funkce uchovatele hodnoty. Znamená, že ekonomické subjekty drží určitou část svého majetku v podobě peněz jako vysoce likvidní formy majetku. Formy struktury jsou různé a působí na ní celá řada faktorů (výnos, splatnost, riziko, atd.). Úspory => investice.
\end{enumerate}

\textbf{Úrok je nákladem na držbu peněz} - budou-li úrokové sazby vysoké, je držba peněz nákladná, jelikož znamená ztrátu úroku, které by přinesly.

\section*{Finanční instituce}

Instituce finančního trhu jsou subjekty, které podnikají s penězi.

Instituce finančního trhu se liší charakterem poskytovaných služeb a také, zda mohou být zakládány bez povolení státu (např. nebankovní instituce poskytující úvěry) či zda pro svojí činnost potřebují \textbf{licenci} (např. banky, pojišťovny, makléřské společnosti, burzy apod.).

\textbf{Česká národní banka}, \textbf{Komerční banky}, \textbf{Družstevní záložny} (spořitelní a úvěrová družstva), \textbf{Stavební spořitelny}, \textbf{Investiční společnosti}, investiční fondy a podílové fondy, Pojišťovny, aj.

Na mezinárodní úrovni působí finanční instituce, které plní regulační, dohledovou a stabilizační funkci. Jsou to např. Evropská centrální banka (pravidla finančního trhu), Mezinárodní měnový fond (stabilizace) nebo Evropská bank pro obnovu a rozvoj (pomoc zemím s přechodem na tržní ekonomiku).

\section*{Cenný papír}

Cenné papír je písemnou formou zachycený právní vztah mezi dvěma (i více) subjekty, tato písemnost má určité náležitosti stanovené zákonem, a pak může být samostatně obchodovatelná.

Cenné papíry peněžního trhu zachycují vztah dlužnický (věřitel $\times$ dlužník)

\section*{Peněžní trh - charakteristika}

Finanční trh vzniká na základě střetu poptávky a nabídky, kde je poptávkou investování a nabídkou spoření. Jinými slovy poptávajícími jsou ti, kteří poptávají peníze a nabízejícími jsou ti, kteří peníze půjčují, tedy nabízí. Na peněžním trhu se obchoduje s krátkodobými vklady a úvěry ve formě krátkodobých CP se splatností do 1 roku

\subsection*{Šek}

Je CP, kterým výstavce šeku dává příkaz bance, aby osobě uvedené na šeku nebo doručiteli zaplatila z jeho účtu částku,
na kterou je šek vystavený. Šek je splatný na viděnou (po předložení) do 8 dnů v téže zemi.

Podstatné náležitosti šeku:
\begin{itemize}
    \item označení CP slovem šek
    \item částka (číslicí i slovy)
    \item jméno toho, kdo má platit (banka)
    \item místo, kde se má platit
    \item datum a místo vystavení šeku
    \item podpis výstavce
\end{itemize}

Použití šeku:
\begin{itemize}
    \item proplacení v hotovosti
    \item zaplacení šekem
    \item zúčtování šeku
    \item proplacení či jeho vklad na účet
\end{itemize}

\subsection*{Směnky}
Směnka je obchodovatelný cenný papír, který slouží k platbě nebo jako zajišťovací nástroj. Jedná se o nejjednodušší a často nejrychlejší formu úvěru. Jasně z ní vyplývají závazkové vztahy a tím je ideálním prostředkem pro obchodování.

\paragraph{Druhy směnek}
Základním členění směnek je na \textbf{vlastní} a \textbf{cizí}. Jednoduše řečeno, pokud osoba, která směnku vystavila, se zaváže 
zaplatit v daném termínu a místě, jde o směnku vlastní. Směnka vlastní obsahuje slovo \uv{zaplatím}. Za směnku cizí musí zaplatit někdo jiný, než ten kdo jí vystavil a vyskytuje se termín \uv{zaplaťte}.

[Obr. Struktura vztahů směnek]

Náležitosti směnky:
\begin{itemize}    
    \item označení, že se jedná o směnku (doslovné označení v textu listiny)
    \item příkaz zaplatit danou finanční sumu (bez dalších podmínek)
    \item jméno a adresa toho, kdo má směnku zaplatit
    \item informace o splatnosti
    \item informace o místě zaplacení
    \item jméno věřitele, tedy toho, jemuž má být zaplaceno
    \item datum a místo vystavení směnky
    \item podpis výstavce směnky
\end{itemize}

\subsection*{Pokladniční poukázky}
Emituje stát prostřednictvím ČNB, je to CP, který slouží ke krytí přechodného schodku (deficitu) státního rozpočtu
\begin{itemize}  
    \item určen pro obchodování na mezibankovním trhu pro velké investory
    \item nominální hodnota jedné poukázky bývá 10mil. Kč, objem jedné emise se pohybuje v řádech miliard korun
    \item o tyto CP je velký zájem, protože stát v pozici dlužníka je velmi solidní zárukou navrácení peněz včetně úrokového výnosu
\end{itemize}

\subsection*{Depozitní certifikát}
Vydává banka (dlužník) a potvrzuje jím přijetí jednorázového termínovaného vkladu od klienta (věřitele). Certifikáty nakupují podniky, občané, banky a jiné subjekty. Jsou vystavovány na částku 10 tisíc Kč. Čím vyšší je jejich nominální hodnota tím vyšší je úrok.

\section*{Kapitálový trh}

Obchoduje se s dlouhodobými termínovanými vklady, úvěry a zdroji financí ve formě dlouhodobých CP se splatností nad 1 rok

\paragraph{Podoby CP}
\begin{itemize}
    \item Listinné - je vydán v podobě listiny
    \item Zaknihované - neexistují fyzicky, jsou evidovány v elektronické podobě na účtu majitelů
\end{itemize}

\paragraph{Členění CP}
\begin{enumerate}
    \item Podle právního nároku
        \begin{itemize}
            \item dlužnické -- představují nárok na splacení dluhu ( např. dluhopisy, směnky),
            \item majetkové -- představují podíl na majetku (např. akcie, podílové listy),
            \item dispoziční -- představují právo nakládat s určitým majetkem (např. konosament, skladištní list).
        \end{itemize}
    \item Podle doby splatnosti
        \begin{itemize}
            \item peněžní -- krátkodobé, splatné do 1 roku,
            \item kapitálové -- dlouhodobé, splatné za dobu delší než 1 rok.
        \end{itemize}
    \item Podle podoby
        \begin{itemize}
            \item zaknihované -- mají jen podobu zápisu v určitém registru (nyní Centrální depozitář cenných papírů, a. s.),
            \item listinné -- mají podobu speciální listiny.
        \end{itemize}
    \item Podle formy
        \begin{itemize}
            \item Cenné papíry na doručitele -- jméno majitele není na cenném papíru uvedeno, převádějí se pouhým předáním novému majiteli.
            \item Cenné papíry na řad -- jméno majitele cenného papíru je na něm uvedeno, někdy s doložkou „na řad“ (zákonné ji obsahovat nemusí), převádějí se rubopisem a předáním. Uvedením doložky \uv{nikoli na řad} se cenný papír na řad změní v cenný papír na jméno.
            \item Cenné papíry na jméno -- jméno majitele cenného papíru je na něm uvedeno, někdy s doložkou „nikoli na řad“, převádějí se smlouvou o postoupení pohledávky a předáním.
        \end{itemize}
    \item Podle obchodovatelnosti
        \begin{itemize}
            \item Obchodovatelní -- nakupují se a prodávají na sekundárním trhu. Jsou převoditelné.
            \item Neobchodovatelné -- emitent zakazuje jejich nákup a prodej na sekundárním trhu. Prodávají se jen na trhu primárním, např. Vkladní knížky.
        \end{itemize}
    \item Podle emise
        \begin{itemize}
            \item CP hromadně vydávané -- např. Akcie, dluhopisy
            \item CP individuálně vydávané -- např. Směnka, šek
        \end{itemize}
\end{enumerate}

\section*{Akcie}

Akcie je majetkovým cenným papírem a její vlastník se stává spolumajitelem firmy. Souček akcií tvoří základní kapitál.

\paragraph{Práva akcionáře, jako společníka}
\begin{itemize}
    \item podílet se na zisku (výplata dividend)
    \item podílet se na řízení společnosti (hlasovat na valné hromadě akcionářů)
    \item podílet se na likvidačním zůstatku společnosti, pokud jde firma do likvidace
\end{itemize}

\paragraph{Akcie může být emitována}
\begin{itemize}
    \item Přímo za nominální hodnotu,
    \item S emisním ážiem (za vyšší cenu než nominální),
    \item S emisním disážiem (za nižší cenu než nominální) - současný zákon o obchodních korporacích neumožňuje.
\end{itemize}

Rozlišujeme:
\begin{enumerate}
    \item Kmenové akcie
        \begin{itemize}
            \item \textbf{Na jméno} - převoditelné rubopisem neboli indosamentem,
            \item \textbf{Na majitele} - volně převoditelné, na doručitele, od 2017 pouze jako zaknihovaný CP nebo imobilizovaný CP.
        \end{itemize}
    \item Speciální
        \begin{itemize}
            \item \textbf{Zaměstnanecké} (pro zaměstnance firmy, akcie vystaveny na jméno, max. 5\% objemu emise akcií),
            \item \textbf{Prioritní akcie} (přednostní výplata dividendy, ovšem majitel nemá právo hlasovat na valné hromadě).    
        \end{itemize}
\end{enumerate}

Členění dle fyzické podoby:
\begin{itemize}
    \item  Akcie \textbf{materializované} (listinné - fyzicky vytištěné na papíře - mají plášť, kuponový arch pro výplaty dividend a talón pro získání nového kuponového archu, když je předchozí vyčerpán)
    \item  Akcie \textbf{dematerializované} (zaknihované - většinou jako údaje v paměti počítače).
    \item  \textbf{Dividenda} je příjmem z kapitálového majetku a jste povinni dle zákona o dani z příjmu fyzických osob ji zdanit srážkovou daní 15\%.
\end{itemize}

Skládá se ze dvou částí:
\begin{enumerate}
    \item plášť -- obchodní jméno a sídlo společnosti, číselné označení akcie, jmenovitá hodnota akcie, označení, zda je akcie na jméno nebo na majitele, výše ZK, počet akcií v době vydání akcie, datum vydání akcie, podpisy dvou členů představenstva
    \item jednotlivé kupóny se odstřihávají (detašují) a vyplácí se podíl na zisku, který se nazývá dividenda (vyjadřuje se v \%). Dividenda by měla být vyšší než úrok v bankách, v opačném případě by občané ukládali svůj kapitál do bank -- poslední ústřižek kupónového archu je talón -- poukázka na nový kupónový arch
\end{enumerate}

\section*{Podílové listy}

Jsou vydávány podílovými fondy, tyto fondy zakládají investiční společnosti nebo banky, které tak shromažďují peníze od investorů (občanů, firem) formou prodeje svých podílových listů. Takto získané prostředky se investují do nákupu různých druhů CP, tzv. portfolio a tím rozloží možné riziko. V současnosti existují v ČR pouze tzv. otevřené podílové fondy. To znamená že podílník má právo kdykoliv svůj podílový list prodat zpět podílovému fondu.

Jaké potencionální zhodnocení fond nabízí, závisí především na aktivech, do kterých investuje.

\paragraph{Rozeznáváme fondy}
\begin{enumerate}
    \item akciové 
    \item dluhopisové
    \item peněžního trhu
    \item smíšené
    \item zajištění - minimální zhodnocení investic
\end{enumerate}

\section*{Obligace}

Dlouhodobý úvěrový CP, v němž se vydavatel zavazuje jeho majiteli splatit dlužnou nominální částku a vyplácet výnosy k určitému datu. Splatnost je zpravidla pevně stanovena. Vydavatel si emisí obligací opatřuje finanční zdroje převážně na uskutečnění svých investičních záměrů. Obligace podléhají při emisi schválení ministerstvem financí -- emise obligací se pohybují v řádek stovek milionů a jejich emitenti bývají největšími ekonomickými subjekty v národním hospodářství.

\paragraph{Členění obligací z hlediska emitentů}
\begin{enumerate}
    \item státní - emituje je vláda
    \item komunální - emitují územně samosprávné celky, např. obce
    \item bankovní
    \item podnikové - vyšší úročení, vyšší riziko
\end{enumerate}

\section*{Hypoteční zástavní listy}

Emituje banka, která má od ČNB povolení k této činnosti. Jejich prodejem banka získá prostředky, které dále používá. Tyto listy jsou kryty (ručeny) splátkami hypotečních úvěrů a zastavenými nemovitostmi. Vzhledem k dokonalému jištění je nízké úročení. Bývají veřejné obchodovatelné, doba splatnosti 5 let.

\paragraph{Hypotéční zástavní listy mají podobu}
\begin{itemize}
    \item listinnou (cenný papír materializovaný)
    \item zaknihovanou (cenný papír dematerializovaný)
    \item Mohou býti neveřejně obchodovatelné a veřejně obchodovatelné
\end{itemize}

\section*{Deriváty}

Využívají se nejen u CP kapitálového trhu, ale i u deviz. Předmětem koupě nebo prodeje je pouze určité právo, nikoliv věc hmatatelná. Jejich podstatou je forma termínovaného obchodu, tzn. že dochází k určitému zpoždění mezi sjednáním obchodu a jeho plněním. V době sjednání jsou jasně definovány podmínky obchodu, tedy cena a datum plnění. V den splatnosti dochází k dodání finančních instrumentů a jejich proplácení.

Mají funkci:
\begin{enumerate}
    \item zajišťovací - proti rizikům výkyvu kurzu
    \item spekulativní - spekuluji, abych vydělal. V praxi běžnější
\end{enumerate}

\paragraph{Druhy derivátů}
\begin{description}
    \item[Forwardy, Futures] pevné termínované obchody k určitému budoucímu datu za předem dohodnutou cenu
    \item[Swapy] prodej CP či deviz v aktuálním kurzu se současným podpisem smlouvy o budoucím odkoupení zpět za předem dohodnutou cenu
    \item[Opce] je právo k budoucímu datu uskutečnit obchod s CP nebo devizami za předem sjednanou cenu
\end{description}

\section*{Investování volných peněžních prostředků}

Kam uložit peníze? 13 základních investičních možností
\begin{enumerate}
    \item Bankovní účty
    \item Podílové fondy
    \item ETF (ETF jsou fondy obchodované na burzách. Jejich výhodou oproti klasickým podílovým fondům je, že můžete fond koupit a prodat kdykoli a vždy vidíte jeho aktuální cenu (u klasických podílových fondů je hodnota podílového listu přeceňována nejčastěji jednou denně, občas i jednou týdně). Zároveň jsou zpravidla levnější na pořízení i na správu (mají nižší poplatky).
    \item Investiční certifikáty
    \item Penzijní připojištění
    \item Stavební spoření
    \item Životní pojištění
    \item Akcie
    \item Dluhopisy
    \item Nemovitosti
    \item Zlato
    \item Ropa a další komodity
    \item FOREX (Na Forexu se obchoduje s měnovými páry - tedy dvojicemi měn. Posiluje-li jedna měna, cena měnového páru roste. Posiluje-li druhá z páru, cena klesá. I na tomto trhu se obchoduje s pákovým efektem a investice jsou tak značně rizikové. \uv{Vyčistit} účet je mnohem snadnější než ho zhodnotit do závratných výšek)
\end{enumerate}

\section*{Trh drahých kovů}
Je pro nás jen doplňkovým trhem, protože prostřednictvím drahých kovů probíhá jen min. finančních transakcí.

Za nejdůležitější trhy drahých kovů jsou všeobecně považovány trhy zlata a stříbra, přičemž sem bývají zahrnovány i trhy platiny a palladia.

Nejdůležitější institucí je burza drahých kovů v Londýně.

\section*{Devizový trh}

Na devizovém trhu je stejně jako na ostatních trzích cena určována poptávkou a nabídkou po dané komoditě. Obchoduje se zde s měnami různých zemí. Střetávají se zde zájmy kupujících a prodávajících různé měnové jednotky. Na základě poptávky a nabídky je určen poměr v jakém se dané množství měny smění. Devizové trhy tvoří banky, finanční firmy atd.

Poptávka a nabídka po měně dané země bývá zpravidla určena vývojem mezinárodního obchodu, vývojem úrokových měr v zemi, inflačním očekáváním, platební bilancí země, celkovým stavem ekonomiky. Naše měna je volně směnitelná od roku 1995.
