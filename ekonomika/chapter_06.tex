\chapter{Majetková a kapitálová výstavba podniku}

Členění majetku:
[Obr. 2 Členění majetku firmy podle druhů (aktiva firmy)]

Zdroje krytí majetku:
[Obr. 3 Majetek firmy podle zdrojů jeho krytí (pasiva)]
\begin{description}
    \item[ROZVAHA] je statický pohled na majetek, má počáteční a konečný stav \par
        \begin{center}
            {\Large AKTIVA = PASIVA}
        \end{center}
    \item[AKTIVA] Majetek firmy. To, co firma vlastní.
    \item[PASIVA] Zdroje financování. Z čeho byl majetek zaplacen.
\end{description}

\begin{table}[h]
    \caption{Rozvaha k 1.1.2016}
    \begin{tabular}{ p{8cm} | p{8cm} }
        Aktiva &
        Pasiva \\ \hline
        \begin{enumerate}
            \item Stálá aktiva
                \begin{itemize}
                    \item dlouhodobý hmotný majetek
                    \item dlouhodoby nehmotný majetek
                    \item douhodoby finanční majetek
                \end{itemize}
            \item Oběžná aktiva
                \begin{itemize}
                    \item zásoby
                    \item bankovní účty
                    \item penize v pokladně
                    \item pohledávky
                    \item []
                \end{itemize}
            \item Ostatní aktiva 
        \end{enumerate} &
        \begin{enumerate}
            \item Vlastní zdroje
                \begin{itemize}
                    \item základní kapitál
                    \item fondy
                    \item zisky
                \end{itemize}
            \item Cizí zdroje
                \begin{itemize}
                    \item dlouhodobé úvěry
                    \item krátkodobé úvěry
                    \item dodavatelé
                    \item zaměstnanci
                    \item státní rozpočet
                \end{itemize}
            \item Ostatní pasiva
        \end{enumerate} \\ \hline
        AKTIVA CELKEM & PASIVA CELKEM \\
    \end{tabular}
\end{table}

\textbf{Způsoby pořízení dlouhodobého majetku}:
\begin{itemize}
    \item Nákup - nového nebo již použitého DM
    \item Vlastní výroba - stavební firma si postaví novou výrobní halu
    \item Darovánií - stát může darovat ekologické zařízení
    \item Převod z osobního majetku podnikatele - truhlář vloží do firmy svoji garáž
    \item Vklad majetku společníky - dceřina společnost - do které vloží budovu se sklady
    \item Novým zjištěním - jde o majetek, který v UCE nebyl dosud zachycen
    \item Finanční leasing
\end{itemize}

\textbf{Finanční leasing}: DM si pronajmeme a splácíme a po splacení za symbolickou cenu odkoupíme do vlastnictví.
\textbf{Operativní leasing}: DM si zapůjčíme, platíme pronájem a po ukončení nájmu majetek vrátíme - nestane se našim vlastnictvím.
\textbf{Výhody finančního leasingu}:
\begin{itemize}
    \item Firma si může koupit DM, i když na něj nemá finance, pokud ví, že tato investice si bude sama vydělávat na své uhrazení
    \item Investice pořízená formou FL se rychleji dostane do nákladů
\end{itemize}

\textbf{Nevýhody finančního leasingu}:
\begin{itemize}
    \item V případě, že firma se dostane do finančních potíží a přestane splácet řádně splátky, leasingová společnost si vezme DM zpět a již uhrazené splátky propadají
    \item V případě zcizení nebo zničení DM před konečným splacením hradí pojišťovna pojistné leasing. spol., ta pokryje své náklady, a teprve zbytek uhradí nájemci
\end{itemize}

\textbf{Způsoby vyřazení dlouhodobého majetku}:
\begin{enumerate}
    \item likvidace
    \item prodej
    \item manko nebo škoda
    \item darování
    \item přeřazení do osobního užívání
\end{enumerate}

\textbf{Oceňování a odepisování majetku}:
Majetek oceňuje firma vstupní cenou, která se liší podle způsobu pořízení majetku.
Hmotný majetek:
\begin{itemize}
    \item \textbf{Pořizovací cenou} - při nákupu od dodavatele \par PC = cena pořízení + náklady související s pořízením (doprava, montáž, inflace)
    \item \textbf{Reprodukční pořizovací cenou} - v případě, že nemá firma doklad o hodnotě majetku, určí cenu odhadce
    \item \textbf{Cenou ve vlastních nákladech} - v případě, že si firma sama vyrobí DM. \par Firma sečte všechny náklady, ale nesmí započítat zisk.
\end{itemize}

\textbf{Nehmotný majetek} oceňujeme stejně jako hmotný dlouhodobý majetek:
\begin{itemize}
    \item pořizovací cenou
    \item reprodukční pořizovací cenou
    \item cenou ve vlastních nákladech
\end{itemize}

\textbf{Finanční majetek}:
Oceňujeme pořizovací cenou včetně přímých nákladů souvisejících s pořízením

\textbf{Odepisování majetku}:
Odepisování znamená, že hodnotu majetku přenášíme do nákladů firmy postupně několik let prostřednictvím ročních odpisů. Odepisujeme hmotný a nehmotný dlouhodobý majetek.

\textbf{Opotřebování majetku}:
\begin{itemize}
    \item \textbf{Fyzické} používáním se součástky ničí, nepoužíváním součástky rezavějí\ldots
    \item \textbf{Morální} i fyzicky skvěle zachovalý stroj může být technicky zastaralý\ldots
\end{itemize}

\textbf{Druhy odpisů}:
\begin{itemize}
    \item \textbf{Odpisy účetní} Jsou upraveny zákonem o účetnictví. Tyto odpisy mají vyjadřovat co nejobjektivněji skutečnou míru opotřebení toho kterého DM ve firmě. S jejich pomocí dosáhneme dobrého přehledu o skutečné výši majetku firmy.
    \item \textbf{Odpisy daňové} Tyto odpisy slouží jako daňový doklad a ovlivňují výši daní z příjmu podnikatele. Metody výpočtu těchto odpisů jsou závazně stanoveny státem a zakotveny v zákonu o daní z příjmu.
\end{itemize}

\begin{center}
Zůstatková cena = vstupní cena - oprávky
Oprávky = jsou součtem doposud provedených odpisů	
\end{center}

[Tabulka Odpisové třídy a doba odpisu]

\textbf{Funkce odpisů}:
\begin{itemize}
    \item \textbf{Funkce nákladová} - pomocí odpisů přenášíme hodnotu DM do nákladů
    \item \textbf{Funkce zdrojová} - odpisy jsou pro firmu zdrojem financí
    \item \textbf{Funkce fiskální} - odpisy ovlivňují výši příjmů státního rozpočtu z daně z příjmu
    \item \textbf{Funkce rozvojová} - umožní-li stát podnikatelům rychle odepisovat DM, stimuluje je tím k rychlejší obměně strojního vybavení a k zavádění moderních technologií, které umožní rozvoj firem a tím i celého hospodářství
\end{itemize}

\textbf{Evidence a reprodukce majetku}:
Důvody evidence DM:
\begin{itemize}
    \item Kontrola majetku-inventarizace
    \item Odepisování majetku-účetnictví a daně
    \item Přehled o finanční hodnotě firmy
    \item Úhrady škod na majetku pojišťovnou
\end{itemize}

Základní evidenci provádíme na \textbf{inventárních kartách}.

Povinně karta musí obsahovat inventární číslo přidělené DM, zvolený způsob odepisování, vstupní cenu, jednotlivé roční daňové odpisy.

Při pořízení DM navíc vypisujeme zápis o pořízení DM a při vyřazení pak zápis o vyřazení.

\textbf{Reprodukce majetku}:
Údržba, opravy a nákup dlouhodobé majetku.

\textbf{Reprodukci členíme}:
\begin{itemize}
    \item Reprodukce prostá - stroj nahradíme strojem stejného výkonu
    \item Reprodukce rozšířená - stroj nahradíme jedním strojem s vyšším výkonem nebo více stroji se stejným výkonem
    \item Reprodukce zůžená - stroj nahradíme strojem s menším výkonem nebo nenahradíme vůbec
\end{itemize}


