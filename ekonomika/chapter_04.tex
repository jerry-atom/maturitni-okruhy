\chapter{Občanský zákoník a zákon o obchodních korporacích}

\begin{description}
    \item[Podnikatel] Samostatně vykonává na vlastní účet a odpovědnost výdělečnou činnost živnostenským nebo obdobným způsobem se záměrem Činit tak soustavně za účelem dosažení zisku.
\end{description}

\section*{Zákon o obchodních korporacích}
Upravuje obchodní společnosti a družstva.
    \begin{description}
        \item[Obchodní firma] Název, pod kterým je podnikatel zapsán v obchodním rejstříku. Podnikatel je povinen činit právní úkony pod svou firmou.
            \begin{enumerate}
                \item Fyzické osoby mají povinně jako obchodní firmu své jméno a příjmení. (Ivo Večerka, popřípadě dodatek Ivo Večerka - truhlář)
                \item Právnické osoby mají jméno, pod kterým jsou zapsána v obchodním rejstříku. Součástí je i dodatek označující jejich právní formu. (Kaufland, v.o.s., Globus ČR k.s., Jednota Kladno, družstvo)
            \end{enumerate}
        \item[Obchodní závod] Organizovaný soubor jmění, který vytvořil podnikatel a slouží k provozování jeho činnosti.
        \item[Sídlo podnikatele] Adresa zapsána do obchodního rejstříku. (místo, kde má obchodní závod nebo bydliště)
        \item[Jednání podnikatele]
            \begin{itemize}
                \item []
                \item FO - jednají osobně nebo prostřednictvím zástupce
                \item PO - jednají prostřednictvím statutárních orgánů (jednatel, představenstvo\ldots) nebo prostřednictvím zástupce
            \end{itemize}
        \item[Prokura] širší plná moc.
            \begin{itemize}
                \item zmocňuje prokuristu ke všem právním úkonům (kromě prodeje nemovitosti)
                \item lze ji udělit pouze FO (spolumajiteli nebo zaměstnanci firmy)
                \item prokurista musí být zapsán v obchodním rejstříku
                \item praxi bývá prokurista druhý nejdůležitější muž firmy
            \end{itemize}
        \item[Prokurista] zástupce firmy s velmi širokou plnou mocí (prokurou)
        \item[Obchodní tajemství] Tvoří konkurenčně významné, určitelné, ocenitelné a v příslušných obchodních kruzích běžně nedostupné skutečnosti související se závodem a jejichž vlastník zajišťuje ve svém zájmu odpovídajícím způsobem jejich utajení.
        \item[Obchodní rejstřík] Je veřejný rejstřík, do kterého se zapisují zákonem stanovené údaje týkající se podnikatelů nebo organizačních složek jejich obchodních závodů, o nichž to stanoví zákon.
        \item[Konstitutivní funkce OR] Zápisem do OR vznikají právnické osoby. \par Do obchodního rejstříku se zapisují:
            \begin{enumerate}
                \item obchodní společnosti, družstva a jiné právnické osoby-povinně
                \item zahraniční osoby a jejich závody-povinně
                \item FO s trvalým pobytem v ČR, se zapíše do OR povinně nebo na vlastní žádost (chtějí mít prokuristu)			
            \end{enumerate}
        \item[Účetnictví podnikatelů] Podnikatelé vedou účetnictví v rozsahu a způsobem stanoveným zákonem (zákon o účetnictví), který v paragrafu 1 říká:
            \begin{itemize}
                \item Podnikatelé zapsaní v OR vedou povinně účetnictví
                \item FOsobratem nad 25 milionů vedou povinně účetnictví
                \item Ostatní podnikatelé mohou vést účetnictví dobrovolně nebo vedou daň.evidenci
            \end{itemize}
        \item[Účetní období]
            \begin{itemize}
                \item []
                \item Účetním obdobím je nepřetržitě po sobě jdoucích 12 měsíců. Buď se shoduje s kalendářním rokem nebo je hospodářským rokem (zemědělci-od sklizně do sklizně)
                \item Při přechodu na jiné účetní období může být délka delší nebo kratší než 12 měsíců
                \item Při vzniku nebo zániku obchodních společností, může být delší až 15 měsíců nebo kratší
                \item V případě přeměn obchodních společností, může být kratší nebo neomezeně delší
            \end{itemize}
        \item[Hospodářská soutěž (konkurence)] Je souběžná snaha subjektů na trhu určitého druhu zboží nebo služeb, cílem je dosažení určitých výhod před ostatními v oblasti hospodářských užitků.\par Zneužití účasti v hospodářské soutěži:
            \begin{enumerate}
                \item nekalá soutěž - upravuje občanský zákoník
                \begin{itemize}
                    \item Klamavá reklama
                    \item Podplácení
                    \item Zlehčování
                    \item Srovnávací reklama
                    \item Porušení obchodního tajemství
                \end{itemize}
                \item nedovolené omezování hospodářské soutěže - upravuje zákon o ochraně hosp.soutěže
                \begin{itemize}
                    \item Dohody soutěžitelů omezující hospodářskou soutěž
                    \item Dohody o sloučení soutěžitelů vedoucí k omezení konkurence
                    \item Zneužití monopolního nebo dominantního postavení soutěžitelů
                \end{itemize}
            \end{enumerate}
        \item[Dvoustupňové zahájení podnikání PO]
            \begin{enumerate}
                \item []
                \item ustavení (založení) obchodní společnosti (sepsání smlouvy mezi společníky)
                \item vznik obchodní společnosti (zápis do obchodního rejstříku)			
            \end{enumerate}
            V mezičase si společnost vyřizuje živnostenské oprávnění
        \item[Dvoustupňové ukončení podnikání PO]
            \begin{enumerate}
                \item []
                \item zrušení obchodní společnosti (s likvidací, bez likvidace, insolvenční řízení a úpadek)
                \item zánik obchodní společnosti (výmaz z obchodního rejstříku)
            \end{enumerate}
        \item[Zrušení s likvidací]
            \begin{itemize}
                \item []
                \item neexistuje právní nástupce (firma ukončením likvidace přestane existovat)
                \item statutární orgány jsou nahrazeny likvidátorem
                \item zjistí-li likvidátor, že je společnost v úpadku (předlužená \ldots), podá insolvenční návrh
            \end{itemize}
        \item[Zrušení bez likvidace]
            \begin{itemize}
                \item []
                \item existuje právní nástupce (sloučení firem, prodej firmy, rozdělení firmy)
                \item majetek přechází na právního nástupce
                \item firma udělá účetní závěrku, tím vyčíslí svůj majetek, závazky a pohledávky, které převezme nástupnická firma- původní firma zaniká
            \end{itemize}
        \item[Řešení úpadku]
            \begin{itemize}
                \item []
                \item věřitelé firmy mohou podat k soudu návrh na řešení úpadku dlužníka
                \item v případě prohlášení konkurzu je majetek firmy rozprodán a z peněz jsou hrazeny dluhy
                \item o platbě dluhů rozhoduje insolvenční soud
            \end{itemize}
    \end{description}

\section*{Ekonomika a neziskový sektor}
    \begin{description}
        \item[Neziskový sektor] Cílem subjektů neziskového sektoru není podnikat (dosahovat zisk), ale zajišťovat jiné funkce ve společnosti. V omezené míře však mohou subjekty NS i podnikat.
        \item[Principy rozdělování]
            \begin{itemize}
                \item []
                \item v ziskovém sektoru rozdělujeme výsledný produkt podle množství, kvality a tržní úspěšnosti práce
                \item v neziskovém sektoru rozdělujeme podle potřeb
            \end{itemize}
        \item[Státní neziskové organizace]
            \begin{itemize}
                \item []
                \item Státní školství
                \item Státní zdravotnictví
                \item Instituce na ochranu Životního prostředí, kulturních památek
                \item Celá oblast státní správy
            \end{itemize}
        \item[Nestátní neziskové organizace]
            \begin{itemize}
                \item []
                \item Církevní organizace (Centra pro rodinu, charity, semináře)
                \item Spolky (sdružení občanů)
                \item Ústavy
                \item Fundace (nadace, nadační fondy)
                \item Politické strany
            \end{itemize}
    \end{description}
