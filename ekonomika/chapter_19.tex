\chapter{Bankovní služby}

Úvěr - základní charakteristika

Úvěr je podobně jako zápůjčka formou dočasného postoupení peněžních prostředků věřitelem
na principu návratnosti dlužníkovi, který je ochoten za tuto půjčku po uplynutí nebo ještě v
průběhu doby splatnosti zaplatit určitý úrok.

Pojmy - úrok, úroková sazba, jistina, úrokové období, RPSN, likvidita, zadlužení,
předlužení, dluhová past, konsolidace úvěrů, bonita, registr dlužníků, oddlužení,
exekuce

Urok je peněžitá odměna za půjčení peněz.

Uroková sazba je procentní vyjádření zvýšení půjčené částky za určité časové období.
Uroková sazba určuje kolik z jistiny musí dlužník za předem smluvně stanovenou dobu
věřiteli za půjčku či úvěr zaplatit.

Jistina je základní peněžní částka, která byla půjčena nebo která tvořila vklad.
Úrokové období - Období, během kterého jsou připisovány úroky.

RPSN (roční procentní sazba nákladů) je číslo, které má umožnit spotřebiteli lépe vyhodnotit
výhodnost nebo nevýhodnost poskytovaného úvěru.

Likvidita - Likviditu lze definovat také jako míru schopnosti podniku přeměnit svá aktiva na
peněžní prostředky a těmi krýt včas, v požadované podobě a na požadovaném místě všechny
své splatné závazky, a to při minimálních nákladech.

Zadlužení Zadluženost je ekonomický pojem, který označuje skutečnost, že podnik používá
pro financování svých aktiv cizí kapitál.

Předlužení O předlužení jde tehdy, jestliže osoba má více věřitelů a jestliže její splatné
závazky jsou vyšší než její majetek; do ocenění dlužníkova majetku se zahrne i očekávaný
výnos z pokračující podnikatelské činnosti, lze-li příjem převyšující náklady při pokračování
podnikatelské činnosti důvodně předpokládat.

Zadlužení není předlužení. Zatímco zadlužení je normální, zdravé, umožňuje si pořídit nové
věci, vybavení a pomáhá ekonomice, předlužení je negativní patologická forma zadlužení,
která vede k ekonomickému zhroucení dlužníka.

Dluhová past je známější a používanější termín pro dluhovou spirálu. Jedná se o stav, kdy
není jednotlivec anebo 1 celá rodina schopna splácet závazky věřitelům. Což vede k penále,
anebo dalším závazkům formou braní si dalších úvěrů.

Konsolidace úvěrů, též konsolidace půjček, označuje sloučení více půjček do jedné.

Bonita je převrácená hodnota úvěrového rizika a vyjadřuje důvěryhodnost ekonomického
subjektu (firmy, jednotlivce, ale 1 obce nebo státu) na finančním trhu.

74
\newpage
Registr dlužníků (úvěrový registr) je databáze evidující zejména osoby a organizace, které
Jsou pozadu ve splácení svých závazků.

Oddlužení je zákonným prostředkem, kterým mohou dlužníci především fyzické osoby řešit
situaci, kdy mají více věřitelů a nedokáží plnit své splatné závazky. Oddlužením se řeší
úpadek fyzické či právnické osoby nepodnikatele.

Exekuce je nástroj, kterým je možné po někom, kdo dluží, domoci zaplacení dluhu (nebo i
jiných povinností, které neplní - nejen těch finančních), který dlužník dobrovolně neuhradil,
ačkoli měl a ačkoli mu takovou povinnost uložil (nejčastěji) soud v soudním rozhodnutí.

Druhy úvěrů podle splatnosti - kontokorentní, eskontní, akceptační, revolvingový,
lombardní, hypoteční úvěr, emisní úvěr, spotřební půjčky občanům

1. krátkodobé se splatností do 1 roku:

- kontokorentní úvěr - kombinace běžného účtu s možností čerpat krátkodobý úvěr do výše
úvěrového limitu stanoveného ve smlouvě o zřízení kontokorentního účtu

- eskontní úvěr - souvisí s eskontem (odkoupením) směnky klienta před dobou splatnosti
bankou. Nesplatí-l1 dlužník bance směnku v termínu splatnosti, žádá banka úhradu od
posledního majitele směnky (klienta, který směnku prodal)

- akceptační úvěr - banka neposkytne klientovi přímo peníze, ale akceptuje cizí směnku
vystavenou klientem - příjemcem akceptačního úvěru a tím se stává hlavním směnečným
dlužníkem. Akceptační úvěr dává banka jen nejlepším a ověřeným klientům

- revolvingový úvěr - banka umožňuje klientovi opakované čerpání úvěru. Základní
podmínkou je předešlé splacení úvěru.

- lombardní úvěr - úvěr jištěný zástavou movité věci (cenné papíry, zboží \ldots)

2. střednědobé a dlouhodobé - se splatností od 1 do 10 let

- hypoteční úvěr - jištěný hypotékou (zástavou nemovitostí)
- emisní úvěr - spojená s emisí dlouhodobých cenných papírů, např podnikových obligací
- spotřební půjčky občanům

Speciální formy úvěru - faktoring, forfaiting

Faktoring - odkup krátkodobých pohledávek (do 1 roku)
Forfaiting - odkup dlouhodobých pohledávek (nad 1 rok)

Druhy úvěru podle poskytované měny
Jsou úvěry v domácí nebo cizí měně (devizové)

Devizový úvěr

- Jde o požadavek komitenta o úvěr u zahraniční banky. Tyto zahraniční banky mají většinou
nižší úrokové sazby, na druhé straně však hrozí kursové riziko. Nehledě k tomu, že zahraniční
banky vyžadují záruku od českých bank.

- Také naše banky dávají účelové devizové úvěry. Vlastní postup spočívá v tom, že
zahraničnímu výstavci zaplatí česká banka z úvěru, který získá u cizí banky v tzv. úvěrová
linka. Zahraniční banky tak podporují vlastní ekonomiku.

7
\newpage
Druhy úvěru podle zajištěnosti

WWT?

Nezajištěný úvěr je z pohledu bamky velmi rizikový. Proto je tedy i dražší. Tento typ úvěru
se vyskytuje velmi vzácně a je poskytován pouze těm nejspolehlivějším klientům (VIP
klienti).

Zajištěný úvěr

Většina věřitelů chce mít 100 % jistotu, že své peníze dostane zpět. Proto požadují nějaké
zajištění. Tím se znatelně sníží riziko toho, že by se o půjčené peníze přišlo. Jinak věřitel
finance neposkytne. Banka zajišťuje své úvěry zástavou majetku dlužníka (movitý i
nemovitý). V případě, že pak dlužník řádně nesplácí, zástava bude bankou zabavena. Ručí se
formou zástavního práva k věci, popřípadě jde o zajištění další osobou. Zajištěné úvěry bývají
levnější než nezajištěné (nižší úroky). Navíc si věřitel může nechat prověřit klientovu bonitu
nebo stanovit pro úvěr limit. Mezi zajištěné úvěry může patřit úvěr hypotéční, spotřebitelský
nebo nebankovní úvěr na vyšší částku. Stejně tak sem spadají 1 neúčelové půjčky nebo
krátkodobý úvěr (zajištěný zástavou movitých věcí). Méně známé jsou lombardní a eskontní
úvěry, kdy je úvěr zajištěn cennými papíry, šperky či drahými kovy.

Druhy úvěrů podle způsobu čerpání

- jednorázové
- postupné
- před zápisem zástavního práva

Druhy úvěrů podle účelu a subjektů
1. úvěry pro podnikatelské účely

a. účelově - v úvěr. smlouvě je uvedeno, na co je úvěr poskytován = zásoby, pohledávky
b. neúčelově - otevření úvěr. Linky

2. úvěry občanům

- na nákup nemovitého 1 movitého majetku, užívají se zde krátko-, dlouho- i střednědobé
úvěry

- účelově zaměřené - na dům, rekonstrukci, nákup auta = dlohodobá spotřeba

- osobní - na překlenutí dočasného nedostatku financí, převodem na BŮ nebo v hotovosti
- kontokorentní - stanoven úvěrový rámec, podle kterého můžou občané čerpat úvěr na
bězném, na sporožirovém účtě \ldots

- úvěrové karty

3. mezibankovní úvěry

Banka může mít krátkodobě nevyrovnanou peněžní pozici z těchto důvodů:
- větší výběr vkladů z účtů zákazníků banky
- zákazníci banky včas nesplácí bance úvěry

76
\newpage
Dlouhodobě nevyrovnaná pozice:
- klienti u dané banky převážně čerpají úvěry
- klienti převážně ukládají vklady a banka to půjčuje jiným bankám

4. další úvěry
Hlavně obcím a městům, jedná se o středně a dlouhodobé investiční úvěry, krátkodobé
provozní úvěry na překlenutí nedostatku finančních prostředků.

Cíle banky při poskytování úvěru

1. Výnosnost - banka musí stanovit takové úroky z úvěru, aby vydělala. Úrok může být
stanoven pevný nebo pohyblivý.
2. Návratnost - banka se musí zajistit proti případnému nesplacení závazku dlužníka.

Postup při poskytnutí úvěru

Před poskytnutím úvěru si banka ověřuje Bonitu klienta, likviditu jeho majetku, podnikatelský
záměr.

Toto však ještě nestačí, ve většině případů chce banka určité záruky. jištění úvěru, a to ve
formě:

- zástavy nemovitosti (hypotéční úvěr)

- zástavy movitosti - cenné papíry, stroje, zásoby, \ldots (lombardní úvěr)

- ručitelé - ručit mohou 1 podnikatelské subjekty za podnikatelské úvěry, ale takové subjekty
se těžko shánějí

- vinkulace vkladu - zablokování vkladu dlužníka na účtu či na knížce ve prospěch věřitele
pro případ nesplacení

- postoupení pohledávek apod.

Uvěrová smlouva

obsah: název, jméno klienta, název banky, účel a výše úvěru, způsob čerpání, zajištění úvěru,
postup v případě neschopnosti klienta platit splátky, práva a povinnosti banky 1 klienta,
datum, místo, rizika a platné podpisy obou stran + přílohy

Doba splatnosti úvěru se počítá ode dne prvního čerpání úvěru

Kontrola dodržování podmínek úvěrové smlouvy
- sleduje se dodržování výše a termínů splátek úvěrů a úroků, pokud se nedodržují je dlužná
částka převedena na účet úvěrů neuhrazených ve lhůtě a je úročena vyšší sankční sazbou.

Další služby, které poskytují banky
Založení a vedení účtu - Běžné účty, termínované účty, devizové účty atd.

Bezhotovostní platební styk:

v tuzemsku - příkaz k úhradě - trvalý příkaz k úhradě
- příkaz k inkasu - trvalý příkaz k inkasu

JI