\chapter{Ekonomické pojmy}

\begin{description}
    \item[Ekonomie] společenská věda, zabývající se ekonomickými vztahy mezi lidmi
    \begin{itemize}
        \item jako věda se zabývá společenskou realitou zvanou ekonomika, ekonomická praxe
        \item využívá poznatků psychologie, sociologie, demografie, historie, managementu, marketingu, práva, politiky, hospodářského zeměpisu, etiky a také filozofie
    \end{itemize}
    \item[Ekonomika] se zabývá procesem, chováním jednotlivých subjektů, ekonomická teorie
    \item[Makroekonomie] se zabývá ekonomii státu, zkoumáním ekonomického systému jako celku, sleduje vztahy mezi agregátními veličinami
    \item[Mikroekonomie] se zabývá ekonomií firem (malé, střední, velké), domácností apod.
    \item[Ekonomický systém]
    \begin{description}
        \item[]
        \item[ZVYKOVÝ] rozhodoval vůdce kmene-náčelník, podle svých schopností
        \item[PŘÍKAZOVÝ] rozhodovala politická špička
        \item[TRŽNÍ] rozhoduje trh a jeho zákony
    \end{description}
    \item[Tři základní otázky]
    \begin{enumerate}
        \item[]
        \item Co a kolik se má ve společnosti vyrábět?
        \item Jak vyrábět,jakou technologií a jakými výrobními faktory?(práce,přírodní zdroje, kapitál)
        \item Jak se rozdělí to, co bylo vyrobeno?
    \end{enumerate}
    \item[Zákon vzácnosti] Co nemůžeme mít kdykoliv v jakémkoliv množství, je vzácné. Naše potřeby jsou neomezené, ale zdroje jsou omezené.
    \item[Zákon ekonomie času] Říká, že lidé musí šetřit svůj čas - snažit se v kratším čase dosáhnout stejné produkce nebo ve stejném
    čase dosáhnout vyšší produkce. Tomu říkáme zvyšování produktivity práce. Obětovaná příležitost.
    \item[Teorie potřeb] Každý z nás má jiné potřeby. Potřeba je pociťovaný nedostatek a hnací motor ekonomiky. Dělíme:
    \begin{itemize}
        \item Hmotné (jíst, bydlet, oblékat se\ldots)
        \item Nehmotné (svoboda, přátelství, kulturní zážitek\ldots)
        \item Zbytné (kaviár, loď\ldots)
        \item Nezbytné (spánek, voda, vzduch\ldots)
        \item Individuální (poslech rádia)
        \item Současné (nynější), budoucí (pozdější)
    \end{itemize}
    Potřeby uspokojujeme pomocí statků a služeb:
    
    \begin{tabular}{| p{8cm} | p{8cm} |}
        \hline
        \textbf{STATKY - určité předměty} & \textbf{SLUŽBY - cizí činnosti} \\
        \hline
        Hmotné (jídlo, byt, oblečení) \newline Nehmotné (vlastnosti, dovednosti, znalosti) \newline Volné (sluneční světlo, vzduch, déšť) \newline Ekonomické (světlo žárovky) \newline Kapitálové (budovy, stroje) \newline Spotřební (potraviny, uhlí, auta) & Věcné (oprava obuvi, vymalování bytu) \newline Osobní (kosmetika, lékař, kadeřník) \newline Zvířecí (kosmetika, lékař, manikůra) \\
        \hline
    \end{tabular}
    \item[Hospodářský proces] Členíme jej na fáze:
    \begin{enumerate}
        \item výroba
        \item rozdělování a přerozdělování
        \item směna
        \item spotřeba
    \end{enumerate}
    
    \item[Výroba] je činnost, při které člověk přetváří přírodu ve statky. Výrobu může provádět jednotlivec nebo celé výrobní firmy. \newline K výrobě potřebují mít čtyři základní předpoklady - \emph{VÝROBNÍ FAKTORY}.
    \begin{enumerate}
        \item PRÁCE - cenou je mzda
        \item PŘÍRODNÍ ZDROJE - především půdu
        \item KAPITÁL - je zisk nebo úrok
        \item INFORMACE
    \end{enumerate}
    \item[Rozdělování a přerozdělování]
    \item[Směna]
    \item[Spotřeba]
    \item[Práce] je cílevědomá lidská činnost vytvářející statky a služby. Práce je vzácný výrobní faktor, což ovlivňuje jeho cenu na trhu práce, tuto cenu nazýváme MZDOU.
    \item[Mzda reálná] mzda, vypovídá o skutečné hodnotě výdělku, co si zaměstnanec může pořídit.
    \item[Mzda nominální] mzda, kterou si pracovník vydělal jako zaměstnanec, když vynaložil svou pracovní sílu, jeho příjem peněz
    \item[Faktory ovlivňující výši mzdy]
    \begin{itemize}
        \item[]
        \item kvalifikace
        \item poptávka na trhu práce
        \item tržní úspěšnost
    \end{itemize}
    \item[Dělba práce] jednotlivec nevyrábí všechno, soustředí se pouze na určitou výrobu, jednotlivci jsou na sobě závislí
    \item[Specializace] zaměření na určitý okruh činnosti
    \item[Kooperace] vzájemná spolupráce
    \item[Přírodní zdroje (půda)] Při prodeji dosahuje vlastník tržní cenu. Při nájmu dosahuje pozemkovou rentu.
    \item[Pozemková renta]
    \begin{enumerate}[label=(\alph*)]
        \item[]
        \item absolutní renta - vyplývá z monopolu vlastnictví půdy a mají ji veškeré pozemky
        \item diferenční renta - závislá na kvalitě půdy a její vhodnost pro zemědělství-bonita půdy
    \end{enumerate}
    
    \item[Kapitál] peníze přinášející další peníze. Má 2 ceny:
    \begin{description}
        \item[Úrok] je cena vloženého (např. do banky) kapitálu
        \item[Zisk] je cena, kterou očekává vlastník při aktivním podnikání
    \end{description}
    \item[Hranice produkčních možností] Pokud společnost nemůže vyrábět se svými zdroji více určitých výrobků, aniž by snížila výrobu jiných výrobků. Společnost vyrábí efektivně, pokud se pohybuje na hranici produkčních možností
    \item[Spotřeba] Spotřeba je závěrečná fáze hospodářského procesu. Spotřebou statků a služeb uspokojujeme naše potřeby.
    \item[Spotřeba výrobní] Je to spotřeba firem (stroje, kancelářský papír, materiál).
    \item[Spotřeba konečná] Je spotřeba lidí, občanů, domácností (jídlo, pití, bydlení).
\end{description}

